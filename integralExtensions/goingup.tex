\documentclass{ximera}



\usepackage{tikz-cd}
\usepackage[sans]{dsfont}

\DefineVerbatimEnvironment{macaulay2}{Verbatim}{numbers=left,frame=lines,label=Macaulay2,labelposition=topline}

%%% This next bit of code defines all our theorem environments
\makeatletter
\let\c@theorem\relax
\let\c@corollary\relax
\makeatother

\let\definition\relax
\let\enddefinition\relax

\let\theorem\relax
\let\endtheorem\relax

\let\proposition\relax
\let\endproposition\relax

\let\exercise\relax
\let\endexercise\relax

\let\question\relax
\let\endquestion\relax

\let\remark\relax
\let\endremark\relax

\let\corollary\relax
\let\endcorollary\relax


\let\example\relax
\let\endexample\relax


\let\lemma\relax
\let\endlemma\relax

\newtheoremstyle{SlantTheorem}{\topsep}{\topsep}%%% space between body and thm
		{\slshape}                      %%% Thm body font
		{}                              %%% Indent amount (empty = no indent)
		{\bfseries\sffamily}            %%% Thm head font
		{}                              %%% Punctuation after thm head
		{3ex}                           %%% Space after thm head
		{\thmname{#1}\thmnumber{ #2}\thmnote{ \bfseries(#3)}}%%% Thm head spec
\theoremstyle{SlantTheorem}
\newtheorem{theorem}{Theorem}
\newtheorem{definition}[theorem]{Definition}
\newtheorem{proposition}[theorem]{Proposition}
%% \newtheorem*{dfnn}{Definition}
%% \newtheorem{ques}{Question}[theorem]
\newtheorem{lemma}[theorem]{Lemma}
%% \newtheorem*{war}{WARNING}
%% \newtheorem*{cor}{Corollary}
%% \newtheorem*{eg}{Example}
\newtheorem*{remark}{Remark}
\newtheorem*{touchstone}{Touchstone}
\newtheorem{corollary}{Corollary}[theorem]
\newtheorem*{example}{Example}


\newtheoremstyle{Exercise}{\topsep}{\topsep} %%% space between body and thm
		{}                           %%% Thm body font
		{}                           %%% Indent amount (empty = no indent)
		{\bfseries}                  %%% Thm head font
		{)}                          %%% Punctuation after thm head
		{ }                          %%% Space after thm head
		{\thmnumber{#2}\thmnote{ \bfseries(#3)}}%%% Thm head spec
\theoremstyle{Exercise}
\newtheorem{exercise}{}[theorem]

%% \newtheoremstyle{Question}{\topsep}{\topsep} %%% space between body and thm
%% 		{\bfseries}                  %%% Thm body font
%% 		{3ex}                        %%% Indent amount (empty = no indent)
%% 		{}                           %%% Thm head font
%% 		{}                           %%% Punctuation after thm head
%% 		{}                           %%% Space after thm head
%% 		{\thmnumber{#2}\thmnote{ \bfseries(#3)}}%%% Thm head spec
\newtheoremstyle{Question}{3em}{3em} %%% space between body and thm
		{\large\bfseries}                           %%% Thm body font
		{3ex}                           %%% Indent amount (empty = no indent)
		{\bfseries}                  %%% Thm head font
		{}                          %%% Punctuation after thm head
		{ }                          %%% Space after thm head
		{}%%% Thm head spec
\theoremstyle{Question}
\newtheorem*{question}{}



\renewcommand{\tilde}{\widetilde}
\renewcommand{\bar}{\overline}
\renewcommand{\hat}{\widehat}
\newcommand{\N}{\mathbb N}
\newcommand{\Z}{\mathbb Z}
\newcommand{\R}{\mathbb R}
\newcommand{\Q}{\mathbb Q}
\newcommand{\C}{\mathbb C}
\newcommand{\V}{\mathbb V}
\newcommand{\I}{\mathbb I}
\newcommand{\A}{\mathbb A}
\newcommand{\iso}{\simeq}
\newcommand{\ph}{\varphi}
\newcommand{\Cf}{\mathcal{C}}
\newcommand{\IZ}{\mathrm{Int}(\Z)}
\newcommand{\dsum}{\oplus}
\newcommand{\directsum}{\coprod}
\newcommand{\union}{\bigcup}
\renewcommand{\i}{\mathfrak}
\renewcommand{\a}{\mathfrak{a}}
\renewcommand{\b}{\mathfrak{b}}
\newcommand{\m}{\mathfrak{m}}
\newcommand{\p}{\mathfrak{p}}
\newcommand{\q}{\mathfrak{q}}
\newcommand{\dfn}{\textbf}
\let\hom\relax
\DeclareMathOperator{\ann}{Ann}
\DeclareMathOperator{\h}{ht}
\DeclareMathOperator{\hom}{Hom}
\DeclareMathOperator{\spec}{Spec}
\DeclareMathOperator{\supp}{Supp}
\DeclareMathOperator{\ass}{Ass}
\DeclareMathOperator{\ff}{Frac}
\DeclareMathOperator{\im}{Im}
\DeclareMathOperator{\syz}{Syz}
\DeclareMathOperator{\gr}{Gr}
\renewcommand{\ker}{\mathop{\mathrm{Ker}}\nolimits}
\newcommand{\lps}{[\hspace{-0.25ex}[}
\newcommand{\rps}{]\hspace{-0.25ex}]}
\newcommand{\into}{\hookrightarrow}
\newcommand{\onto}{\twoheadrightarrow}
\newcommand{\tensor}{\otimes}
\newcommand{\x}{\mathbf{x}}
\newcommand{\X}{\mathbf X}
\newcommand{\Y}{\mathbf Y}
\renewcommand{\k}{\boldsymbol{\kappa}}
\renewcommand{\emptyset}{\varnothing}
\renewcommand{\qedsymbol}{$\blacksquare$}
\renewcommand{\l}{\ell}
\newcommand{\1}{\mathds{1}}
\newcommand{\lto}{\mathop{\longrightarrow\,}\limits}
\newcommand{\rad}{\sqrt}
\renewcommand{\vec}{\mathbf}
\renewcommand{\phi}{\varphi}
\renewcommand{\epsilon}{\varepsilon}
\renewcommand{\subset}{\subseteq}
\renewcommand{\supset}{\supseteq}
\newcommand{\macaulay}{\textsl{Macaulay2}}
\newcommand{\invlim}{\varprojlim}


%\renewcommand{\proofname}{Sketch of Proof}


\renewenvironment{proof}[1][Proof]
  {\begin{trivlist}\item[\hskip \labelsep \itshape \bfseries #1{}\hspace{2ex}]\upshape}
{\qed\end{trivlist}}

\newenvironment{sketch}[1][Sketch of Proof]
  {\begin{trivlist}\item[\hskip \labelsep \itshape \bfseries #1{}\hspace{2ex}]\upshape}
{\qed\end{trivlist}}



\makeatletter
\renewcommand\section{\@startsection{paragraph}{10}{\z@}%
                                     {-3.25ex\@plus -1ex \@minus -.2ex}%
                                     {1.5ex \@plus .2ex}%
                                     {\normalfont\large\sffamily\bfseries}}
\renewcommand\subsection{\@startsection{subparagraph}{10}{\z@}%
                                    {3.25ex \@plus1ex \@minus.2ex}%
                                    {-1em}%
                                    {\normalfont\normalsize\sffamily\bfseries}}
\makeatother

%% Fix weird index/bib issue.
\makeatletter
\gdef\ttl@savemark{\sectionmark{}}
\makeatother


\author{Bart Snapp}

\title{Going-up}

\begin{document}
\begin{abstract}
  We analyize the lying-over and going-up properties. Sources and references:
  \cite{iK1966}.
\end{abstract}
\maketitle


We begin with another example of Eisenbud's phrase: ``ideals maximal
with respect to some property have an uncanny tendency to be prime.''


\begin{proposition}
  Let $R\subset S$ be rings and let $\p\in\spec(R)$. Consider the set
  \[
  \mathcal{S} = \{I\subset S: \text{$I$ is an ideal of $S$ and $(R-\p)\cap I = \emptyset$}\}.
  \]
  If $S$ has a maximal element $P$, then $P\in\spec(S)$.
  \begin{proof}
    Let $P$ be a maximal element of $\mathcal{S}$. Seeking a
    contradiction, suppose that $xy\in P$ but $x\notin P$ and $y\notin
    P$. This means that $x\in R-\p$ and $y\in R-\p$, but this means
    that $xy\in R-\p$, a contradiction. Hence $P\in\spec(S)$.
  \end{proof}
\end{proposition}

\begin{exercise}
  If $R\subset S$ and $P\in\spec(S)$, show that $P\cap R\in\spec(R)$.
\end{exercise}


\begin{proposition}
  Let $R\subset S$ be rings and $\p\in\spec(R)$. The ring $S$ has the
  going-up property over $R$ if and only if $P\in\spec(S)$ is maximal in
  \[
  \mathcal{S} = \{I\subset S: \text{$I$ is an ideal of $S$ and
    $(R-\p)\cap I = \emptyset$}\}
  \]
  implies that $P\cap R = \p$.
  \begin{proof}
    $(\Rightarrow)$ Suppose $P\in\spec{S}$ is maximal in
    $\mathcal{S}$. We must show that the going-up property holds for
    $\p$.

    First note that $P\cap R\in\spec(R)$. Note, by the construction of
    $P$, $P\cap R$ cannot properly contain $\p$. Hence if $P\cap
    R\subset \p$, then enlarge $P\cap R$ to $\p$. By going-up, there
    exists $P'\in\spec(S)$ with $P'\cap R = \p$. Since $P$ is maximal
    in $\mathcal{S}$, we gave that $P=P'$.

    $(\Leftarrow)$ Suppose that $\p'\subset\p\in\spec(R)$ and that
    $P'\in\spec(S)$ with $P'\cap R = \p'$. Note, $P'\in\mathcal{S}$,
    taking the maximal element $P\in\mathcal{S}$, we have that $P\cap
    R = \p$. Hence $R\subset S$ has the going up property.
  \end{proof}
\end{proposition}


\begin{corollary}
  If a pair of rings $R\subset S$ has the going-up property, then they
  have the lying-over property.
  \begin{proof}
    We can always find a maximal element of sets like $\mathcal{S}$,
    so by our work above, going-up implies lying-over.
  \end{proof}
\end{corollary}


\begin{proposition}
  Let $R\subset S$ be rings and $\p\in\spec(R)$. The ring $S$ is
  incomparable over $R$ if and only if $P\in\spec(S)$ $P\cap R =\p$
  implies $P$ is maximal in
  \[
  \mathcal{S} = \{I\subset S: \text{$I$ is an ideal of $S$ and
    $(R-\p)\cap I = \emptyset$}\}.
  \]
\end{proposition}


\begin{theorem}
  Let $R\subset S$ be rings with $S$ integral over $R$. In this case,
  \begin{enumerate}
  \item The pair $R\subset S$ has the lying-over property.
  \item The pair $R\subset S$ has the incomparable property.
  \item The pair $R\subset S$ has the going-up-property.
  \end{enumerate}
\end{theorem}



\end{document}
