\documentclass{ximera}



\usepackage{tikz-cd}
\usepackage[sans]{dsfont}

\let\oldbibliography\thebibliography%% to compact bib
\renewcommand{\thebibliography}[1]{%
  \oldbibliography{#1}%
  \setlength{\itemsep}{0pt}%
}


\DefineVerbatimEnvironment{macaulay2}{Verbatim}{numbers=left,frame=lines,label=Macaulay2,labelposition=topline}

%%% This next bit of code defines all our theorem environments
\makeatletter
\let\c@theorem\relax
\let\c@corollary\relax
\makeatother

\let\definition\relax
\let\enddefinition\relax

\let\theorem\relax
\let\endtheorem\relax

\let\proposition\relax
\let\endproposition\relax

\let\exercise\relax
\let\endexercise\relax

\let\question\relax
\let\endquestion\relax

\let\remark\relax
\let\endremark\relax

\let\corollary\relax
\let\endcorollary\relax


\let\example\relax
\let\endexample\relax

\let\warning\relax
\let\endwarning\relax

\let\lemma\relax
\let\endlemma\relax

\newtheoremstyle{SlantTheorem}{\topsep}{\topsep}%%% space between body and thm
		{\slshape}                      %%% Thm body font
		{}                              %%% Indent amount (empty = no indent)
		{\bfseries\sffamily}            %%% Thm head font
		{}                              %%% Punctuation after thm head
		{3ex}                           %%% Space after thm head
		{\thmname{#1}\thmnumber{ #2}\thmnote{ \bfseries(#3)}}%%% Thm head spec
\theoremstyle{SlantTheorem}
\newtheorem{theorem}{Theorem}
\newtheorem{definition}[theorem]{Definition}
\newtheorem{proposition}[theorem]{Proposition}
%% \newtheorem*{dfnn}{Definition}
%% \newtheorem{ques}{Question}[theorem]
\newtheorem{lemma}[theorem]{Lemma}
%% \newtheorem*{war}{WARNING}
%% \newtheorem*{cor}{Corollary}
%% \newtheorem*{eg}{Example}
\newtheorem*{remark}{Remark}
\newtheorem*{touchstone}{Touchstone}
\newtheorem{corollary}{Corollary}[theorem]
\newtheorem*{example}{Example}
\newtheorem*{warning}{WARNING}


\newtheoremstyle{Exercise}{\topsep}{\topsep} %%% space between body and thm
		{}                           %%% Thm body font
		{}                           %%% Indent amount (empty = no indent)
		{\bfseries}                  %%% Thm head font
		{)}                          %%% Punctuation after thm head
		{ }                          %%% Space after thm head
		{\thmnumber{#2}\thmnote{ \bfseries(#3)}}%%% Thm head spec
\theoremstyle{Exercise}
\newtheorem{exercise}{}[theorem]

%% \newtheoremstyle{Question}{\topsep}{\topsep} %%% space between body and thm
%% 		{\bfseries}                  %%% Thm body font
%% 		{3ex}                        %%% Indent amount (empty = no indent)
%% 		{}                           %%% Thm head font
%% 		{}                           %%% Punctuation after thm head
%% 		{}                           %%% Space after thm head
%% 		{\thmnumber{#2}\thmnote{ \bfseries(#3)}}%%% Thm head spec
\newtheoremstyle{Question}{3em}{3em} %%% space between body and thm
		{\large\bfseries}                           %%% Thm body font
		{3ex}                           %%% Indent amount (empty = no indent)
		{\bfseries}                  %%% Thm head font
		{}                          %%% Punctuation after thm head
		{ }                          %%% Space after thm head
		{}%%% Thm head spec
\theoremstyle{Question}
\newtheorem*{question}{}



\renewcommand{\tilde}{\widetilde}
\renewcommand{\bar}{\overline}
\renewcommand{\hat}{\widehat}
\newcommand{\N}{\mathbb N}
\newcommand{\Z}{\mathbb Z}
\newcommand{\R}{\mathbb R}
\newcommand{\Q}{\mathbb Q}
\newcommand{\C}{\mathbb C}
\newcommand{\V}{\mathbb V}
\newcommand{\I}{\mathbb I}
\newcommand{\A}{\mathbb A}
\newcommand{\iso}{\simeq}
\newcommand{\ph}{\varphi}
\newcommand{\Cf}{\mathcal{C}}
\newcommand{\IZ}{\mathrm{Int}(\Z)}
\newcommand{\dsum}{\oplus}
\newcommand{\directsum}{\bigoplus}
\newcommand{\union}{\bigcup}
\renewcommand{\i}{\mathfrak}
\renewcommand{\a}{\mathfrak{a}}
\renewcommand{\b}{\mathfrak{b}}
\newcommand{\m}{\mathfrak{m}}
\newcommand{\p}{\mathfrak{p}}
\newcommand{\q}{\mathfrak{q}}
\newcommand{\dfn}[1]{\textbf{#1}\index{#1}}
\let\hom\relax
\DeclareMathOperator{\ann}{Ann}
\DeclareMathOperator{\h}{ht}
\DeclareMathOperator{\hom}{Hom}
\DeclareMathOperator{\Span}{Span}
\DeclareMathOperator{\spec}{Spec}
\DeclareMathOperator{\maxspec}{MaxSpec}
\DeclareMathOperator{\supp}{Supp}
\DeclareMathOperator{\ass}{Ass}
\DeclareMathOperator{\ff}{Frac}
\DeclareMathOperator{\im}{Im}
\DeclareMathOperator{\syz}{Syz}
\DeclareMathOperator{\gr}{Gr}
\renewcommand{\ker}{\mathop{\mathrm{Ker}}\nolimits}
\newcommand{\coker}{\mathop{\mathrm{Coker}}\nolimits}
\newcommand{\lps}{[\hspace{-0.25ex}[}
\newcommand{\rps}{]\hspace{-0.25ex}]}
\newcommand{\into}{\hookrightarrow}
\newcommand{\onto}{\twoheadrightarrow}
\newcommand{\tensor}{\otimes}
\newcommand{\x}{\mathbf{x}}
\newcommand{\X}{\mathbf X}
\newcommand{\Y}{\mathbf Y}
\renewcommand{\k}{\boldsymbol{\kappa}}
\renewcommand{\emptyset}{\varnothing}
\renewcommand{\qedsymbol}{$\blacksquare$}
\renewcommand{\l}{\ell}
\newcommand{\1}{\mathds{1}}
\newcommand{\lto}{\mathop{\longrightarrow\,}\limits}
\newcommand{\rad}{\sqrt}
\newcommand{\hf}{H}
\newcommand{\hs}{H\!S}
\newcommand{\hp}{H\!P}
\renewcommand{\vec}{\mathbf}
\renewcommand{\phi}{\varphi}
\renewcommand{\epsilon}{\varepsilon}
\renewcommand{\subset}{\subseteq}
\renewcommand{\supset}{\supseteq}
\newcommand{\macaulay}{\textsl{Macaulay2}}
\newcommand{\invlim}{\varprojlim}


%\renewcommand{\proofname}{Sketch of Proof}


\renewenvironment{proof}[1][Proof]
  {\begin{trivlist}\item[\hskip \labelsep \itshape \bfseries #1{}\hspace{2ex}]\upshape}
{\qed\end{trivlist}}

\newenvironment{sketch}[1][Sketch of Proof]
  {\begin{trivlist}\item[\hskip \labelsep \itshape \bfseries #1{}\hspace{2ex}]\upshape}
{\qed\end{trivlist}}



\makeatletter
\renewcommand\section{\@startsection{paragraph}{10}{\z@}%
                                     {-3.25ex\@plus -1ex \@minus -.2ex}%
                                     {1.5ex \@plus .2ex}%
                                     {\normalfont\large\sffamily\bfseries}}
\renewcommand\subsection{\@startsection{subparagraph}{10}{\z@}%
                                    {3.25ex \@plus1ex \@minus.2ex}%
                                    {-1em}%
                                    {\normalfont\normalsize\sffamily\bfseries}}
\makeatother

%% Fix weird index/bib issue.
\makeatletter
\gdef\ttl@savemark{\sectionmark{}}
\makeatother


\makeatletter
%% no number for refs
\newcommand\frontstyle{%
  \def\activitystyle{activity-chapter}
  \def\maketitle{%
    \addtocounter{titlenumber}{1}%
                    {\flushleft\small\sffamily\bfseries\@pretitle\par\vspace{-1.5em}}%
                    {\flushleft\LARGE\sffamily\bfseries\@title \par }%
                    {\vskip .6em\noindent\textit\theabstract\setcounter{problem}{0}\setcounter{sectiontitlenumber}{0}}%
                    \par\vspace{2em}
                    \phantomsection\addcontentsline{toc}{section}{\textbf{\@title}}%
                  }}
\makeatother


\author{Bart Snapp}

\title{Going-down}

\begin{document}
\begin{abstract}
  We analyze the going-down property. Sources and references:
  \cite{AM1969}.
\end{abstract}
\maketitle

The going-down property also is closely related to integral
extensions. Let's recall the going-down property:
\begin{description}
\item[Going-down]\index{going-down} If $\p_1,\p_2\in\spec(R)$ with $\p_1\subset\p_2$ and
  $P_2\cap R = \p_2$, then there exists $P_1$ with $P_1\subset P_2$
  and $P_1\cap R = \p_1$.
  \[
  \begin{tikzcd}[row sep=0em,column sep=0em]
    R    & \subset & S \\
    \cup &         & \cup \\
    \p_2 & \subset & P_2 \\
    \cup &         & \cup \\
    \p_1 & \subset & \exists P_1  
  \end{tikzcd}
  \]
\end{description}


We will strengthen Proposition~\ref{P:localint}.


\begin{proposition}[Integral closure and localization]\label{P:localintS}
  Let $R\subset S$ with $A$ being the integral closure of $R$
  in $S$. If $U\subset R$ is a multiplicatively closed set, then
  $U^{-1}A$ is the integral closure of $U^{-1}R$ in $U^{-1}S$.
  \begin{proof}
    First note that by Proposition~\ref{P:localint}, $U^{-1} A$ is
    integral over $U^{-1} R$. Now suppose there is an element $s/u\in
    U^{-1} S$, that is integral over $R$.  We must show that this
    element is in $S^{-1} A$. Consider the integral dependence
    equation for $s/u$ over $S^{-1} R$
    \[
    (s/u)^{n} + \left(r_{n-1}/u_{n-1}\right)(s/u)^{n-1}+\cdots +  \left(r_{1}/u_{1}\right)(s/u) + r_{0}/u_{0} = 0
    \]
    where $r_i\in R$ and $u_i\in U$. Clear denominators, by
    multiplying by $(uu_{n-1}\cdots u_0)^n$, then this is the integral
    dependence equation for $su_{n-1}\cdots u_0$. Since $A$ is the
    integral closure of $R$, we see that
    \[
    su_{n-1}\cdots u_0\in A
    \]
    and hence $s/u\in U^{-1} A$.
  \end{proof}
\end{proposition}


\begin{definition}
  Let $A$ be a domain. The \dfn{integral closure} of $A$, without
  reference to another ring, means the integral closure of $A$ in its
  field of fractions $\ff(A)$. In this case we write $\tilde{A}$ for
  the integral closure.
\end{definition}



\begin{proposition}
  Let $A$ be an integral domain, the following are equivalent:
  \begin{enumerate}
  \item $A$ is integrally closed.
  \item $A_\p$ is integrally closed for all $\p\in\spec(A)$.
  \item $A_\m$ is integrally closed for all maximal ideals.
  \end{enumerate}
  \begin{proof}
    (a)$\Rightarrow$(b)$\wedge$(c) Follows from Proposition~\ref{P:localintS}.

    (b)$\vee$(c)$\Rightarrow$(a) Consider $\iota: A\into\tilde{A}$ via
    the identity map. $A$ is integrally closed if and only if $\iota$
    is surjective. However, if $A_\p\into\tilde{A}_\p$ is surjective
    for all prime ideals (similarly maximal ideals), then
    $A\into\tilde{A}$ is surjective.
  \end{proof}
\end{proposition}


\begin{definition}
   Let $R\subset S$ with $I\subset R$ an ideal. The \dfn{integral
     closure of an ideal} $I$ in $S$ is the set of all elements of $S$
   that are integral over $I$.
\end{definition}


\begin{proposition}\label{P:intideal}
  Let $R\subset A\subset S$ be rings with $A$ being the integral
  closure of $R$ in $S$. Let $I$ be an ideal of $R$. Then
  \[
  \{\text{integral closure of $I$ in $S$}\} = \rad{IA}.
  \]
  \begin{proof}
    $(\subset)$ Suppose that $x$ is integral over $I$. Then
    \[
    x^n + a_{n-1}x^{n-1} + \dots + a_1 x + a_0 = 0
    \]
    for $a_i\in I$. Hence, $x\in A$, and $x^n\in IA$, so $x\in
    \rad{IA}$.

    $(\supset)$ Suppose that $x\in\rad{IA}$. This means that
    \[
    x^n = \sum_{i=1}^m a_i x_i
    \]
    where $a_i\in I$ and $x_i\in A$. Set $M = R[x_1,\dots,
      x_m]$. Since each $x_i$ is integral over $R$, $M$ is finitely
    generated over $R$. Now, $x^n M \subset I M$, hence by the
    determinant trick (Proposition~\ref{P:determinanttrick}) where
    $\phi$ is multiplication by $x^n$, we have that $x^n$ is integral
    over $R$, and we are done.
  \end{proof}
\end{proposition}


\begin{proposition}\label{P:intff}
  Let $A\subset B$ be integral domains with $A = \tilde{A}$. Let $b\in
  B$ be integral over an ideal $I\subset A$.
  \begin{enumerate}
  \item The element $b$ is algebraic over $\ff(A)$.
  \item If $x^n + a_{n-1}x^{n-1} + \dots + a_1 x + a_0$ is a minimal
    polynomial for $b$ over $\ff(A)$, then $a_i\in \rad{I}$.
  \end{enumerate}
  \begin{proof}
    It follows directly that $b$ is algebraic over $\ff(A)$.

    Let $\ff(A)\subset K$ an extension field containing all of the
    conjugates of $x$, call them $x_1,\dots, x_n$. Each $x_i$ is
    integral over $A$ via the same integral dependence equation. The
    coefficients of the integral dependence equation are polynomials
    in the $x_i$. By Proposition~\ref{P:intideal}, they are integral
    over $I$, and are in $\rad{I}$.
  \end{proof}
\end{proposition}


\begin{theorem}[Going-down]
  If $A\subset B$ are integral domains, $A=\tilde{A}$, and $B$ is
  integral over $A$, then the going-down property holds.
  \begin{proof}
    We have the following setup:
    \[
    \begin{tikzcd}[row sep=0em,column sep=0em]
      A=\tilde{A}    & \subset & B \\
      \cup &         & \cup \\
      \p_2 & \subset & P_2 \\
      \cup &         & \cup \\
      \p_1 & \subset & \exists P_1  
    \end{tikzcd}
    \]
    We must show that $P_1$ exists. We claim that $P_1 = \p_1 B_{P_2}$.

    Consider $x\in \p_1 B_{P_2}$. This means that $x = y/u$ where
    $u\in B-P_2$ and $y\in \p_1 B$. By Proposition~\ref{P:intideal},
    $y$ is integral over $\p_1$, and by Proposition~\ref{P:intff}, its
    minimal polynomial over $\ff(A)$ is
    \[
    y^n + p_{n-1}y^{n-1} + \dots + p_1 y + p_0 = 0
    \]
    where each $p_i\in\p_1$.

    Suppose further that $x\in \p_1 B_{P_2}\cap A$. Now since $x =
    y/u$, we have $u = y/x$, where $x\in\ff(A)$. Now the minimal
    polynomial for $u$ can be found by dividing the polynomial above
    by $x^n$ to get
    \[
    u^n + q_{n-1} u^{n-1} + \dots + q_1 u + q_0 = 0
    \]
    where $q_i = p_i/x^i$. Thus
    \[
    x^i q_i = p_i \in \p_1.
    \]
    But $u$ is integral over $A$, so each $q_i\in A$. Now suppose that
    $x\notin\p_1$, then the equation immediately above shows that
    $q_i\in\p_1$, and so the equation above that shows
    \[
    u^n\in \p_1 B \subset \p_2 B \subset P_2
    \]
    and so $u\in P_2$, a contradiction. Hence $x\in \p_1$, and so
    $\p_1 B_{P_2}\cap A = \p_1$.
  \end{proof}
\end{theorem}



\begin{corollary}
  Let $R\subset S$ have the going-down property. If $\p\in\spec{R}$
  and $P\in\spec(S)$, then $\h(P)= \h(\p)$.
\end{corollary}



For some interesting extra reading check out:
\begin{itemize}
\item \link[\textit{Prime ideals and integral dependence}, I.S.\ Cohen and A.\ Seidenberg, Bulletin of the American Mathematical Society, 52, 1946, 252--261]{https://doi-org.proxy.lib.ohio-state.edu/10.1090/S0002-9904-1946-08552-3}.
\end{itemize}


\end{document}
