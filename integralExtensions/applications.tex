\documentclass{ximera}



\usepackage{tikz-cd}
\usepackage[sans]{dsfont}

\let\oldbibliography\thebibliography%% to compact bib
\renewcommand{\thebibliography}[1]{%
  \oldbibliography{#1}%
  \setlength{\itemsep}{0pt}%
}


\DefineVerbatimEnvironment{macaulay2}{Verbatim}{numbers=left,frame=lines,label=Macaulay2,labelposition=topline}

%%% This next bit of code defines all our theorem environments
\makeatletter
\let\c@theorem\relax
\let\c@corollary\relax
\makeatother

\let\definition\relax
\let\enddefinition\relax

\let\theorem\relax
\let\endtheorem\relax

\let\proposition\relax
\let\endproposition\relax

\let\exercise\relax
\let\endexercise\relax

\let\question\relax
\let\endquestion\relax

\let\remark\relax
\let\endremark\relax

\let\corollary\relax
\let\endcorollary\relax


\let\example\relax
\let\endexample\relax

\let\warning\relax
\let\endwarning\relax

\let\lemma\relax
\let\endlemma\relax

\newtheoremstyle{SlantTheorem}{\topsep}{\topsep}%%% space between body and thm
		{\slshape}                      %%% Thm body font
		{}                              %%% Indent amount (empty = no indent)
		{\bfseries\sffamily}            %%% Thm head font
		{}                              %%% Punctuation after thm head
		{3ex}                           %%% Space after thm head
		{\thmname{#1}\thmnumber{ #2}\thmnote{ \bfseries(#3)}}%%% Thm head spec
\theoremstyle{SlantTheorem}
\newtheorem{theorem}{Theorem}
\newtheorem{definition}[theorem]{Definition}
\newtheorem{proposition}[theorem]{Proposition}
%% \newtheorem*{dfnn}{Definition}
%% \newtheorem{ques}{Question}[theorem]
\newtheorem{lemma}[theorem]{Lemma}
%% \newtheorem*{war}{WARNING}
%% \newtheorem*{cor}{Corollary}
%% \newtheorem*{eg}{Example}
\newtheorem*{remark}{Remark}
\newtheorem*{touchstone}{Touchstone}
\newtheorem{corollary}{Corollary}[theorem]
\newtheorem*{example}{Example}
\newtheorem*{warning}{WARNING}


\newtheoremstyle{Exercise}{\topsep}{\topsep} %%% space between body and thm
		{}                           %%% Thm body font
		{}                           %%% Indent amount (empty = no indent)
		{\bfseries}                  %%% Thm head font
		{)}                          %%% Punctuation after thm head
		{ }                          %%% Space after thm head
		{\thmnumber{#2}\thmnote{ \bfseries(#3)}}%%% Thm head spec
\theoremstyle{Exercise}
\newtheorem{exercise}{}[theorem]

%% \newtheoremstyle{Question}{\topsep}{\topsep} %%% space between body and thm
%% 		{\bfseries}                  %%% Thm body font
%% 		{3ex}                        %%% Indent amount (empty = no indent)
%% 		{}                           %%% Thm head font
%% 		{}                           %%% Punctuation after thm head
%% 		{}                           %%% Space after thm head
%% 		{\thmnumber{#2}\thmnote{ \bfseries(#3)}}%%% Thm head spec
\newtheoremstyle{Question}{3em}{3em} %%% space between body and thm
		{\large\bfseries}                           %%% Thm body font
		{3ex}                           %%% Indent amount (empty = no indent)
		{\bfseries}                  %%% Thm head font
		{}                          %%% Punctuation after thm head
		{ }                          %%% Space after thm head
		{}%%% Thm head spec
\theoremstyle{Question}
\newtheorem*{question}{}



\renewcommand{\tilde}{\widetilde}
\renewcommand{\bar}{\overline}
\renewcommand{\hat}{\widehat}
\newcommand{\N}{\mathbb N}
\newcommand{\Z}{\mathbb Z}
\newcommand{\R}{\mathbb R}
\newcommand{\Q}{\mathbb Q}
\newcommand{\C}{\mathbb C}
\newcommand{\V}{\mathbb V}
\newcommand{\I}{\mathbb I}
\newcommand{\A}{\mathbb A}
\newcommand{\iso}{\simeq}
\newcommand{\ph}{\varphi}
\newcommand{\Cf}{\mathcal{C}}
\newcommand{\IZ}{\mathrm{Int}(\Z)}
\newcommand{\dsum}{\oplus}
\newcommand{\directsum}{\bigoplus}
\newcommand{\union}{\bigcup}
\renewcommand{\i}{\mathfrak}
\renewcommand{\a}{\mathfrak{a}}
\renewcommand{\b}{\mathfrak{b}}
\newcommand{\m}{\mathfrak{m}}
\newcommand{\p}{\mathfrak{p}}
\newcommand{\q}{\mathfrak{q}}
\newcommand{\dfn}[1]{\textbf{#1}\index{#1}}
\let\hom\relax
\DeclareMathOperator{\ann}{Ann}
\DeclareMathOperator{\h}{ht}
\DeclareMathOperator{\hom}{Hom}
\DeclareMathOperator{\Span}{Span}
\DeclareMathOperator{\spec}{Spec}
\DeclareMathOperator{\maxspec}{MaxSpec}
\DeclareMathOperator{\supp}{Supp}
\DeclareMathOperator{\ass}{Ass}
\DeclareMathOperator{\ff}{Frac}
\DeclareMathOperator{\im}{Im}
\DeclareMathOperator{\syz}{Syz}
\DeclareMathOperator{\gr}{Gr}
\renewcommand{\ker}{\mathop{\mathrm{Ker}}\nolimits}
\newcommand{\coker}{\mathop{\mathrm{Coker}}\nolimits}
\newcommand{\lps}{[\hspace{-0.25ex}[}
\newcommand{\rps}{]\hspace{-0.25ex}]}
\newcommand{\into}{\hookrightarrow}
\newcommand{\onto}{\twoheadrightarrow}
\newcommand{\tensor}{\otimes}
\newcommand{\x}{\mathbf{x}}
\newcommand{\X}{\mathbf X}
\newcommand{\Y}{\mathbf Y}
\renewcommand{\k}{\boldsymbol{\kappa}}
\renewcommand{\emptyset}{\varnothing}
\renewcommand{\qedsymbol}{$\blacksquare$}
\renewcommand{\l}{\ell}
\newcommand{\1}{\mathds{1}}
\newcommand{\lto}{\mathop{\longrightarrow\,}\limits}
\newcommand{\rad}{\sqrt}
\newcommand{\hf}{H}
\newcommand{\hs}{H\!S}
\newcommand{\hp}{H\!P}
\renewcommand{\vec}{\mathbf}
\renewcommand{\phi}{\varphi}
\renewcommand{\epsilon}{\varepsilon}
\renewcommand{\subset}{\subseteq}
\renewcommand{\supset}{\supseteq}
\newcommand{\macaulay}{\textsl{Macaulay2}}
\newcommand{\invlim}{\varprojlim}


%\renewcommand{\proofname}{Sketch of Proof}


\renewenvironment{proof}[1][Proof]
  {\begin{trivlist}\item[\hskip \labelsep \itshape \bfseries #1{}\hspace{2ex}]\upshape}
{\qed\end{trivlist}}

\newenvironment{sketch}[1][Sketch of Proof]
  {\begin{trivlist}\item[\hskip \labelsep \itshape \bfseries #1{}\hspace{2ex}]\upshape}
{\qed\end{trivlist}}



\makeatletter
\renewcommand\section{\@startsection{paragraph}{10}{\z@}%
                                     {-3.25ex\@plus -1ex \@minus -.2ex}%
                                     {1.5ex \@plus .2ex}%
                                     {\normalfont\large\sffamily\bfseries}}
\renewcommand\subsection{\@startsection{subparagraph}{10}{\z@}%
                                    {3.25ex \@plus1ex \@minus.2ex}%
                                    {-1em}%
                                    {\normalfont\normalsize\sffamily\bfseries}}
\makeatother

%% Fix weird index/bib issue.
\makeatletter
\gdef\ttl@savemark{\sectionmark{}}
\makeatother


\makeatletter
%% no number for refs
\newcommand\frontstyle{%
  \def\activitystyle{activity-chapter}
  \def\maketitle{%
    \addtocounter{titlenumber}{1}%
                    {\flushleft\small\sffamily\bfseries\@pretitle\par\vspace{-1.5em}}%
                    {\flushleft\LARGE\sffamily\bfseries\@title \par }%
                    {\vskip .6em\noindent\textit\theabstract\setcounter{problem}{0}\setcounter{sectiontitlenumber}{0}}%
                    \par\vspace{2em}
                    \phantomsection\addcontentsline{toc}{section}{\textbf{\@title}}%
                  }}
\makeatother


\author{Bart Snapp}

\title{Applications of integral extensions}

\begin{document}
\begin{abstract}
  We give examples and applications of integral extensions. Sources
  and references: \cite{GP2008,jpS2000}.
\end{abstract}
\maketitle



\section{UFD's are integrally closed}

\begin{theorem}
  Let $A$ be a UFD, then $A = \tilde{A}$.
  \begin{proof}
    Let $\alpha \in \ff(A)$ with $\alpha = a/b$ and $a,b\in A$ where
    $a$ and $b$ have no common factors.  We must show that if there is
    an integral relation for $\alpha$, then $\alpha\in A$. For $c_i\in
    A$, suppose we have
    \begin{align*}
      \alpha^n + c_{n-1} \alpha^{n-1}+ \dots + c_1 \alpha + c_0 &= 0 \\
      (a/b)^n + c_{n-1} (a/b)^{n-1}+ \dots + c_1 (a/b) + c_0 &= 0 \\
      a^n + c_{n-1} a^{n-1}b+ \dots + c_1 ab^{n-1} + c_0b^n &= 0
    \end{align*}
    but this implies that $b|a$, implying that $b$ is a unit in $A$,
    and hence $\alpha = a/b\in A$.
  \end{proof}
\end{theorem}


The word ``normal'' is synomonous with ``integrally closed'' in algebra and algebraic geometry.

\begin{definition}
  A \dfn{normal domain} is an integrally closed domain.
\end{definition}

\begin{corollary}
  $\Z$ is normal, and so is $k[x_1,\dots,x_n]$ where $k$ is a field.
\end{corollary}


\begin{definition}
  A \dfn{discrete valuation ring} also known as a \dfn{DVR} is a local
  PID.
\end{definition}

\begin{example}
  $\Z_{(p)}$, $k[X]_{(X)}$, and $k\lps X\rps$ are discrete valuation
  rings.
\end{example}


In this text, we require local rings to be Noetherian.


\begin{proposition}
  Let $(A,\m)$ be a local domain. The following are equivalent:
  \begin{enumerate}
  \item $A$ is a discrete valuation ring.
  \item $A$ is a normal domain and $\h(\p)= 1$ for all nonzero prime
    ideals in $\spec(A)$.
  \item $A$ is a normal domain and the exists nonzero $m\in \m$ such
    that $\m\in\ass(A/m A)$.
  \item The ideal $\m$ is principal and nonzero.
  \end{enumerate}
  \begin{proof}
    (a)$\Rightarrow$(b) If $A$ is a DVR, then it is a PDF, hence it is
    a UFD, and it is integrally closed.  Since $A$ is local $\h(\m) =
    1$.

    (b)$\Rightarrow$(c) Consider nonzero $m\in \m$, so we have
    $\m\in\ass(A/m A)$.

    (c)$\Rightarrow$(d) Since $\m\in\ass(A/m A)$ there is an injection
    \[
    A/\m \into A/m A
    \]
    Let $\bar{x}$ be the image of $1$ under this map. In this case,
    \[
    x\in A \quad\text{but}\quad x\notin m A.
    \]
    On the other hand, $\m x \subset m A$. Hence $\m x m^{-1} \subset
    A$. Note, $xm^{-1}\notin A$, otherwise, $x\in mA$. Now we claim
    that $xm^{-1}\notin\m$. If this were the case, then since $\m$ is
    finitely generated, $xm^{-1}$ would be integral over $A$, which is
    not the case since $A = \tilde{A}$.

    So, since $\m x m^{-1} \subset A$, there must be $u = axm^{-1}$
    invertible in $\m$. 
     
  \end{proof}
\end{proposition}







\section{Lying-over}

Let's witness lying-over with \macaulay.  Consider $R=\Q[x,y]$ and the
ideal $(x)\subset \Q[x,y]$, and $S = \Q[x,y,z]/(z^2 -xz -1)$. Note,
this is an integral extension.

We can use the \macaulay\ command \texttt{minimalPrimes} to find the
minimal elements of $\ass_S(R/xR)$. These will be the primes lying
over $(x)$ in $\spec(S)$.
\begin{macaulay2}
S = QQ[x,y,z]/(z^2-x*z-1);
minimalPrimes ideal x
\end{macaulay2}

From this code, we see that primes $(z-1,x)\subset S$ and
$(z+1,x)\subset S$ are the primes lying over $(x)$.


\begin{macaulay2}
S =ZZ/101[x,y]/(y^2-x^2)
minimalPrimes ideal (0_S)


restart
S =ZZ/101[x,y]/(x*y-1)
minimalPrimes ideal (x-1)
\end{macaulay2}


\end{document}
