\documentclass{ximera}



\usepackage{tikz-cd}
\usepackage[sans]{dsfont}

\let\oldbibliography\thebibliography%% to compact bib
\renewcommand{\thebibliography}[1]{%
  \oldbibliography{#1}%
  \setlength{\itemsep}{0pt}%
}


\DefineVerbatimEnvironment{macaulay2}{Verbatim}{numbers=left,frame=lines,label=Macaulay2,labelposition=topline}

%%% This next bit of code defines all our theorem environments
\makeatletter
\let\c@theorem\relax
\let\c@corollary\relax
\makeatother

\let\definition\relax
\let\enddefinition\relax

\let\theorem\relax
\let\endtheorem\relax

\let\proposition\relax
\let\endproposition\relax

\let\exercise\relax
\let\endexercise\relax

\let\question\relax
\let\endquestion\relax

\let\remark\relax
\let\endremark\relax

\let\corollary\relax
\let\endcorollary\relax


\let\example\relax
\let\endexample\relax

\let\warning\relax
\let\endwarning\relax

\let\lemma\relax
\let\endlemma\relax

\newtheoremstyle{SlantTheorem}{\topsep}{\topsep}%%% space between body and thm
		{\slshape}                      %%% Thm body font
		{}                              %%% Indent amount (empty = no indent)
		{\bfseries\sffamily}            %%% Thm head font
		{}                              %%% Punctuation after thm head
		{3ex}                           %%% Space after thm head
		{\thmname{#1}\thmnumber{ #2}\thmnote{ \bfseries(#3)}}%%% Thm head spec
\theoremstyle{SlantTheorem}
\newtheorem{theorem}{Theorem}
\newtheorem{definition}[theorem]{Definition}
\newtheorem{proposition}[theorem]{Proposition}
%% \newtheorem*{dfnn}{Definition}
%% \newtheorem{ques}{Question}[theorem]
\newtheorem{lemma}[theorem]{Lemma}
%% \newtheorem*{war}{WARNING}
%% \newtheorem*{cor}{Corollary}
%% \newtheorem*{eg}{Example}
\newtheorem*{remark}{Remark}
\newtheorem*{touchstone}{Touchstone}
\newtheorem{corollary}{Corollary}[theorem]
\newtheorem*{example}{Example}
\newtheorem*{warning}{WARNING}


\newtheoremstyle{Exercise}{\topsep}{\topsep} %%% space between body and thm
		{}                           %%% Thm body font
		{}                           %%% Indent amount (empty = no indent)
		{\bfseries}                  %%% Thm head font
		{)}                          %%% Punctuation after thm head
		{ }                          %%% Space after thm head
		{\thmnumber{#2}\thmnote{ \bfseries(#3)}}%%% Thm head spec
\theoremstyle{Exercise}
\newtheorem{exercise}{}[theorem]

%% \newtheoremstyle{Question}{\topsep}{\topsep} %%% space between body and thm
%% 		{\bfseries}                  %%% Thm body font
%% 		{3ex}                        %%% Indent amount (empty = no indent)
%% 		{}                           %%% Thm head font
%% 		{}                           %%% Punctuation after thm head
%% 		{}                           %%% Space after thm head
%% 		{\thmnumber{#2}\thmnote{ \bfseries(#3)}}%%% Thm head spec
\newtheoremstyle{Question}{3em}{3em} %%% space between body and thm
		{\large\bfseries}                           %%% Thm body font
		{3ex}                           %%% Indent amount (empty = no indent)
		{\bfseries}                  %%% Thm head font
		{}                          %%% Punctuation after thm head
		{ }                          %%% Space after thm head
		{}%%% Thm head spec
\theoremstyle{Question}
\newtheorem*{question}{}



\renewcommand{\tilde}{\widetilde}
\renewcommand{\bar}{\overline}
\renewcommand{\hat}{\widehat}
\newcommand{\N}{\mathbb N}
\newcommand{\Z}{\mathbb Z}
\newcommand{\R}{\mathbb R}
\newcommand{\Q}{\mathbb Q}
\newcommand{\C}{\mathbb C}
\newcommand{\V}{\mathbb V}
\newcommand{\I}{\mathbb I}
\newcommand{\A}{\mathbb A}
\newcommand{\iso}{\simeq}
\newcommand{\ph}{\varphi}
\newcommand{\Cf}{\mathcal{C}}
\newcommand{\IZ}{\mathrm{Int}(\Z)}
\newcommand{\dsum}{\oplus}
\newcommand{\directsum}{\bigoplus}
\newcommand{\union}{\bigcup}
\renewcommand{\i}{\mathfrak}
\renewcommand{\a}{\mathfrak{a}}
\renewcommand{\b}{\mathfrak{b}}
\newcommand{\m}{\mathfrak{m}}
\newcommand{\p}{\mathfrak{p}}
\newcommand{\q}{\mathfrak{q}}
\newcommand{\dfn}[1]{\textbf{#1}\index{#1}}
\let\hom\relax
\DeclareMathOperator{\ann}{Ann}
\DeclareMathOperator{\h}{ht}
\DeclareMathOperator{\hom}{Hom}
\DeclareMathOperator{\Span}{Span}
\DeclareMathOperator{\spec}{Spec}
\DeclareMathOperator{\maxspec}{MaxSpec}
\DeclareMathOperator{\supp}{Supp}
\DeclareMathOperator{\ass}{Ass}
\DeclareMathOperator{\ff}{Frac}
\DeclareMathOperator{\im}{Im}
\DeclareMathOperator{\syz}{Syz}
\DeclareMathOperator{\gr}{Gr}
\renewcommand{\ker}{\mathop{\mathrm{Ker}}\nolimits}
\newcommand{\coker}{\mathop{\mathrm{Coker}}\nolimits}
\newcommand{\lps}{[\hspace{-0.25ex}[}
\newcommand{\rps}{]\hspace{-0.25ex}]}
\newcommand{\into}{\hookrightarrow}
\newcommand{\onto}{\twoheadrightarrow}
\newcommand{\tensor}{\otimes}
\newcommand{\x}{\mathbf{x}}
\newcommand{\X}{\mathbf X}
\newcommand{\Y}{\mathbf Y}
\renewcommand{\k}{\boldsymbol{\kappa}}
\renewcommand{\emptyset}{\varnothing}
\renewcommand{\qedsymbol}{$\blacksquare$}
\renewcommand{\l}{\ell}
\newcommand{\1}{\mathds{1}}
\newcommand{\lto}{\mathop{\longrightarrow\,}\limits}
\newcommand{\rad}{\sqrt}
\newcommand{\hf}{H}
\newcommand{\hs}{H\!S}
\newcommand{\hp}{H\!P}
\renewcommand{\vec}{\mathbf}
\renewcommand{\phi}{\varphi}
\renewcommand{\epsilon}{\varepsilon}
\renewcommand{\subset}{\subseteq}
\renewcommand{\supset}{\supseteq}
\newcommand{\macaulay}{\textsl{Macaulay2}}
\newcommand{\invlim}{\varprojlim}


%\renewcommand{\proofname}{Sketch of Proof}


\renewenvironment{proof}[1][Proof]
  {\begin{trivlist}\item[\hskip \labelsep \itshape \bfseries #1{}\hspace{2ex}]\upshape}
{\qed\end{trivlist}}

\newenvironment{sketch}[1][Sketch of Proof]
  {\begin{trivlist}\item[\hskip \labelsep \itshape \bfseries #1{}\hspace{2ex}]\upshape}
{\qed\end{trivlist}}



\makeatletter
\renewcommand\section{\@startsection{paragraph}{10}{\z@}%
                                     {-3.25ex\@plus -1ex \@minus -.2ex}%
                                     {1.5ex \@plus .2ex}%
                                     {\normalfont\large\sffamily\bfseries}}
\renewcommand\subsection{\@startsection{subparagraph}{10}{\z@}%
                                    {3.25ex \@plus1ex \@minus.2ex}%
                                    {-1em}%
                                    {\normalfont\normalsize\sffamily\bfseries}}
\makeatother

%% Fix weird index/bib issue.
\makeatletter
\gdef\ttl@savemark{\sectionmark{}}
\makeatother


\makeatletter
%% no number for refs
\newcommand\frontstyle{%
  \def\activitystyle{activity-chapter}
  \def\maketitle{%
    \addtocounter{titlenumber}{1}%
                    {\flushleft\small\sffamily\bfseries\@pretitle\par\vspace{-1.5em}}%
                    {\flushleft\LARGE\sffamily\bfseries\@title \par }%
                    {\vskip .6em\noindent\textit\theabstract\setcounter{problem}{0}\setcounter{sectiontitlenumber}{0}}%
                    \par\vspace{2em}
                    \phantomsection\addcontentsline{toc}{section}{\textbf{\@title}}%
                  }}
\makeatother


\author{Bart Snapp}

\title{Normal domains}

\begin{document}
\begin{abstract}
  We introduce discrete valuation rings, Dedekind domains, and give a
  criterion for a ring to be integrally closed in its field of
  fractions. Sources and references: \cite{jpS2000}.
\end{abstract}
\maketitle



The word ``normal'' is synonymous with ``integrally closed'' in
algebra and algebraic geometry.

\begin{definition}
  A \dfn{normal domain} is an integrally closed domain.
\end{definition}



\begin{theorem}
  Let $A$ be a UFD, then $A = \tilde{A}$.
  \begin{proof}
    Let $\alpha \in \ff(A)$ with $\alpha = a/b$ and $a,b\in A$ where
    $a$ and $b$ have no common factors.  We must show that if there is
    an integral relation for $\alpha$, then $\alpha\in A$. For $c_i\in
    A$, suppose we have
    \begin{align*}
      \alpha^n + c_{n-1} \alpha^{n-1}+ \dots + c_1 \alpha + c_0 &= 0 \\
      (a/b)^n + c_{n-1} (a/b)^{n-1}+ \dots + c_1 (a/b) + c_0 &= 0 \\
      a^n + c_{n-1} a^{n-1}b+ \dots + c_1 ab^{n-1} + c_0b^n &= 0
    \end{align*}
    but this implies that $b|a$, implying that $b$ is a unit in $A$,
    and hence $\alpha = a/b\in A$.
  \end{proof}
\end{theorem}

\begin{corollary}
  $\Z$ is normal, and so is $k[x_1,\dots,x_n]$ where $k$ is a field.
\end{corollary}


\begin{definition}
  A \dfn{discrete valuation ring} also known as a \dfn{DVR} is a local
  PID.
\end{definition}

\begin{example}
  $\Z_{(p)}$, $k[X]_{(X)}$, and $k\lps X\rps$ are discrete valuation
  rings.
\end{example}


In this text, we require local rings to be Noetherian.


\begin{proposition}[Characterization of DVRs]
  Let $(A,\m)$ be a local domain. The following are equivalent:
  \begin{enumerate}
  \item $A$ is a discrete valuation ring.
  \item $A$ is a normal domain and $\h(\p)= 1$ for all nonzero prime
    ideals in $\spec(A)$.
  \item $A$ is a normal domain and the exists nonzero $m\in \m$ such
    that $\m\in\ass(A/m A)$.
  \item The ideal $\m$ is principal and nonzero.
  \end{enumerate}
  \begin{proof}
    (a)$\Rightarrow$(b) If $A$ is a DVR, then it is a PID, hence it is
    a UFD, and it is integrally closed.  Since $A$ is local $\h(\m) =
    1$.

    (b)$\Rightarrow$(c) Consider nonzero $m\in \m$, so we have
    $\m\in\ass(A/m A)$.

    (c)$\Rightarrow$(d) Since $\m\in\ass(A/m A)$ there is an injection
    \[
    A/\m \into A/m A
    \]
    Let $\bar{x}$ be the image of $1$ under this map. In this case,
    \[
    x\in A \quad\text{but}\quad x\notin m A.
    \]
    On the other hand, $\m x \subset m A$. Hence $\m x m^{-1} \subset
    A$. Note, $xm^{-1}\notin A$, otherwise, $x\in mA$.

    Now we claim that $\m xm^{-1}\not\subset\m$. If it were the case
    that $\m xm^{-1}\subset \m\subset A$, then since $\m$ is finitely
    generated, we may use the determinant trick
    (Proposition~\ref{P:determinanttrick}) where $\phi$ is
    multiplication by $xm^{-1}$, to see that $xm^{-1}$ is integral
    over $A$, which is not the case since $A = \tilde{A}$.

    So, since $\m xm^{-1} \not\subset\m$ there must $g\in \m$ such
    that $u = gxm^{-1}\notin\m$ is invertible in $A$.  Now given any
    element $e\in\m$, we may write
    \begin{align*}
      e &= e xm^{-1} g^{-1} x^{-1} m g \\
      &= (exm^{-1}) u^{-1} g,
    \end{align*}
    and since $\m xm^{-1}\subset A$, $exm^{-1}\in A$, and $u^{-1}\in
    A$, so $\m = g A$.

    (d)$\Rightarrow$(a) We must show that $A$ is a PID. Suppose that
    $\m = gA$. So $\m^n = g^n A$, and by Corollary~\ref{C:intmax}, we
    have that $\bigcap \m^n = 0$. This means for any nonzero element
    $a\in A$, there exists $n$ such that $a\in g^n A$ but not in
    $g^{n+1}A$. Hence $a = g^nu$ where $u\notin\m$, and is hence
    invertible in $A$. So $a A = g^n A$. Thus every ideal can be
    broken into a sum of principal ideals, and each of these is of
    the form $g^i A$, hence this sum is also a principal ideal.
  \end{proof}
\end{proposition}


We will now give a criterion for a ring to be a normal domain.

\begin{theorem}[Criterion for normality]
  Let $A$ be Noetherian, then $A$ is a normal domain if and only if
  both of the following hold:
  \begin{enumerate}
  \item $A_\p$ is a DVR for all primes $\p$ of height $1$.
  \item For all $0 \neq m \in A$, if $\p \in \ass(A/mA)$, then $\h(\p) = 1$.
  \end{enumerate}
  \begin{proof}
    $(\Rightarrow)$ Suppose that $A$ is normal, then $A_\p$ is also
    normal. If $\h(\p) = 1$, then the previous proposition shows that
    $A_\p$ is a DVR.

    If $m\ne 0$, and $\p \in \ass(A_\p/mA_\p)$, then apply the previous
    proposition to see that $\h(\p) = 1$.

    $(\Leftarrow)$ Suppose that (a) and (b) hold above. Consider
    $b/a\in\ff(A)$ where $b/a\in A_\p$ for all $\p\in\spec(A)$. Now
    $b\in a A_\p$, so $b\in \p\in \ass(A_\p/aA_\p)$ so $\h(\p) = 1$
    for all $\p$.  So $b\in a A$. Hence $A = \bigcap A_\p$ where $\p$
    is a prime ideal of height $1$. Since each of the $A_\p$ are
    normal, so is $A$.
  \end{proof}
\end{theorem}


\begin{corollary}
  Let $A$ be normal and $\p$ be a prime ideal of height $1$. If
  $\ass(A/I) = \{\p\}$, then $I$ is a \dfn{symbolic power} of $\p$,
  meaning
  \[
  I = \p^{(n)} = \p^n A_\p \cap A.
  \]
  for  $n\ge 1$.
\end{corollary}




%% other things:
%% Galois actions on primes
%% More geometry

%% \begin{macaulay2}
%% S =ZZ/101[x,y]/(y^2-x^2)
%% minimalPrimes ideal (0_S)


%% restart
%% S =ZZ/101[x,y]/(x*y-1)
%% minimalPrimes ideal (x-1)
%% \end{macaulay2}


\end{document}
