\documentclass{ximera}



\usepackage{tikz-cd}
\usepackage[sans]{dsfont}

\let\oldbibliography\thebibliography%% to compact bib
\renewcommand{\thebibliography}[1]{%
  \oldbibliography{#1}%
  \setlength{\itemsep}{0pt}%
}


\DefineVerbatimEnvironment{macaulay2}{Verbatim}{numbers=left,frame=lines,label=Macaulay2,labelposition=topline}

%%% This next bit of code defines all our theorem environments
\makeatletter
\let\c@theorem\relax
\let\c@corollary\relax
\makeatother

\let\definition\relax
\let\enddefinition\relax

\let\theorem\relax
\let\endtheorem\relax

\let\proposition\relax
\let\endproposition\relax

\let\exercise\relax
\let\endexercise\relax

\let\question\relax
\let\endquestion\relax

\let\remark\relax
\let\endremark\relax

\let\corollary\relax
\let\endcorollary\relax


\let\example\relax
\let\endexample\relax

\let\warning\relax
\let\endwarning\relax

\let\lemma\relax
\let\endlemma\relax

\newtheoremstyle{SlantTheorem}{\topsep}{\topsep}%%% space between body and thm
		{\slshape}                      %%% Thm body font
		{}                              %%% Indent amount (empty = no indent)
		{\bfseries\sffamily}            %%% Thm head font
		{}                              %%% Punctuation after thm head
		{3ex}                           %%% Space after thm head
		{\thmname{#1}\thmnumber{ #2}\thmnote{ \bfseries(#3)}}%%% Thm head spec
\theoremstyle{SlantTheorem}
\newtheorem{theorem}{Theorem}
\newtheorem{definition}[theorem]{Definition}
\newtheorem{proposition}[theorem]{Proposition}
%% \newtheorem*{dfnn}{Definition}
%% \newtheorem{ques}{Question}[theorem]
\newtheorem{lemma}[theorem]{Lemma}
%% \newtheorem*{war}{WARNING}
%% \newtheorem*{cor}{Corollary}
%% \newtheorem*{eg}{Example}
\newtheorem*{remark}{Remark}
\newtheorem*{touchstone}{Touchstone}
\newtheorem{corollary}{Corollary}[theorem]
\newtheorem*{example}{Example}
\newtheorem*{warning}{WARNING}


\newtheoremstyle{Exercise}{\topsep}{\topsep} %%% space between body and thm
		{}                           %%% Thm body font
		{}                           %%% Indent amount (empty = no indent)
		{\bfseries}                  %%% Thm head font
		{)}                          %%% Punctuation after thm head
		{ }                          %%% Space after thm head
		{\thmnumber{#2}\thmnote{ \bfseries(#3)}}%%% Thm head spec
\theoremstyle{Exercise}
\newtheorem{exercise}{}[theorem]

%% \newtheoremstyle{Question}{\topsep}{\topsep} %%% space between body and thm
%% 		{\bfseries}                  %%% Thm body font
%% 		{3ex}                        %%% Indent amount (empty = no indent)
%% 		{}                           %%% Thm head font
%% 		{}                           %%% Punctuation after thm head
%% 		{}                           %%% Space after thm head
%% 		{\thmnumber{#2}\thmnote{ \bfseries(#3)}}%%% Thm head spec
\newtheoremstyle{Question}{3em}{3em} %%% space between body and thm
		{\large\bfseries}                           %%% Thm body font
		{3ex}                           %%% Indent amount (empty = no indent)
		{\bfseries}                  %%% Thm head font
		{}                          %%% Punctuation after thm head
		{ }                          %%% Space after thm head
		{}%%% Thm head spec
\theoremstyle{Question}
\newtheorem*{question}{}



\renewcommand{\tilde}{\widetilde}
\renewcommand{\bar}{\overline}
\renewcommand{\hat}{\widehat}
\newcommand{\N}{\mathbb N}
\newcommand{\Z}{\mathbb Z}
\newcommand{\R}{\mathbb R}
\newcommand{\Q}{\mathbb Q}
\newcommand{\C}{\mathbb C}
\newcommand{\V}{\mathbb V}
\newcommand{\I}{\mathbb I}
\newcommand{\A}{\mathbb A}
\newcommand{\iso}{\simeq}
\newcommand{\ph}{\varphi}
\newcommand{\Cf}{\mathcal{C}}
\newcommand{\IZ}{\mathrm{Int}(\Z)}
\newcommand{\dsum}{\oplus}
\newcommand{\directsum}{\bigoplus}
\newcommand{\union}{\bigcup}
\renewcommand{\i}{\mathfrak}
\renewcommand{\a}{\mathfrak{a}}
\renewcommand{\b}{\mathfrak{b}}
\newcommand{\m}{\mathfrak{m}}
\newcommand{\p}{\mathfrak{p}}
\newcommand{\q}{\mathfrak{q}}
\newcommand{\dfn}[1]{\textbf{#1}\index{#1}}
\let\hom\relax
\DeclareMathOperator{\ann}{Ann}
\DeclareMathOperator{\h}{ht}
\DeclareMathOperator{\hom}{Hom}
\DeclareMathOperator{\Span}{Span}
\DeclareMathOperator{\spec}{Spec}
\DeclareMathOperator{\maxspec}{MaxSpec}
\DeclareMathOperator{\supp}{Supp}
\DeclareMathOperator{\ass}{Ass}
\DeclareMathOperator{\ff}{Frac}
\DeclareMathOperator{\im}{Im}
\DeclareMathOperator{\syz}{Syz}
\DeclareMathOperator{\gr}{Gr}
\renewcommand{\ker}{\mathop{\mathrm{Ker}}\nolimits}
\newcommand{\coker}{\mathop{\mathrm{Coker}}\nolimits}
\newcommand{\lps}{[\hspace{-0.25ex}[}
\newcommand{\rps}{]\hspace{-0.25ex}]}
\newcommand{\into}{\hookrightarrow}
\newcommand{\onto}{\twoheadrightarrow}
\newcommand{\tensor}{\otimes}
\newcommand{\x}{\mathbf{x}}
\newcommand{\X}{\mathbf X}
\newcommand{\Y}{\mathbf Y}
\renewcommand{\k}{\boldsymbol{\kappa}}
\renewcommand{\emptyset}{\varnothing}
\renewcommand{\qedsymbol}{$\blacksquare$}
\renewcommand{\l}{\ell}
\newcommand{\1}{\mathds{1}}
\newcommand{\lto}{\mathop{\longrightarrow\,}\limits}
\newcommand{\rad}{\sqrt}
\newcommand{\hf}{H}
\newcommand{\hs}{H\!S}
\newcommand{\hp}{H\!P}
\renewcommand{\vec}{\mathbf}
\renewcommand{\phi}{\varphi}
\renewcommand{\epsilon}{\varepsilon}
\renewcommand{\subset}{\subseteq}
\renewcommand{\supset}{\supseteq}
\newcommand{\macaulay}{\textsl{Macaulay2}}
\newcommand{\invlim}{\varprojlim}


%\renewcommand{\proofname}{Sketch of Proof}


\renewenvironment{proof}[1][Proof]
  {\begin{trivlist}\item[\hskip \labelsep \itshape \bfseries #1{}\hspace{2ex}]\upshape}
{\qed\end{trivlist}}

\newenvironment{sketch}[1][Sketch of Proof]
  {\begin{trivlist}\item[\hskip \labelsep \itshape \bfseries #1{}\hspace{2ex}]\upshape}
{\qed\end{trivlist}}



\makeatletter
\renewcommand\section{\@startsection{paragraph}{10}{\z@}%
                                     {-3.25ex\@plus -1ex \@minus -.2ex}%
                                     {1.5ex \@plus .2ex}%
                                     {\normalfont\large\sffamily\bfseries}}
\renewcommand\subsection{\@startsection{subparagraph}{10}{\z@}%
                                    {3.25ex \@plus1ex \@minus.2ex}%
                                    {-1em}%
                                    {\normalfont\normalsize\sffamily\bfseries}}
\makeatother

%% Fix weird index/bib issue.
\makeatletter
\gdef\ttl@savemark{\sectionmark{}}
\makeatother


\makeatletter
%% no number for refs
\newcommand\frontstyle{%
  \def\activitystyle{activity-chapter}
  \def\maketitle{%
    \addtocounter{titlenumber}{1}%
                    {\flushleft\small\sffamily\bfseries\@pretitle\par\vspace{-1.5em}}%
                    {\flushleft\LARGE\sffamily\bfseries\@title \par }%
                    {\vskip .6em\noindent\textit\theabstract\setcounter{problem}{0}\setcounter{sectiontitlenumber}{0}}%
                    \par\vspace{2em}
                    \phantomsection\addcontentsline{toc}{section}{\textbf{\@title}}%
                  }}
\makeatother


\title{Monomials and orderings}

\begin{document}
\begin{abstract}
  We set up definitions and notation for the study of Groebner
  bases. Sources and references: \cite{CLO2007}.
\end{abstract}
\maketitle

Throughout this chapter our ring will be $R=k[x_1,\dots,x_n] =
k[\x]$. We will denote the whole numbers by $\omega =
\{0,1,2,3,\dots\}$.


\begin{definition}
  A \dfn{monomial} in $k[x_1,\dots,x_n]$ is a product of the form
  \[
  \x^{\valpha} = x_1^{\alpha_1}\cdot x_2^{\alpha_2}  \cdots x_n^{\alpha_n}.
  \]
  where $\valpha = (\alpha_1,\dots,\alpha_n) \in
  \omega^n$.  A \dfn{polynomial} is finite sum of monomials with
  coefficients $c_\alpha\in k$,
  \[
  F = \sum_{\mathrm{finite}} c_{\valpha} \x^{\valpha}.
  \]
\end{definition}


\begin{definition}
  Let $R = k[\x]$ and consider a polynomial
  \[
  F =  \sum_{\mathrm{finite}} c_{\valpha} \x^{\valpha}.
  \]
  \begin{itemize}
    \item The \dfn{degree} of a monomial $\x^{\valpha}$ is
      $\valpha=(\alpha_1,\dots,\alpha_n)\in \omega^n$.
    \item The \dfn{total degree} of a monomial $\x^{\valpha}$ is
      $|\boldsymbol{\alpha}| = \sum_{i=1}^n \alpha_i$.
    \item Each $c_{\valpha}$ is the \dfn{coefficient} of the
      monomial $\x^{\valpha}$.
    \item If $c_{\valpha} \ne 0$, then
      $c_{\valpha}\x^{\valpha}$ is a \dfn{term} of
      $F$.
    \item The \dfn{total degree} of $F$, is the maximum total degree
      of the monomials in $F$.
  \end{itemize}
\end{definition}



\begin{definition}
  A \dfn{monomial ordering} $(k[\x],>)$ is a relation on the set of
  monomials in $k[\x]$ such that:
  \begin{enumerate}
  \item If $\x^{\valpha}$ and $\x^{\vbeta}$ are
    monomials in $R$, then exactly one of the three statements
      \[
      \x^{\valpha} > \x^{\vbeta},\quad \x^{\valpha} = \x^{\vbeta},
      \quad \x^{\vbeta} > \x^{\valpha}.
      \]
      holds. This means that ordering $>$ is a total ordering.
    \item If $\x^{\valpha}>\x^{\vbeta}$ and
      $\vgamma\in\omega^n$, then
      $\x^{\valpha+\vgamma}>\x^{\vbeta+\vgamma}$.
    \item Every nonempty subset of the set of monomials of $k[\x]$ has
      a least element with respect to $>$. This means that the
      monomials of $k[\x]$ are well-ordered.
  \end{enumerate}
\end{definition}

Now let's look at some explicit monomial orderings.


\begin{definition}[Lexicographic order]\index{lexicographic order}
  Consider monomials $\x^\valpha,\x^\vbeta\in k[\x]$. We say
  \[
  \x^\valpha \underset{\mathtt{Lex}}{>}\x^\vbeta
  \]
  when the left-most nonzero entry of $\valpha-\vbeta$ is positive.
\end{definition}

The computer algebra system \macaulay\ will write the terms of a
polynomial in decreasing order relative to the specified monomial
order.

\begin{macaulay2}
ZZ/101[x,y,MonomialOrder=>Lex];
y^3+ x*y + x + y + 1
\end{macaulay2}

Here we see:
\[
xy \underset{\mathtt{Lex}}{>} x \underset{\mathtt{Lex}}{>} y^3
\underset{\mathtt{Lex}}{>} y \underset{\mathtt{Lex}}{>} 1.
\]



\begin{definition}[Graded reverse lexicographic order]\index{graded reverse lexicographic order}
  Consider monomials $\x^\valpha,\x^\vbeta\in k[\x]$. We say
  \[
  \x^\valpha \underset{\mathtt{GRevLex}}{>}\x^\vbeta
  \]
  when $|\valpha| > |\vbeta|$ or $|\valpha| = |\vbeta|$ and the
  right-most nonzero entry of $\valpha-\vbeta$ is negative.
\end{definition}

The order $\mathtt{GRevLex}$ is the default ordering in \macaulay.

\begin{macaulay2}
ZZ/101[x,y]
y^3+ x*y + x + y + 1
\end{macaulay2}

Here we see:
\[
y^3 \underset{\mathtt{GRevLex}}{>} xy \underset{\mathtt{GRevLex}}{>} x  \underset{\mathtt{GRevLex}}{>} y \underset{\mathtt{GRevLex}}{>} 1
\]

\begin{definition}[Weight order]\index{weight order}
  Consider monomials $\x^\valpha,\x^\vbeta\in k[x_1,\dots, x_n]$ and
  let $\vec{w}$ be a vector in $\Z^n$. We say
  \[
  \x^\valpha \underset{\mathtt{w}}{>}\x^\vbeta
  \]
  when $\vec{w}\dotp \valpha > \vec{w}\dotp \vbeta$ or $\vec{w}\dotp
  \valpha = \vec{w}\dotp \vbeta$ and $\x^\valpha
  \underset{\mathtt{GRevLex}}{>}\x^\vbeta$.
\end{definition}

Weighted orderings are quite versatile. Here is a basic example with
$\vec{w} = (0,3)$ in \macaulay.

\begin{macaulay2}
ZZ/101[x,y,MonomialOrder=>{Weights=>{0,3}}]
y^3+ x*y + x + y + 1
\end{macaulay2}

Here we see:
\[
y^3 \underset{\mathtt{w}}{>} xy \underset{\mathtt{w}}{>} y \underset{\mathtt{w}}{>} x \underset{\mathtt{w}}{>} 1
\]

\begin{proposition}
  Each of the orderings above defines a monomial ordering.
  \begin{sketch}
    Just check the conditions.
  \end{sketch}
\end{proposition}



\end{document}
