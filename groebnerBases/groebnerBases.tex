\documentclass{ximera}



\usepackage{tikz-cd}
\usepackage[sans]{dsfont}

\let\oldbibliography\thebibliography%% to compact bib
\renewcommand{\thebibliography}[1]{%
  \oldbibliography{#1}%
  \setlength{\itemsep}{0pt}%
}


\DefineVerbatimEnvironment{macaulay2}{Verbatim}{numbers=left,frame=lines,label=Macaulay2,labelposition=topline}

%%% This next bit of code defines all our theorem environments
\makeatletter
\let\c@theorem\relax
\let\c@corollary\relax
\makeatother

\let\definition\relax
\let\enddefinition\relax

\let\theorem\relax
\let\endtheorem\relax

\let\proposition\relax
\let\endproposition\relax

\let\exercise\relax
\let\endexercise\relax

\let\question\relax
\let\endquestion\relax

\let\remark\relax
\let\endremark\relax

\let\corollary\relax
\let\endcorollary\relax


\let\example\relax
\let\endexample\relax

\let\warning\relax
\let\endwarning\relax

\let\lemma\relax
\let\endlemma\relax

\newtheoremstyle{SlantTheorem}{\topsep}{\topsep}%%% space between body and thm
		{\slshape}                      %%% Thm body font
		{}                              %%% Indent amount (empty = no indent)
		{\bfseries\sffamily}            %%% Thm head font
		{}                              %%% Punctuation after thm head
		{3ex}                           %%% Space after thm head
		{\thmname{#1}\thmnumber{ #2}\thmnote{ \bfseries(#3)}}%%% Thm head spec
\theoremstyle{SlantTheorem}
\newtheorem{theorem}{Theorem}
\newtheorem{definition}[theorem]{Definition}
\newtheorem{proposition}[theorem]{Proposition}
%% \newtheorem*{dfnn}{Definition}
%% \newtheorem{ques}{Question}[theorem]
\newtheorem{lemma}[theorem]{Lemma}
%% \newtheorem*{war}{WARNING}
%% \newtheorem*{cor}{Corollary}
%% \newtheorem*{eg}{Example}
\newtheorem*{remark}{Remark}
\newtheorem*{touchstone}{Touchstone}
\newtheorem{corollary}{Corollary}[theorem]
\newtheorem*{example}{Example}
\newtheorem*{warning}{WARNING}


\newtheoremstyle{Exercise}{\topsep}{\topsep} %%% space between body and thm
		{}                           %%% Thm body font
		{}                           %%% Indent amount (empty = no indent)
		{\bfseries}                  %%% Thm head font
		{)}                          %%% Punctuation after thm head
		{ }                          %%% Space after thm head
		{\thmnumber{#2}\thmnote{ \bfseries(#3)}}%%% Thm head spec
\theoremstyle{Exercise}
\newtheorem{exercise}{}[theorem]

%% \newtheoremstyle{Question}{\topsep}{\topsep} %%% space between body and thm
%% 		{\bfseries}                  %%% Thm body font
%% 		{3ex}                        %%% Indent amount (empty = no indent)
%% 		{}                           %%% Thm head font
%% 		{}                           %%% Punctuation after thm head
%% 		{}                           %%% Space after thm head
%% 		{\thmnumber{#2}\thmnote{ \bfseries(#3)}}%%% Thm head spec
\newtheoremstyle{Question}{3em}{3em} %%% space between body and thm
		{\large\bfseries}                           %%% Thm body font
		{3ex}                           %%% Indent amount (empty = no indent)
		{\bfseries}                  %%% Thm head font
		{}                          %%% Punctuation after thm head
		{ }                          %%% Space after thm head
		{}%%% Thm head spec
\theoremstyle{Question}
\newtheorem*{question}{}



\renewcommand{\tilde}{\widetilde}
\renewcommand{\bar}{\overline}
\renewcommand{\hat}{\widehat}
\newcommand{\N}{\mathbb N}
\newcommand{\Z}{\mathbb Z}
\newcommand{\R}{\mathbb R}
\newcommand{\Q}{\mathbb Q}
\newcommand{\C}{\mathbb C}
\newcommand{\V}{\mathbb V}
\newcommand{\I}{\mathbb I}
\newcommand{\A}{\mathbb A}
\newcommand{\iso}{\simeq}
\newcommand{\ph}{\varphi}
\newcommand{\Cf}{\mathcal{C}}
\newcommand{\IZ}{\mathrm{Int}(\Z)}
\newcommand{\dsum}{\oplus}
\newcommand{\directsum}{\bigoplus}
\newcommand{\union}{\bigcup}
\renewcommand{\i}{\mathfrak}
\renewcommand{\a}{\mathfrak{a}}
\renewcommand{\b}{\mathfrak{b}}
\newcommand{\m}{\mathfrak{m}}
\newcommand{\p}{\mathfrak{p}}
\newcommand{\q}{\mathfrak{q}}
\newcommand{\dfn}[1]{\textbf{#1}\index{#1}}
\let\hom\relax
\DeclareMathOperator{\ann}{Ann}
\DeclareMathOperator{\h}{ht}
\DeclareMathOperator{\hom}{Hom}
\DeclareMathOperator{\Span}{Span}
\DeclareMathOperator{\spec}{Spec}
\DeclareMathOperator{\maxspec}{MaxSpec}
\DeclareMathOperator{\supp}{Supp}
\DeclareMathOperator{\ass}{Ass}
\DeclareMathOperator{\ff}{Frac}
\DeclareMathOperator{\im}{Im}
\DeclareMathOperator{\syz}{Syz}
\DeclareMathOperator{\gr}{Gr}
\renewcommand{\ker}{\mathop{\mathrm{Ker}}\nolimits}
\newcommand{\coker}{\mathop{\mathrm{Coker}}\nolimits}
\newcommand{\lps}{[\hspace{-0.25ex}[}
\newcommand{\rps}{]\hspace{-0.25ex}]}
\newcommand{\into}{\hookrightarrow}
\newcommand{\onto}{\twoheadrightarrow}
\newcommand{\tensor}{\otimes}
\newcommand{\x}{\mathbf{x}}
\newcommand{\X}{\mathbf X}
\newcommand{\Y}{\mathbf Y}
\renewcommand{\k}{\boldsymbol{\kappa}}
\renewcommand{\emptyset}{\varnothing}
\renewcommand{\qedsymbol}{$\blacksquare$}
\renewcommand{\l}{\ell}
\newcommand{\1}{\mathds{1}}
\newcommand{\lto}{\mathop{\longrightarrow\,}\limits}
\newcommand{\rad}{\sqrt}
\newcommand{\hf}{H}
\newcommand{\hs}{H\!S}
\newcommand{\hp}{H\!P}
\renewcommand{\vec}{\mathbf}
\renewcommand{\phi}{\varphi}
\renewcommand{\epsilon}{\varepsilon}
\renewcommand{\subset}{\subseteq}
\renewcommand{\supset}{\supseteq}
\newcommand{\macaulay}{\textsl{Macaulay2}}
\newcommand{\invlim}{\varprojlim}


%\renewcommand{\proofname}{Sketch of Proof}


\renewenvironment{proof}[1][Proof]
  {\begin{trivlist}\item[\hskip \labelsep \itshape \bfseries #1{}\hspace{2ex}]\upshape}
{\qed\end{trivlist}}

\newenvironment{sketch}[1][Sketch of Proof]
  {\begin{trivlist}\item[\hskip \labelsep \itshape \bfseries #1{}\hspace{2ex}]\upshape}
{\qed\end{trivlist}}



\makeatletter
\renewcommand\section{\@startsection{paragraph}{10}{\z@}%
                                     {-3.25ex\@plus -1ex \@minus -.2ex}%
                                     {1.5ex \@plus .2ex}%
                                     {\normalfont\large\sffamily\bfseries}}
\renewcommand\subsection{\@startsection{subparagraph}{10}{\z@}%
                                    {3.25ex \@plus1ex \@minus.2ex}%
                                    {-1em}%
                                    {\normalfont\normalsize\sffamily\bfseries}}
\makeatother

%% Fix weird index/bib issue.
\makeatletter
\gdef\ttl@savemark{\sectionmark{}}
\makeatother


\makeatletter
%% no number for refs
\newcommand\frontstyle{%
  \def\activitystyle{activity-chapter}
  \def\maketitle{%
    \addtocounter{titlenumber}{1}%
                    {\flushleft\small\sffamily\bfseries\@pretitle\par\vspace{-1.5em}}%
                    {\flushleft\LARGE\sffamily\bfseries\@title \par }%
                    {\vskip .6em\noindent\textit\theabstract\setcounter{problem}{0}\setcounter{sectiontitlenumber}{0}}%
                    \par\vspace{2em}
                    \phantomsection\addcontentsline{toc}{section}{\textbf{\@title}}%
                  }}
\makeatother


\title{Gr\"obner bases}

\author{Bart Snapp}

\begin{document}
\begin{abstract}
  We give an algorithm for deciding when a polynomial is a member of
  an ideal.  Sources and references: \cite{CLO2007,hS2003}.
\end{abstract}
\maketitle



\section{The ideal membership problem}

Given a ring and ideal in the ring, we would like to know when an
element is a member of the ideal. We'll need some notation.


\begin{definition}
  Consider a polynomial $f\in k[x_1,\dots, x_n]$ with a monomial order
  $>$.
  \begin{itemize}
    \item The \dfn{multidegree} of $f$ is
    \[
    \multideg(f) := \max\{\valpha: \text{$c_\valpha x^\valpha$ is the largest term w.r.t.\ $>$} \}.
    \]
  \item The \dfn{lead coefficient} of $f$ is $\lc(f) := c_{\multideg(f)}$.
  \item The \dfn{lead monomial} of $f$ is $\lm(f) := \x^{\multideg(f)}$.
  \item The \dfn{lead term} of $f$ is $\lt(f) := \lc(f)\cdot \lm(f)$.
  \item The \dfn{ideal of lead terms} of an ideal $I\subset k[\x]$ is
    the ideal
    \[
    \lt(I) = (\lt(f): f\in I)\subset k[\x].
    \]
  \end{itemize}
\end{definition}

Now suppose we are working in the ring $k[x_1,\dots, x_n]$ and we have an ideal
\[
I = (g_1,\dots, g_m) \subset k[\x].
\]
Here is a basic algorithm that attempts to solve the ideal membership
problem.

\begin{algorithm}[Division algorithm]
  \hfill
  \begin{algorithmic}[1]
    \Procedure{Division}{$f,g_1,\dots,g_m$} 
    \State $\texttt{quotient} = 0$
    \State $\texttt{remainder} = 0$
    \Repeat
    \If{$\lt(g_i) \cdot h_i = \lt(f)$}
    \State $\texttt{quotient} \from \texttt{quotient} + h_i\cdot g_i$
    \State $f \from f - h_i\cdot G_i$
    \Else
    \State $\texttt{remainder} \from \texttt{remainder} + \lt(f)$
    \State $f\from f- \lt(f)$
    \EndIf
    \Until{$f = 0$}
    \Return\{\texttt{quotient}, \texttt{reminder}\}
    \EndProcedure
  \end{algorithmic}
\end{algorithm}

At the end of this algorithm, we can write
\[
f = \sum_{\mathrm{finite}}  h_i\cdot g_i + r.
\]
If the reminder is zero, then $f\in I$. However, this algorithm is
insufficient. Consider an example from \cite{hS2003} where
\begin{align*}
  f &= x^2 - y^2\\
  g_1 &= x^2 + y\\
  g_2 &=xy+x \\
  I &= (g_1,g_2).
\end{align*}
Let's run the division algorithm manually in \macaulay and try to see
if we what happens.

\begin{macaulay2}
R = ZZ/101[x,y];
f = x^2-y^2;
g1 = x^2+y;
g2 = x*y+x;
leadTerm f
leadTerm g1
g%g1
\end{macaulay2}

Using the monomial order \texttt{GRevLex}, $\lt(f) = x^2$ and
$\lt(g_1) = x^2$. Hence, we divide $f$ by $g_1$ and compute the
remainder, which comes out to $-y^2-y$. Since neither $\lt(g_1) =
x^2$ nor $\lt(g_2) = xy$ divide $y^2$, the algorithm concludes with
\[
f  = 1\cdot g_1 + \underbrace{-y^2-y}_{\text{remainder}}.
\]
this would seem to indicate that $F\notin I$. However,
\[
f = x g_1 - y g_2.
\]
so in fact, $f\in I$. Our algorithm didn't work because the lead terms
of $g_1$ and $g_2$ cancelled. To remedy this, we will enlarge our set
of generators to include more lead terms. This leads us to the
definition of a Gr\"obner basis.

\begin{definition}
  A subset $\{g_1,\dots, g_n\}\subset I$ of an ideal is a
  \dfn{Gr\"obner basis} for $I$ if
  \[
  \lt(I) = (\lt(g_1),\dots, \lt(g_n)).
  \]
\end{definition}

\begin{proposition}
  Suppose $(g_1,\dots,g_n)$ is a Gr\"obner basis for $I$. Then
  \[
  I = (g_1,\dots,g_n).
  \]
  \begin{proof}
    Suppose $f\in I$. Using the division algorithm on the ideal
    $(g_1,\dots,g_n)$, there will always be a $g_i$ such that
    $\lt(g_i)|\lt(f)$. Hence $f$ can be expressed as a finite sum
    \[
    f = \sum_\mathrm{finite} h_i g_i.
    \]
  \end{proof}
\end{proposition}

Now suppose that $(g_1,\dots,g_n)$ is a Gr\"obner basis for some
ideal. We can then write:
\[
0 \to S \to k[\x]^n \to k[x]/I \to 0
\]
$S$ is a syzygy, the kernel of this map of free modules.

\begin{definition}
  Let $f,g\in k[\x]$ be nonzero polynomials. The \dfn{S-polynomial} or
  \dfn{syzygy-polynomial} is given by
  \[
  S(f,g) := \frac{\lcm(\lm(f),\lm(g))}{\lt(f)}\cdot f - \frac{\lcm(\lm(f),\lm(g))}{\lt(g)}\cdot g.
  \]
\end{definition}

\begin{macaulay2}
Spair = (g1,g2) ->
lcm(leadMonomial(g1),leadMonomial(g2))/leadTerm(g1)*g1-
lcm(leadMonomial(g1),leadMonomial(g2))/leadTerm(g2)*g2;
R = ZZ/101[x,y];
g1 = x^2+y;
g2 = x*y+x;
Spair(g1,g2)
\end{macaulay2}





\begin{theorem}[Buchberger's Criterion]
  A set $\{g_1,\dots,g_n\}\subset k[\x]$ is Gr\"obner basis if and
  only if for all $i\ne j$
  \[
  S(g_i,g_j) \equiv 0 \pmod{\i g}
  \]
  where $\i g = (g_1,\dots,g_n)$. In this case $g_i$ and $g_j$ are
  called \dfn{S-pairs} or $\dfn{syzygy-pairs}$.
  \begin{proof}
    $(\Rightarrow)$ Suppose that $\{g_1,\dots,g_n\}$ is a Gr\"obner
    basis. Since $S(g_i,g_j)\in (g_1,\dots,g_n)$ we are done.

    $(\Leftarrow)$ Suppose that $f\in \i g$ and that
    \[
    S(g_i,g_j \equiv 0 \pmod{(\i g})
    \]
    for all $i\ne j$. We must show that
    \[
    \lt(f) \in (\lt(g_1),\dots, \lt(g_n)).
    \]
    Since $f\in \i g$, we know
    \[
    f = \sum_{\mathrm{finite}} h_i g_i.
    \]
    At this point there are two cases, either $\lm(f) = \lm(g_i)$, in
    which case we are done, or not. This second case is the one that
    requires attention. We refer the interested reader
    to~\ref{CLO2007}.
  \end{proof}
\end{theorem}




\begin{algorithm}[Buchberger algorithm]
  \hfill
  \begin{algorithmic}[1]
    \Procedure{Gr\"obner basis}{$f_1,\dots,f_m$} 
    \State $G = \{f_1,\dots,f_m\}$
    \Repeat
    \State $G':=G$
    \Until{$G' = G$}
    \EndProcedure
  \end{algorithmic}
\end{algorithm}


NOTE - IF WE WERE WORKING WITH MONOMIALS THIS WOULD BE EASY



\section{Gr\"obner bases}






\end{document}



