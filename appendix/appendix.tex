\documentclass{ximera}



\usepackage{tikz-cd}
\usepackage[sans]{dsfont}

\DefineVerbatimEnvironment{macaulay2}{Verbatim}{numbers=left,frame=lines,label=Macaulay2,labelposition=topline}

%%% This next bit of code defines all our theorem environments
\makeatletter
\let\c@theorem\relax
\let\c@corollary\relax
\makeatother

\let\definition\relax
\let\enddefinition\relax

\let\theorem\relax
\let\endtheorem\relax

\let\proposition\relax
\let\endproposition\relax

\let\exercise\relax
\let\endexercise\relax

\let\question\relax
\let\endquestion\relax

\let\remark\relax
\let\endremark\relax

\let\corollary\relax
\let\endcorollary\relax


\let\example\relax
\let\endexample\relax


\let\lemma\relax
\let\endlemma\relax

\newtheoremstyle{SlantTheorem}{\topsep}{\topsep}%%% space between body and thm
		{\slshape}                      %%% Thm body font
		{}                              %%% Indent amount (empty = no indent)
		{\bfseries\sffamily}            %%% Thm head font
		{}                              %%% Punctuation after thm head
		{3ex}                           %%% Space after thm head
		{\thmname{#1}\thmnumber{ #2}\thmnote{ \bfseries(#3)}}%%% Thm head spec
\theoremstyle{SlantTheorem}
\newtheorem{theorem}{Theorem}
\newtheorem{definition}[theorem]{Definition}
\newtheorem{proposition}[theorem]{Proposition}
%% \newtheorem*{dfnn}{Definition}
%% \newtheorem{ques}{Question}[theorem]
\newtheorem{lemma}[theorem]{Lemma}
%% \newtheorem*{war}{WARNING}
%% \newtheorem*{cor}{Corollary}
%% \newtheorem*{eg}{Example}
\newtheorem*{remark}{Remark}
\newtheorem*{touchstone}{Touchstone}
\newtheorem{corollary}{Corollary}[theorem]
\newtheorem*{example}{Example}


\newtheoremstyle{Exercise}{\topsep}{\topsep} %%% space between body and thm
		{}                           %%% Thm body font
		{}                           %%% Indent amount (empty = no indent)
		{\bfseries}                  %%% Thm head font
		{)}                          %%% Punctuation after thm head
		{ }                          %%% Space after thm head
		{\thmnumber{#2}\thmnote{ \bfseries(#3)}}%%% Thm head spec
\theoremstyle{Exercise}
\newtheorem{exercise}{}[theorem]

%% \newtheoremstyle{Question}{\topsep}{\topsep} %%% space between body and thm
%% 		{\bfseries}                  %%% Thm body font
%% 		{3ex}                        %%% Indent amount (empty = no indent)
%% 		{}                           %%% Thm head font
%% 		{}                           %%% Punctuation after thm head
%% 		{}                           %%% Space after thm head
%% 		{\thmnumber{#2}\thmnote{ \bfseries(#3)}}%%% Thm head spec
\newtheoremstyle{Question}{3em}{3em} %%% space between body and thm
		{\large\bfseries}                           %%% Thm body font
		{3ex}                           %%% Indent amount (empty = no indent)
		{\bfseries}                  %%% Thm head font
		{}                          %%% Punctuation after thm head
		{ }                          %%% Space after thm head
		{}%%% Thm head spec
\theoremstyle{Question}
\newtheorem*{question}{}



\renewcommand{\tilde}{\widetilde}
\renewcommand{\bar}{\overline}
\renewcommand{\hat}{\widehat}
\newcommand{\N}{\mathbb N}
\newcommand{\Z}{\mathbb Z}
\newcommand{\R}{\mathbb R}
\newcommand{\Q}{\mathbb Q}
\newcommand{\C}{\mathbb C}
\newcommand{\V}{\mathbb V}
\newcommand{\I}{\mathbb I}
\newcommand{\A}{\mathbb A}
\newcommand{\iso}{\simeq}
\newcommand{\ph}{\varphi}
\newcommand{\Cf}{\mathcal{C}}
\newcommand{\IZ}{\mathrm{Int}(\Z)}
\newcommand{\dsum}{\oplus}
\newcommand{\directsum}{\coprod}
\newcommand{\union}{\bigcup}
\renewcommand{\i}{\mathfrak}
\renewcommand{\a}{\mathfrak{a}}
\renewcommand{\b}{\mathfrak{b}}
\newcommand{\m}{\mathfrak{m}}
\newcommand{\p}{\mathfrak{p}}
\newcommand{\q}{\mathfrak{q}}
\newcommand{\dfn}{\textbf}
\let\hom\relax
\DeclareMathOperator{\ann}{Ann}
\DeclareMathOperator{\h}{ht}
\DeclareMathOperator{\hom}{Hom}
\DeclareMathOperator{\spec}{Spec}
\DeclareMathOperator{\supp}{Supp}
\DeclareMathOperator{\ass}{Ass}
\DeclareMathOperator{\ff}{Frac}
\DeclareMathOperator{\im}{Im}
\DeclareMathOperator{\syz}{Syz}
\DeclareMathOperator{\gr}{Gr}
\renewcommand{\ker}{\mathop{\mathrm{Ker}}\nolimits}
\newcommand{\lps}{[\hspace{-0.25ex}[}
\newcommand{\rps}{]\hspace{-0.25ex}]}
\newcommand{\into}{\hookrightarrow}
\newcommand{\onto}{\twoheadrightarrow}
\newcommand{\tensor}{\otimes}
\newcommand{\x}{\mathbf{x}}
\newcommand{\X}{\mathbf X}
\newcommand{\Y}{\mathbf Y}
\renewcommand{\k}{\boldsymbol{\kappa}}
\renewcommand{\emptyset}{\varnothing}
\renewcommand{\qedsymbol}{$\blacksquare$}
\renewcommand{\l}{\ell}
\newcommand{\1}{\mathds{1}}
\newcommand{\lto}{\mathop{\longrightarrow\,}\limits}
\newcommand{\rad}{\sqrt}
\renewcommand{\vec}{\mathbf}
\renewcommand{\phi}{\varphi}
\renewcommand{\epsilon}{\varepsilon}
\renewcommand{\subset}{\subseteq}
\renewcommand{\supset}{\supseteq}
\newcommand{\macaulay}{\textsl{Macaulay2}}
\newcommand{\invlim}{\varprojlim}


%\renewcommand{\proofname}{Sketch of Proof}


\renewenvironment{proof}[1][Proof]
  {\begin{trivlist}\item[\hskip \labelsep \itshape \bfseries #1{}\hspace{2ex}]\upshape}
{\qed\end{trivlist}}

\newenvironment{sketch}[1][Sketch of Proof]
  {\begin{trivlist}\item[\hskip \labelsep \itshape \bfseries #1{}\hspace{2ex}]\upshape}
{\qed\end{trivlist}}



\makeatletter
\renewcommand\section{\@startsection{paragraph}{10}{\z@}%
                                     {-3.25ex\@plus -1ex \@minus -.2ex}%
                                     {1.5ex \@plus .2ex}%
                                     {\normalfont\large\sffamily\bfseries}}
\renewcommand\subsection{\@startsection{subparagraph}{10}{\z@}%
                                    {3.25ex \@plus1ex \@minus.2ex}%
                                    {-1em}%
                                    {\normalfont\normalsize\sffamily\bfseries}}
\makeatother

%% Fix weird index/bib issue.
\makeatletter
\gdef\ttl@savemark{\sectionmark{}}
\makeatother


\author{Bart Snapp and Jason McCullough}

\title{Appendix}

\begin{document}
\begin{abstract}
We give several diagrams showing relations between rings.  Adapted
from \cite{sD2008}.
\end{abstract}
\maketitle

\section{Diagram of implications}

\[
\begin{tikzcd}          
              & &         & \text{DVR} \ar[Rightarrow,rrdd]\ar[Rightarrow,ld] &    &                 \\   
& & \text{PID} \ar[Rightarrow,rd]\ar[Rightarrow,ld] &  &                                   & \\
& \text{UFD} \ar[Rightarrow,ld] &   & \text{DD} \ar[Rightarrow,rd] &   & \text{LR} \ar[Rightarrow,ld] \\
\text{ICD} & & & & \text{NR} &                            
\end{tikzcd}                 
\]

\noindent In the diagram above, the abbreviations are as follows:

\begin{itemize}
\item[DVR] Discrete Valuation Ring\index{DVR}\index{discrete valuation ring}
\item[PID] Principal Ideal Domain\index{PID}\index{principal ideal domain}
\item[DD] Dedekind Domain\index{Dedekind domain}
\item[UFD] Unique Factorization Domain\index{UFD}\index{unique factorization domain}
\item[ICD] Integrally Closed Domain, also known as a Normal Domain\index{normal domain}
\item[LR] Local Ring\index{local ring}
\item[NR] Noetherian ring\index{Noetherian!ring}
\end{itemize}


\newpage

\section{Diagram and examples of domains}

All rings are assumed to be domains in the diagram below:
%% \[
%% \includegraphics[width=12cm]{venn.eps}
%% \]

\[
\begin{tikzpicture}
\draw[rounded corners] (10,10) rectangle (12,11);
\node at (11,10.5) {Field \ref{A:field}};

\draw[rounded corners] (9.7,9.7) rectangle (12.3,12);
\node at (11,11.5) {DVR \ref{A:dvr}};

\draw[rounded corners] (9.4,9.4) rectangle (12.6,13);
\node at (11,12.6) {PID \ref{A:pid}};

\draw[rounded corners] (9.1,9.1) rectangle (12.9,14);
\node at (11,13.6) {Dedekind Domain \ref{A:dd}};

\draw[rounded corners] (7,8.1) rectangle (13.8,13.3);
\node at (8,12.6) {UFD \ref{A:ufd}};

\draw[rounded corners] (3.7
,8.8) rectangle (13.2,12.3);
\node at (5.3,11.5) {Local Domain \ref{A:l}};

\draw[rounded corners] (3.4,7.5) rectangle (13.5,15);
\node at (5.3,14.6) {Noetherian Domain \ref{A:n}};

\draw[rounded corners] (6.7,6.7) rectangle (14.1,14.3);
\node at (11,7) {Integrally Closed \ref{A:ic}};

\draw[rounded corners] (3.1,6.4) rectangle (14.4,15.3);
\node at (5.3,7) {Domain \ref{A:domain}};



\end{tikzpicture}
\]


\noindent In the examples below, $k$ represents any field and $p\in\Z$
represents a prime number.

\begin{multicols}{2}
\begin{enumerate}
\item\label{A:domain} Not Noetherian, not integrally closed:
$k[X^2,X^3,Y_1,Y_2,Y_3,\ldots]$.


\item\label{A:ic} Integrally closed, not a UFD, not Noetherian:
  $\Z[2X,2X^2,2X^3,\ldots]$, $k[U,V,Y,Z,X_1,X_2,X_3,\ldots]/(UV -
  YZ)$.  For further information, see \cite{hH1981}.

\item\label{A:ufd} A UFD but not Noetherian: $k[X_1,X_2,X_3,\ldots]$.

\item\label{A:n} Noetherian, not local, not integrally closed: $k[X^2,X^3]$, $\Z[\sqrt{5}]$. 

\item\label{A:l} Local, not integrally closed: $k[X^2,X^3]_{(X^2,X^3)}$, $k\lps X^2, X^3 \rps$.

\item\label{A:dd} A Dedekind domain, not a UFD and hence not local: $\Z[\sqrt{-5}]$.

\item\label{A:pid} A PID but not local: $\Z$, $k[X]$, $\Z[i]$.

\item\label{A:dvr} A DVR, not a field: $\Z_{(p)}$, $k[X]_{(X)}$.

\item\label{A:field} A field: $k$, $\Q$, $\R$, $\C$, $\Z/p\Z$.

\end{enumerate}
\end{multicols}

\newpage

\section{Table of invariances}

The table below summarizes those basic properties of commutative rings
that are and are not preserved under the basic operations on rings.
For example, the symbol \ding{51} that appears in the upper left box
means that if $R$ is Noetherian, then $R[X]$ is Noetherian as well.
An \ding{54} in the table merely means ``not in general.''

\begin{center}
\begin{tabular}{| c || c | c | c | c | c | c | c | c|}
\multicolumn{1}{c}{$R$}&   \multicolumn{1}{c}{$R[X]$} &    \multicolumn{1}{c}{$R\lps X\rps$} &    \multicolumn{1}{c}{$R/I$} &    \multicolumn{1}{c}{$R/\p$} &    \multicolumn{1}{c}{$U^{-1}R$}  & \multicolumn{1}{c}{$R_\p$} &   \multicolumn{1}{c}{$\hat{R}$} &    \multicolumn{1}{c}{$\tilde{R}$}   \\ \hline \hline
Noetherian        & \ding{51}& \ding{51}& \ding{51}& \ding{51}& \ding{51}& \ding{51}& \ding{51} & \ding{54} \\\hline  
local             & \ding{54}& \ding{51}& \ding{51}& \ding{51}& \ding{54} & \ding{51} & \ding{51} & \ding{54}  \\\hline  
local and complete & \ding{54}& \ding{51}& \ding{51}& \ding{51}& \ding{54} & \ding{54} & \ding{51} & \ding{51} \\\hline  
normal domain     & \ding{51}& \ding{51}& \ding{54}& \ding{54}& \ding{51}& \ding{51} & \ding{54}& \ding{51}\\\hline  
Dedekind domain   & \ding{54}& \ding{54}& \ding{54}& \ding{51}& \ding{51}& \ding{51}& \ding{54} & \ding{51} \\\hline  
UFD               & \ding{51}& \ding{54}& \ding{54}& \ding{54}& \ding{51}& \ding{51}& \ding{54} & \ding{51}\\\hline  
PID               & \ding{54}& \ding{54}& \ding{54}& \ding{51}& \ding{51}& \ding{51}& \ding{54}& \ding{51} \\\hline  
DVR or a field    & \ding{54}& \ding{54}& \ding{54}& \ding{51}& \ding{51}& \ding{51} & \ding{51}& \ding{51} \\\hline
\end{tabular}
\end{center}



\noindent In the above table, $\hat{R}$ denotes the completion of $R$ with
respect to some ideal $I$ which is taken to be the unique maximal
ideal if $R$ is local.  Local, as throughout these notes, is taken to
mean Noetherian and local.  For the third and fourth columns $I$
denotes an arbitrary ideal of $R$ while $\p$ denotes a prime ideal.
Lastly, $\tilde{R}$ denotes the integral closure of $R$, which is
assumed to be a domain in this column.



\end{document}
