\documentclass{ximera}



\usepackage{tikz-cd}
\usepackage[sans]{dsfont}

\let\oldbibliography\thebibliography%% to compact bib
\renewcommand{\thebibliography}[1]{%
  \oldbibliography{#1}%
  \setlength{\itemsep}{0pt}%
}


\DefineVerbatimEnvironment{macaulay2}{Verbatim}{numbers=left,frame=lines,label=Macaulay2,labelposition=topline}

%%% This next bit of code defines all our theorem environments
\makeatletter
\let\c@theorem\relax
\let\c@corollary\relax
\makeatother

\let\definition\relax
\let\enddefinition\relax

\let\theorem\relax
\let\endtheorem\relax

\let\proposition\relax
\let\endproposition\relax

\let\exercise\relax
\let\endexercise\relax

\let\question\relax
\let\endquestion\relax

\let\remark\relax
\let\endremark\relax

\let\corollary\relax
\let\endcorollary\relax


\let\example\relax
\let\endexample\relax

\let\warning\relax
\let\endwarning\relax

\let\lemma\relax
\let\endlemma\relax

\newtheoremstyle{SlantTheorem}{\topsep}{\topsep}%%% space between body and thm
		{\slshape}                      %%% Thm body font
		{}                              %%% Indent amount (empty = no indent)
		{\bfseries\sffamily}            %%% Thm head font
		{}                              %%% Punctuation after thm head
		{3ex}                           %%% Space after thm head
		{\thmname{#1}\thmnumber{ #2}\thmnote{ \bfseries(#3)}}%%% Thm head spec
\theoremstyle{SlantTheorem}
\newtheorem{theorem}{Theorem}
\newtheorem{definition}[theorem]{Definition}
\newtheorem{proposition}[theorem]{Proposition}
%% \newtheorem*{dfnn}{Definition}
%% \newtheorem{ques}{Question}[theorem]
\newtheorem{lemma}[theorem]{Lemma}
%% \newtheorem*{war}{WARNING}
%% \newtheorem*{cor}{Corollary}
%% \newtheorem*{eg}{Example}
\newtheorem*{remark}{Remark}
\newtheorem*{touchstone}{Touchstone}
\newtheorem{corollary}{Corollary}[theorem]
\newtheorem*{example}{Example}
\newtheorem*{warning}{WARNING}


\newtheoremstyle{Exercise}{\topsep}{\topsep} %%% space between body and thm
		{}                           %%% Thm body font
		{}                           %%% Indent amount (empty = no indent)
		{\bfseries}                  %%% Thm head font
		{)}                          %%% Punctuation after thm head
		{ }                          %%% Space after thm head
		{\thmnumber{#2}\thmnote{ \bfseries(#3)}}%%% Thm head spec
\theoremstyle{Exercise}
\newtheorem{exercise}{}[theorem]

%% \newtheoremstyle{Question}{\topsep}{\topsep} %%% space between body and thm
%% 		{\bfseries}                  %%% Thm body font
%% 		{3ex}                        %%% Indent amount (empty = no indent)
%% 		{}                           %%% Thm head font
%% 		{}                           %%% Punctuation after thm head
%% 		{}                           %%% Space after thm head
%% 		{\thmnumber{#2}\thmnote{ \bfseries(#3)}}%%% Thm head spec
\newtheoremstyle{Question}{3em}{3em} %%% space between body and thm
		{\large\bfseries}                           %%% Thm body font
		{3ex}                           %%% Indent amount (empty = no indent)
		{\bfseries}                  %%% Thm head font
		{}                          %%% Punctuation after thm head
		{ }                          %%% Space after thm head
		{}%%% Thm head spec
\theoremstyle{Question}
\newtheorem*{question}{}



\renewcommand{\tilde}{\widetilde}
\renewcommand{\bar}{\overline}
\renewcommand{\hat}{\widehat}
\newcommand{\N}{\mathbb N}
\newcommand{\Z}{\mathbb Z}
\newcommand{\R}{\mathbb R}
\newcommand{\Q}{\mathbb Q}
\newcommand{\C}{\mathbb C}
\newcommand{\V}{\mathbb V}
\newcommand{\I}{\mathbb I}
\newcommand{\A}{\mathbb A}
\newcommand{\iso}{\simeq}
\newcommand{\ph}{\varphi}
\newcommand{\Cf}{\mathcal{C}}
\newcommand{\IZ}{\mathrm{Int}(\Z)}
\newcommand{\dsum}{\oplus}
\newcommand{\directsum}{\bigoplus}
\newcommand{\union}{\bigcup}
\renewcommand{\i}{\mathfrak}
\renewcommand{\a}{\mathfrak{a}}
\renewcommand{\b}{\mathfrak{b}}
\newcommand{\m}{\mathfrak{m}}
\newcommand{\p}{\mathfrak{p}}
\newcommand{\q}{\mathfrak{q}}
\newcommand{\dfn}[1]{\textbf{#1}\index{#1}}
\let\hom\relax
\DeclareMathOperator{\ann}{Ann}
\DeclareMathOperator{\h}{ht}
\DeclareMathOperator{\hom}{Hom}
\DeclareMathOperator{\Span}{Span}
\DeclareMathOperator{\spec}{Spec}
\DeclareMathOperator{\maxspec}{MaxSpec}
\DeclareMathOperator{\supp}{Supp}
\DeclareMathOperator{\ass}{Ass}
\DeclareMathOperator{\ff}{Frac}
\DeclareMathOperator{\im}{Im}
\DeclareMathOperator{\syz}{Syz}
\DeclareMathOperator{\gr}{Gr}
\renewcommand{\ker}{\mathop{\mathrm{Ker}}\nolimits}
\newcommand{\coker}{\mathop{\mathrm{Coker}}\nolimits}
\newcommand{\lps}{[\hspace{-0.25ex}[}
\newcommand{\rps}{]\hspace{-0.25ex}]}
\newcommand{\into}{\hookrightarrow}
\newcommand{\onto}{\twoheadrightarrow}
\newcommand{\tensor}{\otimes}
\newcommand{\x}{\mathbf{x}}
\newcommand{\X}{\mathbf X}
\newcommand{\Y}{\mathbf Y}
\renewcommand{\k}{\boldsymbol{\kappa}}
\renewcommand{\emptyset}{\varnothing}
\renewcommand{\qedsymbol}{$\blacksquare$}
\renewcommand{\l}{\ell}
\newcommand{\1}{\mathds{1}}
\newcommand{\lto}{\mathop{\longrightarrow\,}\limits}
\newcommand{\rad}{\sqrt}
\newcommand{\hf}{H}
\newcommand{\hs}{H\!S}
\newcommand{\hp}{H\!P}
\renewcommand{\vec}{\mathbf}
\renewcommand{\phi}{\varphi}
\renewcommand{\epsilon}{\varepsilon}
\renewcommand{\subset}{\subseteq}
\renewcommand{\supset}{\supseteq}
\newcommand{\macaulay}{\textsl{Macaulay2}}
\newcommand{\invlim}{\varprojlim}


%\renewcommand{\proofname}{Sketch of Proof}


\renewenvironment{proof}[1][Proof]
  {\begin{trivlist}\item[\hskip \labelsep \itshape \bfseries #1{}\hspace{2ex}]\upshape}
{\qed\end{trivlist}}

\newenvironment{sketch}[1][Sketch of Proof]
  {\begin{trivlist}\item[\hskip \labelsep \itshape \bfseries #1{}\hspace{2ex}]\upshape}
{\qed\end{trivlist}}



\makeatletter
\renewcommand\section{\@startsection{paragraph}{10}{\z@}%
                                     {-3.25ex\@plus -1ex \@minus -.2ex}%
                                     {1.5ex \@plus .2ex}%
                                     {\normalfont\large\sffamily\bfseries}}
\renewcommand\subsection{\@startsection{subparagraph}{10}{\z@}%
                                    {3.25ex \@plus1ex \@minus.2ex}%
                                    {-1em}%
                                    {\normalfont\normalsize\sffamily\bfseries}}
\makeatother

%% Fix weird index/bib issue.
\makeatletter
\gdef\ttl@savemark{\sectionmark{}}
\makeatother


\makeatletter
%% no number for refs
\newcommand\frontstyle{%
  \def\activitystyle{activity-chapter}
  \def\maketitle{%
    \addtocounter{titlenumber}{1}%
                    {\flushleft\small\sffamily\bfseries\@pretitle\par\vspace{-1.5em}}%
                    {\flushleft\LARGE\sffamily\bfseries\@title \par }%
                    {\vskip .6em\noindent\textit\theabstract\setcounter{problem}{0}\setcounter{sectiontitlenumber}{0}}%
                    \par\vspace{2em}
                    \phantomsection\addcontentsline{toc}{section}{\textbf{\@title}}%
                  }}
\makeatother


\author{Bart Snapp}

\title{Completions}

\begin{document}
\begin{abstract}
  We introduce adic completions. Sources and references:
  \cite{AK2012,AM1969,sD2008}.
\end{abstract}
\maketitle



\section{Cauchy sequences}


We now move from advanced algebra to undergraduate analysis.



\begin{definition}
  A sequence $(c_1,c_2,\dots)$ in $(M,d)$ is called a \dfn{Cauchy
    sequence} if for any $\epsilon>0$, there exists a positive integer
  $N$ such that for all natural numbers $m,n>N$,
  \[
  d(c_m,c_n) < \epsilon.
  \]
\end{definition}

\begin{definition}
  A metric space $(M,d)$ is called \dfn{complete} if every Cauchy
  sequence converges to an element of $X$, meaning if $(c_i)$ is a
  Cauchy sequence, then there is $\l$ in $M$ such that for any
  $\epsilon>0$, there is a positive integer $N$ such that whenever $n>N$
  \[
  d(\l,c_n) < \epsilon.
  \]
\end{definition}

Now, given a metric space, we can construct its \textit{completion} in
the following way. Call a sequences which converges to zero a
\index{null sequence}\dfn{null sequence}. Thus in the adic metric, a
null sequence is a sequence $(x_n)$ such that for every $n$ there
exists $N$ such that $n > N$ implies that $x_n \in I^mM$.  We first
construct $\hat{M}$ as follows:
\[
\hat{M} := \{\text{Cauchy sequences in $M$}\}/\{\text{null sequences in $M$}\}
\]

\begin{definition}
  Given an $R$-module $M$ and an ideal $I\subset R$ such that the
  $I$-adic topology induces a metric, the
  \textbf{$\boldsymbol{I}$-adic completion} is the completion of $M$
  in this space and is denoted $\hat{M}_I$.
\end{definition}



\section{Limits}

\begin{definition}\index{inverse system}
  A family of objects $(X_i)_{i\in \mathcal{I}}$ is a \textbf{inverse
    system indexed by a directed set $\boldsymbol{\mathcal{I}}$} if
  for every $i,j \in \mathcal{I}$ with $i \le j$ there exists a
  morphism $\ph_{j,i}:X_j \to X_ i$ such that:
  \begin{enumerate}
  \item $\ph_{i,i} = \1_{X_i}$ for all $i \in \mathcal{I}$.
  \item For any $i,j,k \in \mathcal{I}$ where $i\le j\le k$, the
    following diagram commutes:
    \[
    \begin{tikzcd}
      X_j \ar[rr,"\ph_{j,i}"] &    & X_i\\
      &  X_\gamma\ar[ul,"\ph_{k,j}"] \ar[ur,"\ph_{k,i}"]  &
    \end{tikzcd}
    \]
  \end{enumerate}
\end{definition}


\begin{definition}\index{inverse limit}\index{limit}
  An \textbf{inverse limit}, which is an example of a \textbf{limit},
  of a inverse system $(X_i)_{i\in \mathcal{I}}$, is an
  object, denoted by $\invlim(X_i)$, with morphisms
  $\ph_i:\invlim(X_i)\to X_i$ such that for all
  $i, j \in \mathcal{I}$ with $i \le j$ we have
  $\ph_{j,i} \circ \ph_j = \ph_i$.  Further, for
  every object $Y$ with compatible morphisms $\psi_i:Y\to
  X_i$, there exists a unique morphism $\ph$ making the diagram
  below commute for all $i\le j$:
\[
\begin{tikzcd}
  Y\ar[ddr,bend right=60,swap,"\psi_i"] \ar[dr,swap,"\psi_j"]  \ar[dashed,rr,"\ph"] &   &  \invlim(X_{i}) \ar[ddl,bend left=60,"\ph_i"] \ar[dl,"\ph_j"]\\
  & X_{j} \ar[d,"\ph_{j,i}"] &\\
  & X_{i}&
\end{tikzcd}
\]
\end{definition}

\begin{example}
  If we consider the category of sets, where the morphisms are set
  inclusion, then given $X_0\supset X_1\supset \cdots \supset
  X_n\supset \cdots$, show that
  \[
  \invlim(X_i) = \bigcap_{i=0}^\infty X_i.
  \]
  \begin{proof}
    Consider the following diagram:
    \[
    \begin{tikzcd}
      Y\ar[ddr,bend right=60,swap,"\psi_i"] \ar[dr,swap,"\psi_j"]  \ar[dashed,rr,"\iota"] &   &  \bigcap_{i=0}^\infty(X_{i}) \ar[ddl,bend left=60,"\iota_i"] \ar[dl,"\iota_j"]\\
      & X_{j} \ar[d,"\iota_{j,i}"] &\\
      & X_{i}&
    \end{tikzcd}
    \]
    where $j>i$, $\iota_i$, $\iota_j$, and $\iota_{j,i}$ are set
    inclusion maps.  We must show that the map
    \[
    \iota: Y \to \bigcap_{i=0}^\infty X_i
    \]
    is unique.  Suppose we have a set $Y$ such that the diagram above
    commutes, we must show that the map $\iota$ is unique.  If $y\in
    Y$ maps to $x\in X_j\subset X_i$, and $x\in X_k$ for $k\ge j$,
    then $x\in\bigcap_{i=0}^\infty X_i$, hence $\iota(y) = x$. This
    map is unique, and hence $\invlim(X_i) = \bigcap_{i=0}^\infty
    X_i$.
  \end{proof}
\end{example}



\begin{example}
  For a given inverse system, $(X_i)_{i\in \mathcal{I}}$,
  show that
  \[
  \invlim(X_i) = \left\{(x_i)_{i\in \mathcal{I}}\in
  \prod_{i\in \mathcal{I}} X_i:\text{if }i\le j,\text{ then }x_i =
  \ph_{j,i}(x_j) \right\}.
  \]
  \begin{proof}
    Let
    \[
    L =
    \left\{(x_i)_{i\in \mathcal{I}}\in \prod_{i\in \mathcal{I}} X_i:\text{if }i\le j,\text{ then }x_i = \ph_{j,i}(x_j) \right\}
    \]
    and consider
    \[
    \begin{tikzcd}
      Y\ar[ddr,bend right=60,swap,"\psi_i"] \ar[dr,swap,"\psi_j"]  \ar[dashed,rr,"\ph"] &   &  L \ar[ddl,bend left=60,"\ph_i"] \ar[dl,"\ph_j"]\\
      & X_{j} \ar[d,"\ph_{j,i}"] &\\
      & X_{i}&
    \end{tikzcd}
    \]
    Suppose we have $Y$ so that the diagram commutes. In this case,
    the commuting triangle on the left, $\psi_i =
    \phi_{j,i}\circ\psi_j$, completely and uniquely determine where
    $y\in Y$ must map.
  \end{proof}
\end{example}




\begin{example}
  Let $\hat{\Z}_{(p)}$ be the Cauchy sequences in $\Z_{(p)}$ modulo the
  null sequences. We claim $\hat{\Z}_{(p)}\iso\invlim(\Z/p^n/\Z)$.
  \begin{proof}
    Consider the following diagram:
    \[
    \begin{tikzcd}
      R\ar[ddr,bend right=60,swap,"\psi_i"] \ar[dr,swap,"\psi_j"]  \ar[dashed,rr,"\ph"] &   &  \hat{\Z}_{(p)} \ar[ddl,bend left=60,"\ph_i"] \ar[dl,"\ph_j"]\\
      & \Z/p^j\Z \ar[d,"\ph_{j,i}"] &\\
      & \Z/p^i\Z&
    \end{tikzcd}
    \]
    We must show when bottom triangles commute, that we can find a
    unique ring homomorphism so that the top triangle commutes. Let
    $(c_0,c_1,\dots)$ be a Cauchy sequence (actually an element) in
    $\hat{\Z}_(p)$. This sequence is compatible with our maps above in
    the sense that
    \[
    \phi_i(\vec{c}) = \phi_{j,i} \circ \phi_j(\vec{c}).
    \]
    Now suppose $R$ is given along with other compatible maps,
    \[
    \psi_i(r) = \phi_{j,i}\circ \psi_j(r)
    \]
    Now, looking at the image of $r$ in $\Z/p^i\Z$, we have
    \[
    (\psi_0(r), \psi_1(r),\dots)
    \]
    which can be viewed as an element of $\hat{\Z}_{(p)}$. This map is
    unique, as if another Cauchy sequence is found, it varies from the
    original by a null sequence, which go to zero in $\hat{\Z}_{(p)}$.
  \end{proof}
\end{example}

\begin{definition}\index{formal power series ring}\index{power series ring}\index{AX@$A\lps X \rps$}
  Given a ring $A$, we can form the ring of \textbf{formal power
    series} in $X_1,\dots,X_n$ over $A$ by considering all infinite
  sums of the form
  \[
  \sum_{i=0}^\infty a_{i_1,\dots, i_n}X_1^{i_1}\cdots
  X_n^{i_n},\qquad\text{where} \qquad a_{i_1,\dots, i_n}\in A.
  \]
  Sums such as these form a ring under the canonical rules for
  summation and product. We denote the ring of formal power series
  over $A$ in $n$ variables by $A\lps X_1,\dots,X_n\rps$.
\end{definition}

\begin{exercise}
  If $B = A[X_1,\dots,X_n]$ and $I = (X_1,\dots,X_n)$. Show $(A/I^t)$
  form an inverse system. Moreover, show that
  \[
  \invlim(B/I^n) \iso A\lps X_1,\dots,X_n\rps.
  \]
\end{exercise}





\begin{proposition} \label{invlimleftexact}
Suppose we have inverse systems $(A_i,\alpha_{j,i})_{i \in \mathcal{I}}$,
$(B_i,\beta_{j,i})_{i \in \mathcal{I}}$, and $(C_i,\gamma_{j,i})_{i \in
  \mathcal{I}}$, over the directed set $\mathcal{I}$ and maps
$(f_i):(A_i) \to (B_i)$ and $(g_i):(B_i)
\to (C_i)$ such that for every $i \in \mathcal{I}$
\[
0 \lto A_i \lto^{f_i} B_i \lto^{g_i} C_i\lto 0
\]
is exact.  Then
\[
0 \lto \invlim(A_i) \lto^{\invlim f_i} \invlim(B_i) \lto^{\invlim g_i} \invlim(C_i)
\]
is exact.  In other words, inverse limit is a left exact functor from
the category of inverse systems of modules over a fixed directed set
to the category of modules.
\begin{proof}
  Consider the diagram below where
  \[
  \begin{tikzcd}
    0 \ar[r] & \prod A_i  \ar[r]  \ar[d,"d_A"] &   \prod B_i \ar[r]  \ar[d,"d_B"] &  \prod C_i  \ar[r] \ar[d,"d_C"] & 0\\
    0 \ar[r] & \prod A_i  \ar[r]  &   \prod B_i \ar[r]  &  \prod C_i  \ar[r] & 0
  \end{tikzcd}
  \]
  where
  \begin{align*}
    d_A &= a_n-\alpha_{n+1,n}(a_{n+1})\\
    d_B &= b_n-\beta_{n+1,n}(b_{n+1})\\
    d_C &= c_n-\gamma_{n+1,n}(c_{n+1}).
  \end{align*}
  In this case, the diagram commutes and we may apply the \index{Snake Lemma}
  Snake Lemma to find the exact sequence
  \[
  0 \to \ker(d_A) \to \ker(d_B) \to \ker(d_C) \to \coker(d_A).
  \]
  Since $\ker(d_A) = \invlim(A_i)$, $\ker(d_B) = \invlim(B_i)$, and
  $\ker(d_C) = \invlim(C_i)$, we have our result.
\end{proof}
\end{proposition}

\begin{remark}
  Note that if $d_A$ is surjective in the proof above, then in fact we
  have a short exact sequence
  \[
  0 \lto \invlim(A_i) \lto^{\invlim \ph_i} \invlim(B_i) \lto^{\invlim \psi_i} \invlim(C_i) \lto 0.
  \]
\end{remark}


\begin{theorem}
  Let $R$ be a ring, $M$ an $R$-module, and suppose that $M$ has an
  $I$-adic metric. Then
  \[
  \hat{M}_I \iso \invlim(M/I^n M).
  \]
  \begin{proof}
    The module $\hat{M}_I$ is the set of Cauchy sequences modulo null
    sequences.  Consider the following diagram:
    \[
    \begin{tikzcd}
      N\ar[ddr,bend right=60,swap,"\psi_i"] \ar[dr,swap,"\psi_j"]  \ar[dashed,rr,"\ph"] &   &  \hat{M}_{I} \ar[ddl,bend left=60,"\ph_i"] \ar[dl,"\ph_j"]\\
      & M/I^jM \ar[d,"\ph_{j,i}"] &\\
      & M/I^iM &
    \end{tikzcd}
    \]
    We must show when bottom triangles commute, that we can find a
    unique ring homomorphism so that the top triangle commutes. Let
    $(m_0,m_1,\dots)$ be a Cauchy sequence (actually an element) in
    $\hat{M}_I$. This sequence is compatible with our maps above in
    the sense that
    \[
    \phi_i(\vec{m}) = \phi_{j,i} \circ \phi_j(\vec{m}).
    \]
    Now suppose $N$ is given along with other compatible maps,
    \[
    \psi_i(n) = \phi_{j,i}\circ \psi_j(n)
    \]
    Now, looking at the image of $n$ in $M/I^iM$, we have
    \[
    (\psi_0(n), \psi_1(n),\dots)
    \]
    which can be viewed as an element of $\hat{M}_{I}$. This map is
    unique, as if another Cauchy sequence is found, it varies from the
    original by a null sequence, which go to zero in $\hat{M}_{I}$.
  \end{proof}
\end{theorem}











\end{document}
