\documentclass{ximera}



\usepackage{tikz-cd}
\usepackage[sans]{dsfont}

\DefineVerbatimEnvironment{macaulay2}{Verbatim}{numbers=left,frame=lines,label=Macaulay2,labelposition=topline}

%%% This next bit of code defines all our theorem environments
\makeatletter
\let\c@theorem\relax
\let\c@corollary\relax
\makeatother

\let\definition\relax
\let\enddefinition\relax

\let\theorem\relax
\let\endtheorem\relax

\let\proposition\relax
\let\endproposition\relax

\let\exercise\relax
\let\endexercise\relax

\let\question\relax
\let\endquestion\relax

\let\remark\relax
\let\endremark\relax

\let\corollary\relax
\let\endcorollary\relax


\let\example\relax
\let\endexample\relax


\let\lemma\relax
\let\endlemma\relax

\newtheoremstyle{SlantTheorem}{\topsep}{\topsep}%%% space between body and thm
		{\slshape}                      %%% Thm body font
		{}                              %%% Indent amount (empty = no indent)
		{\bfseries\sffamily}            %%% Thm head font
		{}                              %%% Punctuation after thm head
		{3ex}                           %%% Space after thm head
		{\thmname{#1}\thmnumber{ #2}\thmnote{ \bfseries(#3)}}%%% Thm head spec
\theoremstyle{SlantTheorem}
\newtheorem{theorem}{Theorem}
\newtheorem{definition}[theorem]{Definition}
\newtheorem{proposition}[theorem]{Proposition}
%% \newtheorem*{dfnn}{Definition}
%% \newtheorem{ques}{Question}[theorem]
\newtheorem{lemma}[theorem]{Lemma}
%% \newtheorem*{war}{WARNING}
%% \newtheorem*{cor}{Corollary}
%% \newtheorem*{eg}{Example}
\newtheorem*{remark}{Remark}
\newtheorem*{touchstone}{Touchstone}
\newtheorem{corollary}{Corollary}[theorem]
\newtheorem*{example}{Example}


\newtheoremstyle{Exercise}{\topsep}{\topsep} %%% space between body and thm
		{}                           %%% Thm body font
		{}                           %%% Indent amount (empty = no indent)
		{\bfseries}                  %%% Thm head font
		{)}                          %%% Punctuation after thm head
		{ }                          %%% Space after thm head
		{\thmnumber{#2}\thmnote{ \bfseries(#3)}}%%% Thm head spec
\theoremstyle{Exercise}
\newtheorem{exercise}{}[theorem]

%% \newtheoremstyle{Question}{\topsep}{\topsep} %%% space between body and thm
%% 		{\bfseries}                  %%% Thm body font
%% 		{3ex}                        %%% Indent amount (empty = no indent)
%% 		{}                           %%% Thm head font
%% 		{}                           %%% Punctuation after thm head
%% 		{}                           %%% Space after thm head
%% 		{\thmnumber{#2}\thmnote{ \bfseries(#3)}}%%% Thm head spec
\newtheoremstyle{Question}{3em}{3em} %%% space between body and thm
		{\large\bfseries}                           %%% Thm body font
		{3ex}                           %%% Indent amount (empty = no indent)
		{\bfseries}                  %%% Thm head font
		{}                          %%% Punctuation after thm head
		{ }                          %%% Space after thm head
		{}%%% Thm head spec
\theoremstyle{Question}
\newtheorem*{question}{}



\renewcommand{\tilde}{\widetilde}
\renewcommand{\bar}{\overline}
\renewcommand{\hat}{\widehat}
\newcommand{\N}{\mathbb N}
\newcommand{\Z}{\mathbb Z}
\newcommand{\R}{\mathbb R}
\newcommand{\Q}{\mathbb Q}
\newcommand{\C}{\mathbb C}
\newcommand{\V}{\mathbb V}
\newcommand{\I}{\mathbb I}
\newcommand{\A}{\mathbb A}
\newcommand{\iso}{\simeq}
\newcommand{\ph}{\varphi}
\newcommand{\Cf}{\mathcal{C}}
\newcommand{\IZ}{\mathrm{Int}(\Z)}
\newcommand{\dsum}{\oplus}
\newcommand{\directsum}{\coprod}
\newcommand{\union}{\bigcup}
\renewcommand{\i}{\mathfrak}
\renewcommand{\a}{\mathfrak{a}}
\renewcommand{\b}{\mathfrak{b}}
\newcommand{\m}{\mathfrak{m}}
\newcommand{\p}{\mathfrak{p}}
\newcommand{\q}{\mathfrak{q}}
\newcommand{\dfn}{\textbf}
\let\hom\relax
\DeclareMathOperator{\ann}{Ann}
\DeclareMathOperator{\h}{ht}
\DeclareMathOperator{\hom}{Hom}
\DeclareMathOperator{\spec}{Spec}
\DeclareMathOperator{\supp}{Supp}
\DeclareMathOperator{\ass}{Ass}
\DeclareMathOperator{\ff}{Frac}
\DeclareMathOperator{\im}{Im}
\DeclareMathOperator{\syz}{Syz}
\DeclareMathOperator{\gr}{Gr}
\renewcommand{\ker}{\mathop{\mathrm{Ker}}\nolimits}
\newcommand{\lps}{[\hspace{-0.25ex}[}
\newcommand{\rps}{]\hspace{-0.25ex}]}
\newcommand{\into}{\hookrightarrow}
\newcommand{\onto}{\twoheadrightarrow}
\newcommand{\tensor}{\otimes}
\newcommand{\x}{\mathbf{x}}
\newcommand{\X}{\mathbf X}
\newcommand{\Y}{\mathbf Y}
\renewcommand{\k}{\boldsymbol{\kappa}}
\renewcommand{\emptyset}{\varnothing}
\renewcommand{\qedsymbol}{$\blacksquare$}
\renewcommand{\l}{\ell}
\newcommand{\1}{\mathds{1}}
\newcommand{\lto}{\mathop{\longrightarrow\,}\limits}
\newcommand{\rad}{\sqrt}
\renewcommand{\vec}{\mathbf}
\renewcommand{\phi}{\varphi}
\renewcommand{\epsilon}{\varepsilon}
\renewcommand{\subset}{\subseteq}
\renewcommand{\supset}{\supseteq}
\newcommand{\macaulay}{\textsl{Macaulay2}}
\newcommand{\invlim}{\varprojlim}


%\renewcommand{\proofname}{Sketch of Proof}


\renewenvironment{proof}[1][Proof]
  {\begin{trivlist}\item[\hskip \labelsep \itshape \bfseries #1{}\hspace{2ex}]\upshape}
{\qed\end{trivlist}}

\newenvironment{sketch}[1][Sketch of Proof]
  {\begin{trivlist}\item[\hskip \labelsep \itshape \bfseries #1{}\hspace{2ex}]\upshape}
{\qed\end{trivlist}}



\makeatletter
\renewcommand\section{\@startsection{paragraph}{10}{\z@}%
                                     {-3.25ex\@plus -1ex \@minus -.2ex}%
                                     {1.5ex \@plus .2ex}%
                                     {\normalfont\large\sffamily\bfseries}}
\renewcommand\subsection{\@startsection{subparagraph}{10}{\z@}%
                                    {3.25ex \@plus1ex \@minus.2ex}%
                                    {-1em}%
                                    {\normalfont\normalsize\sffamily\bfseries}}
\makeatother

%% Fix weird index/bib issue.
\makeatletter
\gdef\ttl@savemark{\sectionmark{}}
\makeatother


\author{Bart Snapp}

\title{Filtrations and gradings}

\begin{document}
\begin{abstract}
  We introduce filtrations. Sources and
  references: \cite{sD2008,jpS2000}.
\end{abstract}
\maketitle


\section{Filtrations}


\begin{definition}\index{filtration}\index{filtered!ring}
  If $R$ is a ring, we call a descending chain of additive subgroups
  \[
  R = R_0 \supset R_1\supset \cdots \supset R_n\supset \cdots
  \]
  a \dfn{filtration} of $R$ if 
  \[
  R_i R_j \subset R_{i+j}.
  \]
  We say that a ring with a filtration is a \textbf{filtered ring}.
\end{definition}



\begin{definition}\index{filtered!module}
  If $R$ is a filtered ring with filtration $(R_n)$ and $M$ is an
  $R$-module, then $M$ is a \dfn{filtered module} if
  \[
  M = M_0  \supset M_1\supset \cdots \supset M_n\supset \cdots
  \]
  is a descending chain of subgroups of $M$ such that 
  \[
  R_i M_j \subset M_{i+j}.
  \]
\end{definition}


We are not agnoistic with regard to our filtration. We are intrerested
in what is called the \dfn{adic} filtration.

\begin{definition}\index{filtration!adic}\index{adic!filtration}
  Let $R$ be a ring, the \textbf{$\boldsymbol{I}$-adic filtration} of
  $R$ is given by
  \[
  R \supset I \supset I^2 \supset \cdots \supset I^n \supset \cdots
  \]
  If $M$ is an $R$-module and $I$ is an ideal of $R$, then the
  \textbf{$\boldsymbol{I}$-adic filtration} is the filtration:
  \[
  M \supset  I M \supset I^2 M \supset \cdots  \supset  I^n M \supset \cdots 
  \]
In other words, the filtration $(M_n)$ is given by $M_n = I^n M$.
\end{definition}



 
\begin{definition}\index{filtration!induced}\index{induced filtration}
  Let $M$ be a filtered $R$-module with filtration $(M_n)$ and let $N$
  be a submodule of $M$. Then setting $N_n= N\cap M_n$ forms a
  filtration for $N$. This is called the \dfn{induced filtration}.
\end{definition}

\begin{definition}\index{filtration!image}\index{image filtration}
  Let $M$ be a filtered $R$-module with filtration $(M_n)$ which
  surjects onto another $R$-module $N$ via a module homomorphism
  \[
  \ph:M\onto N.
  \]
  Setting $N_n = \ph(M_n)$, we obtain the \dfn{image filtration}.
\end{definition}

\begin{definition}\index{filtered!map}
  A module homomorphism $\ph:M\to N$ of filtered modules is called a
  \dfn{filtered map} if
\[
\ph(M_n)\subset N_n.
\]
\end{definition}

\begin{definition}\index{filtered!map!strict}\index{strict filtered map}
  Suppose that $\ph:M\to N$ is a filtered map. Then $\ph$ is called
  \dfn{strict} if:
\[
\underbrace{\ph(M_n)}_{\text{image filtration}} = \underbrace{\ph(M)\cap N_n}_{\text{induced filtration}}
\] 
\end{definition}



\section{Graded rings and modules}


\begin{definition}\index{graded!ring}\index{homogeneous element}\index{homogeneous element!degree}
  A ring $R$ is called a \dfn{graded ring} if it can be written as a
  direct sum of subgroups
  \[
  R=\directsum_{n =0}^\infty R_n,
  \]
  where $R_i R_j \subset R_{i+j}$.  Further, elements of $R_i$ are
  called \dfn{homogeneous} elements of degree $i$.  
\end{definition}

\begin{remark}
  Our canonical example of a graded rings is a polynomial ring over a
  field.
\end{remark}



\begin{definition}\index{irrelevant ideal}\index{irrevelent@$A_+$}
  If $R$ is graded ring:
  \[
  R = R_0\oplus \underbrace{R_1\oplus \cdots R_n\oplus \cdots}_{R_+}
  \]
  Then $R_+$ is called the \dfn{irrelevant ideal} of $R$.
\end{definition}

\begin{exercise}
  Let $R$ be a graded ring.
  \begin{enumerate}
  \item $1_R\in R_0$.
  \item $R_0$ is a ring.
  \item $R$ is Noetherian if and only if $R_0$ is Noetherian and $R_+$
    is a finitely generated ideal of $R$.
  \end{enumerate}
\end{exercise}

\begin{definition}\index{graded!module}
  A module $M$ is called a \dfn{graded module} if it can be written
  as a direct sum of subgroups
  \[
  \directsum_{n =0}^\infty M_n,
  \]
  where $R_iM_j \subset M_{i + j}$.
\end{definition}


%% We will work in the adic filtrations, though more general statements
%% can be made.


%% \begin{definition}\index{Gr1@$\gr(R)$}\index{graded!ring!associated to a filtration}\index{filtered!ring!associated graded ring}
%%   Let $R$ be a ring and $I\subset R$ be a proper ideal. The
%%   \dfn{graded ring associated to the $\boldsymbol{I}$-adic filtration}
%%   is defined to be
%%   \[
%%   \gr_I(R) := \directsum_{n=0}^\infty I^n/I^{n+1}
%%   \]
%%   Similarly, given an $R$-module $M$,  we have the associated graded module
%%   \[
%%   \gr_I(M) := \directsum_{n = 0}^\infty I^nM / I^{n+1}M.
%%   \]
%% \end{definition}






\begin{theorem}[Artin-Rees]\label{L:ArtinRees}\index{Artin-Rees}
  If $R$ is a Noetherian ring with an ideal $I$, and $M$ is a finitely
  generated $R$-module with submodule $N$, then there exists $m\ge 0$
  such that
  \[
  N\cap I^{m+n}M = I^n(N\cap I^m M)
  \]
  for all $n\ge 0$.
  \begin{proof}
    $(\subset)$ Set
    \[
    R^* = R \oplus I \oplus I^2 \oplus \cdots,
    \]
    \[
    M^* = M \oplus IM \oplus I^2M \oplus \cdots,
    \]
    and
    \[
    N^* = N \oplus N \cap IM \oplus N \cap I^2M \oplus \cdots.
    \]
    Since $R$ is Noetherian, $I$ is finitely generated and hence $R^*$
    is Noetherian as we can surject $R[\X]$ onto $R^*$.  Since $M$ is
    finitely generated over $R$, $M^*$ is finitely generated over
    $R^*$, and hence is also Noetherian.  Thus $N^*$ is finitely
    generated over $R^*$.  We may choose generators of $N^*$,
    $\eta_1,\ldots,\eta_k$, such that each is of homogeneous degree
    $d_1,\ldots,d_k$ respectively, that is to say, $\eta_i\in N\cap
    I^{d_i}M$.  Set
    \[
    m = \max\{d_1,\dots, d_k \}.
    \]  
    Suppose $x \in N \cap I^{m+n}M$ for $n \ge 0$.  Then we may write 
    \[
    x =  a_1 \eta_1 + a_2 \eta_2 + \dots + a_k \eta_k
    \]
    where $\deg(a_i) = m + n - d_i \ge n$.  Thus $a_i \in I^n$.  So we can write
    \[
    x = b_1a_1'\eta_1 + \cdots + b_ka_k'\eta_k
    \]
    where $b_i \in I^n$ and $a_i = b_ia_i'$.  Therefore $N \cap I^{m + n}M
    \subset I^n(N \cap I^m M)$.

    $(\supset)$ Write:
    \begin{align*}
      I^n(N\cap I^m M) &= I^n N\cap I^{m+n} M\\
      &\subset N\cap I^{m+n}M.
    \end{align*}
  \end{proof}
\end{theorem}

\end{document}
