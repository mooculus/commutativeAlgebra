\documentclass{ximera}



\usepackage{tikz-cd}
\usepackage[sans]{dsfont}

\let\oldbibliography\thebibliography%% to compact bib
\renewcommand{\thebibliography}[1]{%
  \oldbibliography{#1}%
  \setlength{\itemsep}{0pt}%
}


\DefineVerbatimEnvironment{macaulay2}{Verbatim}{numbers=left,frame=lines,label=Macaulay2,labelposition=topline}

%%% This next bit of code defines all our theorem environments
\makeatletter
\let\c@theorem\relax
\let\c@corollary\relax
\makeatother

\let\definition\relax
\let\enddefinition\relax

\let\theorem\relax
\let\endtheorem\relax

\let\proposition\relax
\let\endproposition\relax

\let\exercise\relax
\let\endexercise\relax

\let\question\relax
\let\endquestion\relax

\let\remark\relax
\let\endremark\relax

\let\corollary\relax
\let\endcorollary\relax


\let\example\relax
\let\endexample\relax

\let\warning\relax
\let\endwarning\relax

\let\lemma\relax
\let\endlemma\relax

\newtheoremstyle{SlantTheorem}{\topsep}{\topsep}%%% space between body and thm
		{\slshape}                      %%% Thm body font
		{}                              %%% Indent amount (empty = no indent)
		{\bfseries\sffamily}            %%% Thm head font
		{}                              %%% Punctuation after thm head
		{3ex}                           %%% Space after thm head
		{\thmname{#1}\thmnumber{ #2}\thmnote{ \bfseries(#3)}}%%% Thm head spec
\theoremstyle{SlantTheorem}
\newtheorem{theorem}{Theorem}
\newtheorem{definition}[theorem]{Definition}
\newtheorem{proposition}[theorem]{Proposition}
%% \newtheorem*{dfnn}{Definition}
%% \newtheorem{ques}{Question}[theorem]
\newtheorem{lemma}[theorem]{Lemma}
%% \newtheorem*{war}{WARNING}
%% \newtheorem*{cor}{Corollary}
%% \newtheorem*{eg}{Example}
\newtheorem*{remark}{Remark}
\newtheorem*{touchstone}{Touchstone}
\newtheorem{corollary}{Corollary}[theorem]
\newtheorem*{example}{Example}
\newtheorem*{warning}{WARNING}


\newtheoremstyle{Exercise}{\topsep}{\topsep} %%% space between body and thm
		{}                           %%% Thm body font
		{}                           %%% Indent amount (empty = no indent)
		{\bfseries}                  %%% Thm head font
		{)}                          %%% Punctuation after thm head
		{ }                          %%% Space after thm head
		{\thmnumber{#2}\thmnote{ \bfseries(#3)}}%%% Thm head spec
\theoremstyle{Exercise}
\newtheorem{exercise}{}[theorem]

%% \newtheoremstyle{Question}{\topsep}{\topsep} %%% space between body and thm
%% 		{\bfseries}                  %%% Thm body font
%% 		{3ex}                        %%% Indent amount (empty = no indent)
%% 		{}                           %%% Thm head font
%% 		{}                           %%% Punctuation after thm head
%% 		{}                           %%% Space after thm head
%% 		{\thmnumber{#2}\thmnote{ \bfseries(#3)}}%%% Thm head spec
\newtheoremstyle{Question}{3em}{3em} %%% space between body and thm
		{\large\bfseries}                           %%% Thm body font
		{3ex}                           %%% Indent amount (empty = no indent)
		{\bfseries}                  %%% Thm head font
		{}                          %%% Punctuation after thm head
		{ }                          %%% Space after thm head
		{}%%% Thm head spec
\theoremstyle{Question}
\newtheorem*{question}{}



\renewcommand{\tilde}{\widetilde}
\renewcommand{\bar}{\overline}
\renewcommand{\hat}{\widehat}
\newcommand{\N}{\mathbb N}
\newcommand{\Z}{\mathbb Z}
\newcommand{\R}{\mathbb R}
\newcommand{\Q}{\mathbb Q}
\newcommand{\C}{\mathbb C}
\newcommand{\V}{\mathbb V}
\newcommand{\I}{\mathbb I}
\newcommand{\A}{\mathbb A}
\newcommand{\iso}{\simeq}
\newcommand{\ph}{\varphi}
\newcommand{\Cf}{\mathcal{C}}
\newcommand{\IZ}{\mathrm{Int}(\Z)}
\newcommand{\dsum}{\oplus}
\newcommand{\directsum}{\bigoplus}
\newcommand{\union}{\bigcup}
\renewcommand{\i}{\mathfrak}
\renewcommand{\a}{\mathfrak{a}}
\renewcommand{\b}{\mathfrak{b}}
\newcommand{\m}{\mathfrak{m}}
\newcommand{\p}{\mathfrak{p}}
\newcommand{\q}{\mathfrak{q}}
\newcommand{\dfn}[1]{\textbf{#1}\index{#1}}
\let\hom\relax
\DeclareMathOperator{\ann}{Ann}
\DeclareMathOperator{\h}{ht}
\DeclareMathOperator{\hom}{Hom}
\DeclareMathOperator{\Span}{Span}
\DeclareMathOperator{\spec}{Spec}
\DeclareMathOperator{\maxspec}{MaxSpec}
\DeclareMathOperator{\supp}{Supp}
\DeclareMathOperator{\ass}{Ass}
\DeclareMathOperator{\ff}{Frac}
\DeclareMathOperator{\im}{Im}
\DeclareMathOperator{\syz}{Syz}
\DeclareMathOperator{\gr}{Gr}
\renewcommand{\ker}{\mathop{\mathrm{Ker}}\nolimits}
\newcommand{\coker}{\mathop{\mathrm{Coker}}\nolimits}
\newcommand{\lps}{[\hspace{-0.25ex}[}
\newcommand{\rps}{]\hspace{-0.25ex}]}
\newcommand{\into}{\hookrightarrow}
\newcommand{\onto}{\twoheadrightarrow}
\newcommand{\tensor}{\otimes}
\newcommand{\x}{\mathbf{x}}
\newcommand{\X}{\mathbf X}
\newcommand{\Y}{\mathbf Y}
\renewcommand{\k}{\boldsymbol{\kappa}}
\renewcommand{\emptyset}{\varnothing}
\renewcommand{\qedsymbol}{$\blacksquare$}
\renewcommand{\l}{\ell}
\newcommand{\1}{\mathds{1}}
\newcommand{\lto}{\mathop{\longrightarrow\,}\limits}
\newcommand{\rad}{\sqrt}
\newcommand{\hf}{H}
\newcommand{\hs}{H\!S}
\newcommand{\hp}{H\!P}
\renewcommand{\vec}{\mathbf}
\renewcommand{\phi}{\varphi}
\renewcommand{\epsilon}{\varepsilon}
\renewcommand{\subset}{\subseteq}
\renewcommand{\supset}{\supseteq}
\newcommand{\macaulay}{\textsl{Macaulay2}}
\newcommand{\invlim}{\varprojlim}


%\renewcommand{\proofname}{Sketch of Proof}


\renewenvironment{proof}[1][Proof]
  {\begin{trivlist}\item[\hskip \labelsep \itshape \bfseries #1{}\hspace{2ex}]\upshape}
{\qed\end{trivlist}}

\newenvironment{sketch}[1][Sketch of Proof]
  {\begin{trivlist}\item[\hskip \labelsep \itshape \bfseries #1{}\hspace{2ex}]\upshape}
{\qed\end{trivlist}}



\makeatletter
\renewcommand\section{\@startsection{paragraph}{10}{\z@}%
                                     {-3.25ex\@plus -1ex \@minus -.2ex}%
                                     {1.5ex \@plus .2ex}%
                                     {\normalfont\large\sffamily\bfseries}}
\renewcommand\subsection{\@startsection{subparagraph}{10}{\z@}%
                                    {3.25ex \@plus1ex \@minus.2ex}%
                                    {-1em}%
                                    {\normalfont\normalsize\sffamily\bfseries}}
\makeatother

%% Fix weird index/bib issue.
\makeatletter
\gdef\ttl@savemark{\sectionmark{}}
\makeatother


\makeatletter
%% no number for refs
\newcommand\frontstyle{%
  \def\activitystyle{activity-chapter}
  \def\maketitle{%
    \addtocounter{titlenumber}{1}%
                    {\flushleft\small\sffamily\bfseries\@pretitle\par\vspace{-1.5em}}%
                    {\flushleft\LARGE\sffamily\bfseries\@title \par }%
                    {\vskip .6em\noindent\textit\theabstract\setcounter{problem}{0}\setcounter{sectiontitlenumber}{0}}%
                    \par\vspace{2em}
                    \phantomsection\addcontentsline{toc}{section}{\textbf{\@title}}%
                  }}
\makeatother


\author{Bart Snapp}

\title{Topology and adic filtrations}

\begin{document}
\begin{abstract}
  We introduce topology to our study of adic filtrations. Sources and
  references: \cite{sD2008,jpS2000}.
\end{abstract}
\maketitle


\section{Topological rings and modules}

\begin{definition}\index{topological!ring}
  A ring $R$ is a \dfn{topological ring} if there exists some topology
  on $R$ such that the maps:
  \begin{align*}
    R\times R &\to R    & R\times R &\to R    \\
    (x,y) &\mapsto x+y  & (x,y) &\mapsto xy   
  \end{align*}
  are all continuous maps.
\end{definition}


\begin{definition}\index{topological!module}
  If $R$ is a topological ring, an $R$-module $M$ is a
  \dfn{topological module} if there exists some topology $M$ such that
  the maps:
  \begin{align*}
    M\times M &\to M    & R\times M &\to M \\
    (m,n) &\mapsto m+n  & (x,m) &\mapsto xm
  \end{align*}
  are both continuous maps.
\end{definition}


Now recall the open-set definition of a topology:

\begin{definition}
  A \dfn{topological space} is a set $X$ with a set of subsets called
  the \dfn{open sets} that satisfy the following properties:
  \begin{enumerate}
  \item The empty set and $X$ are open.
  \item Unions of arbitrary collections of open sets are
    open.
  \item Finite intersections of open sets are open.
  \end{enumerate}
\end{definition}

\begin{exercise}
  Prove the equivalence of the open set definition of a topology and
  the closed set definition of a topology.
\end{exercise}


While there is a more general presentation, we will stick to what we
are most interested in, adic topologies.


\begin{definition}
  Let $R$ be a ring, $I \subset R$ be an ideal, and $M$ an
  $R$-module. The \textbf{$\boldsymbol{I}$-adic topology} on $M$ is
  where $U \subset M$ is defined to be an open set if for each $x\in
  U$, there exists a positive integer $n$ such that
  \[
  x + I^n M \subset U.
  \]
\end{definition}



\begin{proposition}
  Let $R$ be a ring, $I\subset R$ be an ideal, and $M$ be a module
  equipped with the $I$-adic topology. If $N$ is a submodule of $M$,
  the closure $\bar{N}$ of $N$ is equal to
  \[
  \bigcap_{i=1}^\infty (N+I^n M).
  \]
  \begin{proof}
    $(\subset)$ If $x\in \bar{N}$, then it is in every
    neighborhood of $N$, and hence in the intersection.

    $(\supset)$ If $x\notin\bar{N}$, then there is $n$ such that
    $(x+I^nM)\cap N=\emptyset$ or $x\notin (N+I^nM)$.
  \end{proof}
\end{proposition}

\begin{corollary}
  Let $R$ be a ring, $I\subset R$ be an ideal, and $M$ be a module
  equipped with the $I$-adic topology. $M$ is Hausdorff if and only if
  \[
  \bigcap_{i=1}^\infty I^n M  = 0.
  \]
\end{corollary}

\begin{corollary}
  Let $(A,\m)$ be a local ring equipped with the $\m$-adic
  topology. In this case $A$ is Hausdorff.
\end{corollary}





\section{Metric spaces}


Recall the definition of a metric space:

\begin{definition}\index{pseudometric}\index{metric}
  A function of sets $d:M\times M\to [0,\infty)$ is called a \dfn{pseudometric} if:
    \begin{enumerate}
    \item For all $x,y\in M$, $d(x,y) = d(y,x)$.
    \item For all $x,y,z\in M$, $d(x,y) + d(y,z) \ge d(x,z)$.
    \end{enumerate}
    If in addition we have that for all $x,y \in M$, $d(x,y)=0$ if and
    only if $x = y$, then $d$ is called a \dfn{metric}.
\end{definition}

\begin{definition}
  A \dfn{metric space} is a set $M$ and a metric $d:M\times M \to
  [0,\infty)$.
\end{definition}

If $M$ has the $I$-adic topology, one may define a
pseudometric\index{pseudometric} on $M$ as follows: Fix any $c\in
(0,1)$. For any $x,y\in M$ define
\[
d(x,y) := c^n
\]
where $n$ is the integer such that $(x-y)\in I^nM-I^{n+1}M$, if no
such integer exists, then set $d(x,y) = 0$. If $M$ is
Hausdorff\index{Hausdorff}, then we have defined a metric.


\begin{corollary}
  The $\m$-adic topology induces a metric.
  \begin{proof}
    We know from Corollary~\ref{C:intmax}, that given a local ring $(A,\m)$,
    \[
    \bigcap_{n\ge 1} \m^n  = (0).
    \]
  \end{proof}
\end{corollary}

There are other cases where the $I$-adic topology induces a metric.

\begin{corollary}
  Let $R$ be a Noetherian ring and $I \subset \i J(R)$. If $M$ is a
  finitely generated module then
  \[
  \bigcap_{n=1}^\infty I^n M = 0.
  \]
  \begin{proof}
    By Artin-Rees, Theorem~\ref{L:ArtinRees},\index{Artin-Rees} there
    exists $k > 0$ such that for all $n$ we have
    \[
    N = N \cap I^{n + k}M = I^n(N \cap I^kM) = I^nN.
    \]
    Thus by \index{Nakayama's lemma}Nakayama's lemma,
    Theorem~\ref{NAK}, $N = 0$.
  \end{proof}
\end{corollary}


\begin{corollary}
  Let $A$ be a Noetherian domain. If $I$ is a proper ideal of $A$,
  then
  \[
  \bigcap_{n=1}^\infty I^n = 0.
  \]
  \begin{proof}
    Let $J = \bigcap I^n$.  Then we have $J = I J$.  By Nakayama's
    lemma, Theorem \ref{NAK}, there is a $x$ comaximal to $I$ such
    that $x\bigcap_{n=1}^\infty I^n=0$.  Since $A$ is a domain, we see
    $\bigcap_{n=1}^\infty I^n=0$.
  \end{proof}
\end{corollary}


\end{document}
