\documentclass{ximera}



\usepackage{tikz-cd}
\usepackage[sans]{dsfont}

\DefineVerbatimEnvironment{macaulay2}{Verbatim}{numbers=left,frame=lines,label=Macaulay2,labelposition=topline}

%%% This next bit of code defines all our theorem environments
\makeatletter
\let\c@theorem\relax
\let\c@corollary\relax
\makeatother

\let\definition\relax
\let\enddefinition\relax

\let\theorem\relax
\let\endtheorem\relax

\let\proposition\relax
\let\endproposition\relax

\let\exercise\relax
\let\endexercise\relax

\let\question\relax
\let\endquestion\relax

\let\remark\relax
\let\endremark\relax

\let\corollary\relax
\let\endcorollary\relax


\let\example\relax
\let\endexample\relax


\let\lemma\relax
\let\endlemma\relax

\newtheoremstyle{SlantTheorem}{\topsep}{\topsep}%%% space between body and thm
		{\slshape}                      %%% Thm body font
		{}                              %%% Indent amount (empty = no indent)
		{\bfseries\sffamily}            %%% Thm head font
		{}                              %%% Punctuation after thm head
		{3ex}                           %%% Space after thm head
		{\thmname{#1}\thmnumber{ #2}\thmnote{ \bfseries(#3)}}%%% Thm head spec
\theoremstyle{SlantTheorem}
\newtheorem{theorem}{Theorem}
\newtheorem{definition}[theorem]{Definition}
\newtheorem{proposition}[theorem]{Proposition}
%% \newtheorem*{dfnn}{Definition}
%% \newtheorem{ques}{Question}[theorem]
\newtheorem{lemma}[theorem]{Lemma}
%% \newtheorem*{war}{WARNING}
%% \newtheorem*{cor}{Corollary}
%% \newtheorem*{eg}{Example}
\newtheorem*{remark}{Remark}
\newtheorem*{touchstone}{Touchstone}
\newtheorem{corollary}{Corollary}[theorem]
\newtheorem*{example}{Example}


\newtheoremstyle{Exercise}{\topsep}{\topsep} %%% space between body and thm
		{}                           %%% Thm body font
		{}                           %%% Indent amount (empty = no indent)
		{\bfseries}                  %%% Thm head font
		{)}                          %%% Punctuation after thm head
		{ }                          %%% Space after thm head
		{\thmnumber{#2}\thmnote{ \bfseries(#3)}}%%% Thm head spec
\theoremstyle{Exercise}
\newtheorem{exercise}{}[theorem]

%% \newtheoremstyle{Question}{\topsep}{\topsep} %%% space between body and thm
%% 		{\bfseries}                  %%% Thm body font
%% 		{3ex}                        %%% Indent amount (empty = no indent)
%% 		{}                           %%% Thm head font
%% 		{}                           %%% Punctuation after thm head
%% 		{}                           %%% Space after thm head
%% 		{\thmnumber{#2}\thmnote{ \bfseries(#3)}}%%% Thm head spec
\newtheoremstyle{Question}{3em}{3em} %%% space between body and thm
		{\large\bfseries}                           %%% Thm body font
		{3ex}                           %%% Indent amount (empty = no indent)
		{\bfseries}                  %%% Thm head font
		{}                          %%% Punctuation after thm head
		{ }                          %%% Space after thm head
		{}%%% Thm head spec
\theoremstyle{Question}
\newtheorem*{question}{}



\renewcommand{\tilde}{\widetilde}
\renewcommand{\bar}{\overline}
\renewcommand{\hat}{\widehat}
\newcommand{\N}{\mathbb N}
\newcommand{\Z}{\mathbb Z}
\newcommand{\R}{\mathbb R}
\newcommand{\Q}{\mathbb Q}
\newcommand{\C}{\mathbb C}
\newcommand{\V}{\mathbb V}
\newcommand{\I}{\mathbb I}
\newcommand{\A}{\mathbb A}
\newcommand{\iso}{\simeq}
\newcommand{\ph}{\varphi}
\newcommand{\Cf}{\mathcal{C}}
\newcommand{\IZ}{\mathrm{Int}(\Z)}
\newcommand{\dsum}{\oplus}
\newcommand{\directsum}{\coprod}
\newcommand{\union}{\bigcup}
\renewcommand{\i}{\mathfrak}
\renewcommand{\a}{\mathfrak{a}}
\renewcommand{\b}{\mathfrak{b}}
\newcommand{\m}{\mathfrak{m}}
\newcommand{\p}{\mathfrak{p}}
\newcommand{\q}{\mathfrak{q}}
\newcommand{\dfn}{\textbf}
\let\hom\relax
\DeclareMathOperator{\ann}{Ann}
\DeclareMathOperator{\h}{ht}
\DeclareMathOperator{\hom}{Hom}
\DeclareMathOperator{\spec}{Spec}
\DeclareMathOperator{\supp}{Supp}
\DeclareMathOperator{\ass}{Ass}
\DeclareMathOperator{\ff}{Frac}
\DeclareMathOperator{\im}{Im}
\DeclareMathOperator{\syz}{Syz}
\DeclareMathOperator{\gr}{Gr}
\renewcommand{\ker}{\mathop{\mathrm{Ker}}\nolimits}
\newcommand{\lps}{[\hspace{-0.25ex}[}
\newcommand{\rps}{]\hspace{-0.25ex}]}
\newcommand{\into}{\hookrightarrow}
\newcommand{\onto}{\twoheadrightarrow}
\newcommand{\tensor}{\otimes}
\newcommand{\x}{\mathbf{x}}
\newcommand{\X}{\mathbf X}
\newcommand{\Y}{\mathbf Y}
\renewcommand{\k}{\boldsymbol{\kappa}}
\renewcommand{\emptyset}{\varnothing}
\renewcommand{\qedsymbol}{$\blacksquare$}
\renewcommand{\l}{\ell}
\newcommand{\1}{\mathds{1}}
\newcommand{\lto}{\mathop{\longrightarrow\,}\limits}
\newcommand{\rad}{\sqrt}
\renewcommand{\vec}{\mathbf}
\renewcommand{\phi}{\varphi}
\renewcommand{\epsilon}{\varepsilon}
\renewcommand{\subset}{\subseteq}
\renewcommand{\supset}{\supseteq}
\newcommand{\macaulay}{\textsl{Macaulay2}}
\newcommand{\invlim}{\varprojlim}


%\renewcommand{\proofname}{Sketch of Proof}


\renewenvironment{proof}[1][Proof]
  {\begin{trivlist}\item[\hskip \labelsep \itshape \bfseries #1{}\hspace{2ex}]\upshape}
{\qed\end{trivlist}}

\newenvironment{sketch}[1][Sketch of Proof]
  {\begin{trivlist}\item[\hskip \labelsep \itshape \bfseries #1{}\hspace{2ex}]\upshape}
{\qed\end{trivlist}}



\makeatletter
\renewcommand\section{\@startsection{paragraph}{10}{\z@}%
                                     {-3.25ex\@plus -1ex \@minus -.2ex}%
                                     {1.5ex \@plus .2ex}%
                                     {\normalfont\large\sffamily\bfseries}}
\renewcommand\subsection{\@startsection{subparagraph}{10}{\z@}%
                                    {3.25ex \@plus1ex \@minus.2ex}%
                                    {-1em}%
                                    {\normalfont\normalsize\sffamily\bfseries}}
\makeatother

%% Fix weird index/bib issue.
\makeatletter
\gdef\ttl@savemark{\sectionmark{}}
\makeatother


\author{Bart Snapp}

\title{Limits}

\begin{document}
  \begin{abstract}
    We review inverse limits.  Sources and references: \cite{sD2008}.
  \end{abstract}
\maketitle

\subsection{Inverse Limits}

\begin{definition}\index{inverse system}
  A family of objects $(X_\alpha)_{\alpha\in \mathcal{I}}$ is a \textbf{inverse
    system indexed by a directed set $\boldsymbol{\mathcal{I}}$} if for every
  $\alpha,\beta \in \mathcal{I}$ with $\alpha \le \beta$ there exists a
  morphism $\ph_{\alpha\beta}:X_\beta \to X_\alpha$ such that:
  \begin{enumerate}
  \item $\ph_{\alpha\alpha} = \1_{X_\alpha}$ for all $\alpha \in \mathcal{I}$.
  \item For any $\alpha,\beta,\gamma \in \mathcal{I}$ where $\alpha\le \beta\le\gamma$, the following diagram commutes:
    \[
    \begin{tikzcd}
      X_\beta \ar[rr,"\ph_{\alpha\beta}"] &    & X_\alpha\\
      &  X_\gamma\ar[ul,"\ph_{\beta\gamma}"] \ar[ur,"\ph_{\alpha\gamma}"]  &
    \end{tikzcd}
    \]
  \end{enumerate}
\end{definition}


\begin{definition}\index{inverse limit}\index{limit}
  An \textbf{inverse limit}, which is an example of a \textbf{limit},
  of a inverse system $(X_\alpha)_{\alpha\in \mathcal{I}}$, is an
  object, denoted by $\invlim(X_\alpha)$, with morphisms
  $\ph_\alpha:\invlim(X_\alpha)\to X_\alpha$ such that for all
  $\alpha, \beta \in \mathcal{I}$ with $\alpha \le \beta$ we have
  $\ph_{\alpha\beta} \circ \ph_\beta = \ph_\alpha$.  Further, for
  every object $Y$ with compatible morphisms $\psi_\alpha:Y\to
  X_\alpha$, there exists a unique morphism $\ph$ making the diagram
  below commute for all $\alpha\le\beta$:
\[
\begin{tikzcd}
  Y\ar[ddr,bend right=60,swap,"\psi_\alpha"] \ar[dr,swap,"\psi_\beta"]  \ar[dashed,rr,"\ph"] &   &  \invlim(X_{\alpha}) \ar[ddl,bend left=60,"\ph_\alpha"] \ar[dl,"\ph_\beta"]\\
  & X_{\beta} \ar[d,"\ph_{\alpha\beta}"] &\\
  & X_{\alpha}&
\end{tikzcd}
\]
\end{definition}

\begin{example} If we consider the category of sets, where the morphisms are set inclusion, then given $X_0\supset X_1\supset \cdots \supset X_n\supset \cdots$,
\[
\invlim(X_i) = \bigcap_{i=0}^\infty X_i.
\]
\end{example}

\begin{example} The inverse limit can be constructed as follows: For a given inverse system, $(X_\alpha)_{\alpha\in \mathcal{I}}$, write
\[
\invlim(X_\alpha) = \{(x_\alpha)_{\alpha\in \mathcal{I}}:\text{if }\alpha\le \beta,\text{ then }x_\alpha = \ph_{\alpha\beta}(x_\beta) \}\subset \prod_{\alpha\in \mathcal{I}} X_\alpha.
\]
The reader should check that this construction agrees with the definition of an inverse limit.
\end{example}

We now will define the ring of formal power series as it will be very useful in this chapter: 

\begin{definition}\index{formal power series ring}\index{power series ring}\index{AX@$A\lps X \rps$} Given a ring $A$, we can form the ring of \textbf{formal power series} in $X_1,\dots,X_n$ over $A$ by considering all infinite sums of the form
\[
\sum_{i=0}^\infty a_{i_1,\dots, i_n}X_1^{i_1}\cdots X_n^{i_n},\qquad\text{where} \qquad a_{i_1,\dots, i_n}\in A.
\]
Sums such as these form a ring under the canonical rules for summation and product. We denote the ring of formal power series over $A$ in $n$ variables by $A\lps X_1,\dots,X_n\rps$.
\end{definition}

\begin{exercise} If $B = A[X_1,\dots,X_n]$ and $I = (X_1,\dots,X_n)$. Show $(A/I^t)$ form an inverse system. Moreover, show that
\[
\invlim(B/I^n) \iso A\lps X_1,\dots,X_n\rps.
\]
\end{exercise}


\begin{exercise} \label{invlimleftexact}
Suppose we have inverse systems $(A_\alpha)_{\alpha \in \mathcal{I}}$, $(B_\alpha)_{\alpha \in \mathcal{I}}$, and $(C_\alpha)_{\alpha \in \mathcal{I}}$, over the directed set $\mathcal{I}$ and maps $(\ph_\alpha):(A_\alpha) \to (B_\alpha)$ and $(\psi_\alpha):(B_\alpha) \to (C_\alpha)$ such that for every $\alpha \in \mathcal{I}$
\[
0 \lto A_\alpha \lto^{\ph_\alpha} B_\alpha \lto^{\psi_\alpha} C_\alpha\lto 0
\]
is exact.  Then
\[
0 \lto \invlim A_\alpha \lto^{\invlim \ph_\alpha} \invlim B_\alpha \lto^{\invlim \psi_\alpha} \invlim C_\alpha
\]
is exact.  In other words, inverse limit is a left exact functor from the category of inverse systems of modules over a fixed directed set to the category of modules.
\end{exercise}



\end{document}
