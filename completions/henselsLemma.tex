\documentclass{ximera}



\usepackage{tikz-cd}
\usepackage[sans]{dsfont}

\DefineVerbatimEnvironment{macaulay2}{Verbatim}{numbers=left,frame=lines,label=Macaulay2,labelposition=topline}

%%% This next bit of code defines all our theorem environments
\makeatletter
\let\c@theorem\relax
\let\c@corollary\relax
\makeatother

\let\definition\relax
\let\enddefinition\relax

\let\theorem\relax
\let\endtheorem\relax

\let\proposition\relax
\let\endproposition\relax

\let\exercise\relax
\let\endexercise\relax

\let\question\relax
\let\endquestion\relax

\let\remark\relax
\let\endremark\relax

\let\corollary\relax
\let\endcorollary\relax


\let\example\relax
\let\endexample\relax


\let\lemma\relax
\let\endlemma\relax

\newtheoremstyle{SlantTheorem}{\topsep}{\topsep}%%% space between body and thm
		{\slshape}                      %%% Thm body font
		{}                              %%% Indent amount (empty = no indent)
		{\bfseries\sffamily}            %%% Thm head font
		{}                              %%% Punctuation after thm head
		{3ex}                           %%% Space after thm head
		{\thmname{#1}\thmnumber{ #2}\thmnote{ \bfseries(#3)}}%%% Thm head spec
\theoremstyle{SlantTheorem}
\newtheorem{theorem}{Theorem}
\newtheorem{definition}[theorem]{Definition}
\newtheorem{proposition}[theorem]{Proposition}
%% \newtheorem*{dfnn}{Definition}
%% \newtheorem{ques}{Question}[theorem]
\newtheorem{lemma}[theorem]{Lemma}
%% \newtheorem*{war}{WARNING}
%% \newtheorem*{cor}{Corollary}
%% \newtheorem*{eg}{Example}
\newtheorem*{remark}{Remark}
\newtheorem*{touchstone}{Touchstone}
\newtheorem{corollary}{Corollary}[theorem]
\newtheorem*{example}{Example}


\newtheoremstyle{Exercise}{\topsep}{\topsep} %%% space between body and thm
		{}                           %%% Thm body font
		{}                           %%% Indent amount (empty = no indent)
		{\bfseries}                  %%% Thm head font
		{)}                          %%% Punctuation after thm head
		{ }                          %%% Space after thm head
		{\thmnumber{#2}\thmnote{ \bfseries(#3)}}%%% Thm head spec
\theoremstyle{Exercise}
\newtheorem{exercise}{}[theorem]

%% \newtheoremstyle{Question}{\topsep}{\topsep} %%% space between body and thm
%% 		{\bfseries}                  %%% Thm body font
%% 		{3ex}                        %%% Indent amount (empty = no indent)
%% 		{}                           %%% Thm head font
%% 		{}                           %%% Punctuation after thm head
%% 		{}                           %%% Space after thm head
%% 		{\thmnumber{#2}\thmnote{ \bfseries(#3)}}%%% Thm head spec
\newtheoremstyle{Question}{3em}{3em} %%% space between body and thm
		{\large\bfseries}                           %%% Thm body font
		{3ex}                           %%% Indent amount (empty = no indent)
		{\bfseries}                  %%% Thm head font
		{}                          %%% Punctuation after thm head
		{ }                          %%% Space after thm head
		{}%%% Thm head spec
\theoremstyle{Question}
\newtheorem*{question}{}



\renewcommand{\tilde}{\widetilde}
\renewcommand{\bar}{\overline}
\renewcommand{\hat}{\widehat}
\newcommand{\N}{\mathbb N}
\newcommand{\Z}{\mathbb Z}
\newcommand{\R}{\mathbb R}
\newcommand{\Q}{\mathbb Q}
\newcommand{\C}{\mathbb C}
\newcommand{\V}{\mathbb V}
\newcommand{\I}{\mathbb I}
\newcommand{\A}{\mathbb A}
\newcommand{\iso}{\simeq}
\newcommand{\ph}{\varphi}
\newcommand{\Cf}{\mathcal{C}}
\newcommand{\IZ}{\mathrm{Int}(\Z)}
\newcommand{\dsum}{\oplus}
\newcommand{\directsum}{\coprod}
\newcommand{\union}{\bigcup}
\renewcommand{\i}{\mathfrak}
\renewcommand{\a}{\mathfrak{a}}
\renewcommand{\b}{\mathfrak{b}}
\newcommand{\m}{\mathfrak{m}}
\newcommand{\p}{\mathfrak{p}}
\newcommand{\q}{\mathfrak{q}}
\newcommand{\dfn}{\textbf}
\let\hom\relax
\DeclareMathOperator{\ann}{Ann}
\DeclareMathOperator{\h}{ht}
\DeclareMathOperator{\hom}{Hom}
\DeclareMathOperator{\spec}{Spec}
\DeclareMathOperator{\supp}{Supp}
\DeclareMathOperator{\ass}{Ass}
\DeclareMathOperator{\ff}{Frac}
\DeclareMathOperator{\im}{Im}
\DeclareMathOperator{\syz}{Syz}
\DeclareMathOperator{\gr}{Gr}
\renewcommand{\ker}{\mathop{\mathrm{Ker}}\nolimits}
\newcommand{\lps}{[\hspace{-0.25ex}[}
\newcommand{\rps}{]\hspace{-0.25ex}]}
\newcommand{\into}{\hookrightarrow}
\newcommand{\onto}{\twoheadrightarrow}
\newcommand{\tensor}{\otimes}
\newcommand{\x}{\mathbf{x}}
\newcommand{\X}{\mathbf X}
\newcommand{\Y}{\mathbf Y}
\renewcommand{\k}{\boldsymbol{\kappa}}
\renewcommand{\emptyset}{\varnothing}
\renewcommand{\qedsymbol}{$\blacksquare$}
\renewcommand{\l}{\ell}
\newcommand{\1}{\mathds{1}}
\newcommand{\lto}{\mathop{\longrightarrow\,}\limits}
\newcommand{\rad}{\sqrt}
\renewcommand{\vec}{\mathbf}
\renewcommand{\phi}{\varphi}
\renewcommand{\epsilon}{\varepsilon}
\renewcommand{\subset}{\subseteq}
\renewcommand{\supset}{\supseteq}
\newcommand{\macaulay}{\textsl{Macaulay2}}
\newcommand{\invlim}{\varprojlim}


%\renewcommand{\proofname}{Sketch of Proof}


\renewenvironment{proof}[1][Proof]
  {\begin{trivlist}\item[\hskip \labelsep \itshape \bfseries #1{}\hspace{2ex}]\upshape}
{\qed\end{trivlist}}

\newenvironment{sketch}[1][Sketch of Proof]
  {\begin{trivlist}\item[\hskip \labelsep \itshape \bfseries #1{}\hspace{2ex}]\upshape}
{\qed\end{trivlist}}



\makeatletter
\renewcommand\section{\@startsection{paragraph}{10}{\z@}%
                                     {-3.25ex\@plus -1ex \@minus -.2ex}%
                                     {1.5ex \@plus .2ex}%
                                     {\normalfont\large\sffamily\bfseries}}
\renewcommand\subsection{\@startsection{subparagraph}{10}{\z@}%
                                    {3.25ex \@plus1ex \@minus.2ex}%
                                    {-1em}%
                                    {\normalfont\normalsize\sffamily\bfseries}}
\makeatother

%% Fix weird index/bib issue.
\makeatletter
\gdef\ttl@savemark{\sectionmark{}}
\makeatother


\author{Bart Snapp}

\title{Hensel's lemma}

\begin{document}
\begin{abstract}
  We state and prove Hensel's lemma. Sources and references:
  \cite{rA2006}.
\end{abstract}
\maketitle

\begin{theorem}[Hensel's lemma]\index{Hensel's lemma}
  Let $(A,\m,k)$ be a complete local ring and let $f$ be a monic
  polynomial of degree $d$ in $A[X]$. Consider the image of $f$ in a
  polynomial ring over the residue field, $\bar{f}\in k[X]$. If
  \begin{enumerate}
  \item $\bar{f} = \bar{g}\cdot \bar{h}$ in $k[X]$,
  \item with $(\bar{g},\bar{h}) = k[X]$,
  \end{enumerate}
  then $f = g\cdot h$ in $A[X]$, where $g= \bar{g}$ and $h=\bar{h}$
  are monic polynomials in $A[X]$.
  \begin{proof}
    We will work inductively, producing a sequence of polynomials,
    \[
    g_1,g_2,\dots \quad\text{and}\quad h_1,h_2,\dots
    \]
    in $A[X]$ where $\deg(g_i) = c$ and $\deg(h_i) = d-c$, such that
    \[
    f(X) - g_i(X) h_i(X) \in \m^i(X).
    \]

    
    If $i=1$, then by the hypothesis of the theorem, $\bar{f} =
    \bar{g}\cdot \bar{h}$ in $k[X]$. Thus
    \[
    f(X) - g_1(X) h_1(X) \in \m[X].
    \]

    Now assume our inductive construction holds up to $i=n$. Hence we have
    \[
    f(X) - g_n(X) h_n(X) = \sum_{i=0}^n a_i X^i\in \m^n[X].
    \]
    with $a_i\in \m^n$.  Since $(\bar{g}_n, \bar{h}_n) = k[X]$, there
    are polynomials $\bar{u}_i,\bar{v}_i\in k[X]$ such that
    \begin{align*}
      1  &= \bar{g}_n(X)\bar{u}_0(X) + \bar{h}_n(X)\bar{v}_0(X) \\
      X  &= \bar{g}_n(X)\bar{u}_1(X) + \bar{h}_n(X)\bar{v}_1(X) \\
      X^2  &= \bar{g}_n(X)\bar{u}_2(X) + \bar{h}_n(X)\bar{v}_2(X) \\
      &\ \vdots\\
      X^d  &= \bar{g}_n(X)\bar{u}_d(X) + \bar{h}_n(X)\bar{v}_d(X).
    \end{align*}
    Note that $\deg(\bar{u}_i) \le d-c$ and $\deg(\bar{v}_i) \le
    c$. If we lift $\bar{u}_i$ and $\bar{v}_i$, we have that
    \[
    X^i - g_n(X) u_i(X) - h_n(X) v_i(X) \in \m[X]. 
    \]
    Now set
    \begin{align*}
      g_{n+1}(X) &= g_n(X) + \sum_{i=0}^d a_i v_i(X)\\
      h_{n+1}(X) &= h_n(X) + \sum_{i=0}^d a_i u_i(X).
    \end{align*}
    Note that $\bar{g}_{n+1} = \bar{g}_n = g$ and $\bar{h}_{n+1} =
    \bar{h}_n = h$, and that the $\deg(\bar{g}_{n+1}) = c$ and
    $\deg(\bar{h}_{n+1}) = d-c$. We must show that
    \[
    f(X) - g_{n+1}(X) h_{n+1}(X) \in \m^{n+1}(X).
    \]
    Expanding out the element on the left,
    \[
    f(X) - \left(g_n(X) + \sum_{i=0}^d a_i v_i(X)\right)\left(h_n(X) + \sum_{i=0}^d a_i u_i(X)\right)
    \]
    yields
    \[
    \left(f(X) - g_n(X)h_n(X) - \sum_{i=0}^d a_i X^i\right) + \sum_{i=0}^d a_i(X^i - g_n(X) u_i(X) - h_n(X) v_i(X)) - T.
    \]
    By hypothesis, the left-most term is zero, by construction the
    middle term is in $\m^{n+1}[X]$, and the final term is a sum of
    elements of the form $a_ia_ju_i(X)v_i(X)\in\m^{n+1}[X]$.

    Now we will use that our ring is complete. The sequences
    \[
    g_1,g_2,\dots \quad\text{and}\quad h_1,h_2,\dots
    \]
    are such that $g_{n+1}-g_n\in \m^n[X]$ and $h_{n+1}-h_n\in
    \m^n[X]$. Thus the coefficients of each $X^i$ form a Cauchy
    sequence that converges. Let $g$ and $h$ be the polynomials with
    converged coefficients. Since $f$ is monic, the leading
    coefficients of $g$ and $h$ must be inverses of each other. Since
    the coefficients of these polynomials are not in $\m$, we may
    clear denominators to replace $g$ and $h$ with monic polynomials.
  \end{proof}
\end{theorem}







\end{document}
