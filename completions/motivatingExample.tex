\documentclass{ximera}




\usepackage{tikz-cd}
\usepackage[sans]{dsfont}

\let\oldbibliography\thebibliography%% to compact bib
\renewcommand{\thebibliography}[1]{%
  \oldbibliography{#1}%
  \setlength{\itemsep}{0pt}%
}


\DefineVerbatimEnvironment{macaulay2}{Verbatim}{numbers=left,frame=lines,label=Macaulay2,labelposition=topline}

%%% This next bit of code defines all our theorem environments
\makeatletter
\let\c@theorem\relax
\let\c@corollary\relax
\makeatother

\let\definition\relax
\let\enddefinition\relax

\let\theorem\relax
\let\endtheorem\relax

\let\proposition\relax
\let\endproposition\relax

\let\exercise\relax
\let\endexercise\relax

\let\question\relax
\let\endquestion\relax

\let\remark\relax
\let\endremark\relax

\let\corollary\relax
\let\endcorollary\relax


\let\example\relax
\let\endexample\relax

\let\warning\relax
\let\endwarning\relax

\let\lemma\relax
\let\endlemma\relax

\newtheoremstyle{SlantTheorem}{\topsep}{\topsep}%%% space between body and thm
		{\slshape}                      %%% Thm body font
		{}                              %%% Indent amount (empty = no indent)
		{\bfseries\sffamily}            %%% Thm head font
		{}                              %%% Punctuation after thm head
		{3ex}                           %%% Space after thm head
		{\thmname{#1}\thmnumber{ #2}\thmnote{ \bfseries(#3)}}%%% Thm head spec
\theoremstyle{SlantTheorem}
\newtheorem{theorem}{Theorem}
\newtheorem{definition}[theorem]{Definition}
\newtheorem{proposition}[theorem]{Proposition}
%% \newtheorem*{dfnn}{Definition}
%% \newtheorem{ques}{Question}[theorem]
\newtheorem{lemma}[theorem]{Lemma}
%% \newtheorem*{war}{WARNING}
%% \newtheorem*{cor}{Corollary}
%% \newtheorem*{eg}{Example}
\newtheorem*{remark}{Remark}
\newtheorem*{touchstone}{Touchstone}
\newtheorem{corollary}{Corollary}[theorem]
\newtheorem*{example}{Example}
\newtheorem*{warning}{WARNING}


\newtheoremstyle{Exercise}{\topsep}{\topsep} %%% space between body and thm
		{}                           %%% Thm body font
		{}                           %%% Indent amount (empty = no indent)
		{\bfseries}                  %%% Thm head font
		{)}                          %%% Punctuation after thm head
		{ }                          %%% Space after thm head
		{\thmnumber{#2}\thmnote{ \bfseries(#3)}}%%% Thm head spec
\theoremstyle{Exercise}
\newtheorem{exercise}{}[theorem]

%% \newtheoremstyle{Question}{\topsep}{\topsep} %%% space between body and thm
%% 		{\bfseries}                  %%% Thm body font
%% 		{3ex}                        %%% Indent amount (empty = no indent)
%% 		{}                           %%% Thm head font
%% 		{}                           %%% Punctuation after thm head
%% 		{}                           %%% Space after thm head
%% 		{\thmnumber{#2}\thmnote{ \bfseries(#3)}}%%% Thm head spec
\newtheoremstyle{Question}{3em}{3em} %%% space between body and thm
		{\large\bfseries}                           %%% Thm body font
		{3ex}                           %%% Indent amount (empty = no indent)
		{\bfseries}                  %%% Thm head font
		{}                          %%% Punctuation after thm head
		{ }                          %%% Space after thm head
		{}%%% Thm head spec
\theoremstyle{Question}
\newtheorem*{question}{}



\renewcommand{\tilde}{\widetilde}
\renewcommand{\bar}{\overline}
\renewcommand{\hat}{\widehat}
\newcommand{\N}{\mathbb N}
\newcommand{\Z}{\mathbb Z}
\newcommand{\R}{\mathbb R}
\newcommand{\Q}{\mathbb Q}
\newcommand{\C}{\mathbb C}
\newcommand{\V}{\mathbb V}
\newcommand{\I}{\mathbb I}
\newcommand{\A}{\mathbb A}
\newcommand{\iso}{\simeq}
\newcommand{\ph}{\varphi}
\newcommand{\Cf}{\mathcal{C}}
\newcommand{\IZ}{\mathrm{Int}(\Z)}
\newcommand{\dsum}{\oplus}
\newcommand{\directsum}{\bigoplus}
\newcommand{\union}{\bigcup}
\renewcommand{\i}{\mathfrak}
\renewcommand{\a}{\mathfrak{a}}
\renewcommand{\b}{\mathfrak{b}}
\newcommand{\m}{\mathfrak{m}}
\newcommand{\p}{\mathfrak{p}}
\newcommand{\q}{\mathfrak{q}}
\newcommand{\dfn}[1]{\textbf{#1}\index{#1}}
\let\hom\relax
\DeclareMathOperator{\ann}{Ann}
\DeclareMathOperator{\h}{ht}
\DeclareMathOperator{\hom}{Hom}
\DeclareMathOperator{\Span}{Span}
\DeclareMathOperator{\spec}{Spec}
\DeclareMathOperator{\maxspec}{MaxSpec}
\DeclareMathOperator{\supp}{Supp}
\DeclareMathOperator{\ass}{Ass}
\DeclareMathOperator{\ff}{Frac}
\DeclareMathOperator{\im}{Im}
\DeclareMathOperator{\syz}{Syz}
\DeclareMathOperator{\gr}{Gr}
\renewcommand{\ker}{\mathop{\mathrm{Ker}}\nolimits}
\newcommand{\coker}{\mathop{\mathrm{Coker}}\nolimits}
\newcommand{\lps}{[\hspace{-0.25ex}[}
\newcommand{\rps}{]\hspace{-0.25ex}]}
\newcommand{\into}{\hookrightarrow}
\newcommand{\onto}{\twoheadrightarrow}
\newcommand{\tensor}{\otimes}
\newcommand{\x}{\mathbf{x}}
\newcommand{\X}{\mathbf X}
\newcommand{\Y}{\mathbf Y}
\renewcommand{\k}{\boldsymbol{\kappa}}
\renewcommand{\emptyset}{\varnothing}
\renewcommand{\qedsymbol}{$\blacksquare$}
\renewcommand{\l}{\ell}
\newcommand{\1}{\mathds{1}}
\newcommand{\lto}{\mathop{\longrightarrow\,}\limits}
\newcommand{\rad}{\sqrt}
\newcommand{\hf}{H}
\newcommand{\hs}{H\!S}
\newcommand{\hp}{H\!P}
\renewcommand{\vec}{\mathbf}
\renewcommand{\phi}{\varphi}
\renewcommand{\epsilon}{\varepsilon}
\renewcommand{\subset}{\subseteq}
\renewcommand{\supset}{\supseteq}
\newcommand{\macaulay}{\textsl{Macaulay2}}
\newcommand{\invlim}{\varprojlim}


%\renewcommand{\proofname}{Sketch of Proof}


\renewenvironment{proof}[1][Proof]
  {\begin{trivlist}\item[\hskip \labelsep \itshape \bfseries #1{}\hspace{2ex}]\upshape}
{\qed\end{trivlist}}

\newenvironment{sketch}[1][Sketch of Proof]
  {\begin{trivlist}\item[\hskip \labelsep \itshape \bfseries #1{}\hspace{2ex}]\upshape}
{\qed\end{trivlist}}



\makeatletter
\renewcommand\section{\@startsection{paragraph}{10}{\z@}%
                                     {-3.25ex\@plus -1ex \@minus -.2ex}%
                                     {1.5ex \@plus .2ex}%
                                     {\normalfont\large\sffamily\bfseries}}
\renewcommand\subsection{\@startsection{subparagraph}{10}{\z@}%
                                    {3.25ex \@plus1ex \@minus.2ex}%
                                    {-1em}%
                                    {\normalfont\normalsize\sffamily\bfseries}}
\makeatother

%% Fix weird index/bib issue.
\makeatletter
\gdef\ttl@savemark{\sectionmark{}}
\makeatother


\makeatletter
%% no number for refs
\newcommand\frontstyle{%
  \def\activitystyle{activity-chapter}
  \def\maketitle{%
    \addtocounter{titlenumber}{1}%
                    {\flushleft\small\sffamily\bfseries\@pretitle\par\vspace{-1.5em}}%
                    {\flushleft\LARGE\sffamily\bfseries\@title \par }%
                    {\vskip .6em\noindent\textit\theabstract\setcounter{problem}{0}\setcounter{sectiontitlenumber}{0}}%
                    \par\vspace{2em}
                    \phantomsection\addcontentsline{toc}{section}{\textbf{\@title}}%
                  }}
\makeatother


\author{Bart Snapp}

\title{Iteratively solving equations}

\begin{document}
\begin{abstract}
  We start by solving equations in using an iterative method.
\end{abstract}
\maketitle


Suppose you want to solve a polynomial equation $f(x)=0$ in $\Z$. You
have many options, but one option we will look at is based on the
following idea.


\begin{proposition}
  Let $x\in\Z$ and $p$ be a fixed prime in $\Z$. If $x$ is determined
  uniquely by its image in $\Z/p^n\Z$ for $n\ge 1$.
  \begin{sketch}
    Look for $n$ such that $p^{n-1}\le x < p^n$.
  \end{sketch}
\end{proposition}

One intuition for the proposition above is that for $n\gg 0$,
$Z/p^n\Z$ looks a bit like $\Z$ when the elements of $\Z$ are near
zero.

Armed with this fact, suppose you want to solve a terrible equation in
$\Z$. As a toy example, we'll use
\[
x^2-36 x+323 = 0.
\]
Of course, this equation is easy to solve. We'll use a method that can
extend to a more general setting. Start by choosing a prime of $\Z$
where the equation has a solution in $\Z/p \Z$, we use $\Z/5\Z$, and
we'll call our first solution $x_1$. Write with me:
\begin{align*}
  x_1^2- 36x_1 + 323 &\equiv 0 &&\pmod{5}\\
  x_1^2 - x_1 + 3 &\equiv 0 &&\pmod{5}\\
  x_1 &\equiv 2&&\pmod{5}.
\end{align*}
In this case, $x_1 = 5y + 2$ for some $y\in \Z$. So now we will work
in $\Z/5^2\Z$. Write with me:
\begin{align*}
  x_1^2- 36x_1 + 323 &\equiv 0&&\pmod{5^2}\\
  15y+5 &\equiv 0 &&\pmod{5^2}\\
  y &\equiv -2 &&\pmod{5^2}
\end{align*}
Evaluating $x_1 = 5y+2$ at a representative of $y$, we obtain $x_2
\equiv -8 \pmod{5^2}$. This means for some new $y$, $x_2 = 5^2 y
-8$. Now we work in $\Z/5^3\Z$. Write with me:
\begin{align*}
  x_2^2- 36x_2 + 323 &\equiv 0&&\pmod{5^3}\\
  75y+50 &\equiv 0 &&\pmod{5^3}\\
  y &\equiv 1 &&\pmod{5^3}.
\end{align*}
Evaluating $x_2 = 5^2 y-8$ at a representative of $y$, we obtain $x_3
\equiv 17\pmod{5^3}$, so $x_3 = 5^3y+17$ for some new $y$. Now we work
in $\Z/5^4\Z$. Write with me:
\begin{align*}
  x_3^2- 36x_3 + 323 &\equiv 0 &&\pmod{5^4}\\
  375 &\equiv 0 &&\pmod{5^4}\\
  y &\equiv 0 &&\pmod{5^4}.
\end{align*}
Evaluating $x_3 = 5^3 y+17$ at a representative of $y$, we obtain $x_4
\equiv 17 \pmod{5^4}$.  Since we've obtained a residue of $17$ twice
now, let's check it:  
\[
(17)^2-36 (17)+323 = 0.
\]
OK, we've found a root.

\begin{question}
  How exactly does this algorithm work?
\end{question}

Let's try the algorithm now on an equation where we expect it to fail
in finding a solution. Say
\[
x^2 +  1 = 0
\]
over $\Z$. Again we'll choose the prime $p=5$. Write with me:
\begin{align*}
  x_1^2+1 &\equiv 0&&\pmod{5}\\
  x_1 &\equiv 2 &&\pmod{5}.
\end{align*}
In this case, $x_1 = 5y + 2$ for some $y\in \Z$. So now we will work
in $\Z/5^2\Z$. Write with me:
\begin{align*}
  x_1^2+1 &\equiv 0&&\pmod{5^2}\\
  20y+5 &\equiv 0 &&\pmod{5^2}\\
  y &\equiv 1 &&\pmod{5^2}
\end{align*}
Evaluating $x_1 = 5 y+2$ at a representative of $y$, we obtain $x_2
\equiv 7\pmod{5^2}$. So $x_2 = 5^2 y +7$. Now we work in
$\Z/5^3\Z$. Write with me:
\begin{align*}
  x_2^2+1 &\equiv 0&&\pmod{5^3}\\
  100y+50 &\equiv 0 &&\pmod{5^3}\\
  y &\equiv 2 &&\pmod{5^3}
\end{align*}
Evaluating $x_2 = 5^2 y+7$ at a representative of $y$, we obtain $x_3
\equiv 57\pmod{5^3}$. So $x_3 = 5^3 y +57$. Now we work in
$\Z/5^4\Z$. Write with me:
\begin{align*}
  x_3^2+1 &\equiv 0&&\pmod{5^4}\\
  500y+125 &\equiv 0 &&\pmod{5^4}\\
  y &\equiv 1 &&\pmod{5^4}
\end{align*}
Evaluating $x_3 = 5^3 y+57$ at a representative of $y$, we obtain $x_4
\equiv 182\pmod{5^4}$.  We claim that this process can continue. In
essence we have
\[
\begin{tikzcd}[row sep=0em,]
  \cdots \ar[r] & \Z/5^4\Z \ar[r] & \Z/5^3\Z \ar[r] & \Z/5^2\Z \ar[r] & \Z/5\Z \\
  & x_4\equiv 182 \ar[r,mapsto] & x_3\equiv 57 \ar[r,mapsto] & x_2\equiv 7 \ar[r,mapsto] &  x_1\equiv 2
\end{tikzcd}
\]

Let's see if we can figure out what is going on.

\end{document}
