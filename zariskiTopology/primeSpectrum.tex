\documentclass{ximera}



\usepackage{tikz-cd}
\usepackage[sans]{dsfont}

\let\oldbibliography\thebibliography%% to compact bib
\renewcommand{\thebibliography}[1]{%
  \oldbibliography{#1}%
  \setlength{\itemsep}{0pt}%
}


\DefineVerbatimEnvironment{macaulay2}{Verbatim}{numbers=left,frame=lines,label=Macaulay2,labelposition=topline}

%%% This next bit of code defines all our theorem environments
\makeatletter
\let\c@theorem\relax
\let\c@corollary\relax
\makeatother

\let\definition\relax
\let\enddefinition\relax

\let\theorem\relax
\let\endtheorem\relax

\let\proposition\relax
\let\endproposition\relax

\let\exercise\relax
\let\endexercise\relax

\let\question\relax
\let\endquestion\relax

\let\remark\relax
\let\endremark\relax

\let\corollary\relax
\let\endcorollary\relax


\let\example\relax
\let\endexample\relax

\let\warning\relax
\let\endwarning\relax

\let\lemma\relax
\let\endlemma\relax

\newtheoremstyle{SlantTheorem}{\topsep}{\topsep}%%% space between body and thm
		{\slshape}                      %%% Thm body font
		{}                              %%% Indent amount (empty = no indent)
		{\bfseries\sffamily}            %%% Thm head font
		{}                              %%% Punctuation after thm head
		{3ex}                           %%% Space after thm head
		{\thmname{#1}\thmnumber{ #2}\thmnote{ \bfseries(#3)}}%%% Thm head spec
\theoremstyle{SlantTheorem}
\newtheorem{theorem}{Theorem}
\newtheorem{definition}[theorem]{Definition}
\newtheorem{proposition}[theorem]{Proposition}
%% \newtheorem*{dfnn}{Definition}
%% \newtheorem{ques}{Question}[theorem]
\newtheorem{lemma}[theorem]{Lemma}
%% \newtheorem*{war}{WARNING}
%% \newtheorem*{cor}{Corollary}
%% \newtheorem*{eg}{Example}
\newtheorem*{remark}{Remark}
\newtheorem*{touchstone}{Touchstone}
\newtheorem{corollary}{Corollary}[theorem]
\newtheorem*{example}{Example}
\newtheorem*{warning}{WARNING}


\newtheoremstyle{Exercise}{\topsep}{\topsep} %%% space between body and thm
		{}                           %%% Thm body font
		{}                           %%% Indent amount (empty = no indent)
		{\bfseries}                  %%% Thm head font
		{)}                          %%% Punctuation after thm head
		{ }                          %%% Space after thm head
		{\thmnumber{#2}\thmnote{ \bfseries(#3)}}%%% Thm head spec
\theoremstyle{Exercise}
\newtheorem{exercise}{}[theorem]

%% \newtheoremstyle{Question}{\topsep}{\topsep} %%% space between body and thm
%% 		{\bfseries}                  %%% Thm body font
%% 		{3ex}                        %%% Indent amount (empty = no indent)
%% 		{}                           %%% Thm head font
%% 		{}                           %%% Punctuation after thm head
%% 		{}                           %%% Space after thm head
%% 		{\thmnumber{#2}\thmnote{ \bfseries(#3)}}%%% Thm head spec
\newtheoremstyle{Question}{3em}{3em} %%% space between body and thm
		{\large\bfseries}                           %%% Thm body font
		{3ex}                           %%% Indent amount (empty = no indent)
		{\bfseries}                  %%% Thm head font
		{}                          %%% Punctuation after thm head
		{ }                          %%% Space after thm head
		{}%%% Thm head spec
\theoremstyle{Question}
\newtheorem*{question}{}



\renewcommand{\tilde}{\widetilde}
\renewcommand{\bar}{\overline}
\renewcommand{\hat}{\widehat}
\newcommand{\N}{\mathbb N}
\newcommand{\Z}{\mathbb Z}
\newcommand{\R}{\mathbb R}
\newcommand{\Q}{\mathbb Q}
\newcommand{\C}{\mathbb C}
\newcommand{\V}{\mathbb V}
\newcommand{\I}{\mathbb I}
\newcommand{\A}{\mathbb A}
\newcommand{\iso}{\simeq}
\newcommand{\ph}{\varphi}
\newcommand{\Cf}{\mathcal{C}}
\newcommand{\IZ}{\mathrm{Int}(\Z)}
\newcommand{\dsum}{\oplus}
\newcommand{\directsum}{\bigoplus}
\newcommand{\union}{\bigcup}
\renewcommand{\i}{\mathfrak}
\renewcommand{\a}{\mathfrak{a}}
\renewcommand{\b}{\mathfrak{b}}
\newcommand{\m}{\mathfrak{m}}
\newcommand{\p}{\mathfrak{p}}
\newcommand{\q}{\mathfrak{q}}
\newcommand{\dfn}[1]{\textbf{#1}\index{#1}}
\let\hom\relax
\DeclareMathOperator{\ann}{Ann}
\DeclareMathOperator{\h}{ht}
\DeclareMathOperator{\hom}{Hom}
\DeclareMathOperator{\Span}{Span}
\DeclareMathOperator{\spec}{Spec}
\DeclareMathOperator{\maxspec}{MaxSpec}
\DeclareMathOperator{\supp}{Supp}
\DeclareMathOperator{\ass}{Ass}
\DeclareMathOperator{\ff}{Frac}
\DeclareMathOperator{\im}{Im}
\DeclareMathOperator{\syz}{Syz}
\DeclareMathOperator{\gr}{Gr}
\renewcommand{\ker}{\mathop{\mathrm{Ker}}\nolimits}
\newcommand{\coker}{\mathop{\mathrm{Coker}}\nolimits}
\newcommand{\lps}{[\hspace{-0.25ex}[}
\newcommand{\rps}{]\hspace{-0.25ex}]}
\newcommand{\into}{\hookrightarrow}
\newcommand{\onto}{\twoheadrightarrow}
\newcommand{\tensor}{\otimes}
\newcommand{\x}{\mathbf{x}}
\newcommand{\X}{\mathbf X}
\newcommand{\Y}{\mathbf Y}
\renewcommand{\k}{\boldsymbol{\kappa}}
\renewcommand{\emptyset}{\varnothing}
\renewcommand{\qedsymbol}{$\blacksquare$}
\renewcommand{\l}{\ell}
\newcommand{\1}{\mathds{1}}
\newcommand{\lto}{\mathop{\longrightarrow\,}\limits}
\newcommand{\rad}{\sqrt}
\newcommand{\hf}{H}
\newcommand{\hs}{H\!S}
\newcommand{\hp}{H\!P}
\renewcommand{\vec}{\mathbf}
\renewcommand{\phi}{\varphi}
\renewcommand{\epsilon}{\varepsilon}
\renewcommand{\subset}{\subseteq}
\renewcommand{\supset}{\supseteq}
\newcommand{\macaulay}{\textsl{Macaulay2}}
\newcommand{\invlim}{\varprojlim}


%\renewcommand{\proofname}{Sketch of Proof}


\renewenvironment{proof}[1][Proof]
  {\begin{trivlist}\item[\hskip \labelsep \itshape \bfseries #1{}\hspace{2ex}]\upshape}
{\qed\end{trivlist}}

\newenvironment{sketch}[1][Sketch of Proof]
  {\begin{trivlist}\item[\hskip \labelsep \itshape \bfseries #1{}\hspace{2ex}]\upshape}
{\qed\end{trivlist}}



\makeatletter
\renewcommand\section{\@startsection{paragraph}{10}{\z@}%
                                     {-3.25ex\@plus -1ex \@minus -.2ex}%
                                     {1.5ex \@plus .2ex}%
                                     {\normalfont\large\sffamily\bfseries}}
\renewcommand\subsection{\@startsection{subparagraph}{10}{\z@}%
                                    {3.25ex \@plus1ex \@minus.2ex}%
                                    {-1em}%
                                    {\normalfont\normalsize\sffamily\bfseries}}
\makeatother

%% Fix weird index/bib issue.
\makeatletter
\gdef\ttl@savemark{\sectionmark{}}
\makeatother


\makeatletter
%% no number for refs
\newcommand\frontstyle{%
  \def\activitystyle{activity-chapter}
  \def\maketitle{%
    \addtocounter{titlenumber}{1}%
                    {\flushleft\small\sffamily\bfseries\@pretitle\par\vspace{-1.5em}}%
                    {\flushleft\LARGE\sffamily\bfseries\@title \par }%
                    {\vskip .6em\noindent\textit\theabstract\setcounter{problem}{0}\setcounter{sectiontitlenumber}{0}}%
                    \par\vspace{2em}
                    \phantomsection\addcontentsline{toc}{section}{\textbf{\@title}}%
                  }}
\makeatother


\author{Bart Snapp}

\title{The prime spectrum}


\begin{document}
\begin{abstract}
  We introduce the prime spectrum. Sources and references:
  \cite{AM1969}.
\end{abstract}
\maketitle



\begin{definition}\index{nilpotent}
  An element $x$ of a ring is called \dfn{nilpotent} if there exists
  $n\in \N$ such that $x^n = 0$.
\end{definition}


\begin{exercise}
  Find the nilpotent elements of $\Z/n\Z$ where $n=pq$, $n=pq^2$ and
  $n = \prod_{i=1}^m p_i^{n_i}$; where $p$, $q$, $p_i$ are distinct
  primes of $\Z$.
\end{exercise}


\begin{exercise}
  If $u$ is a unit and $x$ is nilpotent, show that $u+x$ is a
  unit. Hint: Expand $(1+x/u)^{-1}$ as a power series.
\end{exercise}

\begin{definition}\index{radical0@$\rad{0}$}\index{nilradical}
  The set of nilpotent elements of a ring $R$ is called the
  \dfn{nilradical} of $R$. We will use $\rad{0}$ to denote this
  set. 
\end{definition}

\begin{exercise}
  Show that $\rad{0}$ is an ideal.
\end{exercise}


\begin{exercise}
  Let $R$ be a ring, and let $I\subset \rad{0}$. If $\bar{u}\in R/I$
  is a unit, prove that $u\in R$ is a unit.
\end{exercise}


We can play with the nilradical in \macaulay. Note, we need to insist
that $0$ is the zero of our ring:
\begin{macaulay2}
R = ZZ/101[x,y]/ideal(x^4,y^7)
radical ideal 0_R
\end{macaulay2}

We can generalize the idea of the nilradical as follows:

\begin{definition}\index{radical}
  The \dfn{radical} of an ideal $I$ is denoted by $\rad{I}$ and is
  defined to be the set
  \[
  \rad{I} = \{x\in R:x^n\in I\text{ for some }n\in\N\}.
  \]
\end{definition}

\begin{exercise}
  Show that $\rad{I}$ is an ideal.
\end{exercise}

\begin{exercise}
  If $I,J$ are ideals of $R$, and $\p$ is a prime ideal, the following
  hold:
  \begin{enumerate}
  \item $I \subset \rad{I}$.
  \item $\rad{I} =\rad{\rad{I}}$.
  \item $\rad{IJ}= \rad{I\cap J} = \rad{I} \cap \rad{J}$.
  \item $\rad{\p^n} = \p$.
  \end{enumerate}
\end{exercise}

We can play with the radicals in \macaulay:
\begin{macaulay2}
R = ZZ/101[x,y]
I = ideal(x^4,y^7)
J = ideal(y^3 - 3*y^2*x + 3*y*x^2 - x^3)
p = ideal(x^2+y^3+1)
isPrime p
rI=radical I
rJ=radical J
isSubset(I,rI)
rI == radical rI
radical(I*J)
radical intersect(I,J)
intersect(ideal rI,rJ)
radical (p^7)
\end{macaulay2}


\begin{proposition}
  Given a ring $R$, the radical of an ideal $I$ is equal to the
  intersection of all the prime ideals that contain $I$.
\end{proposition}

\begin{proof}
  $(\subset)$ Suppose that $x\in \rad{I}$. Then $x^n\in I$ and for
  each prime ideal containing $I$, $x^n\in \i p$. Since $\i p$ is
  prime, $x\in \i p$. Thus
  \[
  \rad{I} \subset \bigcap_{\i p \supset I}\i p.
  \]

  $(\supset)$ By the \index{Correspondence Theorem}Correspondence
  Theorem, the prime ideals of $R$ containing $I$ correspond
  bijectively to the ideals of $R/I$, hence we reduce to the case
  where $I =(0)$.

  Suppose that $x$ is not nilpotent. We'll show that
  \[
  x\notin \bigcap_{\p \supset (0)}\p.
  \]
  Consider the set of ideals of $R$:
  \[
  \mathcal{S} = \{\a: x \notin \rad{\a}\}
  \]
  Note that $(0)\in \mathcal{S}$ and that $\mathcal{S}$ may be ordered
  by inclusion. Now let $\mathcal{C}$ be any chain of ideals in
  $\mathcal{S}$. This chain has an upper bound in $\mathcal{S}$,
  namely the ideal:
  \[
  \bigcup_{\a\in\mathcal{C}} \a
  \]
  Hence by Zorn's Lemma,\index{Zorn's Lemma} $\mathcal{S}$ has a
  maximal element, call it $\p$. We claim that $\p$ is prime. Suppose
  that $a,b\notin\p$. Hence $(a) + \p$ and $(b)+\p$ are ideals not
  contained in $\mathcal{S}$. Thus for some $m,n\in\N$:
  \[
  x^m \in (a) + \p \qquad\text{and}\qquad x^n \in (b) + \p 
  \]
  Moreover,
  \[
  x^{m+n}\in (ab) + \p
  \]
  and so we see that $ab\notin\p$. Hence $\p$ is prime and
  $x\notin\p$. Thus $x$ is not in the intersection of the prime ideals
  of $R$.
\end{proof}

To quote David Eisenbud, ``ideals maximal with respect to some
property have an uncanny tendency to be prime.''



\begin{definition}
  Let $R$ be a ring, the \dfn{prime spectrum} of $R$ is the set of all
  prime ideals of $R$. We denote the prime spectrum by $\spec(R)$.
\end{definition}

\begin{definition}
  Let $R$ be a ring, then $V(E)\subset \spec(R)$ is the set of all
  prime ideals that contain $E$.
\end{definition}

\begin{exercise}
  Show that if $I\subset J$, then $V(J) \subset V(I)$.
\end{exercise}

\begin{proposition}
  The set $\spec(R)$ is a topological space with closed sets
  \[
  \{ V(E):E\subset R\}.
  \]
  This topology is called the \dfn{Zariski topology}.
  \begin{sketch}
    Check the conditions for a topology.
  \end{sketch}
\end{proposition}

\begin{exercise}%% From AM1969, Ch 1 (16)
  Draw pictures of:
  \begin{enumerate}
  \item $\spec(\Z)$
  \item $\spec(\R)$
  \item $\spec(\C[x])$
  \item $\spec(\R[x])$
  \item $\spec(\Z[x])$
  \end{enumerate}
\end{exercise}


\begin{exercise}
  For each $f\in R$, set
  \[
  U_f := \spec(R) - V(f).
  \]
  Show that the sets $U_f$ form a basis for $\spec(R)$, that is show:
  \begin{enumerate}
    \item For each $\p\in\spec(R)$, there is some $f\in R$ such that
      $\p\in U_f$.
    \item If $\p\in U_f \cap U_g$, then there is $h\in R$ such that
      $U_h\subseteq U_f\cap U_g$, and $\p\in U_h$.
  \end{enumerate}
\end{exercise}


\begin{exercise}%% From AM1969, Ch 1 (17)
  Let $U_f$ be defined as above. Prove that:
  \begin{enumerate}
  \item $U_f \cap U_g = U_{fg}$
  \item $U_f = \emptyset$ if and only if $f$ is nilpotent.
  \item $U_f = \spec(R)$ if and only if $f$ is a unit.
  \item $U_f = U_g$ if and only if $\rad{f} = \rad{g}$.
  \item $\spec(R)$ is quasi-compact (every open covering of $\spec(R)$
    has a finite subcovering, but $\spec(R)$ might not be Hausdorff).
  \end{enumerate}
\end{exercise}


\begin{exercise}%% From AM1969, Ch 1 (21)
  Let $\phi:R \to S$ be a homomorphism of rings. For $P\in\spec(S)$,
  note that $\phi^{-1}(P) \in \spec(R)$. Hence there is an induced map
  \[
  \phi^*:\spec(S) \to \spec(R). 
  \]
  Prove that:
  \begin{enumerate}
  \item $\phi^*$ is continuous.
  \item If $\a$ is an ideal of $R$, then ${\phi^*}^{-1}(V(\a)) =
    V(\phi(\a) S)$.
  \item If $\b$ is an ideal of $S$, then $\bar{\phi^*(V(\b))} =
    V(\phi^{-1}(\b)R)$.
  \item If $\phi$ is surjective, then $\phi^*$ is a homeomorphism from
    $\spec(S)$ to $V(\ker(\phi))$.
  \item $\spec(R)$ and $\spec(R/\rad{0})$ are homeomorphic.
  \item If $\phi$ is injective, then $\phi^*(\spec(S))$ is dense in
    $\spec(R)$.
  \item $\phi^*(\spec(S))$ is dense in $\spec(R)$ if and only if
    $\ker(\phi)\subseteq \rad{0}$.
  \item If
    \[
    R\lto^\phi S \lto^\psi T,
    \]
    then
    \[
    \begin{tikzcd}[column sep=2em]
      \spec(T) \ar[r, "\psi^*"] \ar[rr, "(\phi \circ \psi)^*"', bend right=20]
      & \spec(S) \ar[r, "\phi^*"]
      & \spec(R).
    \end{tikzcd}
    \]
  \item Let $R$ be an integral domain with a single prime ideal $\p$.
    Define
    \begin{align*}
      \phi:R &\to R/\p \times \ff(R)\\
      x &\mapsto (\bar{x}, x).
    \end{align*}
    Show that $\phi^*$ is bijective, but not a homeomorphism from
    $\spec(R/\p \times \ff(R))$ to $\spec(R)$.
  \end{enumerate}
\end{exercise}


\begin{exercise}%% From AM1969, Ch 1 (19)
  Show that $V(I)$ is irreducible if and only if $\rad{I}$ is a prime
  ideal.
\end{exercise}



I think we have another pressing question:

\begin{question}
  What is the connection between affine space and the prime spectrum?
\end{question}


\end{document}
