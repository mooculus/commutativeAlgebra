\documentclass{ximera}



\usepackage{tikz-cd}
\usepackage[sans]{dsfont}

\let\oldbibliography\thebibliography%% to compact bib
\renewcommand{\thebibliography}[1]{%
  \oldbibliography{#1}%
  \setlength{\itemsep}{0pt}%
}


\DefineVerbatimEnvironment{macaulay2}{Verbatim}{numbers=left,frame=lines,label=Macaulay2,labelposition=topline}

%%% This next bit of code defines all our theorem environments
\makeatletter
\let\c@theorem\relax
\let\c@corollary\relax
\makeatother

\let\definition\relax
\let\enddefinition\relax

\let\theorem\relax
\let\endtheorem\relax

\let\proposition\relax
\let\endproposition\relax

\let\exercise\relax
\let\endexercise\relax

\let\question\relax
\let\endquestion\relax

\let\remark\relax
\let\endremark\relax

\let\corollary\relax
\let\endcorollary\relax


\let\example\relax
\let\endexample\relax

\let\warning\relax
\let\endwarning\relax

\let\lemma\relax
\let\endlemma\relax

\newtheoremstyle{SlantTheorem}{\topsep}{\topsep}%%% space between body and thm
		{\slshape}                      %%% Thm body font
		{}                              %%% Indent amount (empty = no indent)
		{\bfseries\sffamily}            %%% Thm head font
		{}                              %%% Punctuation after thm head
		{3ex}                           %%% Space after thm head
		{\thmname{#1}\thmnumber{ #2}\thmnote{ \bfseries(#3)}}%%% Thm head spec
\theoremstyle{SlantTheorem}
\newtheorem{theorem}{Theorem}
\newtheorem{definition}[theorem]{Definition}
\newtheorem{proposition}[theorem]{Proposition}
%% \newtheorem*{dfnn}{Definition}
%% \newtheorem{ques}{Question}[theorem]
\newtheorem{lemma}[theorem]{Lemma}
%% \newtheorem*{war}{WARNING}
%% \newtheorem*{cor}{Corollary}
%% \newtheorem*{eg}{Example}
\newtheorem*{remark}{Remark}
\newtheorem*{touchstone}{Touchstone}
\newtheorem{corollary}{Corollary}[theorem]
\newtheorem*{example}{Example}
\newtheorem*{warning}{WARNING}


\newtheoremstyle{Exercise}{\topsep}{\topsep} %%% space between body and thm
		{}                           %%% Thm body font
		{}                           %%% Indent amount (empty = no indent)
		{\bfseries}                  %%% Thm head font
		{)}                          %%% Punctuation after thm head
		{ }                          %%% Space after thm head
		{\thmnumber{#2}\thmnote{ \bfseries(#3)}}%%% Thm head spec
\theoremstyle{Exercise}
\newtheorem{exercise}{}[theorem]

%% \newtheoremstyle{Question}{\topsep}{\topsep} %%% space between body and thm
%% 		{\bfseries}                  %%% Thm body font
%% 		{3ex}                        %%% Indent amount (empty = no indent)
%% 		{}                           %%% Thm head font
%% 		{}                           %%% Punctuation after thm head
%% 		{}                           %%% Space after thm head
%% 		{\thmnumber{#2}\thmnote{ \bfseries(#3)}}%%% Thm head spec
\newtheoremstyle{Question}{3em}{3em} %%% space between body and thm
		{\large\bfseries}                           %%% Thm body font
		{3ex}                           %%% Indent amount (empty = no indent)
		{\bfseries}                  %%% Thm head font
		{}                          %%% Punctuation after thm head
		{ }                          %%% Space after thm head
		{}%%% Thm head spec
\theoremstyle{Question}
\newtheorem*{question}{}



\renewcommand{\tilde}{\widetilde}
\renewcommand{\bar}{\overline}
\renewcommand{\hat}{\widehat}
\newcommand{\N}{\mathbb N}
\newcommand{\Z}{\mathbb Z}
\newcommand{\R}{\mathbb R}
\newcommand{\Q}{\mathbb Q}
\newcommand{\C}{\mathbb C}
\newcommand{\V}{\mathbb V}
\newcommand{\I}{\mathbb I}
\newcommand{\A}{\mathbb A}
\newcommand{\iso}{\simeq}
\newcommand{\ph}{\varphi}
\newcommand{\Cf}{\mathcal{C}}
\newcommand{\IZ}{\mathrm{Int}(\Z)}
\newcommand{\dsum}{\oplus}
\newcommand{\directsum}{\bigoplus}
\newcommand{\union}{\bigcup}
\renewcommand{\i}{\mathfrak}
\renewcommand{\a}{\mathfrak{a}}
\renewcommand{\b}{\mathfrak{b}}
\newcommand{\m}{\mathfrak{m}}
\newcommand{\p}{\mathfrak{p}}
\newcommand{\q}{\mathfrak{q}}
\newcommand{\dfn}[1]{\textbf{#1}\index{#1}}
\let\hom\relax
\DeclareMathOperator{\ann}{Ann}
\DeclareMathOperator{\h}{ht}
\DeclareMathOperator{\hom}{Hom}
\DeclareMathOperator{\Span}{Span}
\DeclareMathOperator{\spec}{Spec}
\DeclareMathOperator{\maxspec}{MaxSpec}
\DeclareMathOperator{\supp}{Supp}
\DeclareMathOperator{\ass}{Ass}
\DeclareMathOperator{\ff}{Frac}
\DeclareMathOperator{\im}{Im}
\DeclareMathOperator{\syz}{Syz}
\DeclareMathOperator{\gr}{Gr}
\renewcommand{\ker}{\mathop{\mathrm{Ker}}\nolimits}
\newcommand{\coker}{\mathop{\mathrm{Coker}}\nolimits}
\newcommand{\lps}{[\hspace{-0.25ex}[}
\newcommand{\rps}{]\hspace{-0.25ex}]}
\newcommand{\into}{\hookrightarrow}
\newcommand{\onto}{\twoheadrightarrow}
\newcommand{\tensor}{\otimes}
\newcommand{\x}{\mathbf{x}}
\newcommand{\X}{\mathbf X}
\newcommand{\Y}{\mathbf Y}
\renewcommand{\k}{\boldsymbol{\kappa}}
\renewcommand{\emptyset}{\varnothing}
\renewcommand{\qedsymbol}{$\blacksquare$}
\renewcommand{\l}{\ell}
\newcommand{\1}{\mathds{1}}
\newcommand{\lto}{\mathop{\longrightarrow\,}\limits}
\newcommand{\rad}{\sqrt}
\newcommand{\hf}{H}
\newcommand{\hs}{H\!S}
\newcommand{\hp}{H\!P}
\renewcommand{\vec}{\mathbf}
\renewcommand{\phi}{\varphi}
\renewcommand{\epsilon}{\varepsilon}
\renewcommand{\subset}{\subseteq}
\renewcommand{\supset}{\supseteq}
\newcommand{\macaulay}{\textsl{Macaulay2}}
\newcommand{\invlim}{\varprojlim}


%\renewcommand{\proofname}{Sketch of Proof}


\renewenvironment{proof}[1][Proof]
  {\begin{trivlist}\item[\hskip \labelsep \itshape \bfseries #1{}\hspace{2ex}]\upshape}
{\qed\end{trivlist}}

\newenvironment{sketch}[1][Sketch of Proof]
  {\begin{trivlist}\item[\hskip \labelsep \itshape \bfseries #1{}\hspace{2ex}]\upshape}
{\qed\end{trivlist}}



\makeatletter
\renewcommand\section{\@startsection{paragraph}{10}{\z@}%
                                     {-3.25ex\@plus -1ex \@minus -.2ex}%
                                     {1.5ex \@plus .2ex}%
                                     {\normalfont\large\sffamily\bfseries}}
\renewcommand\subsection{\@startsection{subparagraph}{10}{\z@}%
                                    {3.25ex \@plus1ex \@minus.2ex}%
                                    {-1em}%
                                    {\normalfont\normalsize\sffamily\bfseries}}
\makeatother

%% Fix weird index/bib issue.
\makeatletter
\gdef\ttl@savemark{\sectionmark{}}
\makeatother


\makeatletter
%% no number for refs
\newcommand\frontstyle{%
  \def\activitystyle{activity-chapter}
  \def\maketitle{%
    \addtocounter{titlenumber}{1}%
                    {\flushleft\small\sffamily\bfseries\@pretitle\par\vspace{-1.5em}}%
                    {\flushleft\LARGE\sffamily\bfseries\@title \par }%
                    {\vskip .6em\noindent\textit\theabstract\setcounter{problem}{0}\setcounter{sectiontitlenumber}{0}}%
                    \par\vspace{2em}
                    \phantomsection\addcontentsline{toc}{section}{\textbf{\@title}}%
                  }}
\makeatother


\author{Bart Snapp}

\title{Affine space}


\begin{document}
\begin{abstract}
  We introduce affine space. Sources and references:
  \cite{mR1995, SKKT2002}.
\end{abstract}
\maketitle

\begin{definition}
  Let $k$ be a field and let $\vec{F} = \{F_{i}\}_{i\in I}$ be polynomials
  $k[\X]$, an \dfn{affine algebraic variety} is the set
  \[
  \V(\vec{F}) := \{\x: \text{$F_i(\x) = 0$ for all $i\in I$}\}.
  \]
\end{definition}


\begin{exercise}
  Give examples and nonexamples of algebraic varieties.
\end{exercise}





\begin{exercise}
  Let $\mathsf{M}$ be a $m\times m$ matrix with entries in some field
  $k$ and $\det(\mathsf{M}) \ne 0$. Is $\V( \mathsf{M} \vec{F}) =
  \V(\vec{F})$?
\end{exercise}


\begin{exercise}
  Show that if $I\subset J$, then $\V(J) \subset \V(I)$.
\end{exercise}

\begin{exercise}
  Let $F,G\in k[\X]$. Show that $\V(F) \cap\V(G) = \V(\{F,G\})$.
\end{exercise}


\begin{exercise}
  Let $F,G\in k[\X]$. Show that $\V(F) \cup\V(G) = \V(F\cdot G)$.
\end{exercise}





\begin{definition}
  Let $V$ be an algebraic variety defined by some polynomials in
  $k[\X]$. Define
  \[
  \I(V) = \{F\in k[\X]:F[\x] = 0 \text{ for } \x \in V\}.
  \]
\end{definition}

\begin{exercise}
  Given an algebraic variety defined by polynomials in $k[\X]$, show
  that $\I(V)$ is an ideal of $k[\X]$.
\end{exercise}


\begin{exercise}
  Let $\vec{F} = \{F_i\}_{i\in I}\subset k[\X]$. Show that every
  algebraic variety $\V(\vec{F})$ is the set of common zeros of
  finitely many polynomials in $k[\X]$.
\end{exercise}

\begin{exercise}
  Let $V$ be an algebraic variety defined by elements of $k[\X]$ and
  let $I$ be an ideal of $k[\X]$. Which of the following are true?
  \begin{enumerate}
  \item $\V(\I(V)) = V$
  \item $\I(\V(I)) = I$
  \end{enumerate}
\end{exercise}


This brings us to a question:


\begin{question}
  To what extent are varieties and ideals dual?
\end{question}

We will need to take some time to answer this question. Until then, we
press on.




\begin{proposition}
  The space $k^n$ where closed sets are defined to be the algebraic
  varieties is a topological space. This topology is called the
  \dfn{Zariski topology}, and when $k^n$ is equipped with this
  topology, we denote it $\A_k^n$, or simply $\A^n$.
  \begin{sketch}
    Check the conditions for a topology.
  \end{sketch}
\end{proposition}

\begin{exercise}
  Let $F\in k[X_1,\dots,X_n]$. Show that $F:\A^n\to \A^1$ is
  continuous. Hint: First show that the only closed subsets of $\A^1$
  are finite subsets and all of $\A^1$.
\end{exercise}


\begin{definition}
  Given an algebraic variety $\V(\vec{F})$, the \dfn{ring of regular
    functions} on $\V(\vec{F})$, is $k[\vec{X}]/(\vec{F})$.
\end{definition}

\begin{exercise}
  Show that the closed subsets of $\V(\vec{F})$ are of the form
  \[
  \{x\in \V(\vec{F}): G(x) = 0\text{ for all $G \in I$ an ideal of $k[\vec{X}]/(\vec{F})$}\}.
  \]
\end{exercise}


\begin{exercise}
  Given $G\in k[\vec{X}]/(\vec{F})$, show that $G: \V(\vec{F})\to
  \A^1$ is continuous with respect to the Zariski topology.
\end{exercise}



\begin{definition}
  Let $U\subseteq\A^m$ and $V\subseteq\A^n$ be affine algebraic
  varieties. A map $\vec{F} = (F_1,\dots,F_n)$
  \[
  \vec{F}: U\to V
  \]
  is a \dfn{morphism} of algebraic varieties if $F_1,\dots,F_n\in k[\X]/\I(U)$.
\end{definition}

\begin{exercise}
  Prove that every morphism of varieties is continuous with respect to
  the Zariski topology.
\end{exercise}

\begin{exercise}
   Let $U\subseteq\A^m$ and $V\subseteq\A^n$ be affine algebraic varieties. Show that a
   morphism $\vec{F}: U\to V$ induces a map of $k$-algebras
   \begin{align*}
     \vec{F}^*:  k[\X]/\I(V) &\to k[\X]/\I(U)\\
     G &\mapsto G\circ \vec{F}.
    \end{align*}
\end{exercise}


\begin{proposition}%% From mR1995 5.7
  Show that an algebraic variety $V$ is irreducible if and only if
  $\I(V)$ is prime.
  %% \begin{sketch}
    
  %% \end{sketch}
\end{proposition}


\begin{example}
	Let's look at a simple explicit example of an induced map of $k$-algebras given a map of varieties, as in exercise 5.2. Consider the varieties
	\[
	U=\{(x,y)\in\mathbb C^2:y-x=0\}, \quad\text{and}\quad
	V=\{(x,y)\in\mathbb C^2:y-2x=0\}
	\]
	and the map $\vec F:U\to V:(x,y)\mapsto(x,2y)$. Following the notation of exercise 5.2,
	\[
	\I(U)=(y-x), \quad\text{and}\quad
	\I(V)=(y-2x)
	\]
	so we wish to calculate
	\[
	\vec F^*:\mathbb C[x,y]/(y-2x)\to\mathbb C[x,y]/(y-x).
	\]
	Trying to write this map abstractly, we get $[G(x,y)]\mapsto[G(\vec F(x,y)]=[G(x,2y)]$. We can explicitly map $[G(x,y)]=[x^2+y^3x]$ by
	\begin{equation*}
	\begin{tikzcd}
	x^2+y^3x \arrow[leftrightarrow]{r}{=}\arrow[mapsto]{d}{\vec F^*} & x^2+8x^4 \arrow[leftrightarrow]{r}{=}\arrow[mapsto]{d}{\vec F^*} & \tfrac14y^2+\tfrac12y^4 \arrow[mapsto]{d}{\vec F^*}\\
	x^2+8y^3x \arrow[leftrightarrow]{r}{=} & x^2+8x^4 \arrow[leftrightarrow]{r}{=} & y^2+8y^4
	\end{tikzcd}
	\end{equation*}
	so we see that our above interpretation of the map $\vec F^*$ really is well-defined.
\end{example}


This last propositon suggests that we might want to look into the set
of prime ideas of a ring.


\end{document}
