\documentclass{ximera}



\usepackage{tikz-cd}
\usepackage[sans]{dsfont}

\let\oldbibliography\thebibliography%% to compact bib
\renewcommand{\thebibliography}[1]{%
  \oldbibliography{#1}%
  \setlength{\itemsep}{0pt}%
}


\DefineVerbatimEnvironment{macaulay2}{Verbatim}{numbers=left,frame=lines,label=Macaulay2,labelposition=topline}

%%% This next bit of code defines all our theorem environments
\makeatletter
\let\c@theorem\relax
\let\c@corollary\relax
\makeatother

\let\definition\relax
\let\enddefinition\relax

\let\theorem\relax
\let\endtheorem\relax

\let\proposition\relax
\let\endproposition\relax

\let\exercise\relax
\let\endexercise\relax

\let\question\relax
\let\endquestion\relax

\let\remark\relax
\let\endremark\relax

\let\corollary\relax
\let\endcorollary\relax


\let\example\relax
\let\endexample\relax

\let\warning\relax
\let\endwarning\relax

\let\lemma\relax
\let\endlemma\relax

\newtheoremstyle{SlantTheorem}{\topsep}{\topsep}%%% space between body and thm
		{\slshape}                      %%% Thm body font
		{}                              %%% Indent amount (empty = no indent)
		{\bfseries\sffamily}            %%% Thm head font
		{}                              %%% Punctuation after thm head
		{3ex}                           %%% Space after thm head
		{\thmname{#1}\thmnumber{ #2}\thmnote{ \bfseries(#3)}}%%% Thm head spec
\theoremstyle{SlantTheorem}
\newtheorem{theorem}{Theorem}
\newtheorem{definition}[theorem]{Definition}
\newtheorem{proposition}[theorem]{Proposition}
%% \newtheorem*{dfnn}{Definition}
%% \newtheorem{ques}{Question}[theorem]
\newtheorem{lemma}[theorem]{Lemma}
%% \newtheorem*{war}{WARNING}
%% \newtheorem*{cor}{Corollary}
%% \newtheorem*{eg}{Example}
\newtheorem*{remark}{Remark}
\newtheorem*{touchstone}{Touchstone}
\newtheorem{corollary}{Corollary}[theorem]
\newtheorem*{example}{Example}
\newtheorem*{warning}{WARNING}


\newtheoremstyle{Exercise}{\topsep}{\topsep} %%% space between body and thm
		{}                           %%% Thm body font
		{}                           %%% Indent amount (empty = no indent)
		{\bfseries}                  %%% Thm head font
		{)}                          %%% Punctuation after thm head
		{ }                          %%% Space after thm head
		{\thmnumber{#2}\thmnote{ \bfseries(#3)}}%%% Thm head spec
\theoremstyle{Exercise}
\newtheorem{exercise}{}[theorem]

%% \newtheoremstyle{Question}{\topsep}{\topsep} %%% space between body and thm
%% 		{\bfseries}                  %%% Thm body font
%% 		{3ex}                        %%% Indent amount (empty = no indent)
%% 		{}                           %%% Thm head font
%% 		{}                           %%% Punctuation after thm head
%% 		{}                           %%% Space after thm head
%% 		{\thmnumber{#2}\thmnote{ \bfseries(#3)}}%%% Thm head spec
\newtheoremstyle{Question}{3em}{3em} %%% space between body and thm
		{\large\bfseries}                           %%% Thm body font
		{3ex}                           %%% Indent amount (empty = no indent)
		{\bfseries}                  %%% Thm head font
		{}                          %%% Punctuation after thm head
		{ }                          %%% Space after thm head
		{}%%% Thm head spec
\theoremstyle{Question}
\newtheorem*{question}{}



\renewcommand{\tilde}{\widetilde}
\renewcommand{\bar}{\overline}
\renewcommand{\hat}{\widehat}
\newcommand{\N}{\mathbb N}
\newcommand{\Z}{\mathbb Z}
\newcommand{\R}{\mathbb R}
\newcommand{\Q}{\mathbb Q}
\newcommand{\C}{\mathbb C}
\newcommand{\V}{\mathbb V}
\newcommand{\I}{\mathbb I}
\newcommand{\A}{\mathbb A}
\newcommand{\iso}{\simeq}
\newcommand{\ph}{\varphi}
\newcommand{\Cf}{\mathcal{C}}
\newcommand{\IZ}{\mathrm{Int}(\Z)}
\newcommand{\dsum}{\oplus}
\newcommand{\directsum}{\bigoplus}
\newcommand{\union}{\bigcup}
\renewcommand{\i}{\mathfrak}
\renewcommand{\a}{\mathfrak{a}}
\renewcommand{\b}{\mathfrak{b}}
\newcommand{\m}{\mathfrak{m}}
\newcommand{\p}{\mathfrak{p}}
\newcommand{\q}{\mathfrak{q}}
\newcommand{\dfn}[1]{\textbf{#1}\index{#1}}
\let\hom\relax
\DeclareMathOperator{\ann}{Ann}
\DeclareMathOperator{\h}{ht}
\DeclareMathOperator{\hom}{Hom}
\DeclareMathOperator{\Span}{Span}
\DeclareMathOperator{\spec}{Spec}
\DeclareMathOperator{\maxspec}{MaxSpec}
\DeclareMathOperator{\supp}{Supp}
\DeclareMathOperator{\ass}{Ass}
\DeclareMathOperator{\ff}{Frac}
\DeclareMathOperator{\im}{Im}
\DeclareMathOperator{\syz}{Syz}
\DeclareMathOperator{\gr}{Gr}
\renewcommand{\ker}{\mathop{\mathrm{Ker}}\nolimits}
\newcommand{\coker}{\mathop{\mathrm{Coker}}\nolimits}
\newcommand{\lps}{[\hspace{-0.25ex}[}
\newcommand{\rps}{]\hspace{-0.25ex}]}
\newcommand{\into}{\hookrightarrow}
\newcommand{\onto}{\twoheadrightarrow}
\newcommand{\tensor}{\otimes}
\newcommand{\x}{\mathbf{x}}
\newcommand{\X}{\mathbf X}
\newcommand{\Y}{\mathbf Y}
\renewcommand{\k}{\boldsymbol{\kappa}}
\renewcommand{\emptyset}{\varnothing}
\renewcommand{\qedsymbol}{$\blacksquare$}
\renewcommand{\l}{\ell}
\newcommand{\1}{\mathds{1}}
\newcommand{\lto}{\mathop{\longrightarrow\,}\limits}
\newcommand{\rad}{\sqrt}
\newcommand{\hf}{H}
\newcommand{\hs}{H\!S}
\newcommand{\hp}{H\!P}
\renewcommand{\vec}{\mathbf}
\renewcommand{\phi}{\varphi}
\renewcommand{\epsilon}{\varepsilon}
\renewcommand{\subset}{\subseteq}
\renewcommand{\supset}{\supseteq}
\newcommand{\macaulay}{\textsl{Macaulay2}}
\newcommand{\invlim}{\varprojlim}


%\renewcommand{\proofname}{Sketch of Proof}


\renewenvironment{proof}[1][Proof]
  {\begin{trivlist}\item[\hskip \labelsep \itshape \bfseries #1{}\hspace{2ex}]\upshape}
{\qed\end{trivlist}}

\newenvironment{sketch}[1][Sketch of Proof]
  {\begin{trivlist}\item[\hskip \labelsep \itshape \bfseries #1{}\hspace{2ex}]\upshape}
{\qed\end{trivlist}}



\makeatletter
\renewcommand\section{\@startsection{paragraph}{10}{\z@}%
                                     {-3.25ex\@plus -1ex \@minus -.2ex}%
                                     {1.5ex \@plus .2ex}%
                                     {\normalfont\large\sffamily\bfseries}}
\renewcommand\subsection{\@startsection{subparagraph}{10}{\z@}%
                                    {3.25ex \@plus1ex \@minus.2ex}%
                                    {-1em}%
                                    {\normalfont\normalsize\sffamily\bfseries}}
\makeatother

%% Fix weird index/bib issue.
\makeatletter
\gdef\ttl@savemark{\sectionmark{}}
\makeatother


\makeatletter
%% no number for refs
\newcommand\frontstyle{%
  \def\activitystyle{activity-chapter}
  \def\maketitle{%
    \addtocounter{titlenumber}{1}%
                    {\flushleft\small\sffamily\bfseries\@pretitle\par\vspace{-1.5em}}%
                    {\flushleft\LARGE\sffamily\bfseries\@title \par }%
                    {\vskip .6em\noindent\textit\theabstract\setcounter{problem}{0}\setcounter{sectiontitlenumber}{0}}%
                    \par\vspace{2em}
                    \phantomsection\addcontentsline{toc}{section}{\textbf{\@title}}%
                  }}
\makeatother


\author{Bart Snapp}

\title{The nullstellensatz}


\begin{document}
\begin{abstract}
  We prove Noether normalization, the weak nullstellensatz and the
  strong nullstellensatz. Sources and references:
  \cite{AM1969,dE1995,hM1986,mR1995}.
\end{abstract}
\maketitle


At this point we should believe there is a correspondence between:
affine algebraic varieties and ideals; and affine space and the prime
spectrum. Let's make this precise. We start by proving Noether
normalization. We will prove it now for infinite fields. Later we will
give a general proof. See \cite{iK1966,gK2011,eK1991} for alternate
proofs at this level.


\begin{proposition}\label{P:finiteext}
  Let $R$ be a ring, if $f$ is a monic polynomial in $R[X]$, then
  $R[X]/(f)$ is a finitely generated module over $R$.
  \begin{proof}
    If $f(X) = X^n + a_{n-1}X^{n-1} + \dots + a_1 X + a_0$, then
    \[
    X^n = -a_{n-1}X^{n-1} - \dots - a_1 X - a_0
    \]
    in $R[X]/(f)$. So, $R[X]/(f)$ is generated by
    $\{1,X,\dots,X^{n-1}\}$.
  \end{proof}
\end{proposition}

\begin{proposition}
  Let $R$, $S$, and $T$ be rings. If $S$ is a finitely generated
  $R$-module, and $T$ is a finitely generated $S$-module, then $T$ is
  a finitely generated $R$-module.
\end{proposition}


\begin{theorem}[Noether normalization]
  Let $k$ be an infinite field and $A$ be a finitely generated
  $k$-algebra. There exist $\alpha_1,\dots,\alpha_m\in A$ such that
  \begin{enumerate}
  \item $\alpha_i$ are algebraically independent over $k$.
  \item $A$ is module-finite over $k[\boldsymbol{\alpha}]$.
  \end{enumerate}
  In other words, 
  \[
  k\subset k[\boldsymbol{\alpha}] \subset A,
  \]
  where $k[\boldsymbol{\alpha}]$ is a polynomial ring over $k$, and
  $A$ is a finitely generated as a $k[\boldsymbol{\alpha}]$-module.
  \begin{sketch}
    Proceed by induction on the number of generators
    $\zeta_1,\dots,\zeta_n$ of $A$ as a $k$-algebra. Consider
    \[
    k \subset B \subset k[\zeta] = A
    \]
    Decide what $B$ must be when $\zeta$ is algebraic over $k$, and
    what $B$ must be when $\zeta$ is transcendental over $k$.

    Now suppose that we know this theorem for $A$ generated by less
    than $n$ generators. We need to show this holds for $n$
    generators. Set
    \[
    A = k[\zeta_1,\dots,\zeta_n] = k[\boldsymbol{\zeta}].
    \]
    Consider the canonical map
    \[
    k[\X] \to k[\boldsymbol{\zeta}]
    \]
    If this map is injective, then $k\subset k[\boldsymbol{\zeta}]$ is
    a polynomial extension.

    If the map above is not injective, then there exists $F\in k[\X]$
    that maps to zero.


    If $F$ is monic, say in $X_n$, then
    \[
    k[\zeta_1,\dots,\zeta_{n-1}] \subset k[\zeta_1,\dots,\zeta_n]
    \]
    is a module finite extension.


    If $F$ is not monic, we will change variables. Write
    \begin{align*}
      X_i &\mapsto X_i + a_i X_n\\
      X_n &\mapsto X_n
    \end{align*}
    where
    \begin{align*}
      \xi_i &= \zeta_i - a_i \zeta_n\\
      \xi_n &= \zeta_n
    \end{align*}
    Now,
    \[
    k[\X] \to k[\boldsymbol{\xi}]  = k[\boldsymbol{\zeta}]
    \]
    with
    \[
    F(X_1 + a_1 X_n,X_2 + a_2 X_n,\dots , X_n) \mapsto 0.
    \]

    We claim that we can choose $a_i$ such that
    \[
    F(X_1 + a_1 X_n,X_2 + a_2 X_n,\dots , X_{n-1} + a_{n-1},X_n)
    \]
    is monic in $X_n$.
    (More will be given later)
  \end{sketch}
\end{theorem}

Let's see an example in \macaulay. Below, we will set $k = \Q$ and $A
= \Q[x]/(xy+yz^2,x^2y-xz^3)$.

\begin{macaulay2}
loadPackage "NoetherNormalization"
setRandomSeed(42);
R = QQ[x,y,z,MonomialOrder=>Lex];
I = ideal(x*y+y*z^2,x^2*y-x*z^3);
A = R/I;
(phi,J,X)=noetherNormalization(A)
phi(lift(x,R))
phi(lift(y,R))
phi(lift(z,R))
transpose gens gb J
\end{macaulay2}

Now we have $I = (xy+yz^2,x^2y-xz^3)$, and
$A=\Q[x,y,z]/I$. \macaulay\ gives a map of rings $\phi$ and the image
of $I$ under $\phi$, we'll call it $J$. We now have
\[
A=\Q[x,y,z]/I  \iso \Q[4x+y+z,x,y]/J.
\]
From our final computation, we witness a monic polynomial generating
$y$ over $\Q[z]$, namely $y^7+y^6z-3y^6-2y^5z+y^4z^2$, and a polynomial
with an invertible lead coefficient generating $x$ over $\Q[y,z]$,
namely $4x^2+xy^2+xy+xz$. Hence 
\[
\Q \subset \Q[z] \subset \Q[4x+y+z,x,y]/J
\]
where the first extension is a polynomial extension and the second is
module finite.




Now we prove the so-called ``weak nullstellensatz.'' Again we will
insist that the field be infinite so that we can apply the version of
Noether normalization above. Later, when we prove Noether
normalization in generality, a more general version of the weak
nullstellensatz will also hold.

\begin{theorem}[Weak nullstellensatz]
  Let $k$ be a(n infinite) field, and $L$ be a $k$-algebra such that
  \begin{enumerate}
  \item $L$ is finitely generated as a $k$-algebra.
  \item $L$ is a field.
  \end{enumerate}
  In this case, $L$ is an algebraic extension of $k$.
\end{theorem}

\begin{corollary}
  Let $k$ be an algebraically closed field. An ideal $I$ is maximal in
  $k[\X]$ if and only if $I= (X_1-a_1,\dots,X_n-a_n)$ for some $a_i\in
  k$.
\begin{sketch}
  $(\Rightarrow)$ If $I$ is maximal then $k[\X] / I$ is a field. By
  the weak nullstellansatz, it is an algebraic extension of $k$ where
  $k$ is an algebraically closed field. Therefore, $k[\X] / I = k$.

  $(\Leftarrow)$ The evalutaion map is a surjective homomorphism to
  $k$. So, $I$ is maximal.
\end{sketch}
\end{corollary}




\begin{theorem}[Nullstellensatz]
  Let $k$ be an algebraically closed field with $I \subsetneq k[X_1,\dots,X_n]$.
  \begin{enumerate}
  \item $\V(I) \ne \emptyset$.
  \item $\I(\V(I)) = \rad{I}$.
  \end{enumerate}
\begin{sketch}
Assume k is an algebraically closed field and $I$ is a proper ideal.
  \begin{enumerate}
    \item $I$ is contained in a maximal ideal. Use the corollary above
      to finish.
    \item $(\subset)$ Consider $F(\X)\in\I(\V(I))$. We want to prove
      $F(\X)\in\rad{I}$. We will use the \index{Rabinowitch trick}``Rabinowitch Trick.'' This
      means, we'll introduce a new variable $Y$ and consider the ideal
      $J= I + (F(\X)\cdot Y-1)\subset k[\X,Y]$. Observe that $\V(J) =
      \emptyset$ and conclude from part (a) that $J=k[\X,Y]$. Write
      \[
      1= G_0(\X,Y)\cdot (F(\X)\cdot Y-1) H_i(\X) + \sum_{i=1}^m G_i(\X,Y)
      \cdot H_i(\X)
      \]
      where $H_i(\X) \in I$ and $G_i(\X,Y) \in k[\X,Y]$.  Working in
      the field of fractions, set $Y=\frac{1}{F(\X)}$. Clear
      denominators to see that a power of $F(\X)$ is an element of $I$.

      $(\supset)$ Note that that in $k[\X]$, $F(\x) = 0$ if and only
      if $F(\x)^n = 0$ where $\x\in k$.
  \end{enumerate}
\end{sketch}
\end{theorem}

\begin{remark}%https://mathoverflow.net/questions/90661/the-rabinowitz-trick
  To make the ``Rabinowitch Trick'' more intuitative, consider a
  paraphrase of \link[Martin Brandenburg's answer on MathOverflow]{https://mathoverflow.net/questions/90661/the-rabinowitz-trick}:
\begin{quote}
  We want to prove that $F(\X)$ is nilpotent in $k[\X]/I$, or in other
  words, that the localization $(k[\X]/I)_{F(\X)}$ vanishes. We have that the rings
  \[
  (k[\X]/I)_{F(\X)}\iso k[\X,Y]/(I,F(\X)\cdot Y-1)
  \]
  are isomorphic as rings as $Y$ is congruent to $F(X)^{-1}$ in the
  right hand ring.  But, clearly $\V(I,F(\X)\cdot Y-1)=\emptyset$ and
  therefore the weak nullstellensatz implies that $(I,F(\X)\cdot
  Y-1)=(1)$, meaning that the quotient vanishes, and hence $F(\X)$ is
  nilpotent.
\end{quote}
\end{remark}



\begin{corollary}
  Let $k$ be an algebraically closed field and $\vec{F} =
  F_1,\dots,F_n\in k[x]$. In this case,
  \[
  \vec{F}(\x) \ne 0
  \]
  for all $\x\in k^m$ if and only if there exist $G_i\in k[\X]$ such that
  \[
  1 = \sum_{i=1}^m G_i F_i.
  \]
\end{corollary}

\begin{corollary}
  Let $k$ be an algebraically closed field and $A$ be a $k$-algebra.
  In this case,
  \[
  A = k[\X]/\I(V) 
  \]
  if and only if $\rad{0_A} = (0)$ and $A$ is finitely generated as a
  $k$-algebra.
\end{corollary}


\begin{corollary}
  Let $k$ be an algebraically closed field and $V\subset\A^n$. An ideal $I$ is maximal in
  $k[\X]/\I(V)$ if and only if $I= (X_1-a_1,\dots,X_n-a_n)$ for some $a_i\in
  k$.
\end{corollary}

\begin{corollary}
  The category of affine algebraic varieties and morphisms is
  equivalent to the category of $k$-algebras of the form
  $k[\X]/\rad{I}$ and $k$-algebra homomorphisms with the arrows
  reversed.
\end{corollary}







\end{document}
