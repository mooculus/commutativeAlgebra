\documentclass{ximera}



\usepackage{tikz-cd}
\usepackage[sans]{dsfont}

\let\oldbibliography\thebibliography%% to compact bib
\renewcommand{\thebibliography}[1]{%
  \oldbibliography{#1}%
  \setlength{\itemsep}{0pt}%
}


\DefineVerbatimEnvironment{macaulay2}{Verbatim}{numbers=left,frame=lines,label=Macaulay2,labelposition=topline}

%%% This next bit of code defines all our theorem environments
\makeatletter
\let\c@theorem\relax
\let\c@corollary\relax
\makeatother

\let\definition\relax
\let\enddefinition\relax

\let\theorem\relax
\let\endtheorem\relax

\let\proposition\relax
\let\endproposition\relax

\let\exercise\relax
\let\endexercise\relax

\let\question\relax
\let\endquestion\relax

\let\remark\relax
\let\endremark\relax

\let\corollary\relax
\let\endcorollary\relax


\let\example\relax
\let\endexample\relax

\let\warning\relax
\let\endwarning\relax

\let\lemma\relax
\let\endlemma\relax

\newtheoremstyle{SlantTheorem}{\topsep}{\topsep}%%% space between body and thm
		{\slshape}                      %%% Thm body font
		{}                              %%% Indent amount (empty = no indent)
		{\bfseries\sffamily}            %%% Thm head font
		{}                              %%% Punctuation after thm head
		{3ex}                           %%% Space after thm head
		{\thmname{#1}\thmnumber{ #2}\thmnote{ \bfseries(#3)}}%%% Thm head spec
\theoremstyle{SlantTheorem}
\newtheorem{theorem}{Theorem}
\newtheorem{definition}[theorem]{Definition}
\newtheorem{proposition}[theorem]{Proposition}
%% \newtheorem*{dfnn}{Definition}
%% \newtheorem{ques}{Question}[theorem]
\newtheorem{lemma}[theorem]{Lemma}
%% \newtheorem*{war}{WARNING}
%% \newtheorem*{cor}{Corollary}
%% \newtheorem*{eg}{Example}
\newtheorem*{remark}{Remark}
\newtheorem*{touchstone}{Touchstone}
\newtheorem{corollary}{Corollary}[theorem]
\newtheorem*{example}{Example}
\newtheorem*{warning}{WARNING}


\newtheoremstyle{Exercise}{\topsep}{\topsep} %%% space between body and thm
		{}                           %%% Thm body font
		{}                           %%% Indent amount (empty = no indent)
		{\bfseries}                  %%% Thm head font
		{)}                          %%% Punctuation after thm head
		{ }                          %%% Space after thm head
		{\thmnumber{#2}\thmnote{ \bfseries(#3)}}%%% Thm head spec
\theoremstyle{Exercise}
\newtheorem{exercise}{}[theorem]

%% \newtheoremstyle{Question}{\topsep}{\topsep} %%% space between body and thm
%% 		{\bfseries}                  %%% Thm body font
%% 		{3ex}                        %%% Indent amount (empty = no indent)
%% 		{}                           %%% Thm head font
%% 		{}                           %%% Punctuation after thm head
%% 		{}                           %%% Space after thm head
%% 		{\thmnumber{#2}\thmnote{ \bfseries(#3)}}%%% Thm head spec
\newtheoremstyle{Question}{3em}{3em} %%% space between body and thm
		{\large\bfseries}                           %%% Thm body font
		{3ex}                           %%% Indent amount (empty = no indent)
		{\bfseries}                  %%% Thm head font
		{}                          %%% Punctuation after thm head
		{ }                          %%% Space after thm head
		{}%%% Thm head spec
\theoremstyle{Question}
\newtheorem*{question}{}



\renewcommand{\tilde}{\widetilde}
\renewcommand{\bar}{\overline}
\renewcommand{\hat}{\widehat}
\newcommand{\N}{\mathbb N}
\newcommand{\Z}{\mathbb Z}
\newcommand{\R}{\mathbb R}
\newcommand{\Q}{\mathbb Q}
\newcommand{\C}{\mathbb C}
\newcommand{\V}{\mathbb V}
\newcommand{\I}{\mathbb I}
\newcommand{\A}{\mathbb A}
\newcommand{\iso}{\simeq}
\newcommand{\ph}{\varphi}
\newcommand{\Cf}{\mathcal{C}}
\newcommand{\IZ}{\mathrm{Int}(\Z)}
\newcommand{\dsum}{\oplus}
\newcommand{\directsum}{\bigoplus}
\newcommand{\union}{\bigcup}
\renewcommand{\i}{\mathfrak}
\renewcommand{\a}{\mathfrak{a}}
\renewcommand{\b}{\mathfrak{b}}
\newcommand{\m}{\mathfrak{m}}
\newcommand{\p}{\mathfrak{p}}
\newcommand{\q}{\mathfrak{q}}
\newcommand{\dfn}[1]{\textbf{#1}\index{#1}}
\let\hom\relax
\DeclareMathOperator{\ann}{Ann}
\DeclareMathOperator{\h}{ht}
\DeclareMathOperator{\hom}{Hom}
\DeclareMathOperator{\Span}{Span}
\DeclareMathOperator{\spec}{Spec}
\DeclareMathOperator{\maxspec}{MaxSpec}
\DeclareMathOperator{\supp}{Supp}
\DeclareMathOperator{\ass}{Ass}
\DeclareMathOperator{\ff}{Frac}
\DeclareMathOperator{\im}{Im}
\DeclareMathOperator{\syz}{Syz}
\DeclareMathOperator{\gr}{Gr}
\renewcommand{\ker}{\mathop{\mathrm{Ker}}\nolimits}
\newcommand{\coker}{\mathop{\mathrm{Coker}}\nolimits}
\newcommand{\lps}{[\hspace{-0.25ex}[}
\newcommand{\rps}{]\hspace{-0.25ex}]}
\newcommand{\into}{\hookrightarrow}
\newcommand{\onto}{\twoheadrightarrow}
\newcommand{\tensor}{\otimes}
\newcommand{\x}{\mathbf{x}}
\newcommand{\X}{\mathbf X}
\newcommand{\Y}{\mathbf Y}
\renewcommand{\k}{\boldsymbol{\kappa}}
\renewcommand{\emptyset}{\varnothing}
\renewcommand{\qedsymbol}{$\blacksquare$}
\renewcommand{\l}{\ell}
\newcommand{\1}{\mathds{1}}
\newcommand{\lto}{\mathop{\longrightarrow\,}\limits}
\newcommand{\rad}{\sqrt}
\newcommand{\hf}{H}
\newcommand{\hs}{H\!S}
\newcommand{\hp}{H\!P}
\renewcommand{\vec}{\mathbf}
\renewcommand{\phi}{\varphi}
\renewcommand{\epsilon}{\varepsilon}
\renewcommand{\subset}{\subseteq}
\renewcommand{\supset}{\supseteq}
\newcommand{\macaulay}{\textsl{Macaulay2}}
\newcommand{\invlim}{\varprojlim}


%\renewcommand{\proofname}{Sketch of Proof}


\renewenvironment{proof}[1][Proof]
  {\begin{trivlist}\item[\hskip \labelsep \itshape \bfseries #1{}\hspace{2ex}]\upshape}
{\qed\end{trivlist}}

\newenvironment{sketch}[1][Sketch of Proof]
  {\begin{trivlist}\item[\hskip \labelsep \itshape \bfseries #1{}\hspace{2ex}]\upshape}
{\qed\end{trivlist}}



\makeatletter
\renewcommand\section{\@startsection{paragraph}{10}{\z@}%
                                     {-3.25ex\@plus -1ex \@minus -.2ex}%
                                     {1.5ex \@plus .2ex}%
                                     {\normalfont\large\sffamily\bfseries}}
\renewcommand\subsection{\@startsection{subparagraph}{10}{\z@}%
                                    {3.25ex \@plus1ex \@minus.2ex}%
                                    {-1em}%
                                    {\normalfont\normalsize\sffamily\bfseries}}
\makeatother

%% Fix weird index/bib issue.
\makeatletter
\gdef\ttl@savemark{\sectionmark{}}
\makeatother


\makeatletter
%% no number for refs
\newcommand\frontstyle{%
  \def\activitystyle{activity-chapter}
  \def\maketitle{%
    \addtocounter{titlenumber}{1}%
                    {\flushleft\small\sffamily\bfseries\@pretitle\par\vspace{-1.5em}}%
                    {\flushleft\LARGE\sffamily\bfseries\@title \par }%
                    {\vskip .6em\noindent\textit\theabstract\setcounter{problem}{0}\setcounter{sectiontitlenumber}{0}}%
                    \par\vspace{2em}
                    \phantomsection\addcontentsline{toc}{section}{\textbf{\@title}}%
                  }}
\makeatother


\author{Bart Snapp}

\title{The dimension theorem}

%%
%% TODO: Fix verbiage in induction proofs of lemmas for dimension theorem.
%%

\begin{document}
\begin{abstract}
  We prove the dimension theorem. Sources and
  references: \cite{sD2008,hM1986}.
\end{abstract}
\maketitle


In this section, we will show that if $A$ is a local ring and $M$ is a
finitely generated $A$-module, then
\[
\dim(M) = d(M) = s(M). 
\]

We will prove this in three (or four) lemmas.

\begin{lemma}[$\boldsymbol{\dim(M)\le d(M)}$]\label{L:dimd}
  Let $(A,\m)$ be a local ring and $M$ be a finitely generated
  $A$-module. In this case
  \[
  \dim(M)\le d(M).
  \]
  \begin{proof}
    Proceeding by induction on $d(M)$. Suppose that $d(M) = 0$. Then 
    \[
    d(M) = P_M(n) = \l(M/\m^n M)
    \]
    is constant for $n$ sufficiently large. Thus for large enough $n$,
    \[
    \l(M/\m^n M) = \l(M/\m^{n+1}M).
    \]
    This shows us that $\l(\m^n M/\m^{n+1}M) = 0$. Hence $\m^n
    M/\m^{n+1}M = 0$. So,
    \[
    \m^n M = \m^{n+1}M = \m(\m^n M)
    \]
    and hence by Nakayama's lemma, Theorem~\ref{NAK},
    \index{Nakayama's lemma} $\m^n M =0$ and hence $\m^n\subset\ann(M)$. Since
    \[
    \dim(M) = \dim(A/\ann(M))  = \dim(A/\sqrt{\ann(M)}) = \dim(A/\m),
    \]   
    we see that $\dim(M) = 0$. 
    
    Now assume $d(M) = n> 0$. Take any maximal chain 
    \[
    \p_0\subsetneq \p_1 \subsetneq \cdots \subsetneq \p_m
    \]
    in $\supp(M)$.  We need to show $m\le n$. We know that $\p_0$ is a
    minimal element of $\supp(M)$ and hence by
    Theorem~\ref{T:AssassinsMinimal}, is a minimal element in
    $\ass(M)$.  So we have an injection
    \[
    A/\p_0\into M.
    \]
    Set $N = A/\p_0$.  By Corollary~\ref{C:sub}, $d(N) \le d(M)$.  So
    it suffices to show that $m \le d(N)$.  Let $x \in \p_1 - \p_0$.
    Consider the short exact sequence
    \[
    0\lto N \lto^{x} N \lto N/xN \lto 0.
    \]
    By Proposition~\ref{P:AddHS}, we have that
    \[
    P_N(n) = P_N(n) + P_{N/xN}(n) - r(n),
    \]
    where $\deg(r) < \deg(P_N) = d(N)$.  But then
    \[
    P_{N/xN}(n) = r(n).
    \]
    Thus $d(N/xN) = \deg(r) < n$.  By the inductive hypothesis,
    $\dim(N/xN) \le d(N/xN)$.  Since
    \[
    \p_1 \subsetneq \cdots \subsetneq \p_m
    \]
    is a strict chain of prime ideals in $\supp(N/xN)$, we have that
    \[
    m - 1 \le \dim(N/xN) \le d(N/xN) \le n - 1.
    \]
    Hence $\dim(M) = m \le n = d(M)$.
  \end{proof}
\end{lemma}



\begin{lemma}[$\boldsymbol{d(M)\le s(M)}$]\label{L:ds}
  Let $(A,\m)$ be a local ring and $M$ be a finitely generated
  $A$-module. In this case
  \[
  d(M)\le s(M).
  \]
  \begin{proof}
    Proceed by induction on $s(M)$. If $s(M) = 0$, then
    $\l(M)< \infty$. This means that $P_M(n)$ is bounded, and hence
    $d(M) = 0$.

    Now suppose that $s(M) = m >0$. Consider a system of
    parameters $x_1,\dots, x_m\in \m$. In this case
    \[
    \l\left( \frac{M}{(x_1,\dots,x_m)M} \right)<\infty.
    \]
    Set $M_i = \frac{M}{(x_1,\dots,x_i)M}$. So we have that
    \[
    s(M_i) = s(M)-i.
    \]
    Now note
    \[
    \l(M_1/\m^n M_1) = \l\left(\frac{M}{x_1 M + \m^n M}\right).
    \]
    From the short exact sequence
    \[
    0\lto \frac{x_1M}{x_1 M \cap \m^n M} \lto  M/\m^n M \lto \frac{M}{x_1 M + \m^n M}\lto 0
    \]
    we see that
    \[
    \l\left(\frac{M}{x_1 M + \m^n M}\right) = \l(M/\m^n M) - \l \left(\frac{M}{x_1 M \cap \m^n M}\right).
    \]
    Note,
    \[
    \frac{M}{(\m^nM:_M x_1)}\iso \frac{M}{x_1 M \cap \m^n M}
    \]
    via the map $m\mapsto x_1 m$. Finally note
    \[
    M/\m^{n-1}M \onto \frac{M}{x_1 M \cap \m^n M},
    \]
    hence $\l(M/\m^{n-1}M)\ge \l\left(\frac{M}{x_1 M \cap \m^n
      M}\right)$. Putting this together we see
    \[
    \l(M_1/\m^n M_1) \ge \l(M/\m^n M) - \l(M/\m^{n-1}M)
    \]
    So, $d(M_1) \ge d(M) - 1$. Repeating this process we find
    \[
    d(M_m) \ge d(M) -m.
    \]
    Since $s(M_m) = 0$, we have $d(M_m) = 0$ so, $m\ge d(M)$.
  \end{proof}
\end{lemma}





\begin{lemma}[Prime avoidance]\label{L:PA}\index{prime avoidance}
Let $R$ be a ring and $I$ be an ideal of $R$.  If $\p_1,\ldots,\p_n$
are prime ideals such that
\[
I \not\subseteq \p_i \qquad\text{for all $i$,}
\]
then 
\[
I \not\subset \bigcup_{i = 1}^n \p_i.
\]
\begin{proof}
  Proceed by induction on the number of prime ideals. If there is only
  one prime ideal, then we are done.

  Suppose there are two prime ideals. In this case, suppose
  $x_1\notin\p_1$ and $x_2\notin\p_2$. Then if $x_2\notin \p_1$ we
  are done, and if $x_2\in \p_1$, $x_1+x_2\notin \p_1$.  If $x_1\notin
  \p_2$ we are done, and if $x_1\in \p_2$, $x_1+x_2\notin
  \p_1$. Regardless we have found an element not in $\p_1\cup \p_2$.
  Hence $x_1+x_2\notin\p_1\cup\p_2$.
  
  Suppose the number of prime ideals is $n>2$, and that
  $x_i\notin\p_i$. In this case,
  \[
  x_1+ x_2x_3\dots x_n \notin \p_1\cup \union_{i=2}^n \p_i.
  \]
  This proves the lemma. Note, we actually don't need each $\p_i$ to
  be prime. It is sufficient that either each $\p_i$ is an ideal and
  $R$ contains and infinite field, or that at most two of the $\p_i$
  are not prime, see~\cite{dE1995}.
\end{proof}
\end{lemma}


\begin{remark}
  The above lemma is called \textit{prime avoidance} as if
  $I\not\subset \p_i$ for all $i$, then there is some element of $a
  \in I$ which \textit{avoids} being contained in any $\p_i$.
\end{remark}




\begin{lemma}[$\boldsymbol{s(M)\le\dim(M)}$]\label{L:sdim}
  Let $(A,\m)$ be a local ring and $M$ be a finitely generated
  $A$-module. In this case
  \[
  s(M)\le \dim(M).
  \]
  \begin{proof}
    Proceed by induction on $\dim(M)$. If $\dim(M) = 0$,
    \begin{align*}
      \dim(M) = 0 &\Leftrightarrow \ass(M)=\{\m\}, \\
      &\Leftrightarrow \l_A(M) <\infty,
    \end{align*}
    and so we see $s(M) = 0$.
    
    Now assume that $\dim(M)=n>0$. Let $\p_i$ be the minimal prime
    ideals of $\supp(M)$. By Theorem~\ref{T:AssassinsMinimal}, these
    primes are also minimal in $\ass(M)$ and there are only finitely
    many such primes. By the \index{prime avoidance}prime avoidance
    lemma, we may pick $x\in\m$ such that $x$ is not in any of these
    minimal primes. Thus
    \[
    \dim(M/xM) <\dim(M)
    \]
    and thus by induction, $s(M/xM)\le\dim(M/xM)$. But $s(M)-1\le s(M/xM)$ and so
    \[
    s(m) \le s(M/xM) +1 \le \dim(M/xM) +1 \le \dim(M).
    \]
  \end{proof}
\end{lemma}



\begin{theorem}[The Dimension Theorem]\label{T:dimension}\index{Dimension Theorem}
  Let $A$ be a local ring and $M$ be a finitely generated
  $A$-module. Then
  \[
  \dim(M) = d(M) = s(M).
  \]
  \begin{proof}
    By Lemma~\ref{L:dimd}, Lemma~\ref{L:ds}, and Lemma~\ref{L:sdim} we have shown that
    \[
    \dim(M)\le d(M) \le s(M) \le \dim(M),
    \]
    and hence $\dim(M) = d(M) = s(M)$.
  \end{proof}
\end{theorem}



\begin{corollary}
  If $(A,\m)$ is a local ring, then it is has finite dimension.
  \begin{proof}
    Since $A$ is Noetherian, $\m$ is finitely generated, thus
    $s(A)=\dim(A)$ is less than or equal to the number of generators of
    $\m$.
  \end{proof}
\end{corollary}

\begin{remark}
  By a example of Nagata, there are infinite dimensional Noetherian
  rings, see~\cite{mN1962}.
\end{remark}


\begin{corollary}
  If $(A,\m)$ is a local ring and $M$ is a finitely generated
  $A$-module, then
  \[
  \dim_A(M) =\dim_{\hat{A}}(\hat{M}).
  \]
  \begin{proof}
    We know from Corollary~\ref{C:quotcomp}
    \[
    M/\m^n M \iso \hat{M}/\hat{\m^n} \hat{M} \iso \hat{M}/ \hat{\m}^n\hat{M},
    \]
    and so $P_\m(M,n) = P_{\hat{\m}}(\hat{M},n)$.
  \end{proof}
\end{corollary}



\begin{corollary}
  If $(A,\m)$ is a local ring, then 
  \[
  \dim(A) = \min\{i: (a_1,\dots,a_i) = \a\text{ where $\ass(A/\a) = \{\m\}$}\}.
  \]
  \begin{proof}
    Let $\dim(A) = n$, then there exists $x_1,\dots,x_n$ a system of
    parameters of $A$ such that $\l(A/\x) <\infty$. Thus
    $\ass(A/(x_1,\dots,x_n)) = \{\m\}$.  Since $s(A) = n$, we see that
    we cannot obtain $y_1,\dots,y_t$ with $t<n$ such that
    $\l(A/\mathbf{y})<\infty$.  So for any $y_1,\ldots,y_t$ with $t <
    n$, $\ass(A/(y_1,\ldots,y_t)) \ne \{\m\}$.
  \end{proof}
\end{corollary}


\begin{corollary}
  If $R$ is a Noetherian ring, not necessarily local, consider any
  decreasing chain of prime ideals in $R$
  \[
  \p_0\supsetneq \p_1\supsetneq \cdots \supsetneq \p_i\supsetneq \cdots
  \]
  Then there exists $n$ such that $\p_n = \p_{n+1} = \cdots$.
  \begin{proof}
    If we localize at $\p_0$, then $\dim(R_{\p_0}) <\infty$.  So for some $n$
    \[
    \p_n R_{\p_0} = \p_{n+1} R_{\p_0} = \cdots
    \]
    and so $\p_n = \p_{n+1} = \cdots$.
\end{proof}
\end{corollary}


\begin{corollary}
  Given rings $R\subset S$, where $S$ is integral over $R$, we have
  that $\dim(R) = \dim(S)$.
\end{corollary}



Recall the definition of \textit{height}. If $A$ is a Noetherian ring
and $\p$ is a prime ideal of $A$, then the \textbf{height} of $\p$ is
\[
\h(\p) = \sup\left\{n : \begin{minipage}{15em}there exists a chain of prime ideals $\p_0\subsetneq \p_1 \subsetneq \cdots \subsetneq \p_{n-1}\subsetneq \p_n =\p$\end{minipage} \right\}.
\]
Note that $\h(\p) = \dim(A_\p)$.


\begin{definition}\index{height}\index{hti@$\h(I)$}
  If $R$ is a Noetherian ring and $I$ is any ideal of $R$, then the \textbf{height} of $I$ is 
  \begin{align*}
    \h(I) &= \inf\{\h(\p): I\subset\p\} \\
    &= \inf\{\h(\p): \p \text{ is minimal in $\ass(R/I)$}\}.
  \end{align*}
\end{definition}

\begin{definition}\index{coheight}\index{coht@$\mathrm{coht}(I)$}
  If $R$ is a Noetherian ring and $I$ is any ideal, then the
  \textbf{coheight}, denoted $\mathrm{coht}(I)$ is defined as
  \[
  \mathrm{coht}(I) = \dim(R/I).
  \]
\end{definition}

\begin{exercise}
  Check that for any ideal $I$, $\h(I)+\dim(A/I)\le\dim(A)$. 
\end{exercise}

\begin{warning}
  Even if $I$ is a prime ideal, the above inequality may be strict.
\end{warning}

\begin{corollary}[Krull's ideal theorem]\label{C:KIT}\index{Krull's ideal theorem}
  Let $R$ be a Noetherian ring and $\p$ is a prime ideal.  Then
  $\h(\p)\le n$ if and only if there exist $a_1,\dots,a_n\in\p$ such
  that $\p$ is a minimal prime containing $(a_1,\dots,a_n)$.
  \begin{proof}
    Since $\h(\p) = \dim(R_\p)$, $\dim(R_\p)\le n$ if and only if there exists 
    \[
    \frac{x_1}{u_1}, \dots, \frac{x_n}{u_n}
    \]
    in $A_\p$ such that 
    \[
    \ass\left(\frac{x_1}{u_1}, \dots, \frac{x_n}{u_n}\right) = \{\p\}.
    \]
    This is the case if and only if there exist $x_1,\dots, x_n\in \p$
    and $\p$ contains $(x_1,\dots,x_n)$ minimally.
  \end{proof}
\end{corollary}


\begin{corollary}[Krull's principal ideal theorem]\index{Krull's principal ideal theorem}
  Let $R$ be a Noetherian ring and $x$ be an element of $R$ which is
  not a unit or a zerodivisor. Then every prime which contains $(x)$
  minimally has height $1$.
  \begin{proof}
    By Corollary~\ref{C:KIT} the minimal prime containing $(x)$ has
    height at most one. The prime $\p$ containing $(x)$ cannot have
    height zero, as then $\p\in\ass(R)$ by
    Theorem~\ref{T:AssassinsMinimal} and hence then every element of
    $\p$ is a zerodivisor, which is not the case.
  \end{proof}
\end{corollary}

\begin{remark}
  The above corollary to the Dimension Theorem is sometimes called
  \textbf{Krull's hauptidealsatz}.\index{Krull's hauptidealsatz}
\end{remark}

\begin{corollary}\label{C:nzdd-1}
  Let $(A,\m)$ be a local ring and $x$ be an element of $\m$ which is
  not a zerodivisor. Then $\dim(A/xA) = \dim(A)-1$.
  \begin{proof}
    This follows from the previous corollary and the definition of
    dimension.
  \end{proof}
\end{corollary}


\begin{exercise}
  If $(A,\m)$ is a local ring and $M$ is a finitely generated
  $A$-module with $x_1,\dots,x_i\in\m$, then
  \[
  \dim(M/(x_1,\dots,x_i)M)\ge \dim(M) -i.
  \]
  Equality holds if and only if $x_1,\dots,x_i$ form part of a system
  of parameters for $M$.
\end{exercise}


\begin{example}
  A ring $A$ is Artinian if and only if $\dim(A) =
  0$. \index{Artinian!ring}
\end{example}

\begin{example}
  A PID has dimension one.
\end{example}

\begin{example}
  $\dim(\Z) = 1$.
\end{example}

\begin{example}
  If $k$ is a field, then $\dim(k[X]) = 1$.
\end{example}

\begin{example}
  If $k$ is a field, then $\dim(k\lps x \rps) = 1$.
\end{example}


\section{Polynomials over Noetherian rings}

Now we will investigate the dimension of $R[X]$, where $R$ is a
Noetherian ring.


\begin{lemma}\label{L:FT1}
  If $R$ is a Noetherian ring and $P_1\subsetneq P_2$ are two prime
  ideals of $R[X]$ such that
  \[
  P_1\cap R = P_2\cap R = \p,
  \]
  then $P_1 = \p[X]$.
\end{lemma}

\begin{proof}
  Seeking a contradiction, suppose 
  \[
  \p[X]\subsetneq P_1\subsetneq P_2 
  \]
  and so working in $(R/\p)[X]$,  
  \[
  (0) \subsetneq (P_1/\p)[X] \subsetneq (P_2/\p)[X]
  \]
  Set $U = R-\p$.  Since $P_1 \cap U = P_2 \cap U = \emptyset$, and
  since localizations are exact, we have
  \[
  (0) \subsetneq U^{-1}(P_1/\p)[X] \subsetneq U^{-1}(P_2/\p)[X]
  \]
  is a strict chain of prime ideals in $(R_\p/\p R_\p)[X] =
  U^{-1}(R/\p)[X]$.  But this contradicts that $\dim(k[X]) = 1$ when
  $k$ is a field as $k[X]$ is a PID, hence all primes are principal
  and so are of height one or zero.  Thus $P_1 = \p[X]$.
\end{proof}


\begin{lemma}\label{L:FT2}
  If $R$ is Noetherian and $I$ is an ideal of $R$, then
  \[
  \h(I) = \h(I\cdot R[X]).
  \]
  \begin{proof}
    By the definition of assassins,
    \[
    \ass\left(\frac{R[X]}{IR[X]}\right) = \{\p_i[X]: \p_i\in\ass(R/I)\},
    \]
    as $R/I$ injects into $(R/I) [X]$. So it is enough to show that if
    $\p$ is a prime ideal in $R$, then
    \[
    \h(\p) = \h(\p[X]).
    \]
    Suppose that $\h(\p) = n$. Then there exist $a_1,\dots,a_n\in \p$
    such that $\p\supset(a_1,\dots,a_n)$ minimally. By
    Theorem~\ref{T:AssassinsMinimal}, we see that
    $\p[X]\supset(a_1,\dots,a_n)[X]$ minimally. Thus $\h(\p[X])\le n$.
    
    On the other hand, if 
    \[
    \p_0 \subsetneq \p_1 \subsetneq \cdots \subsetneq \p_n = \p
    \] 
    is a chain of prime ideals where $\h(\p) = n$, then 
    \[
    \p_0[X] \subsetneq \p_1[X] \subsetneq \cdots \subsetneq \p_n[X] = \p[X]
    \]
    is a chain of prime ideals in $R[X]$. Thus $\h(\p[X])\ge \h(\p)$
    and so we see that $\h(\p[X]) = \h(\p)$.
\end{proof}
\end{lemma}




\begin{theorem}
  If $R$ is a Noetherian ring, $\dim(R[X]) = \dim(R) + 1$.
  \begin{proof}
    First note that given a chain of primes $\p_0 \subsetneq \cdots
    \subsetneq \p_n$ in $R$, we can construct the chain of primes
    \[
    \p_0 R[X] \subsetneq \cdots \subsetneq \p_n R[X] \subsetneq \p_n
    R[X] + X R[X]
    \]
    in $R[X]$.  It is then clear that $\dim(R[X]) \ge \dim(R) + 1$, so
    it is enough to show $\dim(R[X]) \le\dim(R) + 1$.


    If $\dim(R) = \infty$, then there is nothing to prove. We will
    proceed by induction on the dimension of $R$. If $\dim(R) = 0$,
    then for $P_i$ a prime ideal in $R[X]$ write $\p_i = P_i\cap R$.
    So if
    \[
    P_0\subsetneq P_1\subsetneq \cdots \subsetneq P_n
    \]
    then since $\dim(R) = 0$, we have
    \[
    \p_0 = \p_1 = \cdots = \p_n.
    \]
    Thus by Lemma~\ref{L:FT1}, we have 
    \[
    \p_0[x] = P_0 = P_1 = \cdots = P_{n-1}\subsetneq P_n.
    \]
    Thus $n\le 1$ and so we see that $\dim(R[X]) = 1$.
    
    Now suppose that $\dim(R) = n >0$ and let
    \[
    P_0 \subsetneq P_1 \subsetneq \cdots \subsetneq P_n
    \]
    be a strict chain of prime ideals in $R$. We must show $\dim(R[X])
    \le\dim(R) + 1$.


    
    
    Case 1: Set $\p_i = P_i \cap R$ and suppose that $\p_{n-1}
    \subsetneq \p_n$. Then
    \[
    \dim(R_{\p_{n-1}})< \dim(R).
    \]
    By induction we then have that $\dim(R_{\p_{n-1}}[X]) =
    \dim(R_{\p_{n-1}})+1\le \dim(R)$. In $R_{\p_{n-1}}[X]$ we have a
    strict chain
    \[
    P_0R_{\p_{n-1}} \subsetneq P_1R_{\p_{n-1}} \subsetneq \cdots
    \subsetneq P_{n-1}R_{\p_{n-1}}
    \] 
    and thus 
    \[
    n-1 \le \dim(R_{\p_{n-1}}[X])\le \dim(R).
    \]
    Thus $n \le\dim(R) +1$ and so  $\dim(R[X]) = \dim(R) + 1$.
    
    Case 2: Suppose that $\p_{n-1} = \p_n$. By Lemma~\ref{L:FT1}, we
    have $P_{n-1} = \p_{n-1}[X]$.  By Lemma~\ref{L:FT2}, we have
    $\h(\p_{n-1}[X])=\h(\p_{n-1})$.  Thus
    \[
    \dim(R) \ge \h(\p_{n-1}) = \h(P_{n-1}) \ge n-1.
    \]
    Thus $n\le \dim(R) + 1$ and so we see that $\dim(R[X]) = \dim(R) +1$.
  \end{proof}
\end{theorem}


\begin{exercise}
  If $R$ is Noetherian, show that
  \[
  \dim(R\lps X \rps) = \dim(R) + 1
  \]
  Hint: Does every maximal ideal in $R\lps X \rps$ contain $X$?
\end{exercise}


\begin{corollary}
  We have that if $k$ is a field, then
  \begin{align*}
    \dim(k[X_1,\ldots,X_n]) &= \dim(k\lps X_1,\dots,X_n\rps) = n, \\
    \dim(\Z[X_1,\ldots,X_n]) &= \dim(\Z\lps X_1,\dots,X_n\rps) = n+1, \\
    \dim(\Z_{(p)}[X_1,\ldots,X_n]) &= \dim(\Z_{(p)}\lps X_1,\dots,X_n\rps) = n+1.
  \end{align*}
\end{corollary}


Now let's revisit Noether normalization.

\begin{corollary}
  Let $k$ be an infinite field and $A$ be a finitely generated
  $k$-algebra. There exist $\alpha_1,\dots,\alpha_m\in A$ such that
  \[
  \dim(k[\alpha_1,\dots,\alpha_m]) = \dim(A) = m.
  \]
\end{corollary}





\end{document}
