\documentclass{ximera}



\usepackage{tikz-cd}
\usepackage[sans]{dsfont}

\let\oldbibliography\thebibliography%% to compact bib
\renewcommand{\thebibliography}[1]{%
  \oldbibliography{#1}%
  \setlength{\itemsep}{0pt}%
}


\DefineVerbatimEnvironment{macaulay2}{Verbatim}{numbers=left,frame=lines,label=Macaulay2,labelposition=topline}

%%% This next bit of code defines all our theorem environments
\makeatletter
\let\c@theorem\relax
\let\c@corollary\relax
\makeatother

\let\definition\relax
\let\enddefinition\relax

\let\theorem\relax
\let\endtheorem\relax

\let\proposition\relax
\let\endproposition\relax

\let\exercise\relax
\let\endexercise\relax

\let\question\relax
\let\endquestion\relax

\let\remark\relax
\let\endremark\relax

\let\corollary\relax
\let\endcorollary\relax


\let\example\relax
\let\endexample\relax

\let\warning\relax
\let\endwarning\relax

\let\lemma\relax
\let\endlemma\relax

\newtheoremstyle{SlantTheorem}{\topsep}{\topsep}%%% space between body and thm
		{\slshape}                      %%% Thm body font
		{}                              %%% Indent amount (empty = no indent)
		{\bfseries\sffamily}            %%% Thm head font
		{}                              %%% Punctuation after thm head
		{3ex}                           %%% Space after thm head
		{\thmname{#1}\thmnumber{ #2}\thmnote{ \bfseries(#3)}}%%% Thm head spec
\theoremstyle{SlantTheorem}
\newtheorem{theorem}{Theorem}
\newtheorem{definition}[theorem]{Definition}
\newtheorem{proposition}[theorem]{Proposition}
%% \newtheorem*{dfnn}{Definition}
%% \newtheorem{ques}{Question}[theorem]
\newtheorem{lemma}[theorem]{Lemma}
%% \newtheorem*{war}{WARNING}
%% \newtheorem*{cor}{Corollary}
%% \newtheorem*{eg}{Example}
\newtheorem*{remark}{Remark}
\newtheorem*{touchstone}{Touchstone}
\newtheorem{corollary}{Corollary}[theorem]
\newtheorem*{example}{Example}
\newtheorem*{warning}{WARNING}


\newtheoremstyle{Exercise}{\topsep}{\topsep} %%% space between body and thm
		{}                           %%% Thm body font
		{}                           %%% Indent amount (empty = no indent)
		{\bfseries}                  %%% Thm head font
		{)}                          %%% Punctuation after thm head
		{ }                          %%% Space after thm head
		{\thmnumber{#2}\thmnote{ \bfseries(#3)}}%%% Thm head spec
\theoremstyle{Exercise}
\newtheorem{exercise}{}[theorem]

%% \newtheoremstyle{Question}{\topsep}{\topsep} %%% space between body and thm
%% 		{\bfseries}                  %%% Thm body font
%% 		{3ex}                        %%% Indent amount (empty = no indent)
%% 		{}                           %%% Thm head font
%% 		{}                           %%% Punctuation after thm head
%% 		{}                           %%% Space after thm head
%% 		{\thmnumber{#2}\thmnote{ \bfseries(#3)}}%%% Thm head spec
\newtheoremstyle{Question}{3em}{3em} %%% space between body and thm
		{\large\bfseries}                           %%% Thm body font
		{3ex}                           %%% Indent amount (empty = no indent)
		{\bfseries}                  %%% Thm head font
		{}                          %%% Punctuation after thm head
		{ }                          %%% Space after thm head
		{}%%% Thm head spec
\theoremstyle{Question}
\newtheorem*{question}{}



\renewcommand{\tilde}{\widetilde}
\renewcommand{\bar}{\overline}
\renewcommand{\hat}{\widehat}
\newcommand{\N}{\mathbb N}
\newcommand{\Z}{\mathbb Z}
\newcommand{\R}{\mathbb R}
\newcommand{\Q}{\mathbb Q}
\newcommand{\C}{\mathbb C}
\newcommand{\V}{\mathbb V}
\newcommand{\I}{\mathbb I}
\newcommand{\A}{\mathbb A}
\newcommand{\iso}{\simeq}
\newcommand{\ph}{\varphi}
\newcommand{\Cf}{\mathcal{C}}
\newcommand{\IZ}{\mathrm{Int}(\Z)}
\newcommand{\dsum}{\oplus}
\newcommand{\directsum}{\bigoplus}
\newcommand{\union}{\bigcup}
\renewcommand{\i}{\mathfrak}
\renewcommand{\a}{\mathfrak{a}}
\renewcommand{\b}{\mathfrak{b}}
\newcommand{\m}{\mathfrak{m}}
\newcommand{\p}{\mathfrak{p}}
\newcommand{\q}{\mathfrak{q}}
\newcommand{\dfn}[1]{\textbf{#1}\index{#1}}
\let\hom\relax
\DeclareMathOperator{\ann}{Ann}
\DeclareMathOperator{\h}{ht}
\DeclareMathOperator{\hom}{Hom}
\DeclareMathOperator{\Span}{Span}
\DeclareMathOperator{\spec}{Spec}
\DeclareMathOperator{\maxspec}{MaxSpec}
\DeclareMathOperator{\supp}{Supp}
\DeclareMathOperator{\ass}{Ass}
\DeclareMathOperator{\ff}{Frac}
\DeclareMathOperator{\im}{Im}
\DeclareMathOperator{\syz}{Syz}
\DeclareMathOperator{\gr}{Gr}
\renewcommand{\ker}{\mathop{\mathrm{Ker}}\nolimits}
\newcommand{\coker}{\mathop{\mathrm{Coker}}\nolimits}
\newcommand{\lps}{[\hspace{-0.25ex}[}
\newcommand{\rps}{]\hspace{-0.25ex}]}
\newcommand{\into}{\hookrightarrow}
\newcommand{\onto}{\twoheadrightarrow}
\newcommand{\tensor}{\otimes}
\newcommand{\x}{\mathbf{x}}
\newcommand{\X}{\mathbf X}
\newcommand{\Y}{\mathbf Y}
\renewcommand{\k}{\boldsymbol{\kappa}}
\renewcommand{\emptyset}{\varnothing}
\renewcommand{\qedsymbol}{$\blacksquare$}
\renewcommand{\l}{\ell}
\newcommand{\1}{\mathds{1}}
\newcommand{\lto}{\mathop{\longrightarrow\,}\limits}
\newcommand{\rad}{\sqrt}
\newcommand{\hf}{H}
\newcommand{\hs}{H\!S}
\newcommand{\hp}{H\!P}
\renewcommand{\vec}{\mathbf}
\renewcommand{\phi}{\varphi}
\renewcommand{\epsilon}{\varepsilon}
\renewcommand{\subset}{\subseteq}
\renewcommand{\supset}{\supseteq}
\newcommand{\macaulay}{\textsl{Macaulay2}}
\newcommand{\invlim}{\varprojlim}


%\renewcommand{\proofname}{Sketch of Proof}


\renewenvironment{proof}[1][Proof]
  {\begin{trivlist}\item[\hskip \labelsep \itshape \bfseries #1{}\hspace{2ex}]\upshape}
{\qed\end{trivlist}}

\newenvironment{sketch}[1][Sketch of Proof]
  {\begin{trivlist}\item[\hskip \labelsep \itshape \bfseries #1{}\hspace{2ex}]\upshape}
{\qed\end{trivlist}}



\makeatletter
\renewcommand\section{\@startsection{paragraph}{10}{\z@}%
                                     {-3.25ex\@plus -1ex \@minus -.2ex}%
                                     {1.5ex \@plus .2ex}%
                                     {\normalfont\large\sffamily\bfseries}}
\renewcommand\subsection{\@startsection{subparagraph}{10}{\z@}%
                                    {3.25ex \@plus1ex \@minus.2ex}%
                                    {-1em}%
                                    {\normalfont\normalsize\sffamily\bfseries}}
\makeatother

%% Fix weird index/bib issue.
\makeatletter
\gdef\ttl@savemark{\sectionmark{}}
\makeatother


\makeatletter
%% no number for refs
\newcommand\frontstyle{%
  \def\activitystyle{activity-chapter}
  \def\maketitle{%
    \addtocounter{titlenumber}{1}%
                    {\flushleft\small\sffamily\bfseries\@pretitle\par\vspace{-1.5em}}%
                    {\flushleft\LARGE\sffamily\bfseries\@title \par }%
                    {\vskip .6em\noindent\textit\theabstract\setcounter{problem}{0}\setcounter{sectiontitlenumber}{0}}%
                    \par\vspace{2em}
                    \phantomsection\addcontentsline{toc}{section}{\textbf{\@title}}%
                  }}
\makeatother


\author{Bart Snapp}

\title{The dimension theorem}

\begin{document}
\begin{abstract}
  We prove the dimension theorem. Sources and
  references: \cite{sD2008}.
\end{abstract}
\maketitle



\begin{theorem}[The Dimension Theorem]\label{T:dimension}\index{Dimension Theorem} Let $A$ be a local ring and $M$ a finitely generated $A$-module. Then
\[
\dim(M) = d(M) = s(M). 
\]
\end{theorem}

\begin{proof} We will show $\dim(M)\le d(M) \le s(M) \le \dim(M)$. 


\paragraph{$\boldsymbol{\dim(M)\le d(M)}$} Proceeding by induction on $d(M)$. Suppose that $d(M) = 0$. Then 
\[
d(M) = P_\m(M,n) = \l(M/\m^n M)
\]
is constant for $n$ sufficiently large. Thus for large enough $n$, 
\[
\l(M/\m^n M) = \l(M/\m^{n+1}M)
\]
which shows us that $\l(\m^n M/\m^{n+1}M) = 0$. Hence $\m^n M/\m^{n+1}M = 0$. So, 
\[
\m^n M = \m^{n+1}M = \m(\m^n M)
\]
and hence by Nakayama's Lemma, Corollary~\ref{C:NAK},\index{Nakayama's Lemma} $\m^n M =0$ and hence $\m^n\subset\ann_A(M)$. Since
\[
\dim(M) = \dim(A/\ann(M))  = \dim(A/\sqrt{\ann(M)}) = \dim(A/\m),
\]   
we see that $\dim(M) = 0$. 

Now assume $d(M) = n> 0$. Take any maximal chain 
\[
\p_0\subsetneq \p_1 \subsetneq \cdots \subsetneq \p_m
\]
in $\supp(M)$.  We need to show $m\le n$. We know that $\p_0$ is a minimal element of $\supp(M)$ and hence is a minimal element in $\ass_A(M)$.  So we have an injection
\[
A/\p_0\into M.
\]
Set $N = A/\p_0$.  By Corollary~\ref{C:add}, $d(N) \le d(M)$.  So it suffices to show that $m \le d(N)$.  Let $x \in \p_1 - \p_0$.  Consider the short exact sequence
\[
0\lto N \lto^{x} N \lto N/xN \lto 0.
\]
By Proposition~\ref{P:exactHilbert}, we have that
\[P_\m(N,n) = P_\m(N,n) + P_\m(N/xN,n) - R(n),\]
where $\deg(R(n)) < \deg(P\m(N,n)) = d(N)$.  But then
\[P_\m(N/xN,n) = R(n).\]
Thus $d(N/xN) = \deg(P_\m(M,n)) = \deg(R(n)) < n$.  By the inductive hypothesis, $\dim(N/xN) \le d(N/xN)$.  Since
\[\p_1 \subsetneq \cdots \subsetneq \p_m\]
is a strict chain of prime ideals in $\supp(N/xN)$, we have that
\[m - 1 \le \dim(N/xN) \le d(N/xN) \le n - 1.\]
Hence $\dim(M) = m \le n = d(M)$.

\paragraph{$\boldsymbol{d(M)\le s(M)}$} If $n =s(M)$, consider $x_1,\dots,x_n$ a system of parameters for $M$. In this case we have by definition that 
\[
\l_A(M/(x_1,\dots,x_n)M)<\infty.
\] 
Consider $\a = (x_1,\dots,x_n)$. Now if we consider the image of $\a$ in $M/\ann_A(M)$, we have that
\[
\bar{\a} = (\bar{x}_1, \dots, \bar{x}_n)
\]
where $\bar{x}_i$ is the image of $x_i\in M/\ann_A(M)$. By Proposition~\ref{P:dep}, $d(M)$ depends only on $M$ and
\[
\supp(M/\m M) =\supp(M/\a M),
\]
and thus we see that 
\[
d(M) = \deg(P_\m(M,n)) = \deg(P_\a(M,n))\le n,
\]
using that $\a$ is generated by $n$ elements and applying Corollary~\ref{C:Hilbertprops}.



\paragraph{$\boldsymbol{s(M)\le\dim(M)}$} Proceed by induction on $\dim(M)$. Write
\begin{align*}
\dim(M) = 0 &\Leftrightarrow \ass_A(M)=\{\m\}, \\
&\Leftrightarrow \l_A(M) <\infty,
\end{align*}
and so we see $s(M) = 0$.

Now assume that $\dim(M)=n$. Let $\p_i$ be the minimal prime ideals in $\supp(M)$. By Corollary~\ref{C:SA}, these primes are also minimal in $\ass(M)$ so there are only finitely many such primes. By the \index{prime avoidance}Prime Avoidance Lemma, Lemma~\ref{L:PA}, we may pick $x\in\m$ such that $x$ is not in any of these minimal primes. Thus
\[
\dim(M/xM) <\dim(M)
\]
and thus by induction, $s(M/xM)\le\dim(M/xM)$. But $s(M)-1\le s(M/xM)$ and so
\[
s(m) \le s(M/xM) +1 \le \dim(M/xM) +1 \le \dim(M).
\]

Putting the above steps together we have shown 
\[
\dim(M)\le d(M) \le s(M) \le \dim(M),
\]
and hence $\dim(M) = d(M) = s(M)$.
\end{proof}


\begin{corollary} If $(A,\m)$ is a local ring, then it is has finite dimension.
\end{corollary}

\begin{proof} Since $A$ is Noetherian, $\m$ is finitely generated, thus $s(A)=\dim(A)$ is less than or equal to the number of generators of $\m$.
\end{proof}


\begin{corollary} If $(A,\m)$ is a local ring and $M$ is a finitely generated $A$-module, then 
\[
\dim_A(M) =\dim_{\hat{A}}(\hat{M}).
\]
\end{corollary}

\begin{proof} We know from Proposition~\ref{P:ComQuot}
\[
M/\m^n M \iso \hat{M}/\hat{\m^n} \hat{M} \iso \hat{M}/ \hat{\m}^n\hat{M},
\]
and so $P_\m(M,n) = P_{\hat{\m}}(\hat{M},n)$.
\end{proof}


\begin{corollary} If $(A,\m)$ is a local ring, then \index{primary}
\[
\dim(A) = \min\{i: (a_1,\dots,a_i) = \a\text{ where $\a$ is $\m$-primary}\}.
\]
\end{corollary}

\begin{proof} Let $\dim(A) = n$, then there exists $x_1,\dots,x_n$ a system of parameters of $A$ such that $\l(A/\x) <\infty$. Thus $(x_1,\dots,x_n)$ is $\m$-primary.  Since $s(A) = n$, we see that we cannot obtain $y_1,\dots,y_t$ with $t<n$ such that $\l(A/\mathbf{y})<\infty$.  So for any $y_1,\ldots,y_t$ with $t < n$, $(y_1,\ldots,y_t)$ is not $\m$-primary.
\end{proof}


\begin{corollary} If $A$ is a Noetherian ring, not necessarily local, consider any decreasing chain of prime ideals in $A$ 
\[
\p_0\supsetneq \p_1\supsetneq \cdots \supsetneq \p_i\supsetneq \cdots
\]
Then there exists $n$ such that $\p_n = \p_{n+1} = \cdots$.
\end{corollary}

\begin{proof} If we localize at $\p_0$, then $\dim(A_{\p_0}) <\infty$.  So for some $n$
\[
\p_n A_{\p_0} = \p_{n+1} A_{\p_0} = \cdots
\]
and so $\p_n = \p_{n+1} = \cdots$.
\end{proof}

\begin{definition}\index{height}\index{htp@$\h(\p)$} If $A$ is a Noetherian ring and $\p$ is a prime ideal of $A$, then the \textbf{height} of $\p$ is 
\[
\h(\p) = \sup\left\{n : \begin{minipage}{15em}there exists a chain of prime ideals $\p_0\subsetneq \p_1 \subsetneq \cdots \subsetneq \p_{n-1}\subsetneq \p_n =\p$\end{minipage} \right\}.
\]
\end{definition}

\begin{remark} Note that $\h(\p) = \dim(A_\p)$.
\end{remark}

\begin{definition}\index{height}\index{hti@$\h(I)$} If $A$ is a Noetherian ring and $I$ is any ideal of $A$, then the \textbf{height} of $I$ is 
\begin{align*}
\h(I) &= \inf\{\h(\p): I\subset\p\} \\
&= \inf\{\h(\p): \p \text{ is minimal in $\ass_A(A/I)$}\}.
\end{align*}
\end{definition}

\begin{definition}\index{coheight}\index{coht@$\mathrm{coht}(I)$} If $A$ is a Noetherian ring and $I$ is any ideal, then the \textbf{coheight}, denoted $\mathrm{coht}(I)$ is defined as 
\[
\mathrm{coht}(I) = \dim(A/I).
\]
\end{definition}

\begin{exercise} Check that for any ideal $I$, $\h(I)+\dim(A/I)\le\dim(A)$. 
\end{exercise}

\begin{warning}
Even if $I$ is a prime ideal, the above inequality may be strict.
\end{warning}

\begin{corollary}[Krull's Ideal Theorem]\label{C:KIT}\index{Krull's Ideal Theorem} Let $A$ be a Noetherian ring and $\p$ is a prime ideal.  Then $\h(\p)\le n$ if and only if there exist $a_1,\dots,a_n\in\p$ such that $\p$ is a minimal prime containing $(a_1,\dots,a_n)$.
\end{corollary}

\begin{proof} Since $\h(\p) = \dim(A_\p)$, $\dim(A_\p)\le n$ if and only if there exists 
\[
\frac{x_1}{u_1}, \dots, \frac{x_n}{u_n}
\]
in $A_\p$ such that 
\[
\left(\frac{x_1}{u_1}, \dots, \frac{x_n}{u_n}\right)
\]
is $\p A_\p$-primary. This is the case if and only if there exist $x_1,\dots, x_n\in \p$ and $\p$ contains $(x_1,\dots,x_n)$ minimally. 
\end{proof}


\begin{corollary}[Krull's Principal Ideal Theorem]\index{Krull's Principal Ideal Theorem} Let $A$ be a Noetherian ring and $x$ be an element of $A$ which is not a unit or a zerodivisor. Then every prime which contains $(x)$ minimally has height $1$.
\end{corollary}

\begin{proof} By Corollary~\ref{C:KIT} the minimal prime containing $(x)$ has height at most one. The prime $\p$ containing $(x)$ cannot have height zero, as then $\p\in\ass(A)$ by Corollary~\ref{C:SA} and hence then every element of $\p$ is a zero divisor, which is not the case.
\end{proof}

\begin{remark} The above corollary to the Dimension Theorem is sometimes called \textbf{Krull's Hauptidealsatz}.\index{Krull's Hauptidealsatz}
\end{remark}

\begin{corollary}\label{C:nzdd-1} Let $(A,\m)$ be a local ring and $x$ be an element of $\m$ which is not a zerodivisor. Then $\dim(A/xA) = \dim(A)-1$.
\end{corollary}

\begin{proof} This follows from the previous corollary and the definition of dimension.
\end{proof}


\begin{exercise} If $(A,\m)$ is a local ring and $M$ is a finitely generated $A$-module with $x_1,\dots,x_i\in\m$, then 
\[
\dim(M/(x_1,\dots,x_i)M)\ge \dim(M) -i.
\]
Equality holds if and only if $x_1,\dots,x_i$ form part of a system of parameters for $M$.
\end{exercise}

\begin{exercise} Let $A$ be a Noetherian ring of dimension at least $2$. Then $A$ has infinitely many prime ideals of height $1$.
\end{exercise}

\begin{example} A ring $A$ is Artinian if and only if $\dim(A) = 0$. \index{Artinian!ring}
\end{example}

\begin{example} A PID has dimension one.
\end{example}

\begin{example} $\dim(\Z) = 1$.
\end{example}

\begin{example} If $k$ is a field, then $\dim(k[X]) = 1$.
\end{example}

\begin{example} If $k$ is a field, then $\dim(k\lps x \rps) = 1$.
\end{example}


\begin{lemma}\label{L:FT1} If $A$ is a ring and $P_1\subsetneq P_2$ are two prime ideals of $A[X]$ such that
\[
P_1\cap A = P_2\cap A = \p,
\]
then $P_1 = \p[X]$.
\end{lemma}

\begin{proof} Suppose not. Then 
\[
\p[X]\subsetneq P_1\subsetneq P_2 
\]
and so 
\[
(0) \subsetneq (P_1/\p)[X] \subsetneq (P_2/\p)[X]
\]
in $(A/\p) [X]$. Set $U = A-\p$.  Since $P_1 \cap U = P_2 \cap U = \emptyset$, and since localizations are exact, we have
\[
(0) \subsetneq U^{-1}(P_1/\p)[X] \subsetneq U^{-1}(P_2/\p)[X]
\]
is a strict chain of prime ideals in $(A_\p/\p A_\p)[X] = U^{-1}(A/\p)[X]$.  But this contradicts that $\dim(k[X]) = 1$ when $k$ is a field as $k[X]$ is a PID, hence all primes are principal and so are of height one or zero.  Thus $P_1 = \p[X]$.
\end{proof}


\begin{lemma}\label{L:FT2} If $A$ is Noetherian and $I$ is an ideal of $A$, then
\[
\h(I) = \h(I\cdot A[X]).
\]
\end{lemma}

\begin{proof} By the Primary Decomposition Theorem, Theorem~\ref{T:PDT}, 
\[
\ass\left(\frac{A[X]}{IA[X]}\right) = \{\p_i[X]: \p_i\in\ass(A/I)\},
\]
as $A/I$ injects into $A/I [X]$. So it is enough to show that if $\p$ is a prime ideal in $A$, then 
\[
\h(\p) = \h(\p[X]).
\]
Suppose that $\h(\p) = n$. Then there exist $a_1,\dots,a_n\in \p$ such that $\p\supset(a_1,\dots,a_n)$ minimally. By the Primary Decomposition Theorem, Theorem~\ref{T:PDT}, we see that $\p[X]\supset(a_1,\dots,a_n)[X]$ minimally. Thus $\h(\p[X])\le n$.

On the other hand, if 
\[
\p_0 \subsetneq \p_1 \subsetneq \cdots \subsetneq \p_n = \p
\] 
is a chain of prime ideals where $\h(\p) = n$, then 
\[
\p_0[X] \subsetneq \p_1[X] \subsetneq \cdots \subsetneq \p_n[X] = \p[X]
\]
is a chain of prime ideals in $A[X]$. Thus $\h(\p[X])\ge \h(\p)$ and so we see that $\h(\p[X]) = \h(\p)$. 
\end{proof}


\begin{theorem} If $A$ is a Noetherian ring, $\dim(A[X]) = \dim(A) + 1$.
\end{theorem}

\begin{proof} First note that given a chain of primes $\p_0 \subsetneq \cdots \subsetneq \p_n$ in $A$, we can construct the chain of primes $\p_0 A[X] \subsetneq \cdots \subsetneq \p_n A[X] \subsetneq \p_n A[X] + x A[X]$ in $A[X]$.  It is then clear that $\dim(A[X]) \ge \dim(A) + 1$, so it is enough to show $\dim(A[X]) \le\dim(A) + 1$. If $\dim(A) = \infty$, then there is nothing to prove. We will proceed by induction on the dimension of $A$. If $\dim(A) = 0$, then for $P_i$ a prime ideal in $A[X]$ write $\p_i = P_i\cap A$.  So if 
\[
P_0\subsetneq P_1\subsetneq \cdots \subsetneq P_n
\]
then since $\dim(A) = 0$, we have $\p_0 = \p_1 = \cdots = \p_n$. Thus by Lemma~\ref{L:FT1}, we have 
\[
\p_0[x] = P_0 = P_1 = \cdots = P_{n-1}\subsetneq P_n.
\]
Thus $n\le 1$ and so we see that $\dim(A[X]) = 1$.

Now suppose that $\dim(A) = n >0$ and let $P_0 \subsetneq P_1 \subsetneq \cdots \subsetneq P_n$ be a strict chain of prime ideals in $A$.  Set $\p_i = P_i \cap A$.  

Case 1: Suppose that $\p_{n-1} \subsetneq \p_n$. Then
\[
\dim(A_{\p_{n-1}})< \dim(A).
\]
By induction we then have that $\dim(A_{\p_{n-1}}[X]) = \dim(A_{\p_{n-1}})+1\le \dim(A)$. In $A_{\p_{n-1}}[X]$ we have a strict chain
\[
P_0A_{p_{n-1}} \subsetneq P_1A_{p_{n-1}} \subsetneq \cdots \subsetneq P_{n-1}A_{p_{n-1}}
\] 
and thus 
\[
n-1 \le \dim(A_{\p_{n-1}})[X]\le \dim(A). 
\]
Thus $n \le\dim(A) +1$ and so  $\dim(A[X]) = \dim(A) + 1$.

Case 2: Suppose that $\p_{n-1} = \p_n$. By Lemma~\ref{L:FT1}, we have $P_{n-1} = \p_{n-1}[X]$.  By Lemma~\ref{L:FT2}, we have $\h(\p_{n-1}[X])=\h(\p_{n-1})$.  Thus
\[
\dim(A) \ge \h(\p_{n-1}) = \h(P_{n-1}) \ge n-1.
\]
Thus $n\le \dim(A) + 1$ and so we see that $\dim(A[X]) = \dim(A) +1$.
\end{proof}

\begin{exercise} If $A$ is Noetherian, show that 
\[
\dim(A\lps X \rps) = \dim(A) + 1
\]
Hint: Does every maximal ideal in $A\lps X \rps$ contain $X$?
\end{exercise}


\begin{corollary} We have that if $k$ is a field, then
\begin{align*}
\dim(k[X_1,\ldots,X_n]) &= \dim(k\lps X_1,\dots,X_n\rps) = n, \\
\dim(\Z[X_1,\ldots,X_n]) &= \dim(\Z\lps X_1,\dots,X_n\rps) = n+1, \\
\dim(\Z_{(p)}[X_1,\ldots,X_n]) &= \dim(\Z_{(p)}\lps X_1,\dots,X_n\rps) = n+1.
\end{align*}
\end{corollary}



\end{document}
