\documentclass{ximera}



\usepackage{tikz-cd}
\usepackage[sans]{dsfont}

\let\oldbibliography\thebibliography%% to compact bib
\renewcommand{\thebibliography}[1]{%
  \oldbibliography{#1}%
  \setlength{\itemsep}{0pt}%
}


\DefineVerbatimEnvironment{macaulay2}{Verbatim}{numbers=left,frame=lines,label=Macaulay2,labelposition=topline}

%%% This next bit of code defines all our theorem environments
\makeatletter
\let\c@theorem\relax
\let\c@corollary\relax
\makeatother

\let\definition\relax
\let\enddefinition\relax

\let\theorem\relax
\let\endtheorem\relax

\let\proposition\relax
\let\endproposition\relax

\let\exercise\relax
\let\endexercise\relax

\let\question\relax
\let\endquestion\relax

\let\remark\relax
\let\endremark\relax

\let\corollary\relax
\let\endcorollary\relax


\let\example\relax
\let\endexample\relax

\let\warning\relax
\let\endwarning\relax

\let\lemma\relax
\let\endlemma\relax

\newtheoremstyle{SlantTheorem}{\topsep}{\topsep}%%% space between body and thm
		{\slshape}                      %%% Thm body font
		{}                              %%% Indent amount (empty = no indent)
		{\bfseries\sffamily}            %%% Thm head font
		{}                              %%% Punctuation after thm head
		{3ex}                           %%% Space after thm head
		{\thmname{#1}\thmnumber{ #2}\thmnote{ \bfseries(#3)}}%%% Thm head spec
\theoremstyle{SlantTheorem}
\newtheorem{theorem}{Theorem}
\newtheorem{definition}[theorem]{Definition}
\newtheorem{proposition}[theorem]{Proposition}
%% \newtheorem*{dfnn}{Definition}
%% \newtheorem{ques}{Question}[theorem]
\newtheorem{lemma}[theorem]{Lemma}
%% \newtheorem*{war}{WARNING}
%% \newtheorem*{cor}{Corollary}
%% \newtheorem*{eg}{Example}
\newtheorem*{remark}{Remark}
\newtheorem*{touchstone}{Touchstone}
\newtheorem{corollary}{Corollary}[theorem]
\newtheorem*{example}{Example}
\newtheorem*{warning}{WARNING}


\newtheoremstyle{Exercise}{\topsep}{\topsep} %%% space between body and thm
		{}                           %%% Thm body font
		{}                           %%% Indent amount (empty = no indent)
		{\bfseries}                  %%% Thm head font
		{)}                          %%% Punctuation after thm head
		{ }                          %%% Space after thm head
		{\thmnumber{#2}\thmnote{ \bfseries(#3)}}%%% Thm head spec
\theoremstyle{Exercise}
\newtheorem{exercise}{}[theorem]

%% \newtheoremstyle{Question}{\topsep}{\topsep} %%% space between body and thm
%% 		{\bfseries}                  %%% Thm body font
%% 		{3ex}                        %%% Indent amount (empty = no indent)
%% 		{}                           %%% Thm head font
%% 		{}                           %%% Punctuation after thm head
%% 		{}                           %%% Space after thm head
%% 		{\thmnumber{#2}\thmnote{ \bfseries(#3)}}%%% Thm head spec
\newtheoremstyle{Question}{3em}{3em} %%% space between body and thm
		{\large\bfseries}                           %%% Thm body font
		{3ex}                           %%% Indent amount (empty = no indent)
		{\bfseries}                  %%% Thm head font
		{}                          %%% Punctuation after thm head
		{ }                          %%% Space after thm head
		{}%%% Thm head spec
\theoremstyle{Question}
\newtheorem*{question}{}



\renewcommand{\tilde}{\widetilde}
\renewcommand{\bar}{\overline}
\renewcommand{\hat}{\widehat}
\newcommand{\N}{\mathbb N}
\newcommand{\Z}{\mathbb Z}
\newcommand{\R}{\mathbb R}
\newcommand{\Q}{\mathbb Q}
\newcommand{\C}{\mathbb C}
\newcommand{\V}{\mathbb V}
\newcommand{\I}{\mathbb I}
\newcommand{\A}{\mathbb A}
\newcommand{\iso}{\simeq}
\newcommand{\ph}{\varphi}
\newcommand{\Cf}{\mathcal{C}}
\newcommand{\IZ}{\mathrm{Int}(\Z)}
\newcommand{\dsum}{\oplus}
\newcommand{\directsum}{\bigoplus}
\newcommand{\union}{\bigcup}
\renewcommand{\i}{\mathfrak}
\renewcommand{\a}{\mathfrak{a}}
\renewcommand{\b}{\mathfrak{b}}
\newcommand{\m}{\mathfrak{m}}
\newcommand{\p}{\mathfrak{p}}
\newcommand{\q}{\mathfrak{q}}
\newcommand{\dfn}[1]{\textbf{#1}\index{#1}}
\let\hom\relax
\DeclareMathOperator{\ann}{Ann}
\DeclareMathOperator{\h}{ht}
\DeclareMathOperator{\hom}{Hom}
\DeclareMathOperator{\Span}{Span}
\DeclareMathOperator{\spec}{Spec}
\DeclareMathOperator{\maxspec}{MaxSpec}
\DeclareMathOperator{\supp}{Supp}
\DeclareMathOperator{\ass}{Ass}
\DeclareMathOperator{\ff}{Frac}
\DeclareMathOperator{\im}{Im}
\DeclareMathOperator{\syz}{Syz}
\DeclareMathOperator{\gr}{Gr}
\renewcommand{\ker}{\mathop{\mathrm{Ker}}\nolimits}
\newcommand{\coker}{\mathop{\mathrm{Coker}}\nolimits}
\newcommand{\lps}{[\hspace{-0.25ex}[}
\newcommand{\rps}{]\hspace{-0.25ex}]}
\newcommand{\into}{\hookrightarrow}
\newcommand{\onto}{\twoheadrightarrow}
\newcommand{\tensor}{\otimes}
\newcommand{\x}{\mathbf{x}}
\newcommand{\X}{\mathbf X}
\newcommand{\Y}{\mathbf Y}
\renewcommand{\k}{\boldsymbol{\kappa}}
\renewcommand{\emptyset}{\varnothing}
\renewcommand{\qedsymbol}{$\blacksquare$}
\renewcommand{\l}{\ell}
\newcommand{\1}{\mathds{1}}
\newcommand{\lto}{\mathop{\longrightarrow\,}\limits}
\newcommand{\rad}{\sqrt}
\newcommand{\hf}{H}
\newcommand{\hs}{H\!S}
\newcommand{\hp}{H\!P}
\renewcommand{\vec}{\mathbf}
\renewcommand{\phi}{\varphi}
\renewcommand{\epsilon}{\varepsilon}
\renewcommand{\subset}{\subseteq}
\renewcommand{\supset}{\supseteq}
\newcommand{\macaulay}{\textsl{Macaulay2}}
\newcommand{\invlim}{\varprojlim}


%\renewcommand{\proofname}{Sketch of Proof}


\renewenvironment{proof}[1][Proof]
  {\begin{trivlist}\item[\hskip \labelsep \itshape \bfseries #1{}\hspace{2ex}]\upshape}
{\qed\end{trivlist}}

\newenvironment{sketch}[1][Sketch of Proof]
  {\begin{trivlist}\item[\hskip \labelsep \itshape \bfseries #1{}\hspace{2ex}]\upshape}
{\qed\end{trivlist}}



\makeatletter
\renewcommand\section{\@startsection{paragraph}{10}{\z@}%
                                     {-3.25ex\@plus -1ex \@minus -.2ex}%
                                     {1.5ex \@plus .2ex}%
                                     {\normalfont\large\sffamily\bfseries}}
\renewcommand\subsection{\@startsection{subparagraph}{10}{\z@}%
                                    {3.25ex \@plus1ex \@minus.2ex}%
                                    {-1em}%
                                    {\normalfont\normalsize\sffamily\bfseries}}
\makeatother

%% Fix weird index/bib issue.
\makeatletter
\gdef\ttl@savemark{\sectionmark{}}
\makeatother


\makeatletter
%% no number for refs
\newcommand\frontstyle{%
  \def\activitystyle{activity-chapter}
  \def\maketitle{%
    \addtocounter{titlenumber}{1}%
                    {\flushleft\small\sffamily\bfseries\@pretitle\par\vspace{-1.5em}}%
                    {\flushleft\LARGE\sffamily\bfseries\@title \par }%
                    {\vskip .6em\noindent\textit\theabstract\setcounter{problem}{0}\setcounter{sectiontitlenumber}{0}}%
                    \par\vspace{2em}
                    \phantomsection\addcontentsline{toc}{section}{\textbf{\@title}}%
                  }}
\makeatother


\author{Bart Snapp}

\title{An approach based on length}

\begin{document}
\begin{abstract}
  We introduce the Hilbert Function, Hilbert Series, Hilbert
  Polynomial, and the Hilbert-Samuel Polynomial. Sources and
  references: \cite{AK2012,rA2006,hM1986,hS2003}.
\end{abstract}
\maketitle


This approach generalizes the idea of vector-space dimension to modules. 

\section{Length of a module}





\begin{definition}\index{Jordan-H\"older chain}\index{composition series}
  A chain of $R$-modules
  \[
  M = M_0 \supsetneq M_1\supsetneq \dots \supsetneq M_n =0
  \]
  is a \dfn{Jordan-H\"older chain}, also known as a \dfn{composition series},
  if for each $i$, $M_i/M_{i+1}\iso R/\m$ for some maximal
  ideal $\m$ in $R$.
\end{definition}


\begin{example}
  Consider the $\Z$-module, $\Z/30\Z$. This has six different
  composition series:
  \begin{align*}
  &\Z/30\Z \supsetneq 2\Z/30\Z \supsetneq 6\Z/30\Z \supsetneq 0\\
  &\Z/30\Z \supsetneq 2\Z/30\Z \supsetneq 10\Z/30\Z \supsetneq 0\\
  &\Z/30\Z \supsetneq 3\Z/30\Z \supsetneq 6\Z/30\Z \supsetneq 0\\
  &\Z/30\Z \supsetneq 3\Z/30\Z \supsetneq 15\Z/30\Z \supsetneq 0\\
  &\Z/30\Z \supsetneq 5\Z/30\Z \supsetneq 10\Z/30\Z \supsetneq 0\\
  &\Z/30\Z \supsetneq 5\Z/30\Z \supsetneq 15\Z/30\Z \supsetneq 0
  \end{align*}
\end{example}



\begin{definition}
  Let $R$ be a ring and $M$ be an $R$-module. The \dfn{length} of $M$
  is defined as
  \[
  \l(M) := \inf \{n: \text{$M$ has a composition series of length $n$}\}.
  \]
\end{definition}

\begin{example}
  Consider the $\Z$-module, $\Z/30\Z$. From our work above we see $\l(\Z/30\Z) = 3$.
\end{example}



\begin{lemma}\label{L:lensubmod}
  Let $R$ be a ring and $M$ an $R$-module. If $N\subset M$ is a
  submodule of $M$, then $\l(N)\le\l(M)$ with equality holding exactly
  when $N=M$.
  \begin{proof}
    Consider a composition series of length $n$
    \[
    M = M_0 \supsetneq M_1 \supsetneq \cdots \supsetneq M_n = 0.
    \]
    Intersect this with $N$ to find
    \begin{equation}\tag{$\bigstar$}\label{Eq:intcompseries}
    N = (N\cap M_0) \supset (N\cap M_1) \supset \cdots \supset (N\cap M_n) = 0.
    \end{equation}
    By an isomorphism theorem
    \begin{align*}
      \frac{N\cap M_i}{N\cap M_{i+1}} &= \frac{N\cap M_i}{(N\cap M_i)\cap M_{i+1}} \\
      &\iso \frac{(N\cap M_i) + M_{i+1}}{M_{i+1}}.
    \end{align*}
    Since for some maximal ideal $\m\subset R$
    \[
    \frac{(N\cap M_i) + M_{i+1}}{M_{i+1}} \subset \frac{M_i}{M_{i+1}}\iso R/\m
    \]
    we see that
    \[
    \frac{N\cap M_i}{N\cap M_{i+1}}  = 0 \quad \text{or}\quad \frac{N\cap M_i}{N\cap M_{i+1}} \iso R/\m.
    \]
    Thus, we can remove redundant terms from (\ref{Eq:intcompseries})
    to see $\l(N)\le \l(M)$.

    Now we will show that if $N\subset M$ and $\l(N) = \l(M)$, then $N
    = M$. By our previous work, this means that for all $i = 0,\dots,
    n$,
    \[
    (N\cap M_i) + M_{i+1} = M_i
    \]
    and working inductively, we conclude $N=M$.
  \end{proof}
\end{lemma}



\begin{theorem}[Jordan-H\"older]
  Let $R$ be a ring and $M$ an $R$-module. Each composition series for
  $M$ has the same length.
  \begin{proof}
    Consider a composition series
    \[
    M = M_0\supsetneq M_1 \supsetneq \cdots \supsetneq M_n = 0.
    \]
    By the definition of length, $\l(M) \le n$. However by
    Lemma~\ref{L:lensubmod},
    \[
    \l(M) > \l(M_0) > \l(M_1) > \dots > \l(M_n) =0
    \]
    and we see that $\l(M) \ge n$. Hence $\l(M) = n$, all composition
    series have the same length.
  \end{proof}
\end{theorem}


\begin{corollary}
  Let $R$ be a ring, an $R$-module $M$ has finite length if and only
  if $M$ is both Artinian and Noetherian.
  \begin{proof}
    $(\Rightarrow)$ Suppose $\l(M)<\infty$. In this case then any
    chain (ascending or descending) must have length less than or
    equal to $\l(M)$.

    $(\Leftarrow)$ Suppose that $M$ is both Artinian and Noetherian.
    We will construct a finite composition series. Set $M_0 =
    M$. Working inductively if $i\ge 1$, use the fact that $M$ is
    Noetherian to choose a maximal proper submodule $M_{i+1}\subsetneq
    M_i$. Since $M$ is Artinian, this process will stabilize.
  \end{proof}
\end{corollary}



\begin{exercise}
  Let $R$ be a Noetherian ring and $M$ be a finitely generated
  $R$-module. The following are equivalent:
  \begin{enumerate}
  \item $\l(M)<\infty$
  \item $\supp(M)\subset\maxspec(R)$
  \item $\ass(M)\subset\maxspec(R)$
  \end{enumerate}
\end{exercise}

\begin{exercise}
  Let $R$ be a Noetherian ring and $M$ be a finitely generated
  $R$-module of finite length. Prove that $\ass(M) = \supp(M)$.
\end{exercise}


%% \begin{theorem}[Akizuki]
%%   A ring $R$ is Artinian if and only if $R$ is Noetherian and $\dim(R)
%%   = 0$. In this case, $\spec(R)$ is finite.
%%   \begin{proof}
%%     $(\Leftarrow)$ If $\dim(R)$ is zero and 
%%   \end{proof}
%% \end{theorem}



\begin{proposition}[Additivity of Length]\index{length!additive}\label{P:AL}
  Given a short exact sequence of $R$-modules
  \[
  0\to M' \to M\to M'' \to 0
  \]
  such that $\l(M)$ is finite, then length is an additive function, that is,
  \[
  \l(M) = \l(M')+\l(M'').
  \]
  In particular, $\l(M)$ is finite if and only if $\l(M')$ and
  $\l(M'')$ are finite.
  \begin{proof}
    Start by noting that $M''\iso M/M'$. We can thus write a
    composition series for $M''$ as:
    \[
    M'' = M_0/M' \supsetneq M_1/M' \supsetneq \cdots \supsetneq M'/M'  = 0.
    \]
    Now we may look at the numerators, and extend to a composition
    series for $M$:
    \[
    M = M_0 \supsetneq M_1 \supsetneq \cdots \supsetneq M' \supsetneq \cdots \supsetneq M'_n = 0.
    \]
    Thus we see $\l(M) = \l(M')+\l(M'')$.
  \end{proof}
\end{proposition}

\begin{corollary}\label{C:Lles}
  Given an exact sequence of $R$-modules
  \[
  0 \to M_0\to M_1\to \cdots \to M_n \to 0
  \]
  we have that
  \[
  \sum_{i=0}^n (-1)^i \l(M_i) = 0.
  \]
\end{corollary}



\begin{proposition}
  Let $k$ be a field and $V$ be a finite dimensional vector
  space. Then
  \[
  \l(V) = \dim_k(V)
  \]
  \begin{proof}
    If $V$ is a finite dimensional vector space, and
    $\{b_1,\dots,b_n\}$ are a basis, then
    \[
    \Span(b_1,\dots,b_n) \subsetneq \dots \subsetneq \Span_k(b_1) \subsetneq 0
    \]
    is a composition series.
  \end{proof}
\end{proposition}


Hence when $R$ is a field, and $M$ is a finite dimensional vector
space the length of $M$ is equal to the vector space dimension of $M$.




\section{The Hilbert function}

Let $R$ be a finitely generated graded commutative algebra over a
field $k$
\[
R = \directsum_{n=0}^\infty R_n
\]
where $R_0 = k$.  In this case, each component $R_n$ is a $k$-vector
space. It makes sense to look at
\[
\dim_k(R_n).
\]
We can generalize this further for graded commutative algebras over an
Artinian ring $R_0$.



\begin{lemma}
  Let $R$ be a finitely generated graded commutative algebra over an
  Artinian ring $R_0$ and let $M$ be a finitely generated graded
  $R$-module.  In this case, $M_n$ is a finitely generated
  $R_0$-module.
  \begin{proof}
    Exercise.
  \end{proof}
\end{lemma}


Thoughout this section we will often be working with $R$, a finitely
generated graded commutative algebra over an Artinian ring $R_0$, with
a finitely generated graded $R$-module $M$. A canonical example of
this situation is
\[
R = k[x,y].
\]
Here $k = R_0$, our Artinian ring, and $x$ and $y$ are the
generators. These generators are of degree $1$ and hence are in
$R_1$. In this case, $R$ is a finitely generated graded commutative
algebra over an Artinian ring $R_0$.

For the module, we should be thinking of something like
\[
M = k[x,y]/(x^2,y^3).
\]
Now, $M$ is a finitely generated $R$-module. The degree $1$ piece,
$M_1$, is generated by $\{x,y\}$. The degree $2$ piece, $M_2$, is
generated by $\{xy,y^2\}$, and the degree $3$ piece is generated by
$\{xy^2\}$. For degrees $n=4,5,\dots$, our degree $n$ pieces are
$0$. We can see this with a computation in \macaulay.

\begin{macaulay2}
R = ZZ/101[x,y]/(x^2,y^3);
basis(0,R)
basis(1,R)
basis(2,R)
basis(3,R)
basis(4,R)
\end{macaulay2}


Now we define the \textit{Hilbert function}.

\begin{definition}
  Let $R$ be a finitely generated graded commutative algebra over an
  Artinian ring $R_0$. Let $M$ be a finitely generated graded
  $R$-module. The \dfn{Hilbert function} for $M$, $\hf_M(n)$ is given by
  \[
  \hf_M(n) := \l(M_n).
  \]
\end{definition}


\begin{example}
If $(A,\m,k)$ is a local ring, then 
\[
\hf_M(n) = \dim_{k}\left(\m^n M/\m^{n+1} M\right)
\]
as we take the associated graded ring and graded module.
\end{example}



We can compute the Hilbert function using \macaulay. We'll start with
a very basic example:

\begin{macaulay2}
R = ZZ/101[x,y,z];
hilbertFunction(1,R)
hilbertFunction(2,R)
hilbertFunction(3,R)
\end{macaulay2}

Let's do something a bit more complex:

\begin{macaulay2}
R = ZZ/101[x,y,z]/(x^3+y^3+z^3);
hilbertFunction(1,R)
hilbertFunction(2,R)
hilbertFunction(3,R)
\end{macaulay2}


\begin{macaulay2}
R = ZZ/101[x,y,z]/(x^2,y^3,z^4);
hilbertFunction(1,R)
hilbertFunction(2,R)
hilbertFunction(3,R)
hilbertFunction(4,R)
hilbertFunction(5,R)
hilbertFunction(6,R)
hilbertFunction(7,R)
dim(R) 
\end{macaulay2}



The generating function for the Hilbert function is called the \textit{Hilbert series}.

\begin{definition}
  Let $R$ be a finitely generated graded commutative algebra over an
  Artinian ring $R_0$. Let $M$ be a finitely generated graded
  $R$-module. The \dfn{Hilbert series} for $M$, $\hs_M(t)$ is given by
  \begin{align*}
    \hs_M(t) &:= \sum_{n=0}^\infty \hf_M(n) t^n,\\
    &:= \sum_{n=0}^\infty \l(M_n) t^n.
  \end{align*}
\end{definition}

We can compute the Hilbert series using \macaulay. We'll use the same
examples as before:

\begin{macaulay2}
R = ZZ/101[x,y,z];
hilbertSeries(R)
\end{macaulay2}


\begin{macaulay2}
R = ZZ/101[x,y,z]/(x^3+y^3+z^3);
hilbertSeries(R)
\end{macaulay2}

\begin{macaulay2}
R = ZZ/101[x,y,z]/(x^2,y^3,z^4);
hilbertSeries(R)
\end{macaulay2}



\begin{theorem}[Hilbert-Serre]\label{T:HS}
  Let $R$ be a finitely generated graded commutative algebra over an
  Artinian ring $R_0$ with all generators $x_1,\dots,x_r$ belonging to
  $R_1$ and let $M$ be a finitely generated graded $R$-module. In this
  case, the Hilbert series of $M$ is given by
  \[
  \hs_M(t) = \frac{g(t)}{(1-t)^r}
  \]
  where $g(t)\in\Z[t]$.
  \begin{proof}
    Proceed by induction on the number of generators of $R$ over
    $R_0$.

    Basis step: $r =0$. If $r=0$, then $R=R_0$, and hence $R_i = 0$
    for $i>0$. Choose a finite set of generators for $M$. If $d$ is
    the maximum of the degrees of the generators, then 
    \[
    \hs_M(t) = \sum_{n=0}^d \l(M_n) t^n.
    \]
    is a polynomial.

    Now suppose our theorem holds for up to $r-1$ generators. Let
    $x_r$ be a generator in $R_1$ and consider the map defined by
    multiplication by $x_r$
    \[
    0\lto K_n \lto M_n \lto^{x_r} M_{n+1} \lto L_{n+1}\lto 0 
    \]
    where $K_n$ is the kernel and $L_{n+1}$ is the cokernel. Set
    \[
    K = \directsum_{n= 0}^\infty K_n\quad\text{and}\quad L = \directsum_{n= 0}^\infty L_n.
    \]
    Now $K\subset M$ is a submodule, and $L = M/x_r M$, and
    \[
    x_r K = x_r L = 0,
    \]
    hence both $K$ and $L$ are $(R/x_r R)$-modules over a
    $(r-1)$-generated graded ring. Thus we may apply the inductive
    hypothesis to compute $\hs_K(t)$ and $\hs_L(t)$. By
    Corollary~\ref{C:Lles}, we have that
    \[
    \l(K_n)-\l(M_n) + \l(M_{n+1}) -\l(L_{n+1}) = 0.
    \]
    Multiply each term by $t^{n+1}$ to get
    \[
    t^{n+1}\l(K_n)-t^{n+1}\l(M_n) + t^{n+1}\l(M_{n+1}) -t^{n+1}\l(L_{n+1}) = 0
    \]
    Summing this with respect to $n$ we find
    \[
    t\hs_K(t) - t\hs_M(t)  + \hs_M(t) - \hs_L(t) = 0
    \]
    and so
    \begin{align*}
      (1-t)\hs_M(t) &= \hs_L(t) - t\hs_K(t)\\
      \hs_M(t) &= \frac{\hs_L(t) - t\hs_K(t)}{1-t}
    \end{align*}
    and this is our result.
  \end{proof}
\end{theorem}

\begin{remark}
  Above we have a somewhat specialized theorem, since the degrees of
  the generators of $M$ are all $1$. You can find a more general
  version of this theorem in \cite{AM1969,hM1986}.
\end{remark}


\begin{corollary}\label{C:hp}
  Let $R$ be a finitely generated graded commutative algebra over an
  Artinian ring $R_0$ with all generators $x_1,\dots,x_r$ belonging to
  $R_1$ and let $M$ be a finitely generated graded $R$-module. In this
  case, there is a polynomial $\hp_M(X)\in\Q[X]$ such that
  \[
  \hp_M(n) = \hf_M(n) = \l(M_n)
  \]
  for $n\gg 0$. The polynomial $\hp_M(n)$ is called the
  \dfn{Hilbert polynomial} for $M$.
  \begin{proof}
    Start by finding the Hilbert series for the module:
    \[
    \hs_M(t) = \frac{g(t)}{(1-t)^r}
    \]
    It may be the case that $g(1) = 0$, meaning $(1-t)|g(t)$. Cancel
    all factors of $(1-t)$ of $g$ to obtain a new expression
    \[
    \hs_M(t) = \frac{f(t)}{\prod_{i=1}^s(1-t)^d}
    \]
    where $f(1) \ne 0$. If $d=0$, we are done, as $f(t)\in
    \Z[t]$. Note, via Taylor's theorem
    \[
    \frac{1}{(1-t)^d} = \sum_{n=0}^\infty \binom{d+n-1}{d-1}t^n.
    \]
    If $f(t) = a_0 + a_1 t + \dots + a_s t^s$, then multiplying, we
    find the $n$th coefficient to be
    \[
    \l(M_n) = a_0 \binom{d+n-1}{d-1} + a_1 \binom{d+n-2}{d-1} + \dots + a_s \binom{d+n-s-1}{d-1}
    \]
    where we set $\binom{m}{d-1} = 0$ for $m<d-1$. The expression
    above on the right is a polynomial in $n$, call it $\hp_M(n)$:
    \[
    \hp_M(n) = \sum_{i=0}^s a_i \binom{d+n-i-1}{d-1}
    \]
    we may rearrange these terms to see
    \begin{align*}
    \hp_M(n) &= \frac{\sum_{i=0}^s a_i}{(d-1)!}n^{d-1} + (\text{lower degree terms})\\
    &= \frac{f(1)}{(d-1!)}n^{d-1}  + (\text{lower degree terms}).
    \end{align*}
    Hence $\hp_M(X)\in \Q[X]$ and if $n\ge s+1-d$, we have $\l(M_n) =
    \hf_M(n) = \hp_M(n)$.
    \end{proof}
\end{corollary}

\begin{remark}
   If the degrees of the generators of $M$ are not all $1$. You might
   not be able to find a single polynomial representative of the
   Hilbert function.
\end{remark}

Let's see some examples with \macaulay.

\begin{macaulay2}
R = ZZ/101[x,y,z];
hp=hilbertPolynomial(R,Projective=>false)
S = ring hp;
apply(10,i->hilbertFunction(i,R))
apply(10,j->sub(hp,{first gens S=>j}))
\end{macaulay2}


\begin{macaulay2}
R = ZZ/101[x,y,z]/(x^4,y^2+z^2);
hp=hilbertPolynomial(R,Projective=>false)
S = ring hp;
apply(10,i->hilbertFunction(i,R))
apply(10,j->sub(hp,{first gens S=>j}))
dim R
\end{macaulay2}

\section{The Hilbert-Samuel polynomial}

We can extend our work above, by using a canonical grading, the adic
topology. We will work in the $\m$-adic topology over a local ring,
though we could work more generally. Specifically, if we were working
with an ideal such that $\m^n\subset I \subset \m$ (called an
\dfn{ideal of definition}) the theorems below hold. The interested
reader can see $\cite{AM1969,hM1986}$.

\begin{definition}
  Given a local ring $(A,\m)$ and an $A$-module $M$ we can produce the
  associated grading ring:
  \[
  \gr_\m(A) = \directsum_{n=0}^\infty \left(\m^n/\m^{n+1}\right) \quad\text{and}\quad
  \gr_\m(M) = \directsum_{n=0}^\infty \left(\m^nM/\m^{n+1}M\right)
  \]
\end{definition}

In this case, $\gr_\m(M)$ is a graded module over $\gr_\m(A)$. In
particular, the degree zero ring $A_0 = A/\m$ is an Artinian ring.  If
the maximal ideal $\m$ is generated by $m_1,\dots,m_n$, then
$\bar{m}_1,\dots,\bar{m}_n$ are degree $1$ generators of $\gr_\m(A)$
over $A/\m$. Finally note, if $M$ is finitely generated over $A$, then
$\gr_\m(M)$ is finitely generated over $\gr_\m(A)$. This means that
the conclusion of Theorem~\ref{T:HS} and Corollary~\ref{C:hp} apply to
these graded rings.






\begin{proposition}
  Let $(A,\m,k)$ be a local ring and $M$ be a finitely generated
  $A$-module. Then a composition series for $M/\m^{n} M$ is given by
  \[
  M/\m^{n} M\supsetneq \m M/\m^{n} M \supsetneq \dots \supsetneq
  \m^{n-1} M/\m^{n} M\supsetneq 0.
  \]
  \begin{sketch}
    Check the cofactors.
  \end{sketch}
\end{proposition}



\begin{theorem}[Hilbert-Samuel]\label{T:S}
  Let $(A,\m,k)$ be a local ring and $M$ be a finitely generated
  $A$-module. In this case, there is a polynomial $P_\m(X)\in\Q[X]$
  such that
  \[
  P_M(n) = \l(M/\m^n M)
  \]
  for $n\gg 0$. The polynomial $P_\m(n)$ is called the \dfn{Hilbert-Samuel
    polynomial} for $M$.
  \begin{proof}
    From our work in Theorem~\ref{T:HS} and Corollary~\ref{C:hp}, we
    find a polynomial $\hp_M$ such that
    \[
    \hp_{\gr_\m(M)}(n) = \l_{k}(\m^n M/\m^{n+1}) 
    \]
    for $n\gg 0$. Now note that while $M/\m^n$ is an graded
    $A$-module, it is also a graded $k$-module, view it as
    such. Hence
    \[
    \l_{A}(M/\m^n M) = \l_k(M/\m^n M).
    \]
    Consider the short exact sequence
    \[
    0 \to \m^n M/\m^{n+1} M \to M/\m^{n+1}M \to M/ \m^n M\to 0
    \]
    By additivity of length, Proposition~\ref{P:AL}, we see that
    \[
    \l_k(\m^n M/\m^{n+1} M) - \l_k(M/\m^{n+1}M) + \l_k(M/\m^{n+1}M) = 0
    \]
    Since
    \[
    P_M(n) = \l_{A}(M/\m^n M) = \l_k(M/\m^n M),
    \]
    we see that
    \[
    \hp_{\gr_\m(M)} - P_M(n+1) + P_M(n) = 0.
    \]
    This means that $P_M(n)- P_M(n+1)$ is a polynomial of degree at
    most $d-1$. However, this means $P_M(X)\in\Q[X]$ is a polynomial
    of degree $d$.
  \end{proof}
\end{theorem}



\begin{definition}[Second Notion of Dimension]
  Let $(A,\m,k)$ be a local ring and $M$ be a finitely generated
  $A$-module. Then define
  \[
  d(M) := \deg(P_M(X)).
  \]
  where $P_M(X)\in\Q[X]$ is the Hilbert-Samuel polynomial.
\end{definition}


\begin{proposition}[Additivity and Hilbert-Samuel Polynomials]\label{P:AddHS}
  Let $(A,\m)$ be a local ring and $M$ be a finitely generated
  $A$-module. Given a short exact sequence of $A$-modules
  \[
  0\to M' \to M\to M'' \to 0
  \]
  we have that
  \[
  P_{M'}(n) + P_{M''}(n) = P_M(n) +r(n)
  \]
  where $\deg(r)< d(M) = \deg(P_M)$.
  \begin{proof}
    Consider the following short exact sequence
    \[
    0 \to M/(M'\cap \m^n M) \to M/\m^n M \to M''/\m^n M'' \to 0.
    \]
    By additivity of length, Proposition~\ref{P:AL},
    \[\tag{$\bigstar$}
    P_M(n) - P_{M''}(n)  = \l(M/(M'\cap \m^n M)).
    \]
    hence $\l(M/(M'\cap \m^n M))$ is a polynomial for $n\gg 0$. By
    \index{Artin-Rees}Artin-Rees, Theorem~\ref{L:ArtinRees}, 
    \[
    \m (M'\cap \m^n M) = (M'\cap \m^{n+1} M)
    \]
    for $n\gg 0$, say $n>m$.  So
    \[
    \m^{n+m} M' \subset (M'\cap \m^{n+m}M)  = \m^n (M'\cap\m^m M)\subset \m^n M'.
    \]
    Hence
    \[
    \l(M'/\m^{m+n}M')\ge \l(M'/(M'\cap \m^{n+m}M))\ge \l(M'/\m^n M')
    \]
    but the ends of this string of inequalities can be expressed as
    the Hilbert-Samuel polynomial
    \[
    P_{M'}(n+m) \ge \l(M'/(M'\cap \m^{n+m}M)) \ge P_{M'}(n).
    \]
    This means that $\l(M'/(M'\cap \m^{n+m}M))$ is a polynomial whose
    degree is equal to $\deg(P_{M'})$ and that the leading coefficient
    of $\l(M'/(M'\cap \m^{n+m}M))$ is the same as the leading
    coefficient of $P_{M'}(n)$. Set
    \[\tag{$\bigstar\bigstar$}
    r(n) = P_{M'}(n) -\l(M'/(M'\cap \m^{n+m}M))
    \]
    The polynomial $r(n)$ has degree less than $\deg(P_{M})$ since
    \[
    \deg(\l(M'/(M'\cap \m^{n+m}M)))\le \deg(P_M(n)).
    \]
    zCombining $(\bigstar)$ and $(\bigstar\bigstar)$ we find our
    result.
  \end{proof}
\end{proposition}


\begin{corollary}\label{C:sub}
  Let $A$ be a local ring and $N\subset M$ be a finitely generated
  $A$-modules. In this case
  \[
  d(N) \le d(M)
  \]
\end{corollary}




\end{document}
