\documentclass{ximera}



\usepackage{tikz-cd}
\usepackage[sans]{dsfont}

\let\oldbibliography\thebibliography%% to compact bib
\renewcommand{\thebibliography}[1]{%
  \oldbibliography{#1}%
  \setlength{\itemsep}{0pt}%
}


\DefineVerbatimEnvironment{macaulay2}{Verbatim}{numbers=left,frame=lines,label=Macaulay2,labelposition=topline}

%%% This next bit of code defines all our theorem environments
\makeatletter
\let\c@theorem\relax
\let\c@corollary\relax
\makeatother

\let\definition\relax
\let\enddefinition\relax

\let\theorem\relax
\let\endtheorem\relax

\let\proposition\relax
\let\endproposition\relax

\let\exercise\relax
\let\endexercise\relax

\let\question\relax
\let\endquestion\relax

\let\remark\relax
\let\endremark\relax

\let\corollary\relax
\let\endcorollary\relax


\let\example\relax
\let\endexample\relax

\let\warning\relax
\let\endwarning\relax

\let\lemma\relax
\let\endlemma\relax

\newtheoremstyle{SlantTheorem}{\topsep}{\topsep}%%% space between body and thm
		{\slshape}                      %%% Thm body font
		{}                              %%% Indent amount (empty = no indent)
		{\bfseries\sffamily}            %%% Thm head font
		{}                              %%% Punctuation after thm head
		{3ex}                           %%% Space after thm head
		{\thmname{#1}\thmnumber{ #2}\thmnote{ \bfseries(#3)}}%%% Thm head spec
\theoremstyle{SlantTheorem}
\newtheorem{theorem}{Theorem}
\newtheorem{definition}[theorem]{Definition}
\newtheorem{proposition}[theorem]{Proposition}
%% \newtheorem*{dfnn}{Definition}
%% \newtheorem{ques}{Question}[theorem]
\newtheorem{lemma}[theorem]{Lemma}
%% \newtheorem*{war}{WARNING}
%% \newtheorem*{cor}{Corollary}
%% \newtheorem*{eg}{Example}
\newtheorem*{remark}{Remark}
\newtheorem*{touchstone}{Touchstone}
\newtheorem{corollary}{Corollary}[theorem]
\newtheorem*{example}{Example}
\newtheorem*{warning}{WARNING}


\newtheoremstyle{Exercise}{\topsep}{\topsep} %%% space between body and thm
		{}                           %%% Thm body font
		{}                           %%% Indent amount (empty = no indent)
		{\bfseries}                  %%% Thm head font
		{)}                          %%% Punctuation after thm head
		{ }                          %%% Space after thm head
		{\thmnumber{#2}\thmnote{ \bfseries(#3)}}%%% Thm head spec
\theoremstyle{Exercise}
\newtheorem{exercise}{}[theorem]

%% \newtheoremstyle{Question}{\topsep}{\topsep} %%% space between body and thm
%% 		{\bfseries}                  %%% Thm body font
%% 		{3ex}                        %%% Indent amount (empty = no indent)
%% 		{}                           %%% Thm head font
%% 		{}                           %%% Punctuation after thm head
%% 		{}                           %%% Space after thm head
%% 		{\thmnumber{#2}\thmnote{ \bfseries(#3)}}%%% Thm head spec
\newtheoremstyle{Question}{3em}{3em} %%% space between body and thm
		{\large\bfseries}                           %%% Thm body font
		{3ex}                           %%% Indent amount (empty = no indent)
		{\bfseries}                  %%% Thm head font
		{}                          %%% Punctuation after thm head
		{ }                          %%% Space after thm head
		{}%%% Thm head spec
\theoremstyle{Question}
\newtheorem*{question}{}



\renewcommand{\tilde}{\widetilde}
\renewcommand{\bar}{\overline}
\renewcommand{\hat}{\widehat}
\newcommand{\N}{\mathbb N}
\newcommand{\Z}{\mathbb Z}
\newcommand{\R}{\mathbb R}
\newcommand{\Q}{\mathbb Q}
\newcommand{\C}{\mathbb C}
\newcommand{\V}{\mathbb V}
\newcommand{\I}{\mathbb I}
\newcommand{\A}{\mathbb A}
\newcommand{\iso}{\simeq}
\newcommand{\ph}{\varphi}
\newcommand{\Cf}{\mathcal{C}}
\newcommand{\IZ}{\mathrm{Int}(\Z)}
\newcommand{\dsum}{\oplus}
\newcommand{\directsum}{\bigoplus}
\newcommand{\union}{\bigcup}
\renewcommand{\i}{\mathfrak}
\renewcommand{\a}{\mathfrak{a}}
\renewcommand{\b}{\mathfrak{b}}
\newcommand{\m}{\mathfrak{m}}
\newcommand{\p}{\mathfrak{p}}
\newcommand{\q}{\mathfrak{q}}
\newcommand{\dfn}[1]{\textbf{#1}\index{#1}}
\let\hom\relax
\DeclareMathOperator{\ann}{Ann}
\DeclareMathOperator{\h}{ht}
\DeclareMathOperator{\hom}{Hom}
\DeclareMathOperator{\Span}{Span}
\DeclareMathOperator{\spec}{Spec}
\DeclareMathOperator{\maxspec}{MaxSpec}
\DeclareMathOperator{\supp}{Supp}
\DeclareMathOperator{\ass}{Ass}
\DeclareMathOperator{\ff}{Frac}
\DeclareMathOperator{\im}{Im}
\DeclareMathOperator{\syz}{Syz}
\DeclareMathOperator{\gr}{Gr}
\renewcommand{\ker}{\mathop{\mathrm{Ker}}\nolimits}
\newcommand{\coker}{\mathop{\mathrm{Coker}}\nolimits}
\newcommand{\lps}{[\hspace{-0.25ex}[}
\newcommand{\rps}{]\hspace{-0.25ex}]}
\newcommand{\into}{\hookrightarrow}
\newcommand{\onto}{\twoheadrightarrow}
\newcommand{\tensor}{\otimes}
\newcommand{\x}{\mathbf{x}}
\newcommand{\X}{\mathbf X}
\newcommand{\Y}{\mathbf Y}
\renewcommand{\k}{\boldsymbol{\kappa}}
\renewcommand{\emptyset}{\varnothing}
\renewcommand{\qedsymbol}{$\blacksquare$}
\renewcommand{\l}{\ell}
\newcommand{\1}{\mathds{1}}
\newcommand{\lto}{\mathop{\longrightarrow\,}\limits}
\newcommand{\rad}{\sqrt}
\newcommand{\hf}{H}
\newcommand{\hs}{H\!S}
\newcommand{\hp}{H\!P}
\renewcommand{\vec}{\mathbf}
\renewcommand{\phi}{\varphi}
\renewcommand{\epsilon}{\varepsilon}
\renewcommand{\subset}{\subseteq}
\renewcommand{\supset}{\supseteq}
\newcommand{\macaulay}{\textsl{Macaulay2}}
\newcommand{\invlim}{\varprojlim}


%\renewcommand{\proofname}{Sketch of Proof}


\renewenvironment{proof}[1][Proof]
  {\begin{trivlist}\item[\hskip \labelsep \itshape \bfseries #1{}\hspace{2ex}]\upshape}
{\qed\end{trivlist}}

\newenvironment{sketch}[1][Sketch of Proof]
  {\begin{trivlist}\item[\hskip \labelsep \itshape \bfseries #1{}\hspace{2ex}]\upshape}
{\qed\end{trivlist}}



\makeatletter
\renewcommand\section{\@startsection{paragraph}{10}{\z@}%
                                     {-3.25ex\@plus -1ex \@minus -.2ex}%
                                     {1.5ex \@plus .2ex}%
                                     {\normalfont\large\sffamily\bfseries}}
\renewcommand\subsection{\@startsection{subparagraph}{10}{\z@}%
                                    {3.25ex \@plus1ex \@minus.2ex}%
                                    {-1em}%
                                    {\normalfont\normalsize\sffamily\bfseries}}
\makeatother

%% Fix weird index/bib issue.
\makeatletter
\gdef\ttl@savemark{\sectionmark{}}
\makeatother


\makeatletter
%% no number for refs
\newcommand\frontstyle{%
  \def\activitystyle{activity-chapter}
  \def\maketitle{%
    \addtocounter{titlenumber}{1}%
                    {\flushleft\small\sffamily\bfseries\@pretitle\par\vspace{-1.5em}}%
                    {\flushleft\LARGE\sffamily\bfseries\@title \par }%
                    {\vskip .6em\noindent\textit\theabstract\setcounter{problem}{0}\setcounter{sectiontitlenumber}{0}}%
                    \par\vspace{2em}
                    \phantomsection\addcontentsline{toc}{section}{\textbf{\@title}}%
                  }}
\makeatother


\author{Bart Snapp}

\title{A topological approach}

\begin{document}
\begin{abstract}
  We define the dimension of a ring based on its topological
  properties.  Sources and references: \cite{sD2008}.
\end{abstract}
\maketitle

\section{Dimension of a topological space}

Given a topological space, we can define its dimension as follows:


\begin{definition}\index{dimension!topological space}
  Let $X$ be a topological space, the \textbf{dimension} of $X$,
  denoted by $\dim(X)$ is defined as
  \[
  \dim(X) :=\sup\left\{d:
  \begin{minipage}{43ex}
    there exists a chain $X_0\supsetneq \dots \supsetneq X_d$ of
    length $d$ of irreducible closed subsets of $X$,
  \end{minipage}
  \right\}.
  \]
\end{definition}


\begin{definition}\index{Noetherian!topological space}
  If $X$ is a topological space, $X$ is called \textbf{Noetherian} if
  any of the following equivalent conditions hold:
  \begin{enumerate}
  \item Every nonempty family of open sets has a maximal element.
  \item Any increasing chain of open sets terminates.
  \item Every nonempty family of closed sets has a minimal element.
  \item Any decreasing chain of closed sets terminates.
  \end{enumerate}
\end{definition}



\begin{proposition}
  If $X$ is a Noetherian topological space,
  \[
  \dim(X) = \sup\dim(X_i)
  \]
  where $X_i$ is a component of $X$. 
\end{proposition}

\section{Krull dimension of a ring}


\begin{proposition}\index{Noetherian!topological space}\index{Noetherian!ring}
  If $R$ is a Noetherian ring, then $\spec(R)$ is a Noetherian
  topological space.
  \begin{proof}\index{Primary Decomposition Theorem}
    Suppose $R$ is Noetherian, then by primary decomposition of
    ideals, Theorem~\ref{T:PDI}, we have
    \[
    (0) = \q_1\cap \dots \cap \q_r
    \]
    where each $\q_i$ is $\p_i$-primary and so $V(\q_i) =
    V(\p_i)$. Eliminate those $V(\p_i)$ for which $p_i$ is not minimal
    in $\ass(R)$.  Then reindexing we have
    \[
    \spec(R) = V(0) = V(\p_1) \cup\dots \cup V(\p_s). 
    \]
    Since the $V(\p_i)$ are the irreducible components of $\spec(R)$,
    any decreasing chain of closed sets terminates.
  \end{proof}
\end{proposition}



\begin{warning}\index{Noetherian!topological space}\index{Noetherian!ring}
  The converse of the above proposition is \textbf{not} true. We leave
  it as an exercise to show that if $k$ is a field, then
  \[
  R = \frac{k[X_1,\dots,X_n,\dots]}{(X_1^2,\dots,X_n^2,\dots)}
  \]
  is not a Noetherian ring but $\spec(R)$ is a Noetherian topological space.
\end{warning}


\begin{definition}[First Notion of Dimension]\index{dimA@$\dim(A)$}\index{dimM@$\dim(M)$}\index{dimension}\index{Krull dimension}
  If $R$ is a ring, the \textbf{Krull dimension}, denoted by
  $\dim(R)$, is the dimension of the topological space $\spec(R)$. To
  be explicit:
  \[
  \dim(R) = \sup \{ d : \text{there exists $\p_0\subsetneq \p_1
    \subsetneq \dots \subsetneq \p_d$ such that $\p_i \in \spec(R)$} \}.
  \]
  This notion of dimension is often simply referred to as the
  \textbf{dimension} of a ring. If $M$ is an $R$-module, then
  \[
  \dim(M) := \dim(R/\ann(M)).
  \]
\end{definition}


\begin{exercise}\index{support}\index{supp@$\supp(M)$}
  Given a ring $R$ and an $R$-module $M$, show that the dimension of
  $M$ is the dimension of the topological space $\supp(M)\subset
  \spec(R)$.
\end{exercise}

\begin{example}
  While
  \[
  \dim(R) = \sup \{ d : \text{there exists $\p_0\subsetneq \p_1 \subsetneq \dots \subsetneq \p_d$ such that $\p_i \in \spec(R)$} \}.
  \]
  Nagata gives an example of a ring $R$ such that $R$ is Noetherian
  but with infinite dimension in \cite{mN1962}.
\end{example}


\begin{example}
  In the ring $\C[X_1,\dots,X_n]$ we have the chain of prime ideals
  \[
  (0) \subsetneq (X_1) \subsetneq \cdots \subsetneq (X_1,\dots,X_n).
  \]
  Thus the dimension of $\C[X_1,\dots,X_n]$ is at least $n$.
\end{example}


Suppose that a ring $R$ has finite dimension. If $\m_1, \dots, \m_r$
are maximal prime ideals in $\spec(R)$, then $\dim(R) = d$ implies
that there exists a chain of length $d$ ending at one of the $\m_i$'s.
Thus for some $i$,
\[
\dim(R_{\m_i}) = \dim(R). 
\]
Hence we see that some questions about the dimension of a ring can be
reduced to questions about local rings.


\begin{remark}
  We can characterize dimension in two useful ways:
  \begin{enumerate}
  \item $\dim(R) = \sup \dim(R_\m)$ where $\m \in \maxspec(R)$.
  \item $\dim(R) = \sup \dim(R/\p)$ where $\p$ is a minimal prime ideal of $R$. 
  \end{enumerate}
\end{remark}






\end{document}
