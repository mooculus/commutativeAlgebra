\documentclass{ximera}



\usepackage{tikz-cd}
\usepackage[sans]{dsfont}

\let\oldbibliography\thebibliography%% to compact bib
\renewcommand{\thebibliography}[1]{%
  \oldbibliography{#1}%
  \setlength{\itemsep}{0pt}%
}


\DefineVerbatimEnvironment{macaulay2}{Verbatim}{numbers=left,frame=lines,label=Macaulay2,labelposition=topline}

%%% This next bit of code defines all our theorem environments
\makeatletter
\let\c@theorem\relax
\let\c@corollary\relax
\makeatother

\let\definition\relax
\let\enddefinition\relax

\let\theorem\relax
\let\endtheorem\relax

\let\proposition\relax
\let\endproposition\relax

\let\exercise\relax
\let\endexercise\relax

\let\question\relax
\let\endquestion\relax

\let\remark\relax
\let\endremark\relax

\let\corollary\relax
\let\endcorollary\relax


\let\example\relax
\let\endexample\relax

\let\warning\relax
\let\endwarning\relax

\let\lemma\relax
\let\endlemma\relax

\newtheoremstyle{SlantTheorem}{\topsep}{\topsep}%%% space between body and thm
		{\slshape}                      %%% Thm body font
		{}                              %%% Indent amount (empty = no indent)
		{\bfseries\sffamily}            %%% Thm head font
		{}                              %%% Punctuation after thm head
		{3ex}                           %%% Space after thm head
		{\thmname{#1}\thmnumber{ #2}\thmnote{ \bfseries(#3)}}%%% Thm head spec
\theoremstyle{SlantTheorem}
\newtheorem{theorem}{Theorem}
\newtheorem{definition}[theorem]{Definition}
\newtheorem{proposition}[theorem]{Proposition}
%% \newtheorem*{dfnn}{Definition}
%% \newtheorem{ques}{Question}[theorem]
\newtheorem{lemma}[theorem]{Lemma}
%% \newtheorem*{war}{WARNING}
%% \newtheorem*{cor}{Corollary}
%% \newtheorem*{eg}{Example}
\newtheorem*{remark}{Remark}
\newtheorem*{touchstone}{Touchstone}
\newtheorem{corollary}{Corollary}[theorem]
\newtheorem*{example}{Example}
\newtheorem*{warning}{WARNING}


\newtheoremstyle{Exercise}{\topsep}{\topsep} %%% space between body and thm
		{}                           %%% Thm body font
		{}                           %%% Indent amount (empty = no indent)
		{\bfseries}                  %%% Thm head font
		{)}                          %%% Punctuation after thm head
		{ }                          %%% Space after thm head
		{\thmnumber{#2}\thmnote{ \bfseries(#3)}}%%% Thm head spec
\theoremstyle{Exercise}
\newtheorem{exercise}{}[theorem]

%% \newtheoremstyle{Question}{\topsep}{\topsep} %%% space between body and thm
%% 		{\bfseries}                  %%% Thm body font
%% 		{3ex}                        %%% Indent amount (empty = no indent)
%% 		{}                           %%% Thm head font
%% 		{}                           %%% Punctuation after thm head
%% 		{}                           %%% Space after thm head
%% 		{\thmnumber{#2}\thmnote{ \bfseries(#3)}}%%% Thm head spec
\newtheoremstyle{Question}{3em}{3em} %%% space between body and thm
		{\large\bfseries}                           %%% Thm body font
		{3ex}                           %%% Indent amount (empty = no indent)
		{\bfseries}                  %%% Thm head font
		{}                          %%% Punctuation after thm head
		{ }                          %%% Space after thm head
		{}%%% Thm head spec
\theoremstyle{Question}
\newtheorem*{question}{}



\renewcommand{\tilde}{\widetilde}
\renewcommand{\bar}{\overline}
\renewcommand{\hat}{\widehat}
\newcommand{\N}{\mathbb N}
\newcommand{\Z}{\mathbb Z}
\newcommand{\R}{\mathbb R}
\newcommand{\Q}{\mathbb Q}
\newcommand{\C}{\mathbb C}
\newcommand{\V}{\mathbb V}
\newcommand{\I}{\mathbb I}
\newcommand{\A}{\mathbb A}
\newcommand{\iso}{\simeq}
\newcommand{\ph}{\varphi}
\newcommand{\Cf}{\mathcal{C}}
\newcommand{\IZ}{\mathrm{Int}(\Z)}
\newcommand{\dsum}{\oplus}
\newcommand{\directsum}{\bigoplus}
\newcommand{\union}{\bigcup}
\renewcommand{\i}{\mathfrak}
\renewcommand{\a}{\mathfrak{a}}
\renewcommand{\b}{\mathfrak{b}}
\newcommand{\m}{\mathfrak{m}}
\newcommand{\p}{\mathfrak{p}}
\newcommand{\q}{\mathfrak{q}}
\newcommand{\dfn}[1]{\textbf{#1}\index{#1}}
\let\hom\relax
\DeclareMathOperator{\ann}{Ann}
\DeclareMathOperator{\h}{ht}
\DeclareMathOperator{\hom}{Hom}
\DeclareMathOperator{\Span}{Span}
\DeclareMathOperator{\spec}{Spec}
\DeclareMathOperator{\maxspec}{MaxSpec}
\DeclareMathOperator{\supp}{Supp}
\DeclareMathOperator{\ass}{Ass}
\DeclareMathOperator{\ff}{Frac}
\DeclareMathOperator{\im}{Im}
\DeclareMathOperator{\syz}{Syz}
\DeclareMathOperator{\gr}{Gr}
\renewcommand{\ker}{\mathop{\mathrm{Ker}}\nolimits}
\newcommand{\coker}{\mathop{\mathrm{Coker}}\nolimits}
\newcommand{\lps}{[\hspace{-0.25ex}[}
\newcommand{\rps}{]\hspace{-0.25ex}]}
\newcommand{\into}{\hookrightarrow}
\newcommand{\onto}{\twoheadrightarrow}
\newcommand{\tensor}{\otimes}
\newcommand{\x}{\mathbf{x}}
\newcommand{\X}{\mathbf X}
\newcommand{\Y}{\mathbf Y}
\renewcommand{\k}{\boldsymbol{\kappa}}
\renewcommand{\emptyset}{\varnothing}
\renewcommand{\qedsymbol}{$\blacksquare$}
\renewcommand{\l}{\ell}
\newcommand{\1}{\mathds{1}}
\newcommand{\lto}{\mathop{\longrightarrow\,}\limits}
\newcommand{\rad}{\sqrt}
\newcommand{\hf}{H}
\newcommand{\hs}{H\!S}
\newcommand{\hp}{H\!P}
\renewcommand{\vec}{\mathbf}
\renewcommand{\phi}{\varphi}
\renewcommand{\epsilon}{\varepsilon}
\renewcommand{\subset}{\subseteq}
\renewcommand{\supset}{\supseteq}
\newcommand{\macaulay}{\textsl{Macaulay2}}
\newcommand{\invlim}{\varprojlim}


%\renewcommand{\proofname}{Sketch of Proof}


\renewenvironment{proof}[1][Proof]
  {\begin{trivlist}\item[\hskip \labelsep \itshape \bfseries #1{}\hspace{2ex}]\upshape}
{\qed\end{trivlist}}

\newenvironment{sketch}[1][Sketch of Proof]
  {\begin{trivlist}\item[\hskip \labelsep \itshape \bfseries #1{}\hspace{2ex}]\upshape}
{\qed\end{trivlist}}



\makeatletter
\renewcommand\section{\@startsection{paragraph}{10}{\z@}%
                                     {-3.25ex\@plus -1ex \@minus -.2ex}%
                                     {1.5ex \@plus .2ex}%
                                     {\normalfont\large\sffamily\bfseries}}
\renewcommand\subsection{\@startsection{subparagraph}{10}{\z@}%
                                    {3.25ex \@plus1ex \@minus.2ex}%
                                    {-1em}%
                                    {\normalfont\normalsize\sffamily\bfseries}}
\makeatother

%% Fix weird index/bib issue.
\makeatletter
\gdef\ttl@savemark{\sectionmark{}}
\makeatother


\makeatletter
%% no number for refs
\newcommand\frontstyle{%
  \def\activitystyle{activity-chapter}
  \def\maketitle{%
    \addtocounter{titlenumber}{1}%
                    {\flushleft\small\sffamily\bfseries\@pretitle\par\vspace{-1.5em}}%
                    {\flushleft\LARGE\sffamily\bfseries\@title \par }%
                    {\vskip .6em\noindent\textit\theabstract\setcounter{problem}{0}\setcounter{sectiontitlenumber}{0}}%
                    \par\vspace{2em}
                    \phantomsection\addcontentsline{toc}{section}{\textbf{\@title}}%
                  }}
\makeatother


\author{Bart Snapp}

\title{Applications of primary decomposition}

\begin{document}
\begin{abstract}
  We give examples and applications of primary decomposition. Sources
  and references: \cite{AM1969, sD2008, dE1995}.
\end{abstract}
\maketitle



\begin{definition}
  Let $R$ be as ring. An ideal $\q$ is called \dfn{primary}, if $\q\ne
  R$ and
  \[
  xy\in\q \quad \Rightarrow \quad x\in\q\text{ or }y^m\in\q.
  \]
\end{definition}

\begin{exercise}
  Let $R$ be a Noetherian ring and $\q$ be an ideal of $R$. Prove that
  $\q$ is $\p$-primary if and only if $\ass(R/I) = \{\p\}$.
\end{exercise}

\begin{exercise}
  Show that every prime ideal is primary.
\end{exercise}

\begin{exercise}
  Show that if an idea $\q$ is $\p$-primary, then $\rad{\q} = \p$.
\end{exercise}

\begin{exercise}
  Let $k$ be a field, and consider $R=k[x,y,z]/(z^2-xy)$. Let $\p = (\bar{x},\bar{y})$.
  \begin{enumerate}
  \item Show that $\p$ is a prime ideal in $R$.
  \item Show that $\p^2$ is not $\p$-primary.
  \end{enumerate}
\end{exercise}

\begin{exercise}
  Let $R$ be a UFD and $\p = (p)$ where $p$ is a prime element. Show
  that $\p^n$ is a $\p$-primary ideal for all $n>0$.
\end{exercise}

\begin{exercise}
  Prove that if $\a$ is $\p$-primary and $\b$ is $\q$-primary, then
  $\a$ and $\b$ are comaximal.
\end{exercise}



\begin{theorem}[Primary decomposition of ideals]
  Let $R$ be a Noetherian ring, and $I$ be an ideal of $R$. We may write
  \[
  I = \bigcap_{i=1}^n \q_i
  \]
  uniquely and minimally where each $\q$ is \dfn{primary}
  \begin{proof}
    Let $M = R/I$. By Primary decomposition (Theorem~\ref{T:primarydecomp}),
    \[
    I = \bigcap_{i=1}^n \q_i
    \]
    where for each $i$, $\ass(R/\q_i) = \{\p_i\}$, if $i\ne j$, then
    $\p_i\ne\p_j$, and $\ass(R/I) = \{\p_1,\dots,\p_n\}$.
  \end{proof}
\end{theorem}

\begin{theorem}[Fundamental theorem of arithmetic]
  Every integer greater than $1$ can be factored uniquely (up to
  ordering) into primes.
  \begin{proof}
    Let $a\in\Z$. Consider the ideal generated by $a$. From above
    \[
    a\Z = \bigcap_{i=1}^n \q_i
    \]
    where $\ass(\Z/\q_i) = p_i\Z$ and $\ass(\Z/a\Z) = \{p_1\Z,\dots,
    p_n\Z\}$. Since the prime ideas of $\Z$ are principally generated,
    and the primary ideals of $\Z$ are simply prime powers, we are
    done.
  \end{proof}
\end{theorem}

\begin{exercise}%hM1986 page 161
   A Noetherian integral domain $R$ is a UFD if and only if every minimal
   prime ideal is principal.
\end{exercise}

\begin{theorem}[Fundamental theorem of finitely generated abelian groups]
  Every finitely generated Abelian group $G$ is isomorphic to a
  direct sum of cyclic groups of prime-power order and infinite cyclic
  groups,
  \[
  G \iso \Z^n \dsum \directsum_{i=1}^{m} \Z_{p_i^{k_i}},
  \]
  where the $p_i$ are not necessarily distinct prime numbers.
  \begin{proof}
    A finitely generated Abelian group is a $\Z$-module. Consider
    \[
    0 \to \syz_1(G) \to \Z^n \to G \to 0 
    \]
    where $\syz_1(G)$ is the first syzygy\index{syzygy} module, meaning $\syz_1(G) =
    \ker(\Z^n \to G)$. By Primary decomposition (Theorem~\ref{T:primarydecomp}),
    \[
    \syz_1(G) = \bigcap_{i=1}^n N_i
    \]
    where $\ass(\Z^n/N_i) = \{\p_i\}$ and $\ass(G) =
    \{p_1\Z,\dots,p_n\Z\}$. If $\ass(\Z^n/N_i) = \{p_i\Z\}$, this
    means that $N_i = 0$ or $N_i = (\Z/p_i\Z)^j$.
  \end{proof}
\end{theorem}


\begin{theorem}[Primary decomposition and localization]
  Consider a Noetherian ring $R$, a multiplicatively closed subset of
  $S\subset R$, a finitely generated $R$-module $M$, and an
  $R$-submodule $N\subset M$, with a primary decomposition. We may
  renumber the factors $N_i$ so that for $i=1,\dots, m$, $\p_i\cap S
  =\emptyset$, and if $i=m+1,\dots, n$, $\p_i\cap S \ne\emptyset$,
  \[
  S^{-1}N = \bigcap_{i=m}^n S^{-1}N_i
  \]
  where for each $i$, $\ass(S^{-1}(M/N_i))$ contains exactly one prime ideal
  of $S^{-1}R$ and if $i\ne j$, then $\ass(S^{-1}(M/N_i)) \ne \ass(S^{-1}(M/N_j))$.
  \begin{sketch}
    Apply Proposition~\ref{P:assassinlocal}, to Theorem~\ref{T:primarydecomp}.
  \end{sketch}
\end{theorem}




\begin{theorem}[Decomposing varieties]
  Let $V\subset\A^n$ be an affine algebraic variety over an
  algebraically closed field. We may express
  \[
  V = \bigcup_{i=1}^n U_i
  \]
  uniquely and minimally where each $U_i$ is an irreducible variety.
\end{theorem}


\begin{exercise}
  Let $R$ be a Noetherian ring. Prove that $\spec(R)$ is a Noetherian topological space. 
\end{exercise}


Let's see some examples in \macaulay.

\begin{macaulay2}
R = QQ[x,y];
I = ideal (x^2,x*y);
primaryDecomposition I
ass I
\end{macaulay2}

Note, when write \texttt{ass I} in \macaulay\, it is computing
$\ass(R/I)$.

\begin{macaulay2}
R = QQ[x,y,z];
I = ideal(x^2,x*y,x*z);
primaryDecomposition I
ass I
\end{macaulay2}




%% \subsection{Induced map}


%% answer question:

%%   Given a morphism $\Phi\supp(M) \to \supp(N)$, can you induce
%%   \[
%%   \Phi^\sharp : N\to M?
%%   \]


For some interesting extra reading check out:
\begin{itemize}
\item \link[\textit{Direct methods for primary decomposition}, D.\ Eisenbud, C.\ Huneke, W.\ Vasconcelos, Inventiones mathematicae, December 1992]{https://link.springer.com/content/pdf/10.1007\%2FBF01231331.pdf}.

\item \link[\textit{The question of finitely many steps in polynomial ideal theory}, G.\ Hermann, ACM SIGSAM Bulletin, September 1995]{https://dl.acm.org/ft_gateway.cfm?id=307342&ftid=82112&dwn=1&CFID=78879703&CFTOKEN=a9e554efb35114eb-60020924-030F-8CA8-95F3344136DF695C}.


\end{itemize}



\end{document}
