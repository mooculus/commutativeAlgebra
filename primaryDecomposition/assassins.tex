\documentclass{ximera}



\usepackage{tikz-cd}
\usepackage[sans]{dsfont}

\DefineVerbatimEnvironment{macaulay2}{Verbatim}{numbers=left,frame=lines,label=Macaulay2,labelposition=topline}

%%% This next bit of code defines all our theorem environments
\makeatletter
\let\c@theorem\relax
\let\c@corollary\relax
\makeatother

\let\definition\relax
\let\enddefinition\relax

\let\theorem\relax
\let\endtheorem\relax

\let\proposition\relax
\let\endproposition\relax

\let\exercise\relax
\let\endexercise\relax

\let\question\relax
\let\endquestion\relax

\let\remark\relax
\let\endremark\relax

\let\corollary\relax
\let\endcorollary\relax


\let\example\relax
\let\endexample\relax


\let\lemma\relax
\let\endlemma\relax

\newtheoremstyle{SlantTheorem}{\topsep}{\topsep}%%% space between body and thm
		{\slshape}                      %%% Thm body font
		{}                              %%% Indent amount (empty = no indent)
		{\bfseries\sffamily}            %%% Thm head font
		{}                              %%% Punctuation after thm head
		{3ex}                           %%% Space after thm head
		{\thmname{#1}\thmnumber{ #2}\thmnote{ \bfseries(#3)}}%%% Thm head spec
\theoremstyle{SlantTheorem}
\newtheorem{theorem}{Theorem}
\newtheorem{definition}[theorem]{Definition}
\newtheorem{proposition}[theorem]{Proposition}
%% \newtheorem*{dfnn}{Definition}
%% \newtheorem{ques}{Question}[theorem]
\newtheorem{lemma}[theorem]{Lemma}
%% \newtheorem*{war}{WARNING}
%% \newtheorem*{cor}{Corollary}
%% \newtheorem*{eg}{Example}
\newtheorem*{remark}{Remark}
\newtheorem*{touchstone}{Touchstone}
\newtheorem{corollary}{Corollary}[theorem]
\newtheorem*{example}{Example}


\newtheoremstyle{Exercise}{\topsep}{\topsep} %%% space between body and thm
		{}                           %%% Thm body font
		{}                           %%% Indent amount (empty = no indent)
		{\bfseries}                  %%% Thm head font
		{)}                          %%% Punctuation after thm head
		{ }                          %%% Space after thm head
		{\thmnumber{#2}\thmnote{ \bfseries(#3)}}%%% Thm head spec
\theoremstyle{Exercise}
\newtheorem{exercise}{}[theorem]

%% \newtheoremstyle{Question}{\topsep}{\topsep} %%% space between body and thm
%% 		{\bfseries}                  %%% Thm body font
%% 		{3ex}                        %%% Indent amount (empty = no indent)
%% 		{}                           %%% Thm head font
%% 		{}                           %%% Punctuation after thm head
%% 		{}                           %%% Space after thm head
%% 		{\thmnumber{#2}\thmnote{ \bfseries(#3)}}%%% Thm head spec
\newtheoremstyle{Question}{3em}{3em} %%% space between body and thm
		{\large\bfseries}                           %%% Thm body font
		{3ex}                           %%% Indent amount (empty = no indent)
		{\bfseries}                  %%% Thm head font
		{}                          %%% Punctuation after thm head
		{ }                          %%% Space after thm head
		{}%%% Thm head spec
\theoremstyle{Question}
\newtheorem*{question}{}



\renewcommand{\tilde}{\widetilde}
\renewcommand{\bar}{\overline}
\renewcommand{\hat}{\widehat}
\newcommand{\N}{\mathbb N}
\newcommand{\Z}{\mathbb Z}
\newcommand{\R}{\mathbb R}
\newcommand{\Q}{\mathbb Q}
\newcommand{\C}{\mathbb C}
\newcommand{\V}{\mathbb V}
\newcommand{\I}{\mathbb I}
\newcommand{\A}{\mathbb A}
\newcommand{\iso}{\simeq}
\newcommand{\ph}{\varphi}
\newcommand{\Cf}{\mathcal{C}}
\newcommand{\IZ}{\mathrm{Int}(\Z)}
\newcommand{\dsum}{\oplus}
\newcommand{\directsum}{\coprod}
\newcommand{\union}{\bigcup}
\renewcommand{\i}{\mathfrak}
\renewcommand{\a}{\mathfrak{a}}
\renewcommand{\b}{\mathfrak{b}}
\newcommand{\m}{\mathfrak{m}}
\newcommand{\p}{\mathfrak{p}}
\newcommand{\q}{\mathfrak{q}}
\newcommand{\dfn}{\textbf}
\let\hom\relax
\DeclareMathOperator{\ann}{Ann}
\DeclareMathOperator{\h}{ht}
\DeclareMathOperator{\hom}{Hom}
\DeclareMathOperator{\spec}{Spec}
\DeclareMathOperator{\supp}{Supp}
\DeclareMathOperator{\ass}{Ass}
\DeclareMathOperator{\ff}{Frac}
\DeclareMathOperator{\im}{Im}
\DeclareMathOperator{\syz}{Syz}
\DeclareMathOperator{\gr}{Gr}
\renewcommand{\ker}{\mathop{\mathrm{Ker}}\nolimits}
\newcommand{\lps}{[\hspace{-0.25ex}[}
\newcommand{\rps}{]\hspace{-0.25ex}]}
\newcommand{\into}{\hookrightarrow}
\newcommand{\onto}{\twoheadrightarrow}
\newcommand{\tensor}{\otimes}
\newcommand{\x}{\mathbf{x}}
\newcommand{\X}{\mathbf X}
\newcommand{\Y}{\mathbf Y}
\renewcommand{\k}{\boldsymbol{\kappa}}
\renewcommand{\emptyset}{\varnothing}
\renewcommand{\qedsymbol}{$\blacksquare$}
\renewcommand{\l}{\ell}
\newcommand{\1}{\mathds{1}}
\newcommand{\lto}{\mathop{\longrightarrow\,}\limits}
\newcommand{\rad}{\sqrt}
\renewcommand{\vec}{\mathbf}
\renewcommand{\phi}{\varphi}
\renewcommand{\epsilon}{\varepsilon}
\renewcommand{\subset}{\subseteq}
\renewcommand{\supset}{\supseteq}
\newcommand{\macaulay}{\textsl{Macaulay2}}
\newcommand{\invlim}{\varprojlim}


%\renewcommand{\proofname}{Sketch of Proof}


\renewenvironment{proof}[1][Proof]
  {\begin{trivlist}\item[\hskip \labelsep \itshape \bfseries #1{}\hspace{2ex}]\upshape}
{\qed\end{trivlist}}

\newenvironment{sketch}[1][Sketch of Proof]
  {\begin{trivlist}\item[\hskip \labelsep \itshape \bfseries #1{}\hspace{2ex}]\upshape}
{\qed\end{trivlist}}



\makeatletter
\renewcommand\section{\@startsection{paragraph}{10}{\z@}%
                                     {-3.25ex\@plus -1ex \@minus -.2ex}%
                                     {1.5ex \@plus .2ex}%
                                     {\normalfont\large\sffamily\bfseries}}
\renewcommand\subsection{\@startsection{subparagraph}{10}{\z@}%
                                    {3.25ex \@plus1ex \@minus.2ex}%
                                    {-1em}%
                                    {\normalfont\normalsize\sffamily\bfseries}}
\makeatother

%% Fix weird index/bib issue.
\makeatletter
\gdef\ttl@savemark{\sectionmark{}}
\makeatother


\author{Bart Snapp}

\title{Assassins}

\begin{document}
\begin{abstract}
  We introduce associated primes, also known as assassins. Sources and
  references: \cite{sD2008, dE1995, mR1995}.
\end{abstract}
\maketitle

\begin{definition}
  Let $R$ be a ring and $Q$ be a $R$-module. The \dfn{assassins} of
  $Q$, denoted by $\ass(Q)$, is a set such that:
  \begin{itemize}
  \item $\ass(Q)\subset\spec(R)$.
  \item For each $\p\in\ass(Q)$, $Q$ contains a submodule isomorphic
    to $R/\p$.
  \end{itemize}
  The set of prime ideals in $\ass(Q)$ is often called the
  \dfn{associated primes} of $Q$.
\end{definition}

\begin{exercise}
  Let $Q$ be an $R$-module. Show that if $\p\in\ass(Q)$, then there
  exists and element $x\in Q$ such that $\ann(x) = \p$, hence one
  could call $\p$ an \textit{assassin}.
\end{exercise}

\begin{exercise}
  Consider $\Z_6$ as a $\Z$-module. Find the assassins of $\Z_6$.
\end{exercise}

\begin{exercise}
  Let $k$ be a field, consider the $k[x]$-module $k[x]/(x^5)$. Find
  the assassins of $k[x]/(x^5)$.
\end{exercise}

\begin{exercise}
  Let $k$ be a field, consider the $k[x,y]$-module $k[x,y]/(xy)$. Find
  the assassins of $k[x,y]/(xy)$.
\end{exercise}

\begin{exercise}
  Show that if $T$ is an $R$-submodule of the $R$-module $Q$, that
  $\ass(T) \subset \ass(Q)$.
\end{exercise}


\begin{exercise}
  Let $Q$ and $T$ be $R$-modules. Prove that $\ass(Q\dsum T) = \ass(Q)\cup\ass(T)$.
\end{exercise}


\begin{proposition}
  Let $R$ be a Noetherian ring and $Q$ be a finitely generated
  $R$-module. In this case $\ass(Q) \ne \emptyset$.
  \begin{sketch}
    Let $q\in Q$ be nonzero and consider the set of ideals:
    \[
    \{\ann(xq): x\in R\text{ and }xq\ne 0 \}
    \]
    Show this set is nonempty and argue it has a maximal element, call
    it $\p$. Show $\p$ is prime by seeking a contradiction, ending by
    constructing an ideal larger than $\p$, the maximal element.
  \end{sketch}
\end{proposition}


\begin{proposition}
  Let $Q$ be an $R$-module. If $\p\in\ass(Q)$, then $V(\p)\subset
  \supp(Q)$.
  \begin{sketch}
    
  \end{sketch}
\end{proposition}

\begin{proposition}
  If $R$ is Noetherian, then $\ass(Q)$ contains a minimal element of
  $\supp(Q)$.
\end{proposition}


\begin{proposition}
  If $R$ is a Noetherian ring and $Q$ is a finitely generated
  $R$-module, then there is a finite filtration
  \[
  0 = Q_0 \subsetneq Q_1 \subsetneq \dots \subsetneq Q_n = Q 
  \]
  where $Q_i/Q_{i+1} \iso R/\p_i$ and $\ass(Q) \subset
  \{\p_1,\dots,p_n\}$.
\end{proposition}

\begin{corollary}
  Given a Noetherian ring and a finitely generated module $Q$,
  $\ass(Q)$ is finite.
\end{corollary}





\begin{proposition}
  Let $R$ be a ring, $S$ be a multiplicatively closed subset of $R$,
  and $Q$ be an $R$-module. Then
  \[
  \p \in \ass_R(Q) \Leftrightarrow \ass_{S^{-1}}(S^{-1}(Q))
  \]
  whenever $S\cap\p = \emptyset$.
  \begin{sketch}
    $(\Rightarrow)$

    $(\Leftarrow)$
  \end{sketch}
\end{proposition}

\end{document}



