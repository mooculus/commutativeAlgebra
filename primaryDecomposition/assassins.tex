\documentclass{ximera}



\usepackage{tikz-cd}
\usepackage[sans]{dsfont}

\let\oldbibliography\thebibliography%% to compact bib
\renewcommand{\thebibliography}[1]{%
  \oldbibliography{#1}%
  \setlength{\itemsep}{0pt}%
}


\DefineVerbatimEnvironment{macaulay2}{Verbatim}{numbers=left,frame=lines,label=Macaulay2,labelposition=topline}

%%% This next bit of code defines all our theorem environments
\makeatletter
\let\c@theorem\relax
\let\c@corollary\relax
\makeatother

\let\definition\relax
\let\enddefinition\relax

\let\theorem\relax
\let\endtheorem\relax

\let\proposition\relax
\let\endproposition\relax

\let\exercise\relax
\let\endexercise\relax

\let\question\relax
\let\endquestion\relax

\let\remark\relax
\let\endremark\relax

\let\corollary\relax
\let\endcorollary\relax


\let\example\relax
\let\endexample\relax

\let\warning\relax
\let\endwarning\relax

\let\lemma\relax
\let\endlemma\relax

\newtheoremstyle{SlantTheorem}{\topsep}{\topsep}%%% space between body and thm
		{\slshape}                      %%% Thm body font
		{}                              %%% Indent amount (empty = no indent)
		{\bfseries\sffamily}            %%% Thm head font
		{}                              %%% Punctuation after thm head
		{3ex}                           %%% Space after thm head
		{\thmname{#1}\thmnumber{ #2}\thmnote{ \bfseries(#3)}}%%% Thm head spec
\theoremstyle{SlantTheorem}
\newtheorem{theorem}{Theorem}
\newtheorem{definition}[theorem]{Definition}
\newtheorem{proposition}[theorem]{Proposition}
%% \newtheorem*{dfnn}{Definition}
%% \newtheorem{ques}{Question}[theorem]
\newtheorem{lemma}[theorem]{Lemma}
%% \newtheorem*{war}{WARNING}
%% \newtheorem*{cor}{Corollary}
%% \newtheorem*{eg}{Example}
\newtheorem*{remark}{Remark}
\newtheorem*{touchstone}{Touchstone}
\newtheorem{corollary}{Corollary}[theorem]
\newtheorem*{example}{Example}
\newtheorem*{warning}{WARNING}


\newtheoremstyle{Exercise}{\topsep}{\topsep} %%% space between body and thm
		{}                           %%% Thm body font
		{}                           %%% Indent amount (empty = no indent)
		{\bfseries}                  %%% Thm head font
		{)}                          %%% Punctuation after thm head
		{ }                          %%% Space after thm head
		{\thmnumber{#2}\thmnote{ \bfseries(#3)}}%%% Thm head spec
\theoremstyle{Exercise}
\newtheorem{exercise}{}[theorem]

%% \newtheoremstyle{Question}{\topsep}{\topsep} %%% space between body and thm
%% 		{\bfseries}                  %%% Thm body font
%% 		{3ex}                        %%% Indent amount (empty = no indent)
%% 		{}                           %%% Thm head font
%% 		{}                           %%% Punctuation after thm head
%% 		{}                           %%% Space after thm head
%% 		{\thmnumber{#2}\thmnote{ \bfseries(#3)}}%%% Thm head spec
\newtheoremstyle{Question}{3em}{3em} %%% space between body and thm
		{\large\bfseries}                           %%% Thm body font
		{3ex}                           %%% Indent amount (empty = no indent)
		{\bfseries}                  %%% Thm head font
		{}                          %%% Punctuation after thm head
		{ }                          %%% Space after thm head
		{}%%% Thm head spec
\theoremstyle{Question}
\newtheorem*{question}{}



\renewcommand{\tilde}{\widetilde}
\renewcommand{\bar}{\overline}
\renewcommand{\hat}{\widehat}
\newcommand{\N}{\mathbb N}
\newcommand{\Z}{\mathbb Z}
\newcommand{\R}{\mathbb R}
\newcommand{\Q}{\mathbb Q}
\newcommand{\C}{\mathbb C}
\newcommand{\V}{\mathbb V}
\newcommand{\I}{\mathbb I}
\newcommand{\A}{\mathbb A}
\newcommand{\iso}{\simeq}
\newcommand{\ph}{\varphi}
\newcommand{\Cf}{\mathcal{C}}
\newcommand{\IZ}{\mathrm{Int}(\Z)}
\newcommand{\dsum}{\oplus}
\newcommand{\directsum}{\bigoplus}
\newcommand{\union}{\bigcup}
\renewcommand{\i}{\mathfrak}
\renewcommand{\a}{\mathfrak{a}}
\renewcommand{\b}{\mathfrak{b}}
\newcommand{\m}{\mathfrak{m}}
\newcommand{\p}{\mathfrak{p}}
\newcommand{\q}{\mathfrak{q}}
\newcommand{\dfn}[1]{\textbf{#1}\index{#1}}
\let\hom\relax
\DeclareMathOperator{\ann}{Ann}
\DeclareMathOperator{\h}{ht}
\DeclareMathOperator{\hom}{Hom}
\DeclareMathOperator{\Span}{Span}
\DeclareMathOperator{\spec}{Spec}
\DeclareMathOperator{\maxspec}{MaxSpec}
\DeclareMathOperator{\supp}{Supp}
\DeclareMathOperator{\ass}{Ass}
\DeclareMathOperator{\ff}{Frac}
\DeclareMathOperator{\im}{Im}
\DeclareMathOperator{\syz}{Syz}
\DeclareMathOperator{\gr}{Gr}
\renewcommand{\ker}{\mathop{\mathrm{Ker}}\nolimits}
\newcommand{\coker}{\mathop{\mathrm{Coker}}\nolimits}
\newcommand{\lps}{[\hspace{-0.25ex}[}
\newcommand{\rps}{]\hspace{-0.25ex}]}
\newcommand{\into}{\hookrightarrow}
\newcommand{\onto}{\twoheadrightarrow}
\newcommand{\tensor}{\otimes}
\newcommand{\x}{\mathbf{x}}
\newcommand{\X}{\mathbf X}
\newcommand{\Y}{\mathbf Y}
\renewcommand{\k}{\boldsymbol{\kappa}}
\renewcommand{\emptyset}{\varnothing}
\renewcommand{\qedsymbol}{$\blacksquare$}
\renewcommand{\l}{\ell}
\newcommand{\1}{\mathds{1}}
\newcommand{\lto}{\mathop{\longrightarrow\,}\limits}
\newcommand{\rad}{\sqrt}
\newcommand{\hf}{H}
\newcommand{\hs}{H\!S}
\newcommand{\hp}{H\!P}
\renewcommand{\vec}{\mathbf}
\renewcommand{\phi}{\varphi}
\renewcommand{\epsilon}{\varepsilon}
\renewcommand{\subset}{\subseteq}
\renewcommand{\supset}{\supseteq}
\newcommand{\macaulay}{\textsl{Macaulay2}}
\newcommand{\invlim}{\varprojlim}


%\renewcommand{\proofname}{Sketch of Proof}


\renewenvironment{proof}[1][Proof]
  {\begin{trivlist}\item[\hskip \labelsep \itshape \bfseries #1{}\hspace{2ex}]\upshape}
{\qed\end{trivlist}}

\newenvironment{sketch}[1][Sketch of Proof]
  {\begin{trivlist}\item[\hskip \labelsep \itshape \bfseries #1{}\hspace{2ex}]\upshape}
{\qed\end{trivlist}}



\makeatletter
\renewcommand\section{\@startsection{paragraph}{10}{\z@}%
                                     {-3.25ex\@plus -1ex \@minus -.2ex}%
                                     {1.5ex \@plus .2ex}%
                                     {\normalfont\large\sffamily\bfseries}}
\renewcommand\subsection{\@startsection{subparagraph}{10}{\z@}%
                                    {3.25ex \@plus1ex \@minus.2ex}%
                                    {-1em}%
                                    {\normalfont\normalsize\sffamily\bfseries}}
\makeatother

%% Fix weird index/bib issue.
\makeatletter
\gdef\ttl@savemark{\sectionmark{}}
\makeatother


\makeatletter
%% no number for refs
\newcommand\frontstyle{%
  \def\activitystyle{activity-chapter}
  \def\maketitle{%
    \addtocounter{titlenumber}{1}%
                    {\flushleft\small\sffamily\bfseries\@pretitle\par\vspace{-1.5em}}%
                    {\flushleft\LARGE\sffamily\bfseries\@title \par }%
                    {\vskip .6em\noindent\textit\theabstract\setcounter{problem}{0}\setcounter{sectiontitlenumber}{0}}%
                    \par\vspace{2em}
                    \phantomsection\addcontentsline{toc}{section}{\textbf{\@title}}%
                  }}
\makeatother


\author{Bart Snapp}

\title{Assassins}

\begin{document}
\begin{abstract}
  We introduce associated primes, also known as assassins. Sources and
  references: \cite{sD2008, dE1995, mR1995,jpS2000}.
\end{abstract}
\maketitle

\begin{definition}
  Let $R$ be a ring and $Q$ be a $R$-module. The \dfn{assassins} of
  $Q$, denoted by $\ass(Q)$, is a set such that:
  \begin{itemize}
  \item $\ass(Q)\subset\spec(R)$.
  \item For each $\p\in\ass(Q)$, $Q$ contains a submodule isomorphic
    to $R/\p$.
  \end{itemize}
  The set of prime ideals in $\ass(Q)$ is often called the
  \dfn{associated primes} of $Q$.
\end{definition}

\begin{exercise}
  Let $Q$ be an $R$-module. Show there exists $x\in Q$ such that
  $\ann(x) = \p$ if and only if $Q$ contains a submodule isomorphic to
  $R/\p$. This explains why one might call $\p$ an assassin of $Q$.
\end{exercise}

\begin{exercise}
  Consider $\Z_6$ as a $\Z$-module. Find the assassins of $\Z_6$.
\end{exercise}

\begin{exercise}
  Let $k$ be a field, consider the $k[x]$-module $k[x]/(x^5)$. Find
  the assassins of $k[x]/(x^5)$.
\end{exercise}

\begin{exercise}
  Let $k$ be a field, consider the $k[x,y]$-module $k[x,y]/(xy)$. Find
  the assassins of $k[x,y]/(xy)$.
\end{exercise}



\begin{proposition}
  Let $R$ be a Noetherian ring, and $Q$ be a finitely generated
  $R$-module. Consider
  \[
  \mathcal{S} = \{\ann(x): x\in Q\text{ and } x\ne0\}.
  \]
  The maximal elements of $\mathcal{S}$ are prime.
  \begin{proof}
    Since $Q$ is Noetherian, and $\mathcal{S}$ is nonempty, it
    contains $(0)$, $\mathcal{S}$ contains a maximal element, call it
    $\p$.  Let $x$ be a nonzero element of $Q$ such that $\ann(x) =
    \p$.  Suppose $ab\in\p$, and $a\notin\p$.  I this case,
    $\p=\ann(x) \subset \ann(ax)$. Since $\p$ is maximal, $\p =
    \ann(ax)$. However, $b\in\ann(ax) = \p$, hence $\p$ is prime.
  \end{proof}
\end{proposition}

To quote David Eisenbud, ``ideals maximal with respect to some
property have an uncanny tendency to be prime.''


\begin{corollary}[Assassins are nonempty]
  If $Q\ne 0$, then $\ass(Q) \ne \emptyset$.
  \begin{proof}
    Since $\mathcal{S}$ from above is nonempty, it contains maximal
    elements and these elements are prime.
  \end{proof}
\end{corollary}

\begin{corollary}
  If $R$ is a Noetherian ring and $Q$ is a finitely generated
  $R$-module, then there is a finite filtration
  \[
  0 = Q_0 \subsetneq Q_1 \subsetneq \dots \subsetneq Q_n = Q 
  \]
  where $Q_i/Q_{i+1} \iso R/\p_i$, where $\p_i\in\spec(R)$.
  \begin{proof}
    By the corollary above, $Q$ contains a submodule isomorphic to
    $R/\p_1$. This submodule will be $Q_1$. If $Q = Q_1$, we are
    done. Proceed inductively, suppose that we can prove our statement
    for up to $n$ submodules. If $Q \ne Q_n$, then $Q/Q_n$ contains a
    submodule isomorphic to $R/\p_{n+1}$, call it $Q_{n+1}/Q_n$. Since
    $Q$ is Noetherian, this ascending chain stabilizes.
  \end{proof}
\end{corollary}

\begin{exercise}
  Let $R$ be a Noetherian ring and $M$ be a finitely generated $R$-module. Consider the canonical map
  \[
  M\to \prod_{\p\in\ass(M)} M_\p.
  \]
  Prove that this map is injective.
\end{exercise}



\begin{proposition}\label{P:assassinlocal}
  Let $R$ be a Noetherian ring, $S$ be a multiplicatively closed
  subset of $R$, $\p\in\spec(R)$ with $\p\cap S = \emptyset$, and $Q$
  be an $R$-module. Then
  \[
  \p \in \ass_R(Q) \Leftrightarrow S^{-1}\p\in \ass_{S^{-1}R}(S^{-1}Q)
  \]
  whenever $S\cap\p = \emptyset$.
  \begin{proof}
    $(\Rightarrow)$ If $\p\in\ass_R(Q)$, then for some $x\in Q$
    \begin{align*}
    x\cdot p &= 0\\
    \frac{x}{1} \cdot \frac{p}{1} &= 0
    \end{align*}
    so $S^{-1} \p\in \ass(S^{-1}Q)$.

    $(\Leftarrow)$ If $S^{-1}\p\in \ass_{S^{-1}R}(S^{-1}Q)$, then
    there is an injective homomorphism
    \[
    \iota:\frac{S^{-1}R}{S^{-1}\p}\into S^{-1}Q
    \]
    Since $\p$ is finitely generated (see \cite[Proposition 2.10]{dE1995})
    \[
    \hom_{S^{-1}R}\left(\frac{S^{-1}R}{S^{-1}\p},S^{-1}Q\right) = S^{-1}\hom_{R}(R/\p,Q)
    \]
    Since $S\cap\p=0$ we can clear denominators of $\iota$ to find the
    injection we are looking for.
  \end{proof}
\end{proposition}





\begin{theorem}[Assassins are minimal]
  If $R$ is a Noetherian ring and $Q$ is a finitely generated
  $R$-module, consider the finite filtration
  \[
  0 = Q_0 \subsetneq Q_1 \subsetneq \dots \subsetneq Q_n = Q 
  \]
  where $Q_i/Q_{i-1} \iso R/\p_i$. In this case
  \[
  \ass(Q) \subset \{\p_1,\dots,\p_n\}\subset \supp(Q)
  \]
  and each of these sets have the same minimal elements.
  \begin{proof}
    First we'll show that $\{\p_1,\dots,\p_n\}\subset \supp(Q)$ and
    that these sets have the same minimal elements. Localize the
    finite filtration
    \[
    0 = (Q_0)_\p \subsetneq (Q_1)_\p \subsetneq \dots \subsetneq (Q_n)_\p = Q_\p,
    \]
    with the factors being
    \begin{align*}
      (Q_i)_\p/(Q_{i-1})_\p &\iso \left(Q_i/Q_{i-1}\right)_\p\\
      &\iso \left(R/\p_i\right)_\p.
    \end{align*}
    Hence, $Q_\p \ne 0$ if and only if there is $\p_i$ such that
    $\left(R/\p_i\right)_\p \ne 0$. This can only happen if
    $\p_i\subset\p$, hence $\{\p_1,\dots,\p_n\}\subset \supp(Q)$, and
    these sets have the same minimal elements.

    Now we'll show that $\ass(Q)\subset\{\p_1,\dots,\p_n\}$. Let
    $\p\in\ass(Q)$, this means that
    \[
    R/\p \iso N \subset Q.
    \]
    Intersect $N$ with our finite filtration, and consider the
    smallest $i$ such that $N\cap M_i\ne 0$. Write
    \[
    0=N\cap M_{i-1}\to N\cap M_i,
    \]
    and we must have that $R/\p \into (N\cap Q_i)/(N\cap Q_{i-1})$, and so
    \[
    R/\p \into Q_i/Q_{i-1} \iso R/\p_i
    \]
    and so $\p=\p_i$. Thus
    \[
    \ass(Q)\subset\{\p_1,\dots,\p_n\}\subset \supp(Q).
    \]

    Finally, we will show that the minimal elements of $\supp(Q)$ are
    in $\ass(Q)$. Let $\p$ be a minimal element of
    $\supp(Q)$. Localize to obtain
    \[
    \supp(Q_\p) =\{\p R_\p\}.
    \]
    Since $\ass(Q_\p)\ne \emptyset$, and
    $\ass(Q_\p)\subset\supp(Q_\p)$, we have that $\p
    R_\p\in\ass(Q_\p)$. So by Proposition~\ref{P:assassinlocal},
    $\p\in\ass(Q)$.
  \end{proof}
\end{theorem}


\begin{corollary}[Assassins are finite]
  If $R$ is a Noetherian ring and $Q$ is a finitely generated
  $R$-module, then $\ass(Q)$ is finite.
  \begin{sketch}
    Our filtration above is finite.
  \end{sketch}
\end{corollary}


\begin{proposition}\label{P:subsetassassin}
  Let $R$ be a Noetherian ring and $L,M,N$ be finitely generated
  $R$-modules. Given a short exact sequence
  \[
  0 \to L \to M \to N \to 0,
  \]
  $\ass(L) \subset \ass(M) \subset \ass(L) \cup \ass(N)$.
  \begin{proof}
    Since $L\into M$, we see $\ass(L) \subset \ass(M)$.

    Suppose that $\p\in\ass(M)$, meaning $M$ contains a submodule $Q$
    isomorphic to $R/\p$.

    If $Q\cap L \ne \emptyset$, then there exists $x\in Q\cap L$ such
    that $\p = \ann(x)$, so $\p\in\ass(L)$.

    If $Q\cap L = 0$, $Q$ is isomorphic to a submodule of $M/L\iso N$,
    and so $\p\in\ass(N)$.
  \end{proof}
\end{proposition}







\begin{lemma}[Irreducibility and assassins]\label{L:irrassassin}
  Let $R$ be a Noetherian ring and $M$ be finitely generated
  $R$-module. If $N$ is an irreducible submodule of $M$, then
  $\ass(M/N)=\{\p\}$.
  \begin{proof}
    Let $N$ be an irreducible submodule of $M$. Seeking a
    contradiction, suppose that $\ass(M/N) = \{\p,\q\}$. Then there
    are elements $\bar{x}$ and $\bar{y}$ such that $\p=\ann(\bar{x})$
    and $\q=\ann(\bar{y})$. In this case
    \[
    \bar{x}R \cap \bar{y}R = 0,
    \]
    and this means that $(xR +N)\cap(yR+N) = N$, contradicting the
    irreducibility of $N$.
  \end{proof}
\end{lemma}


\begin{lemma}[Intersections and assassins]\label{L:intersectassassin}
  Consider a Noetherian ring $R$ and a finitely generated $R$-module
  $M$ with $R$-submodules $N_1,N_2\subset M$. If
  \[
  \ass(M/N_1) = \{\p\} \quad\text{and}\quad \ass(M/N_2) = \{\p\}
  \]
  then $\ass(M/(N_1\cap N_2) = \{\p\}$.
  \begin{proof}
    The surjection
    \[
    M \to M/N_1\dsum M/N_2
    \]
    has kernel $N = N_1\cap N_2$. Hence
    \[
    M/(N_1\cap N_2)\iso M/N_1\dsum M/N_2
    \]
    We can write a finite filtration,
    \[
    0 \subset M/N_1 \subset M/N_1\dsum M/N_2
    \]
    and so
    \[
    \ass(M/(N_1\cap N_2)) \subset{\p}. 
    \]
  \end{proof}
\end{lemma}


\end{document}



