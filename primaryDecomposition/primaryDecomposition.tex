\documentclass{ximera}



\usepackage{tikz-cd}
\usepackage[sans]{dsfont}

\DefineVerbatimEnvironment{macaulay2}{Verbatim}{numbers=left,frame=lines,label=Macaulay2,labelposition=topline}

%%% This next bit of code defines all our theorem environments
\makeatletter
\let\c@theorem\relax
\let\c@corollary\relax
\makeatother

\let\definition\relax
\let\enddefinition\relax

\let\theorem\relax
\let\endtheorem\relax

\let\proposition\relax
\let\endproposition\relax

\let\exercise\relax
\let\endexercise\relax

\let\question\relax
\let\endquestion\relax

\let\remark\relax
\let\endremark\relax

\let\corollary\relax
\let\endcorollary\relax


\let\example\relax
\let\endexample\relax


\let\lemma\relax
\let\endlemma\relax

\newtheoremstyle{SlantTheorem}{\topsep}{\topsep}%%% space between body and thm
		{\slshape}                      %%% Thm body font
		{}                              %%% Indent amount (empty = no indent)
		{\bfseries\sffamily}            %%% Thm head font
		{}                              %%% Punctuation after thm head
		{3ex}                           %%% Space after thm head
		{\thmname{#1}\thmnumber{ #2}\thmnote{ \bfseries(#3)}}%%% Thm head spec
\theoremstyle{SlantTheorem}
\newtheorem{theorem}{Theorem}
\newtheorem{definition}[theorem]{Definition}
\newtheorem{proposition}[theorem]{Proposition}
%% \newtheorem*{dfnn}{Definition}
%% \newtheorem{ques}{Question}[theorem]
\newtheorem{lemma}[theorem]{Lemma}
%% \newtheorem*{war}{WARNING}
%% \newtheorem*{cor}{Corollary}
%% \newtheorem*{eg}{Example}
\newtheorem*{remark}{Remark}
\newtheorem*{touchstone}{Touchstone}
\newtheorem{corollary}{Corollary}[theorem]
\newtheorem*{example}{Example}


\newtheoremstyle{Exercise}{\topsep}{\topsep} %%% space between body and thm
		{}                           %%% Thm body font
		{}                           %%% Indent amount (empty = no indent)
		{\bfseries}                  %%% Thm head font
		{)}                          %%% Punctuation after thm head
		{ }                          %%% Space after thm head
		{\thmnumber{#2}\thmnote{ \bfseries(#3)}}%%% Thm head spec
\theoremstyle{Exercise}
\newtheorem{exercise}{}[theorem]

%% \newtheoremstyle{Question}{\topsep}{\topsep} %%% space between body and thm
%% 		{\bfseries}                  %%% Thm body font
%% 		{3ex}                        %%% Indent amount (empty = no indent)
%% 		{}                           %%% Thm head font
%% 		{}                           %%% Punctuation after thm head
%% 		{}                           %%% Space after thm head
%% 		{\thmnumber{#2}\thmnote{ \bfseries(#3)}}%%% Thm head spec
\newtheoremstyle{Question}{3em}{3em} %%% space between body and thm
		{\large\bfseries}                           %%% Thm body font
		{3ex}                           %%% Indent amount (empty = no indent)
		{\bfseries}                  %%% Thm head font
		{}                          %%% Punctuation after thm head
		{ }                          %%% Space after thm head
		{}%%% Thm head spec
\theoremstyle{Question}
\newtheorem*{question}{}



\renewcommand{\tilde}{\widetilde}
\renewcommand{\bar}{\overline}
\renewcommand{\hat}{\widehat}
\newcommand{\N}{\mathbb N}
\newcommand{\Z}{\mathbb Z}
\newcommand{\R}{\mathbb R}
\newcommand{\Q}{\mathbb Q}
\newcommand{\C}{\mathbb C}
\newcommand{\V}{\mathbb V}
\newcommand{\I}{\mathbb I}
\newcommand{\A}{\mathbb A}
\newcommand{\iso}{\simeq}
\newcommand{\ph}{\varphi}
\newcommand{\Cf}{\mathcal{C}}
\newcommand{\IZ}{\mathrm{Int}(\Z)}
\newcommand{\dsum}{\oplus}
\newcommand{\directsum}{\coprod}
\newcommand{\union}{\bigcup}
\renewcommand{\i}{\mathfrak}
\renewcommand{\a}{\mathfrak{a}}
\renewcommand{\b}{\mathfrak{b}}
\newcommand{\m}{\mathfrak{m}}
\newcommand{\p}{\mathfrak{p}}
\newcommand{\q}{\mathfrak{q}}
\newcommand{\dfn}{\textbf}
\let\hom\relax
\DeclareMathOperator{\ann}{Ann}
\DeclareMathOperator{\h}{ht}
\DeclareMathOperator{\hom}{Hom}
\DeclareMathOperator{\spec}{Spec}
\DeclareMathOperator{\supp}{Supp}
\DeclareMathOperator{\ass}{Ass}
\DeclareMathOperator{\ff}{Frac}
\DeclareMathOperator{\im}{Im}
\DeclareMathOperator{\syz}{Syz}
\DeclareMathOperator{\gr}{Gr}
\renewcommand{\ker}{\mathop{\mathrm{Ker}}\nolimits}
\newcommand{\lps}{[\hspace{-0.25ex}[}
\newcommand{\rps}{]\hspace{-0.25ex}]}
\newcommand{\into}{\hookrightarrow}
\newcommand{\onto}{\twoheadrightarrow}
\newcommand{\tensor}{\otimes}
\newcommand{\x}{\mathbf{x}}
\newcommand{\X}{\mathbf X}
\newcommand{\Y}{\mathbf Y}
\renewcommand{\k}{\boldsymbol{\kappa}}
\renewcommand{\emptyset}{\varnothing}
\renewcommand{\qedsymbol}{$\blacksquare$}
\renewcommand{\l}{\ell}
\newcommand{\1}{\mathds{1}}
\newcommand{\lto}{\mathop{\longrightarrow\,}\limits}
\newcommand{\rad}{\sqrt}
\renewcommand{\vec}{\mathbf}
\renewcommand{\phi}{\varphi}
\renewcommand{\epsilon}{\varepsilon}
\renewcommand{\subset}{\subseteq}
\renewcommand{\supset}{\supseteq}
\newcommand{\macaulay}{\textsl{Macaulay2}}
\newcommand{\invlim}{\varprojlim}


%\renewcommand{\proofname}{Sketch of Proof}


\renewenvironment{proof}[1][Proof]
  {\begin{trivlist}\item[\hskip \labelsep \itshape \bfseries #1{}\hspace{2ex}]\upshape}
{\qed\end{trivlist}}

\newenvironment{sketch}[1][Sketch of Proof]
  {\begin{trivlist}\item[\hskip \labelsep \itshape \bfseries #1{}\hspace{2ex}]\upshape}
{\qed\end{trivlist}}



\makeatletter
\renewcommand\section{\@startsection{paragraph}{10}{\z@}%
                                     {-3.25ex\@plus -1ex \@minus -.2ex}%
                                     {1.5ex \@plus .2ex}%
                                     {\normalfont\large\sffamily\bfseries}}
\renewcommand\subsection{\@startsection{subparagraph}{10}{\z@}%
                                    {3.25ex \@plus1ex \@minus.2ex}%
                                    {-1em}%
                                    {\normalfont\normalsize\sffamily\bfseries}}
\makeatother

%% Fix weird index/bib issue.
\makeatletter
\gdef\ttl@savemark{\sectionmark{}}
\makeatother


\author{Bart Snapp}

\title{Primary decomposition}

\begin{document}
\begin{abstract}
  We prove the primary decomposition theorem. Sources and references:
  \cite{sD2008,jpS2000}.
\end{abstract}
\maketitle

We are now ready to classifiy irreducible modules based on their
assassins. For your convienence, we've collected key lemmas into one lemma.

\begin{lemma}\label{L:biglempd}
  Let $R$ be a Noetherian ring and $M$ be a finitely generated
  $R$-module with a submodule $N$. The following are true:
  \begin{description}
    \item[Irreducible decomposition] The $R$-module $N\subsetneq M$
      can be written as a finite intersection of irreducible
      submodules of $M$, see Lemma~\ref{L:irrdecomp}.
    \item[Irreducibility and assassins] If $N_i$ is an irreducible
      submodule of $M$, then $\ass(M/N_i)=\{\p_i\}$, see
      Lemma~\ref{L:irrassassin}.
    \item[Intersections and assassins] If $\ass(M/N_1) = \{\p\}$ and
      $\ass(M/N_2) = \{\p\}$ then $\ass(M/(N_1\cap N_2) = \{\p\}$, see
      Lemma~\ref{L:intersectassassin}.
  \end{description}
\end{lemma}





\begin{definition}
  Given a ring $R$, an $R$-module $M$, and an $R$-submodule $N\subset
  M$, a \dfn{primary decomposition} of $N$ is
  \[
  N = \bigcap_{i=1}^n N_i
  \]
  where for each $i$, $\ass(M/N_i)$ contains exactly one prime ideal
  of $R$ and if $i\ne j$, then $\ass(M/N_i) \ne \ass(M/N_j)$.
\end{definition}


\begin{theorem}[Primary decomposition]\label{T:primarydecomp}
  Let $R$ be a Noetherian ring. Given an $R$-module $M$ and an
  $R$-submodule $N\subset M$, there exist $R$-submodules $N_i\subset
  M$ such that
  \[
  N = \bigcap_{i=1}^n N_i
  \]
  where
  \begin{enumerate}
  \item $\ass(M/N_i) = \{\p_i\}$.
  \item If $i\ne j$, then $\p_i\ne\p_j$.
  \item $\ass(M/N) = \{\p_1,\dots,\p_n\}$.
  \end{enumerate}
  \begin{sketch}
    Use Lemma~\ref{L:biglempd} to show parts (a) and (b) of the
    theorem.
    
    What remains is to show that $\ass(M/N) =
    \{\p_1,\dots,\p_n\}$. Use double containment.
    
    $(\subset)$ We will show that $\ass(M/N)
    \subset\{\p_1,\dots,\p_n\}$.
    \begin{align*}
      \frac{M}{N} = \frac{M}{N_1\cap\dots\cap N_n} &\lto^\eta \directsum_{i=1}^n M/N_i\\
        m &\mapsto (m + N_1,\dots, m+ N_n)
    \end{align*}
    Here $\ker(\eta) = N_1\cap\dots\cap N_n = N$, hence $\eta$ is an
    injection. Thus
    \[
    \ass(M/N) \subset \bigcup_{i=1}^n \ass(M/N_i) = \{\p_1,\dots,\p_n\}.
    \]


    $(\supset)$ We will show that $\{\p_1,\dots,\p_n\} \subset
    \ass(M/N)$. WLOG we'll show that $\p_1\in\ass(M/N)$. Note from
    before we have the injection
    \[
    \frac{M}{N} \into \directsum_{i=1}^n M/N_i.
    \]
    Since $\bigcap_{i=2}^n N_i/N \subset \frac{M}{N}$, this map can be
    restricted
    \[
    \bigcap_{i=2}^n N_i/N \into \directsum_{i=1}^n M/N_i.
    \]
    The injective part of this map is
    \[
    \bigcap_{i=2}^n N_i/N \into M/N_1.
    \]
    Since $\ass(M/N_1) = \{\p_1\}$, by
    Proposition~\ref{P:subsetassassin},
    $\p_1\in\ass\left(\bigcap_{i=2}^n N_i/N\right)$, and so applying
    Proposition~\ref{P:subsetassassin} again, we see
    $\p_1\in\ass(M/N)$.
  \end{sketch}
\end{theorem}



\end{document}



